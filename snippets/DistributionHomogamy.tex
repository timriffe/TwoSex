
The age combination of the male and female fertility schedules from
any given year varies greatly from the distribution that would be expected if
age and mother and age of father were selected randomly according to the two
single-sex distributions.

The expected cross-classified age distribution $\textbf{E}B(x,y)$ is defined as:

\begin{equation}
\textbf{E}B_{xy} = \frac{B_x B_y}{\int _{x = \alpha} ^\beta \int _{y = \alpha}
^\beta B_{xy} \; dx \;dy}
\end{equation}

Visual inspection of surfaces of the observed and expected birth counts
confirms they are indeed quite different: The observed surface shows a stronger
homogamy-hypergamy pattern than the expected surface. Structural hypergamy in
the expected surface owes to differences in the reproductive age-span, and will
be measured in following.

Note that that age-preference is an imprecise label for the variety
of preferences that may actually lead to observed age-combination biases. For
instance, preferences may reflect a third variable (e.g. socioeconomic
in nature) that covaries with age, so as to give the appearance of age
preferences. Furthermore, as \citet{bergstrom1989effects, bergstrom1994sweden}
demonstrate, pair matching may just as easily occur as a function of individual
preferences for event (mating, marriage) timing coupled with relative
availability, which follows partly from cohort size. This is consistent with
\citet{bhrolchain2001flexibility}, who concludes that age preferences for
mates are highly adaptive to availability conditions.

Despite this ambiguity in causes behind age combination patterns, we can create an 
index of the strength of hypergamy. 






