% Timothy L. M. Riffe
% Dissertation defense presentation
% The two-sex problem in populations structured by remaining years of life
% dry-run Berkeley demog May 2_, 2013
% final defense presentation June 26, 2013
% slides for demog dry run presentation

% --------------------------------------------------
% preamble
\documentclass{beamer}
\usetheme{Berkeley}
\usecolortheme{dove}
%\usefonttheme{structuresmallcapsserif}
\usepackage{animate}
% little red box for parts of formulas
\usepackage{amsmath}
\newcommand*{\boxedcolor}{red}
\makeatletter
\renewcommand{\boxed}[1]{\textcolor{\boxedcolor}{%
  \fbox{\normalcolor\m@th$\displaystyle#1$}}}
\makeatother
% --------------------------------------------------
% title slide
\title{The two-sex problem in populations structured by remaining years of life}
\author[Tim Riffe]{Timothy L. M. Riffe}
\institute[UAB]{
  Department of Geography \\
  Universitat Aut\`{o}noma de Barcelona \\
  Centre d'Estudis Demogr\`{a}fics \\
}
\date{June 26, 2013}

\begin{document}
%----------- titlepage ----------------------------------------------%

\begin{frame}
  \titlepage
\end{frame}

%----------- slide --------------------------------------------------%

\begin{frame}
  \frametitle{Presentation overview}

\begin{itemize}
  \item Population structure
  \item Structured demographic rates
  \item Population growth
  \item Two sexes and two ages
\end{itemize}

\end{frame}

%----------- section -------------------------------------------------%
\section{Transforming population structure}
%----------- slide --------------------------------------------------%

\begin{frame}
\frametitle{A population structured by age and sex --}
\includegraphics[width=2.8in,height=2.8in]{FiguresStatic/AgeSexGray}
\end{frame}

%----------- slide --------------------------------------------------%

\begin{frame}
\frametitle{-- is heterogeneous with respect to remaining years of life}
\includegraphics[width=2.8in,height=2.8in]{FiguresStatic/AgeSexEyHetero}
\end{frame}

%----------- slide --------------------------------------------------%
% T. Miller's equation

\begin{frame}
\frametitle{We have a formula for this heterogeneity}
\begin{block}{The population age $a$ that will die in $n$ years is, $P_{a,n}$}
 \begin{equation}
   P_{a,n} = P_a \frac{d_{a+n}}{l_a}
 \end{equation}
 \end{block}
 \vskip.5cm
 \begin{description}
 \item[$d_a$:] lifetable death distribution 
 \item[$l_a$:] survival function
 \end{description}
\pause 

Variants of this equation in T. Miller (2001), Vaupel (2009)
*This identity is generally known
\end{frame}

%----------- slide --------------------------------------------------%

\begin{frame}
\frametitle{We have a formula for this heterogeneity}
\begin{block}{Example: The population age $20$ that will die in $10$ years}
 \begin{equation}
   P_{20,10} = P_{20} \frac{d_{30}}{l_{20}} \notag
 \end{equation}
 \end{block}
\end{frame}

%----------- slide --------------------------------------------------%

%\begin{frame}
%\frametitle{We can re-stack such data by remaining years}
%\includegraphics[width=2.8in,height=2.8in]{FiguresStatic/AgeSexEyHetero}
%\end{frame}

% if you can decompose it you can recompose it
%----------- slide --------------------------------------------------%
% first animation age -> ey

\begin{frame}
\frametitle{We can re-stack such data by remaining years}
\animategraphics[controls,width=2.8in,height=2.8in]{14}{Age2eyAnimation/frame}{000}{101}
\end{frame}

%----------- slide --------------------------------------------------%
% eyPyramid (last frame of prior animation)
% NO NEED FOR THIS SLIDE- last frame of prev animation is same
%\begin{frame}
%\frametitle{A population structured by remaining years of life}
%\includegraphics[width=2.8in,height=2.8in]{FiguresStatic/eyPyramid}
%\end{frame}

%----------- slide --------------------------------------------------%
% Transformation equation

\begin{frame}
\frametitle{The thanatological transformation}
\begin{block}{The population with $y$ remaining years of life, $P_y$}
 \begin{equation}
   \onslide<2->{P_y = P_{n} = \int_{a=0}^\infty} P_a \frac{d_{a+n}}{l_a}
   \onslide<2->{\mathrm{d} a}
 \end{equation}
 \end{block}
\onslide<3->{
\vskip1cm
This is the basic age-transformation we propose: 
\vskip.3cm
Chronological $\rightarrow$ Thanatological 
}
\end{frame}

%----------- slide --------------------------------------------------%
% ey population gray

\begin{frame}
\frametitle{A population structured by remaining years of life --}
\includegraphics[width=2.8in,height=2.8in]{FiguresStatic/eyPyramidGray}
\end{frame}

%----------- slide --------------------------------------------------%
% ey poulation with age-heterogeneity

\begin{frame}
\frametitle{-- is heterogeneous with respect to age}
\includegraphics[width=2.8in,height=2.8in]{FiguresStatic/eyPyramidAgeHet}
\end{frame}

%----------- slide --------------------------------------------------%
% ey -> age animation 

\begin{frame}
\frametitle{We can return to age-struture accordingly}
\animategraphics[controls,width=2.8in,height=2.8in]{14}{ey2ageAnimation/frame}{000}{101}
\end{frame}

%----------- section ------------------------------------------------%
\section{Demographic rates}
%----------- slide --------------------------------------------------%

\begin{frame}
\frametitle{Any age-structured data can be transformed}
\animategraphics[controls,width=2.8in,height=2.8in]{20}{Ba2ByAnimation/frame}{000}{101}
\end{frame}

%----------- slide --------------------------------------------------%

\begin{frame}
\frametitle{Remaining-years classified demographic rates}
\begin{block}{Define remaining-years fertility rates, $F_y$}
\begin{equation}
F_y = \frac{B_y}{E_y}\notag
\end{equation}
\end{block}

\vskip.5cm
 \begin{description}
 \item[$B_y$:] births by remaining years of life of mother (or father)
 \item[$E_y$:] exposures by remaining years of life of mother (or father)
 \end{description}
\pause
\end{frame}

%----------- slide --------------------------------------------------%
% males
\begin{frame}
\frametitle{Some typical fertility curves by remaining years of life}
\includegraphics[width=3.2in,height=3.2in]{FiguresStatic/Fym}
\end{frame}

%----------- slide --------------------------------------------------%
% females
\begin{frame}
\frametitle{Some typical fertility curves by remaining years of life}
\includegraphics[width=3.2in,height=3.2in]{FiguresStatic/Fyf}
\end{frame}

%----------- section ------------------------------------------------%
\section{Growth}
%----------- slide --------------------------------------------------%

\begin{frame}
\frametitle{Reproduction}
\begin{figure}
\includegraphics<1>[width=3in,height=3in]{FiguresStatic/eyRepro1}
\includegraphics<2>[width=3in,height=3in]{FiguresStatic/eyRepro2}
\includegraphics<3>[width=3in,height=3in]{FiguresStatic/eyRepro3}
\end{figure}
\end{frame}

%----------- slide --------------------------------------------------%

\begin{frame}
\frametitle{Births}
\begin{center}
Births in year $t$ are incremented to the population according to the death
distribution, $d_a$.
\vskip2pt
\pause
*Thus, the shape of $d_a$ determines the underlying population structure.
\end{center}
\end{frame}

%----------- slide --------------------------------------------------%

\begin{frame}
\frametitle{Births}
\begin{align}
B(t) &= \int_{y=0}^\infty F_y P_y \mathrm{d}y\notag\\
\onslide<2->{&= \int_{y=0}^\infty \int_{a=0}^\infty F_y \boxed{B(t-a)
d_{a+y}}\mathrm{d}a \mathrm{d}y\notag}\notag
\onslide<3->{\intertext{And after many years pass\ldots}}
\onslide<4->{&= \int_{y=0}^\infty \int_{a=0}^\infty F_y \boxed{B(t)e^{-ra}
d_{a+y}}\mathrm{d}a \mathrm{d}y\notag}\notag
\end{align}
\end{frame}

%----------- slide --------------------------------------------------%

\begin{frame}
\frametitle{Renewal}
\center{The relative size of adjacent cohorts becomes constant, $e^{r}$ }
\begin{align}
\onslide<1->{&= \int_{y=0}^\infty \int_{a=0}^\infty F_y B(t)\boxed{e^{-ra}}
d_{a+y}\mathrm{d}a \mathrm{d}y\notag}\notag
\onslide<2->{\intertext{Divide out $B(t)$ to isolate the intrinsic growth
rate, $r$}
1 &= \int_{y=0}^\infty \int_{a=0}^\infty F_y d_{a+y}e^{-ra}\mathrm{d}a
\mathrm{d}y}
\end{align}
\end{frame}

%----------- slide --------------------------------------------------%

\begin{frame}
\frametitle{We can estimate $r$ from data}
stuff
\end{frame}

%----------- section ------------------------------------------------%

\section{Two sexes and two ages}

%----------- slide --------------------------------------------------%

\begin{frame}
\frametitle{A single estimate of $r$ for the population}
words
\end{frame}

%----------- slide --------------------------------------------------%

\end{document}

