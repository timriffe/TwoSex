% Timothy L. M. Riffe
% Dissertation defense presentation
% The two-sex problem in populations structured by remaining years of life
% dry-run Berkeley demog May 2_, 2013
% final defense presentation June 26, 2013
% slides for demog dry run presentation

% --------------------------------------------------
% preamble
\documentclass{beamer}
\usetheme{Berkeley}
\usecolortheme{dove}
\usefonttheme{structuresmallcapsserif}
\usepackage{animate}
% --------------------------------------------------
% title slide
\title{The two-sex problem in populations structured by remaining years of life}
\author[Tim Riffe]{Timothy L. M. Riffe}
\institute[UAB]{
  Department of Geography \\
  Universitat Aut\`{o}noma de Barcelona \\
  Centre d'Estudis Demogr\`{a}fics \\
}
\date{June 26, 2013}

\begin{document}
%----------- titlepage ----------------------------------------------%
\begin{frame}
  \titlepage
\end{frame}
%----------- slide --------------------------------------------------%
\begin{frame}
  \frametitle{Presentation overview}

\begin{itemize}
  \item Population structure
  \item Structured demographic rates
  \item Population growth
  \item Two sexes and two ages
\end{itemize}

\end{frame}
%----------- slide --------------------------------------------------%
\begin{frame}
\frametitle{A population structured by age and sex --}
\includegraphics[width=2.8in,height=2.8in]{FiguresStatic/AgeSexGray}
\end{frame}
%----------- slide --------------------------------------------------%
\begin{frame}
\frametitle{-- is heterogeneous with respect to remaining years of life}
\includegraphics[width=2.8in,height=2.8in]{FiguresStatic/AgeSexEyHetero}
\end{frame}
%----------- slide --------------------------------------------------%
% T. Miller's equation
\begin{frame}
\frametitle{We have a formula for this heterogeneity}
The population age $a$ that will die in $n$ years is, $P_{a,n}$:
 \begin{equation}
   P_{a,n} = P_a \frac{d_{a+n}}{l_a}
 \end{equation}
 \begin{description}
 \item[$d_a$:] lifetable death distribution 
 \item[$l_a$:] survival function
 \end{description}
\pause 

Variants of this equation in T. Miller (2001), Vaupel (2009)
*This identity is generally known
\end{frame}

%----------- slide --------------------------------------------------%
\begin{frame}
\frametitle{We can re-stack such data by remaining years}
\includegraphics[width=2.8in,height=2.8in]{FiguresStatic/AgeSexEyHetero}
\end{frame}
% if you can decompose it you can recompose it
%----------- slide --------------------------------------------------%
% first animation

\begin{frame}
\frametitle{We can re-stack such data by remaining years}
\animategraphics[autoplay,controls,width=2.8in,height=2.8in]{14}{Age2eyAnimation/frame}{000}{101}
\end{frame}

%----------- slide --------------------------------------------------%
% eyPyramid (last frame of prior animation)
\begin{frame}
\frametitle{A population structured by remaining years of life}
\includegraphics[width=2.8in,height=2.8in]{FiguresStatic/eyPyramid}
\end{frame}
%----------- slide --------------------------------------------------%
% Transformation equation

\begin{frame}
\frametitle{The thanatological transformation}
The population with $y$ remaining years of life is:
 \begin{equation}
   P_y = P_{a+n} = \int_{a=0}^\infty P_a \frac{d_{a+n}}{l_a} \mathrm{d} a
 \end{equation}
 
\pause
This is the basic age-transformation: 

Chronological $\rightarrow$ Thanatological 
\end{frame}

%----------- slide --------------------------------------------------%
% ey population gray

\begin{frame}
\frametitle{A population structured by remaining years of life --}
\includegraphics[width=2.8in,height=2.8in]{FiguresStatic/eyPyramidGray}
\end{frame}

%----------- slide --------------------------------------------------%
% ey poulation with age-heterogeneity

\begin{frame}
\frametitle{-- is heterogeneous with respect to age}
\includegraphics[width=2.8in,height=2.8in]{FiguresStatic/eyPyramidAgeHet}
\end{frame}

%----------- slide --------------------------------------------------%
% ey -> age animation 
\begin{frame}
\frametitle{We can return to age-struture accordingly}
\animategraphics[autoplay,controls,width=2.8in,height=2.8in]{14}{ey2ageAnimation/frame}{000}{101}
\end{frame}

%----------- slide --------------------------------------------------%
\begin{frame}
\frametitle{Any age-structured data can be transformed}

*to model renewal in populations structured by remaining years of life, we're
interested in fertility rates, $F_y$:
\begin{equation}
F_y = \frac{B_y}{E_y}
\end{equation}
 \begin{description}
 \item[$B_y$:] births by remaining years of life of mother (or father)
 \item[$E_y$:] exposures by remaining years of life of mother (or father)
 \end{description}
\pause
\end{frame}

%----------- slide --------------------------------------------------%

\begin{frame}
\frametitle{Some typical fertility curves by remaining years of life}
\includegraphics[width=2.8in,height=2.8in]{FiguresStatic/eyPyramidAgeHet}
\end{frame}

%----------- slide --------------------------------------------------%

\begin{frame}
\frametitle{Reproduction in populations structured by remaining years of life}

\end{frame}

%----------- slide --------------------------------------------------%
%----------- slide --------------------------------------------------%
%----------- slide --------------------------------------------------%
%----------- slide --------------------------------------------------%
\end{document}

