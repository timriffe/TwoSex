% Timothy L. M. Riffe
% Dissertation defense presentation
% The two-sex problem in populations structured by remaining years of life
% dry-run Berkeley demog May 2_, 2013
% final defense presentation June 26, 2013
% slides for demog dry run presentation

% --------------------------------------------------
% preamble
\documentclass{beamer}
\usetheme{Berkeley}
\usecolortheme{dove}
%\usefonttheme{structuresmallcapsserif}
\usepackage{animate}
% little red box for parts of formulas
\usepackage{amsmath}
\newcommand*{\boxedcolor}{red}
\makeatletter
\renewcommand{\boxed}[1]{\textcolor{\boxedcolor}{%
  \fbox{\normalcolor\m@th$\displaystyle#1$}}}
\makeatother
% to remove caption prefixes
\usepackage[labelformat=empty]{caption}
\usepackage{subcaption}
\newcommand{\dd}{\; \mathrm{d}}
% --------------------------------------------------
% title slide
\title[2-sex problem and remaining years structure]{The two-sex problem in
populations structured by remaining years of life}
\author[Tim Riffe]{Timothy L. M. Riffe} 
\institute[UAB]{Director: Dr. Albert Esteve Pal\'{o}s \\
\vspace{2em}
Universitat Aut\`{o}noma de Barcelona \\ 
Centre d'Estudis Demogr\`{a}fics }
\date{June 26, 2013}

\begin{document}
%----------- titlepage ----------------------------------------------%

\begin{frame}[plain]
\vspace{3em}
\LARGE The two-sex problem in populations structured by remaining years of life
\\
\vspace{1em}
\large Timothy L. M. Riffe \\
\vspace{1em}
\normalsize Director: Dr. Albert Esteve Pal\'{o}s \\
\vspace{1em}
Universitat Aut\`{o}noma de Barcelona \\ 
Centre d'Estudis Demogr\`{a}fics 
\vspace{5em}
\begin{figure}
        \centering
        \begin{subfigure}[b]{0.3\textwidth}
                \centering
                \includegraphics[width=.7in]{FiguresStatic/UABlogo}
        \end{subfigure}%
        ~ %add desired spacing between images, e. g. ~, \quad, \qquad etc.
          %(or a blank line to force the subfigure onto a new line)
        \begin{subfigure}[b]{0.3\textwidth}
                \centering
                \includegraphics[width=.7in]{FiguresStatic/CEDlogo}
        \end{subfigure}
        ~ %add desired spacing between images, e. g. ~, \quad, \qquad etc.
          %(or a blank line to force the subfigure onto a new line)
        \begin{subfigure}[b]{0.3\textwidth}
                \centering
                \includegraphics[width=.7in]{FiguresStatic/AGAURlogo}
        \end{subfigure}
\end{figure}
\end{frame}

%----------- section ------------------------------------------------%

\section{Two-sex problem}

%----------- slide --------------------------------------------------%

\begin{frame}
  \frametitle{What is the two-sex problem?}
  \onslide<2->\begin{quotation}
  \noindent Disagreement between male and female models and measures of
  reproductivity  \end{quotation}
  \onslide<3->($TFR$, $R_0$, $r$, birth or marriage predictions) \\
  \vspace{2em}
  \onslide<3-> *We examine the intrinsic growth rate, $r$
\end{frame}
%----------- slide --------------------------------------------------%

\begin{frame}
\setbeamercovered{transparent}
\usebeamercolor{normal text}
   \frametitle{Objectives and outline}
   \begin{enumerate}[<+->]
   \item Measure and decompose the two-sex problem in $r$
   \item Compare two-sex methods using data 
   \item Translate renewal model to remaining years structure
   \item Compare renewal models under both definitions of age 
   \item Decompose $r$ sex gap in remaining-years renewal models
   \item Apply two-sex methods to remaining-years populations 
   \end{enumerate}
\end{frame}

\begin{frame}

\usebeamercolor{normal text}
   \frametitle{Objectives and outline}
   \begin{enumerate}
   \item Measure and decompose the two-sex problem in $r$
   \item \textcolor[gray]{.7}{Compare two-sex methods using data}
   \item Translate renewal model to remaining years structure
   \item Compare renewal models under both definitions of age 
   \item Decompose $r$ sex gap in remaining-years renewal models
   \item \textcolor[gray]{.7}{Apply two-sex methods to
   remaining-years populations}
   \end{enumerate}
\end{frame}
%----------- slide --------------------------------------------------%

\begin{frame}
  \frametitle{What is $r$?}
  The intrinsic growth rate, $r$, falls out of this formula
  \begin{equation}
  1 = \int _{a=0}^\infty l_ae^{-ra}f_a \dd a \notag \quad \quad
  \mathrm{\cite{sharpe1911problem}}
  \end{equation}
  $l_a$:  survival \\
  $e^{-ra}$: cohort scalar \\
  $f_a$: fertility \\
  \vspace{1em}
  \onslide<2-> *$f_a$ and $l_a$ can refer to either males or females, usually
  females
\end{frame}

%----------- slide --------------------------------------------------%

\begin{frame}
  \frametitle{Males and females have different values of $r$}
  \vspace{-2em}
  \begin{figure}
  \centering
  \caption*{Single-sex $r$, Spain and US}
  \includegraphics[scale=.55]{FiguresStatic/rSingleSexLotka}
\end{figure}
\end{frame}

%----------- slide --------------------------------------------------%

\begin{frame}
  \frametitle{The sex-gap in $r$ is decomposable}
  \vspace{-2em}
\begin{figure}
\centering
\caption*{Components to sex-gap in $r$; US, 1969-2009}
\includegraphics[scale=.55]{FiguresStatic/DecomprUS}
\end{figure}
\end{frame}

%----------- slide --------------------------------------------------%

\begin{frame}
  \frametitle{The sex-gap in $r$ is decomposable}
  \vspace{-2em}
\begin{figure}
\centering
\caption*{Components to sex-gap in $r$; Spain, 1975-2009}
\includegraphics[scale=.55]{FiguresStatic/DecomprES}
\end{figure}
\end{frame}

%----------- slide --------------------------------------------------%

\begin{frame} 
\setbeamercovered{transparent}
  
  \frametitle{State of the two-sex problem}
  \begin{itemize}[<+->]
    \item \cite{kuczynski1932fertility} pointed it out
    \item \cite{karmel1947relations,karmel1948analysis,
    karmel1948measurement, karmel1948relations} formalized it, and introduced marriage
    \item \cite{pollard1948measurement} first semi-popular solution
    \item \cite{kendall1949stochastic} made it stochastic
    \item \cite{goodman1953population, goodman1967age} weighted mean
    \item \cite{henry1972nuptiality} iterative matrix decomposition
    \item \cite{mcfarland1972comparison, mc1975models} IPF, axiomatic
    approach
  \end{itemize}
\end{frame}

%----------- slide --------------------------------------------------%

\begin{frame}
  \setbeamercovered{transparent}
  \frametitle{State of the two-sex problem}
  \begin{itemize}[<+->]
    \item \cite{schoen1977two, schoen1978standardized, schoen1981harmonic} 
    harmonic mean popularity
    \item \cite{marriage1981warren} parametric, general equilibrium
    \item \cite{pollak1986reformulation} a flexible matrix framework
    \item \cite{hadeler1988models, hadeler1989pair} introduced to epidemiology,
    ODE framework.
    \item \cite{billari2002wedding}
    \item \cite{choo2006estimating} interage spillover effects
    \item Present work: remaining years structure
  \end{itemize}
  
%----------- slide --------------------------------------------------%
\end{frame}
\begin{frame}
  \frametitle{Methods implemented in this dissertation}
  \begin{itemize}
     \item \cite{pollard1948measurement} first parsimonious
     \item \cite{mitra1978derivation} fixed SRB
     \item \cite{goodman1967age} dominace weights
     \item \cite{gupta1978general} joint age
     \item \cite{schoen1981harmonic} joint age means
     \item \cite{mc1975models} IPF
  \end{itemize}
\end{frame}

\begin{frame}
\setbeamercolor{normal text}{fg=gray,bg=}
\setbeamercolor{alerted text}{fg=black,bg=}
  \frametitle{Methods implemented in this dissertation}
  \begin{itemize}
     \item \cite{pollard1948measurement} first parsimonious
     \item \cite{mitra1978derivation} fixed SRB
     \alert{ \item \cite{goodman1967age} dominace weights}
     \item \cite{gupta1978general} joint age
     \alert{ \item \cite{schoen1981harmonic} and means in general}
     \alert{ \item \cite{mc1975models} IPF} 
     \item \alert{ [Riffe, 2013] ratio-adjustment method}
    \end{itemize}
\end{frame}

%----------- section -------------------------------------------------%

\section{Population structure}

%----------- slide --------------------------------------------------%

\begin{frame}
\frametitle{What is population \textit{structure}?}
Any meaningful way to segment demographic rates, most notably age
and sex, but also \ldots \\
\vspace{1em}
\onslide<2-> life stages, parity groups, socioeconomic groups, e.g. education,
i.e. anything that comes to mind when people say \textbf{``demographics''} \\
\vspace{1em}
\onslide<3-> We just saw that sex matters for the classic renewal equation. \\
\vspace{1em}
\onslide<3-> So does the concept of \textbf{age}
\end{frame}

%----------- slide --------------------------------------------------%

\begin{frame}
\frametitle{Demography and age}
All demographic phenomena are a function of age \\
\vspace{1em}
\onslide<2-> Age is also a function of demographic phenomena \\
\begin{figure}
\centering
\includegraphics<1-2>[scale=.7]{FiguresStatic/AgeFiller}
\includegraphics<3>[scale=.7]{FiguresStatic/AgeChrono}
\includegraphics<4>[scale=.7]{FiguresStatic/AgeThano}
\end{figure}
\end{frame}

%----------- slide --------------------------------------------------%

\begin{frame}
\frametitle{Age and demography}
\begin{figure}
\centering
\begin{minipage}{.5\textwidth}
  \centering
  \caption{Chronos}
  \includegraphics[width=.6\linewidth]{FiguresStatic/Chronos}
\end{minipage}%
\begin{minipage}{.5\textwidth}
  \centering
  \caption{Thanatos}
  \includegraphics[width=.5\linewidth]{FiguresStatic/Thanatos}
\end{minipage}
\end{figure}
\end{frame}

%----------- slide --------------------------------------------------%

\begin{frame}
\frametitle{A population structured by age and sex --}
\vskip-.5cm
\includegraphics[width=3in,height=3in]{FiguresStatic/AgeSexGray}
\end{frame}

%----------- slide --------------------------------------------------%

\begin{frame}
\frametitle{-- is heterogeneous with respect to remaining years of life}
\vskip-.5cm
\includegraphics[width=3in,height=3in]{FiguresStatic/AgeSexEyHetero}
\end{frame}

%----------- slide --------------------------------------------------%
% T. Miller's equation

\begin{frame}
\frametitle{We have a formula for this heterogeneity}
\begin{block}{The population age $a$ that will die in $n$ years is, $P_{a,n}$}
 \begin{equation}
   P_{a,n} = P_a \frac{d_{a+n}}{l_a}
 \end{equation}
 \end{block}
 \vskip.5cm
 \begin{description}
 \item[$d_a$:] lifetable death distribution 
 \item[$l_a$:] survival function
 \end{description}
\pause 

Variants of this equation in T. Miller (2001), Vaupel (2009)
*This identity is generally known
\end{frame}

%----------- slide --------------------------------------------------%

%\begin{frame}
%\frametitle{We can re-stack such data by remaining years}
%\includegraphics[width=2.8in,height=2.8in]{FiguresStatic/AgeSexEyHetero}
%\end{frame}

% if you can decompose it you can recompose it
%----------- slide --------------------------------------------------%
% first animation age -> ey

\begin{frame}
\frametitle{We can re-stack such data by remaining years}
\vskip-.5cm
\animategraphics[controls,width=2.8in,height=2.8in]{14}{Age2eyAnimation/frame}{000}{101}
\end{frame}

%----------- slide --------------------------------------------------%
% eyPyramid (last frame of prior animation)
% NO NEED FOR THIS SLIDE- last frame of prev animation is same
%\begin{frame}
%\frametitle{A population structured by remaining years of life}
%\includegraphics[width=2.8in,height=2.8in]{FiguresStatic/eyPyramid}
%\end{frame}

%----------- slide --------------------------------------------------%
% Transformation equation

\begin{frame}
\frametitle{The thanatological transformation}
\begin{block}{The population with $y$ remaining years of life, $P_y$}
 \begin{equation}
   \onslide<2->{P_y = \int_{a=0}^\infty} P_a \frac{d_{a+y}}{l_a}
   \onslide<2->{\dd a}
 \end{equation}
 \end{block}
\onslide<3->{
\vskip1cm
This is the basic age-transformation we propose: 
\vskip.3cm
Chronological $\rightarrow$ Thanatological 
}
\end{frame}

%----------- slide --------------------------------------------------%
% ey population gray

%\begin{frame}
%\frametitle{A population structured by remaining years of life --}
%\vskip-.5cm
%\includegraphics[width=3in,height=3in]{FiguresStatic/eyPyramidGray}
%\end{frame}

%----------- slide --------------------------------------------------%
% ey poulation with age-heterogeneity

%\begin{frame}
%\frametitle{-- is heterogeneous with respect to age}
%\vskip-.5cm
%\includegraphics[width=3in,height=3in]{FiguresStatic/eyPyramidAgeHet}
%\end{frame}

%----------- slide --------------------------------------------------%
% ey -> age animation 

%\begin{frame}
%\frametitle{We can return to age-struture accordingly}
%\vskip-.5cm
%\animategraphics[controls,width=2.8in,height=2.8in]{14}{ey2ageAnimation/frame}{000}{101}
%\end{frame}

%----------- section ------------------------------------------------%

\section[Rates]{Demographic rates}

%----------- slide --------------------------------------------------%

\begin{frame}
\frametitle{Any age-structured data can be restructured like this}
\vskip-.5cm
\animategraphics[controls,width=2.8in,height=2.8in]{20}{Ba2ByAnimation/frame}{000}{101}
\end{frame}

%----------- slide --------------------------------------------------%

\begin{frame}
\frametitle{Remaining-years classified demographic rates}
Define remaining-years fertility rates, $F_y$
\begin{equation}
F_y = \frac{B_y}{E_y}\notag
\end{equation}

\vskip.5cm
 \begin{description}
 \item[$B_y$:] births by remaining years of life of mother/father
 \item[$E_y$:] exposures by remaining years of life of mother/father
 \end{description}
\end{frame}

%----------- slide --------------------------------------------------%

% males
\begin{frame}
\frametitle{Male fertility rates, US, 1975 and 2009}
\vspace{-5em}
%\includegraphics[width=3.2in,height=3.2in]{FiguresStatic/Fym}
\begin{figure}
        \centering
        \begin{subfigure}[b]{0.5\textwidth}
                \centering
                \caption*{ASFR}
                \includegraphics[width=\textwidth]{FiguresStatic/Fx1}
        \end{subfigure}%
        ~ %add desired spacing between images, e. g. ~, \quad, \qquad etc.
          %(or a blank line to force the subfigure onto a new line)
        \begin{subfigure}[b]{0.5\textwidth}
                \centering
                \caption*{$e$SFR}
                \includegraphics[width=\textwidth]{FiguresStatic/Fy1}
        \end{subfigure}
\end{figure}
\end{frame}

%----------- slide --------------------------------------------------%

% females
\begin{frame}
\frametitle{Female fertility rates, US, 1975 and 2009}
\vspace{-5em}
%\includegraphics[width=3.2in,height=3.2in]{FiguresStatic/Fym}
\begin{figure}
        \centering
        \begin{subfigure}[b]{0.5\textwidth}
                \centering
                \caption*{ASFR}
                \includegraphics[width=\textwidth]{FiguresStatic/Fx2}
        \end{subfigure}%
        ~ %add desired spacing between images, e. g. ~, \quad, \qquad etc.
          %(or a blank line to force the subfigure onto a new line)
        \begin{subfigure}[b]{0.5\textwidth}
                \centering
                \caption*{$e$SFR}
                \includegraphics[width=\textwidth]{FiguresStatic/Fy2}
        \end{subfigure}
\end{figure}
\end{frame}

%----------- slide --------------------------------------------------%

% males
\begin{frame}
\frametitle{Male fertility rates, Spain, 1975 and 2009}
\vspace{-5em}
%\includegraphics[width=3.2in,height=3.2in]{FiguresStatic/Fym}
\begin{figure}
        \centering
        \begin{subfigure}[b]{0.5\textwidth}
                \centering
                \caption*{ASFR}
                \includegraphics[width=\textwidth]{FiguresStatic/Fx3}
        \end{subfigure}%
        ~ %add desired spacing between images, e. g. ~, \quad, \qquad etc.
          %(or a blank line to force the subfigure onto a new line)
        \begin{subfigure}[b]{0.5\textwidth}
                \centering
                \caption*{$e$SFR}
                \includegraphics[width=\textwidth]{FiguresStatic/Fy3}
        \end{subfigure}
\end{figure}
\end{frame}

%----------- slide --------------------------------------------------%

% females
\begin{frame}
\frametitle{Female fertility rates, Spain, 1975 and 2009}
\vspace{-5em}
%\includegraphics[width=3.2in,height=3.2in]{FiguresStatic/Fym}
\begin{figure}
        \centering
        \begin{subfigure}[b]{0.5\textwidth}
                \centering
                \caption*{ASFR}
                \includegraphics[width=\textwidth]{FiguresStatic/Fx4}
        \end{subfigure}%
        ~ %add desired spacing between images, e. g. ~, \quad, \qquad etc.
          %(or a blank line to force the subfigure onto a new line)
        \begin{subfigure}[b]{0.5\textwidth}
                \centering
                \caption*{$e$SFR}
                \includegraphics[width=\textwidth]{FiguresStatic/Fy4}
        \end{subfigure}
\end{figure}
\end{frame}

%----------- section ------------------------------------------------%

\section{Renewal}

%----------- slide --------------------------------------------------%

\begin{frame}
\frametitle{Reproduction}
\vspace{-5em}
\begin{figure}
        \centering
        \begin{subfigure}[b]{0.5\textwidth}
                \centering
                \caption*{Chronological}
                \includegraphics[width=\textwidth]{FiguresStatic/exRepro1}
        \end{subfigure}%
        ~ %add desired spacing between images, e. g. ~, \quad, \qquad etc.
          %(or a blank line to force the subfigure onto a new line)
        \begin{subfigure}[b]{0.5\textwidth}
                \centering
                \caption*{Thanatological}
                \includegraphics[width=\textwidth]{FiguresStatic/eyRepro1}
        \end{subfigure}
\end{figure}
\end{frame}
%----------- slide --------------------------------------------------%

\begin{frame}
\frametitle{Reproduction}
\vspace{-5em}
\begin{figure}
        \centering
        \begin{subfigure}[b]{0.5\textwidth}
                \centering
                \caption*{Chronological}
                \includegraphics[width=\textwidth]{FiguresStatic/exRepro2}
        \end{subfigure}%
        ~ %add desired spacing between images, e. g. ~, \quad, \qquad etc.
          %(or a blank line to force the subfigure onto a new line)
        \begin{subfigure}[b]{0.5\textwidth}
                \centering
                \caption*{Thanatological}
                \includegraphics[width=\textwidth]{FiguresStatic/eyRepro2}
        \end{subfigure}
\end{figure}
\end{frame}

%----------- slide --------------------------------------------------%

\begin{frame}
\frametitle{Reproduction}
\vspace{-5em}
\begin{figure}
        \centering
        \begin{subfigure}[b]{0.5\textwidth}
                \centering
                \caption*{Chronological}
                \includegraphics[width=\textwidth]{FiguresStatic/exRepro3}
        \end{subfigure}%
        ~ %add desired spacing between images, e. g. ~, \quad, \qquad etc.
          %(or a blank line to force the subfigure onto a new line)
        \begin{subfigure}[b]{0.5\textwidth}
                \centering
                \caption*{Thanatological}
                \includegraphics[width=\textwidth]{FiguresStatic/eyRepro3}
        \end{subfigure}
\end{figure}
\end{frame}

%----------- slide --------------------------------------------------%

\begin{frame}
\frametitle{Reproduction}
\vspace{-5em}
\begin{figure}
        \centering
        \begin{subfigure}[b]{0.5\textwidth}
                \centering
                \caption*{Chronological}
                \includegraphics[width=\textwidth]{FiguresStatic/exRepro4}
        \end{subfigure}%
        ~ %add desired spacing between images, e. g. ~, \quad, \qquad etc.
          %(or a blank line to force the subfigure onto a new line)
        \begin{subfigure}[b]{0.5\textwidth}
                \centering
                \caption*{Thanatological}
                \includegraphics[width=\textwidth]{FiguresStatic/eyRepro4}
        \end{subfigure}
\end{figure}
\end{frame}

%----------- slide --------------------------------------------------%

\begin{frame}
\frametitle{Births}
\begin{align}
B(t) &= \int_{y=0}^\infty f_y P_y \dd y\notag\\
\onslide<2->{&= \int_{y=0}^\infty \int_{a=0}^\infty f_y \boxed{B(t-a)
d_{a+y}}\dd a \dd y\notag}\notag
\onslide<3->{\intertext{And after many years pass\ldots}}
\onslide<4->{&= \int_{y=0}^\infty \int_{a=0}^\infty f_y \boxed{B(t)e^{-ra}
d_{a+y}}\dd a \dd y\notag}\notag
\end{align}
\end{frame}

%----------- slide --------------------------------------------------%

\begin{frame}
\frametitle{Renewal}
\center{The relative size of adjacent cohorts becomes constant, $e^{r}$ }
\begin{align}
\onslide<1->{B(t) &= \int_{y=0}^\infty \int_{a=0}^\infty f_y
B(t)\boxed{e^{-ra}} d_{a+y}\dd a \dd y\notag}\notag
\onslide<2->{\intertext{Divide out $B(t)$ to isolate the intrinsic growth
rate, $r$}
1 &= \int_{y=0}^\infty \int_{a=0}^\infty f_y d_{a+y}e^{-ra}\dd a\dd y}
\onslide<3->{\intertext{Compare with \cite{sharpe1911problem}}
1 &=
\int_{a=0}^\infty f_al_ae^{-ra}\dd a \notag}
\end{align}
\end{frame}

%----------- slide --------------------------------------------------%

\begin{frame}
\frametitle{Renewal}
Again, relating to \cite{sharpe1911problem}. Chronological renewal:
\begin{align}
1 &= \int_{a=0}^\infty f_al_ae^{-ra}\dd a \notag \\
  &= \int_{a=0}^\infty \int_{y=0}^\infty f_a d_{a+y}e^{-ra} \dd y \dd a \notag
  \\
  \intertext{Thanatological renewal:}
1 &= \int_{y=0}^\infty \int_{a=0}^\infty f_y d_{a+y}e^{-ra} \dd a \dd y
   \notag
 \end{align}
\end{frame}

%----------- slide --------------------------------------------------%

\begin{frame}
\frametitle{Renewal}
\onslide<1-> Vital rates either refer to males or females \\
\vspace{1em}
\onslide<2-> Male $r$ $\ne$ Female $r$  \\
\onslide<3-> \quad (Except in a stationary population) \\
\vspace{1em}
\onslide<4-> Chronological $r$ $\ne$ thanatological $r$ \\
\onslide<5-> \quad (Except in a stationary population)
\end{frame}

%----------- slide --------------------------------------------------%

\begin{frame}
\frametitle{Single-sex intrinsic growth rates, $r$}
\vspace{-2em}
\begin{figure}
      \centering
      \caption*{Lotka's $r$, US 1969-2009}
      \includegraphics[scale = .55]{FiguresStatic/rSingleSex1}
\end{figure}
\end{frame}

\begin{frame}
\frametitle{Single-sex intrinsic growth rates, $r$}
\vspace{-2em}
\begin{figure}
      \centering
      \caption*{Lotka's vs Thanatological $r$, US 1969-2009}
      \includegraphics[scale = .55]{FiguresStatic/rSingleSex2}
\end{figure}
\end{frame}

\begin{frame}
\frametitle{Single-sex intrinsic growth rates, $r$}
\vspace{-2em}
\begin{figure}
      \centering
      \caption*{Lotka's $r$, Spain 1975-2009}
      \includegraphics[scale = .55]{FiguresStatic/rSingleSex3}
\end{figure}
\end{frame}

\begin{frame}
\frametitle{Single-sex intrinsic growth rates, $r$}
\vspace{-2em}
\begin{figure}
      \centering
      \caption*{Lotka's vs Thanatological $r$, Spain 1975-2009}
      \includegraphics[scale = .55]{FiguresStatic/rSingleSex4}
\end{figure}
\end{frame}
% note differences between males vs females and chrono vs thano
%----------- section ------------------------------------------------%

\section{Two sexes and two ages}

%----------- slide --------------------------------------------------%

\begin{frame}
  \frametitle{The sex-gap in thanatological $r$ is decomposable}
  \vspace{-2em}
\begin{figure}
\centering
\caption*{Components to sex-gap in $r$; US, 1969-2009}
\includegraphics[scale=.55]{FiguresStatic/DecomprThanosUS}
\end{figure}
\end{frame}

%----------- slide --------------------------------------------------%

\begin{frame}
  \frametitle{The sex-gap in thanatological $r$ is decomposable}
  \vspace{-2em}
\begin{figure}
\centering
\caption*{Components to sex-gap in $r$; Spain, 1975-2009}
\includegraphics[scale=.55]{FiguresStatic/DecomprThanosES}
\end{figure}
\end{frame}

%----------- slide --------------------------------------------------%

\begin{frame}
\setbeamercovered{transparent}
  \frametitle{Conclusions from the dissertation}
  \begin{itemize}[<+->]
    \item Two-sex problem is usually non-trivial.
    \item Composition of sex-gap changes over time.
    \item Most two-sex methods give similar results.
    \item Any age data can be thanatologically structured.
    \item Thanatological renewal models are a new \textit{family} of models.
    \item Thanatological populations are more stable, less erratic.
    \item Stability carries over to the two-sex problem.
  \end{itemize}
\end{frame}

%----------- slide --------------------------------------------------%

\begin{frame}
\setbeamercovered{transparent}
  \frametitle{Agenda}
  \begin{itemize}[<+->]
    \item Apply two-sex models to wider range of years and populations.
    \item Produce proofs for uniqueness of solutions and ergodicity.
    \item Translate perspective to other terminal events.
    \item Propose cross-discipline applications.
  \end{itemize}
\end{frame}

%----------- slide --------------------------------------------------%

\begin{frame}[plain]
\vspace{3em}
\LARGE The two-sex problem in populations structured by remaining years of life
\\
\vspace{1em}
\large Timothy L. M. Riffe \\
\vspace{1em}
\normalsize Director: Dr. Albert Esteve Pal\'{o}s \\
\vspace{1em}
Universitat Aut\`{o}noma de Barcelona \\ 
Centre d'Estudis Demogr\`{a}fics 
\vspace{5em}
\begin{figure}
        \centering
        \begin{subfigure}[b]{0.3\textwidth}
                \centering
                \includegraphics[width=.7in]{FiguresStatic/UABlogo}
        \end{subfigure}%
        ~ %add desired spacing between images, e. g. ~, \quad, \qquad etc.
          %(or a blank line to force the subfigure onto a new line)
        \begin{subfigure}[b]{0.3\textwidth}
                \centering
                \includegraphics[width=.7in]{FiguresStatic/CEDlogo}
        \end{subfigure}
        ~ %add desired spacing between images, e. g. ~, \quad, \qquad etc.
          %(or a blank line to force the subfigure onto a new line)
        \begin{subfigure}[b]{0.3\textwidth}
                \centering
                \includegraphics[width=.7in]{FiguresStatic/AGAURlogo}
        \end{subfigure}
\end{figure}
\end{frame}

%-----------------------------------------------------------

\begin{frame}[allowframebreaks]
  \bibliographystyle{apalike}
  \bibliography{References}   % Use the BibTeX file ``References.bib''.
\end{frame}

%-----------------------------------------------------------

\end{document}

