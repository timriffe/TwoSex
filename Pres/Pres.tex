% Timothy L. M. Riffe
% Dissertation defense presentation
% The two-sex problem in populations structured by remaining years of life
% dry-run Berkeley demog May 2_, 2013
% final defense presentation June 26, 2013
% slides for demog dry run presentation

% --------------------------------------------------
% preamble
\documentclass{beamer}
\usetheme{Berkeley}
\usecolortheme{dove}
%\usefonttheme{structuresmallcapsserif}
\usepackage{animate}
% little red box for parts of formulas
\usepackage{amsmath}
\newcommand*{\boxedcolor}{red}
\makeatletter
\renewcommand{\boxed}[1]{\textcolor{\boxedcolor}{%
  \fbox{\normalcolor\m@th$\displaystyle#1$}}}
\makeatother
% to remove caption prefixes
\usepackage[labelformat=empty]{caption}
\usepackage{subcaption}
% --------------------------------------------------
% title slide
\title{The two-sex problem in populations structured by remaining years of life}
\author[Tim Riffe]{Timothy L. M. Riffe} 
\institute[UAB]{\raggedright{\underline{Director}: \\ 
\quad Dr. Albert Esteve Pal\'{o}s \\
\underline{Tribunal}: \\ 
\quad Dr. Julio P\'{e}rez D\'{i}az \\ 
\quad Dr. I\~{n}aki Permanyer Ugartemendia\\ 
\quad Dr. Trifon Missov \\
Department of Geography \\ 
Universitat Aut\`{o}noma de Barcelona \\ 
Centre d'Estudis Demogr\`{a}fics }}
\date{June 26, 2013}

\begin{document}
%----------- titlepage ----------------------------------------------%

\begin{frame}
  \titlepage
\end{frame}

%----------- slide --------------------------------------------------%
\section{Intro}
\begin{frame}
  \frametitle{What is the two-sex problem?}
  \onslide<2->\begin{quotation}
  Disagreement between male and female models and measures of
  reproductivity ($TFR$, $R_0$, $r$, birth or marriage predictions)
  \end{quotation}
  \onslide<3-> Formal demographers and others have been working on this for a
  long time -- from Karmel (1947) to Garibaldi and Sobottka (2013)\\
  \onslide<4-> *We examine the intrinsic growth rate, $r$
\end{frame}

%----------- slide --------------------------------------------------%

\begin{frame}
  \frametitle{What is $r$?}
  \onslide<1-> The intrinsic growth rate, $r$, falls out of this formula
  \begin{equation}
  1 = \int _{a=0}^\infty l_ae^{-ra}f_a \mathrm{d} a \notag \quad \quad
  \mathrm{(Lotka, 1911)}
  \end{equation}
  \onslide<2-> $l_a$ is the survival function\\
  \onslide<3-> $e^{-ra}$ scales $l_a$ to account for growth over cohorts\\
  \onslide<4-> $f_a$ are exact fertility probabilities \\
  \onslide<5-> $f_a$ and $l_a$ can refer to either males or females, usually
  females
\end{frame}

%----------- slide --------------------------------------------------%

\begin{frame}
  \frametitle{Males and females have different values of $r$}
  \vspace{-2em}
  \begin{figure}
  \centering
  \caption*{Single-sex $r$, Spain and US}
  \includegraphics[scale=.55]{FiguresStatic/rSingleSexLotka}
\end{figure}
\end{frame}

%----------- slide --------------------------------------------------%

\begin{frame}
\frametitle{What is population \textit{structure}?}
\onslide<2-> Any meaningful way to segment demographic rates, most notably age
and sex, but also \ldots \\
\vspace{1em}
\onslide<3-> life stages, parity groups, socioeconomic groups, e.g. education,
i.e. anything that comes to mind when people say \textbf{``demographics''} \\
\vspace{1em}
\onslide<4-> We just saw that sex matters for the classic renewal equation. \\
\vspace{1em}
\onslide<5-> So does the concept of \textbf{age}
\end{frame}

%----------- slide --------------------------------------------------%

\begin{frame}
\frametitle{Demography and age}
All demographic phenomena are a function of age \\
\vspace{1em}
\onslide<2-> Age is also a function of demographic phenomena \\
\begin{figure}
\centering
\includegraphics<1-2>[scale=.7]{FiguresStatic/AgeFiller}
\includegraphics<3>[scale=.7]{FiguresStatic/AgeChrono}
\includegraphics<4>[scale=.7]{FiguresStatic/AgeThano}
\end{figure}
\end{frame}

%----------- slide --------------------------------------------------%

\begin{frame}
\frametitle{Age and demography}
\begin{figure}
\centering
\begin{minipage}{.5\textwidth}
  \centering
  \caption{Chronos}
  \includegraphics[width=.6\linewidth]{FiguresStatic/Chronos}
\end{minipage}%
\begin{minipage}{.5\textwidth}
  \centering
  \caption{Thanatos}
  \includegraphics[width=.5\linewidth]{FiguresStatic/Thanatos}
\end{minipage}
\end{figure}
\end{frame}


%----------- section -------------------------------------------------%
\section{Transforming population structure}
%----------- slide --------------------------------------------------%

\begin{frame}
\frametitle{A population structured by age and sex --}
\vskip-.5cm
\includegraphics[width=3in,height=3in]{FiguresStatic/AgeSexGray}
\end{frame}

%----------- slide --------------------------------------------------%

\begin{frame}
\frametitle{-- is heterogeneous with respect to remaining years of life}
\vskip-.5cm
\includegraphics[width=3in,height=3in]{FiguresStatic/AgeSexEyHetero}
\end{frame}

%----------- slide --------------------------------------------------%
% T. Miller's equation

\begin{frame}
\frametitle{We have a formula for this heterogeneity}
\begin{block}{The population age $a$ that will die in $n$ years is, $P_{a,n}$}
 \begin{equation}
   P_{a,n} = P_a \frac{d_{a+n}}{l_a}
 \end{equation}
 \end{block}
 \vskip.5cm
 \begin{description}
 \item[$d_a$:] lifetable death distribution 
 \item[$l_a$:] survival function
 \end{description}
\pause 

Variants of this equation in T. Miller (2001), Vaupel (2009)
*This identity is generally known
\end{frame}

%----------- slide --------------------------------------------------%

%\begin{frame}
%\frametitle{We can re-stack such data by remaining years}
%\includegraphics[width=2.8in,height=2.8in]{FiguresStatic/AgeSexEyHetero}
%\end{frame}

% if you can decompose it you can recompose it
%----------- slide --------------------------------------------------%
% first animation age -> ey

\begin{frame}
\frametitle{We can re-stack such data by remaining years}
\vskip-.5cm
\animategraphics[controls,width=2.8in,height=2.8in]{14}{Age2eyAnimation/frame}{000}{101}
\end{frame}

%----------- slide --------------------------------------------------%
% eyPyramid (last frame of prior animation)
% NO NEED FOR THIS SLIDE- last frame of prev animation is same
%\begin{frame}
%\frametitle{A population structured by remaining years of life}
%\includegraphics[width=2.8in,height=2.8in]{FiguresStatic/eyPyramid}
%\end{frame}

%----------- slide --------------------------------------------------%
% Transformation equation

\begin{frame}
\frametitle{The thanatological transformation}
\begin{block}{The population with $y$ remaining years of life, $P_y$}
 \begin{equation}
   \onslide<2->{P_y = P_{n} = \int_{a=0}^\infty} P_a \frac{d_{a+n}}{l_a}
   \onslide<2->{\mathrm{d} a}
 \end{equation}
 \end{block}
\onslide<3->{
\vskip1cm
This is the basic age-transformation we propose: 
\vskip.3cm
Chronological $\rightarrow$ Thanatological 
}
\end{frame}

%----------- slide --------------------------------------------------%
% ey population gray

\begin{frame}
\frametitle{A population structured by remaining years of life --}
\vskip-.5cm
\includegraphics[width=3in,height=3in]{FiguresStatic/eyPyramidGray}
\end{frame}

%----------- slide --------------------------------------------------%
% ey poulation with age-heterogeneity

\begin{frame}
\frametitle{-- is heterogeneous with respect to age}
\vskip-.5cm
\includegraphics[width=3in,height=3in]{FiguresStatic/eyPyramidAgeHet}
\end{frame}

%----------- slide --------------------------------------------------%
% ey -> age animation 

\begin{frame}
\frametitle{We can return to age-struture accordingly}
\vskip-.5cm
\animategraphics[controls,width=2.8in,height=2.8in]{14}{ey2ageAnimation/frame}{000}{101}
\end{frame}

%----------- section ------------------------------------------------%
\section{Demographic rates}
%----------- slide --------------------------------------------------%

\begin{frame}
\frametitle{Any age-structured data can be restructured like this}
\vskip-.5cm
\animategraphics[controls,width=2.8in,height=2.8in]{20}{Ba2ByAnimation/frame}{000}{101}
\end{frame}

%----------- slide --------------------------------------------------%

\begin{frame}
\frametitle{Remaining-years classified demographic rates}
\begin{block}{Define remaining-years fertility rates, $F_y$}
\begin{equation}
F_y = \frac{B_y}{E_y}\notag
\end{equation}
\end{block}

\vskip.5cm
 \begin{description}
 \item[$B_y$:] births by remaining years of life of mother/father
 \item[$E_y$:] exposures by remaining years of life of mother/father
 \end{description}
\pause
\end{frame}

%----------- slide --------------------------------------------------%
% males
\begin{frame}
\frametitle{Male fertility rates, US, 1975 and 2009}
\vspace{-5em}
%\includegraphics[width=3.2in,height=3.2in]{FiguresStatic/Fym}
\begin{figure}
        \centering
        \begin{subfigure}[b]{0.5\textwidth}
                \centering
                \caption*{ASFR}
                \includegraphics[width=\textwidth]{FiguresStatic/Fx1}
        \end{subfigure}%
        ~ %add desired spacing between images, e. g. ~, \quad, \qquad etc.
          %(or a blank line to force the subfigure onto a new line)
        \begin{subfigure}[b]{0.5\textwidth}
                \centering
                \caption*{$e$SFR}
                \includegraphics[width=\textwidth]{FiguresStatic/Fy1}
        \end{subfigure}
\end{figure}
\end{frame}
%----------- slide --------------------------------------------------%
% females
\begin{frame}
\frametitle{Female fertility rates, US, 1975 and 2009}
\vspace{-5em}
%\includegraphics[width=3.2in,height=3.2in]{FiguresStatic/Fym}
\begin{figure}
        \centering
        \begin{subfigure}[b]{0.5\textwidth}
                \centering
                \caption*{ASFR}
                \includegraphics[width=\textwidth]{FiguresStatic/Fx2}
        \end{subfigure}%
        ~ %add desired spacing between images, e. g. ~, \quad, \qquad etc.
          %(or a blank line to force the subfigure onto a new line)
        \begin{subfigure}[b]{0.5\textwidth}
                \centering
                \caption*{$e$SFR}
                \includegraphics[width=\textwidth]{FiguresStatic/Fy2}
        \end{subfigure}
\end{figure}
\end{frame}
%----------- slide --------------------------------------------------%
% males
\begin{frame}
\frametitle{Male fertility rates, Spain, 1975 and 2009}
\vspace{-5em}
%\includegraphics[width=3.2in,height=3.2in]{FiguresStatic/Fym}
\begin{figure}
        \centering
        \begin{subfigure}[b]{0.5\textwidth}
                \centering
                \caption*{ASFR}
                \includegraphics[width=\textwidth]{FiguresStatic/Fx3}
        \end{subfigure}%
        ~ %add desired spacing between images, e. g. ~, \quad, \qquad etc.
          %(or a blank line to force the subfigure onto a new line)
        \begin{subfigure}[b]{0.5\textwidth}
                \centering
                \caption*{$e$SFR}
                \includegraphics[width=\textwidth]{FiguresStatic/Fy3}
        \end{subfigure}
\end{figure}
\end{frame}
%----------- slide --------------------------------------------------%
% females
\begin{frame}
\frametitle{Female fertility rates, Spain, 1975 and 2009}
\vspace{-5em}
%\includegraphics[width=3.2in,height=3.2in]{FiguresStatic/Fym}
\begin{figure}
        \centering
        \begin{subfigure}[b]{0.5\textwidth}
                \centering
                \caption*{ASFR}
                \includegraphics[width=\textwidth]{FiguresStatic/Fx4}
        \end{subfigure}%
        ~ %add desired spacing between images, e. g. ~, \quad, \qquad etc.
          %(or a blank line to force the subfigure onto a new line)
        \begin{subfigure}[b]{0.5\textwidth}
                \centering
                \caption*{$e$SFR}
                \includegraphics[width=\textwidth]{FiguresStatic/Fy4}
        \end{subfigure}
\end{figure}
\end{frame}
%----------- section ------------------------------------------------%
\section{Growth}
%----------- slide --------------------------------------------------%

\begin{frame}
\frametitle{Reproduction}
\vspace{-5em}
\begin{figure}
        \centering
        \begin{subfigure}[b]{0.5\textwidth}
                \centering
                \caption*{Chronological}
                \includegraphics[width=\textwidth]{FiguresStatic/exRepro1}
        \end{subfigure}%
        ~ %add desired spacing between images, e. g. ~, \quad, \qquad etc.
          %(or a blank line to force the subfigure onto a new line)
        \begin{subfigure}[b]{0.5\textwidth}
                \centering
                \caption*{Thanatological}
                \includegraphics[width=\textwidth]{FiguresStatic/eyRepro1}
        \end{subfigure}
\end{figure}
\end{frame}
%----------- slide --------------------------------------------------%

\begin{frame}
\frametitle{Reproduction}
\vspace{-5em}
\begin{figure}
        \centering
        \begin{subfigure}[b]{0.5\textwidth}
                \centering
                \caption*{Chronological}
                \includegraphics[width=\textwidth]{FiguresStatic/exRepro2}
        \end{subfigure}%
        ~ %add desired spacing between images, e. g. ~, \quad, \qquad etc.
          %(or a blank line to force the subfigure onto a new line)
        \begin{subfigure}[b]{0.5\textwidth}
                \centering
                \caption*{Thanatological}
                \includegraphics[width=\textwidth]{FiguresStatic/eyRepro2}
        \end{subfigure}
\end{figure}
\end{frame}
%----------- slide --------------------------------------------------%

\begin{frame}
\frametitle{Reproduction}
\vspace{-5em}
\begin{figure}
        \centering
        \begin{subfigure}[b]{0.5\textwidth}
                \centering
                \caption*{Chronological}
                \includegraphics[width=\textwidth]{FiguresStatic/exRepro3}
        \end{subfigure}%
        ~ %add desired spacing between images, e. g. ~, \quad, \qquad etc.
          %(or a blank line to force the subfigure onto a new line)
        \begin{subfigure}[b]{0.5\textwidth}
                \centering
                \caption*{Thanatological}
                \includegraphics[width=\textwidth]{FiguresStatic/eyRepro3}
        \end{subfigure}
\end{figure}
\end{frame}
%----------- slide --------------------------------------------------%

\begin{frame}
\frametitle{Reproduction}
\vspace{-5em}
\begin{figure}
        \centering
        \begin{subfigure}[b]{0.5\textwidth}
                \centering
                \caption*{Chronological}
                \includegraphics[width=\textwidth]{FiguresStatic/exRepro4}
        \end{subfigure}%
        ~ %add desired spacing between images, e. g. ~, \quad, \qquad etc.
          %(or a blank line to force the subfigure onto a new line)
        \begin{subfigure}[b]{0.5\textwidth}
                \centering
                \caption*{Thanatological}
                \includegraphics[width=\textwidth]{FiguresStatic/eyRepro4}
        \end{subfigure}
\end{figure}
\end{frame}
%----------- slide -----
\begin{frame}
\frametitle{Births}
\begin{center}
Births in year $t$ are incremented to the population according to the death
distribution, $d_a$.
\vskip2pt
\pause
*Thus, the shape of $d_a$ determines the underlying population structure.
\end{center}
\end{frame}

%----------- slide --------------------------------------------------%

\begin{frame}
\frametitle{Births}
\begin{align}
B(t) &= \int_{y=0}^\infty F_y P_y \mathrm{d}y\notag\\
\onslide<2->{&= \int_{y=0}^\infty \int_{a=0}^\infty F_y \boxed{B(t-a)
d_{a+y}}\mathrm{d}a \mathrm{d}y\notag}\notag
\onslide<3->{\intertext{And after many years pass\ldots}}
\onslide<4->{&= \int_{y=0}^\infty \int_{a=0}^\infty F_y \boxed{B(t)e^{-ra}
d_{a+y}}\mathrm{d}a \mathrm{d}y\notag}\notag
\end{align}
\end{frame}

%----------- slide --------------------------------------------------%

\begin{frame}
\frametitle{Renewal}
\center{The relative size of adjacent cohorts becomes constant, $e^{r}$ }
\begin{align}
\onslide<1->{B(t) &= \int_{y=0}^\infty \int_{a=0}^\infty F_y
B(t)\boxed{e^{-ra}} d_{a+y}\mathrm{d}a \mathrm{d}y\notag}\notag
\onslide<2->{\intertext{Divide out $B(t)$ to isolate the intrinsic growth
rate, $r$}
1 &= \int_{y=0}^\infty \int_{a=0}^\infty F_y d_{a+y}e^{-ra}\mathrm{d}a
\mathrm{d}y}
\onslide<3->{\intertext{Compare with Lotka (1911)}1 &= \int_{a=0}^\infty
F_al_ae^{-ra}
\mathrm{d}a}
\end{align}
\onslide<4->{(Vital rates either refer to males or females.)}
\end{frame}

%----------- slide --------------------------------------------------%
\begin{frame}
\frametitle{Renewal}
\onslide<1-> Vital rates either refer to males or females \\
\vspace{1em}
\onslide<2-> Male $r$ $\ne$ Female $r$  \\
\onslide<3-> \quad (Except of course in a stationary population) \\
\vspace{1em}
\onslide<4-> Chronological $r$ $\ne$ thanatological $r$ \\
\onslide<5-> \quad (Except of course in a stationary population)
\end{frame}
%----------- slide --------------------------------------------------%
\begin{frame}
\frametitle{Single-sex intrinsic growth rates, $r$}
\vspace{-2em}
\begin{figure}
      \centering
      \caption*{Lotka's $r$, US 1969-2009}
      \includegraphics[scale = .55]{FiguresStatic/rSingleSex1}
\end{figure}
\end{frame}

\begin{frame}
\frametitle{Single-sex intrinsic growth rates, $r$}
\vspace{-2em}
\begin{figure}
      \centering
      \caption*{Lotka's vs Thanatological $r$, US 1969-2009}
      \includegraphics[scale = .55]{FiguresStatic/rSingleSex2}
\end{figure}
\end{frame}

\begin{frame}
\frametitle{Single-sex intrinsic growth rates, $r$}
\vspace{-2em}
\begin{figure}
      \centering
      \caption*{Lotka's $r$, Spain 1975-2009}
      \includegraphics[scale = .55]{FiguresStatic/rSingleSex3}
\end{figure}
\end{frame}

\begin{frame}
\frametitle{Single-sex intrinsic growth rates, $r$}
\vspace{-2em}
\begin{figure}
      \centering
      \caption*{Lotka's vs Thanatological $r$, Spain 1975-2009}
      \includegraphics[scale = .55]{FiguresStatic/rSingleSex4}
\end{figure}
\end{frame}
% note differences between males vs females and chrono vs thano
%----------- section ------------------------------------------------%

\section{Two sexes and two ages}

%----------- slide --------------------------------------------------%

\begin{frame}
\frametitle{Presentation still under construction}
Follow with decomposition slides, then one simple method, and then summarize
\begin{itemize}
  \onslide<2->{\item Male and female intrinsic rates differ}\onslide<4->{
  (two-sex problem)} \onslide<3->{ \item Chronos versus Thanatos} \onslide<4->{ (two-age problem)}
\end{itemize}
% insert sculpture pics of Chronos and Thanatos
\vskip-.5cm
\onslide<3->{
\begin{figure}
\centering
\begin{minipage}{.5\textwidth}
  \centering
  \caption{Chronos}
  \includegraphics[width=.5\linewidth]{FiguresStatic/Chronos}
\end{minipage}%
\begin{minipage}{.5\textwidth}
  \centering
  \caption{Thanatos}
  \includegraphics[width=.4\linewidth]{FiguresStatic/Thanatos}
\end{minipage}
\end{figure}
}
\end{frame}

%----------- slide --------------------------------------------------%

\begin{frame}
\frametitle{The two-sex problem}
\begin{itemize}
  \item Why do male and female reproductive indices differ?
\end{itemize}
\end{frame}

%----------- slide --------------------------------------------------%
\end{document}

