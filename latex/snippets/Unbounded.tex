
\citet{gupta1973}, upon applying a particular variant of the model presented in
\citep{gupta1972two}, noted several instances of his two-sex 
interactive \textit{intrinsic} growth
rate, $r^\ast$ falling outside the bounds of the male and female
single-sex intrinsic growth rates. This observation, at odds with intuition, was 
justified and explained in terms of changes inter-age partner availability, a
level of complexity missing from two-sex models at the time. Namely, a function designed
to determine the number of births (marriages) to males of age $x$ and females of
age $y$, $M(P_x^M, P_y^F)$, depends also on the relative availability of
partners in other ages of the oposite sex, and on competition from other ages
within the same sex. Das Gupta briefly presented evidence of a massive
\textit{shake-up} in (hypergamously staggered) male and female relative stocks in prime
reproductive ages between 1940 and 1971 as a likely culprit in conditioning
two-sex interaction. 

Mating is neither random (as allowed in
\cite{gupta1972two}, assumed in \cite{gupta1973}, and partially allowed 
in \cite{mitra1976effect}), nor limited to matched single ages (as assumed 
in \cite{karmel1947relations}, \cite{akers1967measuring} or \cite{cabre1997tortulos}), 
but rather is dependant on age- and sex- interactions and availability 
conditions, the function form of which
\citet{coale1972growth} admonishes may not be directly observable or derivable.
\citet{gupta1973} considers the proposition of \citet{coale1972growth}, that 
bracketing is a computation necessity, as a potential axiom, rather than an 
empirical constraint. In a later paper
\citep{gupta1976interactive}


