
By now it has been demonstrated that the vital rates corresponding to an
$e_x$-structured population have a different overall shape and behavior from
those that belong to age-structured populations. This is because 1) $e_x$-classified rates
are calculated over the entire population, 2) $e_x$ fertility rates respond to
both fertility and mortality changes, 3) the underlying $e_x$-structured
population ranges close to its ultimate stable form, which means that
the effects of population structure are typically minor and never abrupt. This
later point will be demonstrated in greater depth later in this dissertation.

One may conceive of reproduction in an $e_x$-structured population without
periodic reversion to the familar ground of age-structured populations.
Intuitively, imagine the two varieties of pyramid that correspond to the
(closed) population in question. 

\subsubsection{visualizing age-structured renewal}
The age-structured pyramid shifts upward by 1 year with
each passing year, with some decrement occurring in each age of life, such that
the essential shape, primarily the fruit of past fertility\footnote{Credit is
owed to Kirk Scott for first imparting me with this heuristic.}, takes two or
more generations to be erased from memory\footnote{Credit is
owed to Anna Cabr for first imparting me with this heuristic.}. Births from the
age structured population fall to the bottom of the pyramid, and are grouped 
together into a cohort. This cohort is heterogenous with respect to future 
age (year) at death, but is
homogenous with respect to the year of birth. We are familiar with
the way the age-structured population model unfolds, as it
reflects both our experience of life and history of demography.

\subsubsection{visualizing $e_x$-structured renewal}
The $e_x$ structured pyramid (see Figures~\ref{fig:exPyrUS},~\ref{fig:exPyrES})
shifts down by one year each year. There are no deaths, except for in 
the bottommost layer, those whose $e_x = 0$. Those with a life
expectancy of 20 move the next year into 19, and so forth, experiencing
increments from newly added births, but no decrement to mortality. The bottom,
$e_x = 0$ (and every $e_x$, for that matter), is heterogenous with 
respect to year of birth (age) but homogenous with respect to remaining 
years of life. Fertility can arise from individuals with nearly any remaining life
expectancy; the age-boundedness of fecundity belongs to the age
perspective of demography. Thus the entire pyramid produces offspring\footnote{The only exception
to this statement is the very top of the $e_x$-pyramid, consisting only of
pre-menarchical girls and pre-semenarchical boys that will have very long
lives}. Births, $B$, are proportioned to the pyramid using the ``radix-1''
deaths distribution, $d_x$; e.g. $P_{e_1}$ is incremented by $d_1 \cdot B$, and
so forth for all ages, adding a new layer whose total over $e_x$ equals $B$. In this way births
increment most heavily around the modal age at death, typically very high in the
pyramid, between 60 and 80, depending on the year and population. Some are
unfortunate and decrement out of the pyramid in the same year as they are
incremented (births where $e_x = 0$).

\subsubsection{The stable age structure of an $e_x$-classified population}
Given the renewal process described above, it is perhaps now intuitive to see
that the stable structure of the $e_x$ population is determined primarily by 
the $d_x$ distribution and the rate of growth of the
population. Indeed, upon transforming fertility rates to the earlier-presented
$e_x$-SFR, one is just a few short steps away from a full Lotka-type renewal
model, namely (for females):

\begin{align}
1 &= \int _{a' = 0}^\infty e^{-ra'} d_{y}^F \left(\;\int _{b = 0}^\infty
F_b^{F-F} \dd b\right)  \dd y' \\
&= \frac{e_x\mathrm{-TFR}^F}{1 + \varsigma} \int _{y' = 0}^\infty e^{-ry'}
d_{y}^F \dd y'
\end{align}
, where $e_x\mathrm{-TFR}^F$ is female $e_x$-TFR \footnote{e.g. as in
Equation~\eqref{eq:exTFR}, using births classified by mothers' $e_x$ and
$e_x$-classified female exposures. $F_b^{F-F}$ would be the same kind of rate
but calculated only with births of girls in the numerator.}, $\varsigma$ is the
sex ratio at birth, $y$ indexes remaining years of life, and $y'$ indexes age from lifetable deaths distribution, $d_x$. In this case $y' = y$. Likewise for males:

\begin{align}
1 &= \frac{\varsigma \cdot e_x\mathrm{-TFR}^M}{1 + \varsigma} \int _{x =
0}^\infty e^{-rx} d_{x'}^M \dd x
\end{align}

The key observation here is that






