
By now it has been demonstrated that the vital rates corresponding to an
$e_x$-structured population have a different overall shape and behavior from
those that belong to age-structured populations. This is because 1) $e_x$-classified rates
are calculated over the entire population, 2) $e_x$ fertility rates respond to
both fertility and mortality changes, 3) the underlying $e_x$-structured
population ranges close to its ultimate stable form, which means that
the effects of population structure are typically minor and never abrupt. This
later point will be demonstrated in greater depth later in this dissertation.

One may conceive of reproduction in an $e_x$-structured population without
periodic reversion to the familar ground of age-structured populations.
Intuitively, imagine the two varieties of pyramid that correspond to the
(closed) population in question. 

\subsubsection{Age-structured renewal}
The age-structured pyramid shifts upward by 1 year with
each passing year, with some decrement occurring in each age of life, such that
the essential shape, primarily the fruit of past fertility\footnote{Credit is
owed to Kirk Scott for first imparting me with this heuristic.}, takes two or
more generations to be erased from memory\footnote{Credit is
owed to Anna Cabre for first imparting me with this heuristic.}. Births from the
age structured population fall to the bottom of the pyramid, and are grouped 
together into a cohort. This cohort is heterogenous with respect to future 
age (year) at death, but is
homogenous with respect to the year of birth. We are familiar with
the way the age-structured population model unfolds, as it
reflects both our experience of life and history of demography.

\subsubsection{$e_x$-structured renewal}
The $e_x$ structured pyramid (see Figures~\ref{fig:exPyrUS},~\ref{fig:exPyrES})
shifts down by one year each year. There are no deaths, except for in 
the bottommost layer, those whose $e_x = 0$. Those with a life
expectancy of 20 move the next year into 19, and so forth, experiencing
increments from newly added births, but no decrement to mortality. The bottom,
$e_x = 0$ (and every $e_x$, for that matter), is heterogenous with 
respect to year of birth (age) but homogenous with respect to remaining 
years of life. Fertility can arise from individuals with nearly any remaining life
expectancy; the age-boundedness of fecundity belongs to the age
perspective of demography. Thus the entire pyramid produces offspring\footnote{The only exception
to this statement is the very top of the $e_x$-pyramid, consisting only of
pre-menarchical girls and pre-semenarchical boys that will have very long
lives}. Births, $B$, are proportioned to the pyramid using the ``radix-1''
deaths distribution, $d_x$; e.g. $P_{e_1}$ is incremented by $d_1 \cdot B$, and
so forth for all ages, adding a new layer whose total over $e_x$ equals $B$. In this way births
increment most heavily around the modal age at death, typically very high in the
pyramid, between 60 and 80, depending on the year and population. Some are
unfortunate and decrement out of the pyramid in the same year as they are
incremented (births where $e_x = 0$). See Figure~\ref{fig:exrenewal} for a
schematic visualization of $e_x$-structured population renewal.

%\begin{landscape}
\begin{figure}
\caption{Schematic renewal diagram for an $e_x$-structured population}
\label{fig:exrenewal}
\includegraphics[scale = .8]{Figures/exRenovationDiagram.pdf}
\end{figure}
%\end{landscape}

\subsubsection{The stable age structure of an $e_x$-classified population}
Given the renewal process described above, it is perhaps now intuitive to see
that the stable structure of the $e_x$-structured population is determined
primarily by the deaths distribution and the rate of growth of the
population. Indeed, upon transforming fertility rates to the earlier-presented
$e$SFR, one is just a few short steps away from a full Lotka-type renewal
model, namely (for females):

\begin{align}
\label{eq:exLotkafemales}
1 &= \int _{y'=0}^\infty \int _{a'=y'}^\infty e^{-ra'} d_{a'}^F f_{y'}^{F-F} \dd
a' \dd y'
\end{align}
, where $a'$ indexes female age, $y'$ indexes female remaining years of life,
$d_{a'}^F$ is the age-distribution of female deaths from the radix-1 period
lifetable, and $f_{y'}^{F-F}$ are exact female-female (mother-daughter)
fertility probablilities by remaining years of life ($e$SFR, see Equation~\eqref{eq:eSFR}). Likewise for males:

\begin{align}
1 &= \int _{y=0}^\infty \int _{a=y}^\infty e^{-ra} d_a^M f_y^{M-M} \dd a \dd y
\end{align}

Equation~\ref{eq:exLotkafemales} is indeed similar to the
original age-structured Lotka equation:

\begin{align}
\label{eq:lotkaorig}
1 &= \int _{a=0}^\infty e^{-ra}l_a f_a \dd a
\end{align}
, where $l_a$ is the survival curve and $f_a$ the fertility curve. First, note
that $l_a$ inside Equation~\ref{eq:lotkaorig} can also be expressed in terms of
$d_a$ (current livings are the sum of future deaths):

\begin{equation}
l_a = \int _{x = a} ^\infty d_x \dd x
\end{equation}
, in which case Equation~\ref{eq:lotkaorig} can be rewritten as:

\begin{align}
\label{eq:lotkadx}
1 &= \int _{a=0}^\infty \int _{b = a}^\infty e^{-ra} d_b f_a \dd b \dd a
\end{align}

All we have changed in in order to derive Equation~\ref{eq:exLotkafemales}
is to turn $l_a$ and $f_a$ sideways, so to speak, multiplying the two vectors
together where they coincide in terms of remaining years instead of in terms of age. This
transformation is a simple change of perspective. $r$ still applies to sucessive 
time steps, but in terms of remaining years of life, it must be applied incrementally 
over the newcomers to
each grouping of remaining years of life, i.e. over the time-layers of the
$e_x$-structured pyramid.

\paragraph{An iterative approach to find $r$}

\citet{coale1957new} offers a fast-converging iterative approach to estimate the
instrinsic growth rate for age-structured populations. For the $e_x$ structured
renewal equation, a similar approach may be designed, with some slight
modifications. The following steps can be followed to estimate $r$ from
Equation~\ref{eq:exLotkafemales}:

\begin{enumerate}
  \item Derive a first rough estimate of the mean remaining years of life at
  reproduction, $\widehat{T^y}$, akin to Lotka's mean generation time, $T$. If
  one assumes a growth rate of $0$, then a good guess will be:
\begin{equation}
\widehat{T^y} = \frac{\int _{y=0}^\infty \int _{a=y}^\infty a d_a f_y \dd a
\dd y}{\int _{y=0}^\infty \int _{a=y}^\infty d_a f_y \dd a \dd y}
\end{equation}
 This value appears to range between 50 and 70\footnote{True $T^y$ is around 10
 years lower, ranging from 40-50.}
  \item A first rough guess at the net reproduction rate, $R_0$ is given by:
 \begin{equation}
  R_0 = \int _{y=0}^\infty \int _{a=y}^\infty d_a f_y \dd a
\dd y
\end{equation}
  \item A first rough estimate of $r$, $r^0$, is given by:
   \begin{equation}
   r^0 = \frac{ln(R_0)}{\widehat{T^y}}
   \end{equation}
  \item plug $r^0$ into Equation~\ref{eq:exLotkafemales} to calculate a
  residual, $\delta^0$
  \item use $\delta^0$ and $\widehat{T^y}$ to calibrate the estimate of $r$
  using:
  \begin{equation}
  r^{1} = r^0 + \frac{\delta^0}{\widehat{T^y} - \frac{\delta^0}{r^0}}
  \end{equation}
  \item repeat step (3) to to derive a new $\delta^i$, then step (4) to refine
  $r^i$, until converging on a stable $r$ after some 30 iterations,
  depending on the degree of precision desired. ($\widehat{T^y}$ is not updated
  in this process).
\end{enumerate}

The above procedure is both faster and more precise than minimizing the absolute
residual of Equation~\ref{eq:exLotkafemales} using a generic
optimizer\footnote{Use of a Newton-Raphson optimizer with analytic objective
and gradient functions may prove even more efficient, but I have not tried
this, since the present routine is more than efficient enough for practical
purposes.}. 

\paragraph{Other reproduction parameters}
A final calculation of $T^y$ is given by:

\begin{equation}
 T^y =  \frac{\int _{y=0}^\infty \int _{a=y}^\infty a e^{-ra} d_a f_y \dd a
\dd y}{\int _{y=0}^\infty \int _{a=y}^\infty e^{-ra} d_a f_y \dd a \dd y}
\end{equation}
, using $r$ from the iterative procedure. The net reproduction rate, $R_0$ is
related by, e.g.:

\begin{equation}
R_0 = e^{r T^y}
\end{equation}

The birth rate, $b$, is given by:

\begin{equation}
b = \frac{1}{\int _{y=0}^\infty \int _{a=y}^\infty e^{-ra} d_a \dd a
\dd y}
\end{equation}

The stable age structure, $c$, where $c_y$ is the
proportion of the stable population with remaining years to live $y$, is given
by:

\begin{equation}
c_y = b \int _{a=y}^\infty e^{-ra} d_a \dd a
\end{equation}

Other possibly interesting stable parameters may be estimated by
similarly translating the various definitions in the glossary of
\citet{coale1972growth} to the present perspective. Before presenting 
results or extending the present one-sex renewal
formula to two-sex linear and non-linear situations, the heart of this 
thesis, we first describe the construction of the Leslie matrix that corresponds to the
present model.
