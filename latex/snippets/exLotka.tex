
By now it has been demonstrated that the vital rates corresponding to an
$e_x$-structured population have a different overall shape and behavior from
those that belong to age-structured populations. This is because 1) $e_x$-classified rates
are calculated over the entire population, 2) $e_x$ fertility rates respond to
both fertility and mortality changes, 3) the underlying $e_x$-structured
population tends to range close to its ultimate stable form, which means that
the effects of population structure are typically minor and never abrupt.

One may conceive of reproduction in an $e_x$-structured population without
periodic reversion to the familar ground of age-structured populations.
Intuitively, imagine the two varieties of pyramid that correspond to the
population in question. 

The age-structured pyramid shifts upward by 1 year with
each passing year, with some decrement occurring in each age of life, such that
the essential shape, mostly the fruit of past fertility, takes two or more
generations to be erased from memory. Births from the age structured population
fall to the bottom of the pyramid, and are lumped together into a cohort. This
cohort is heterogenous with respect to future age (year) at death, but is
homogenous with respect to the year of birth. We are comfortable with
the way the age-structured population model unfolds, as it
reflects our experience of life.

The $e_x$ structured pyramid shifts down by one year each year. There are no
deaths, except for in the bottom layer, those whose $e_x = 0$. Those with a life
expectancy of 20 move the next year into 19, and so forth, with no changes. The
bottom, $e_x = 0$ (and every $e_x$, for that matter), is heterogenous with
respect to year of birth, age, but equal with respect to remaining years of
life. Fertility can arise from individuals with nearly any remaining life
expectancy; the age-boundedness of fecundity belongs only in the age
perspective of demography. Thus the entire pyramid produces offspring\footnote{The only exception
to this statement is the very top of the $e_x$-pyramid, consisting only of
pre-menarchical girls and pre-semenarchical boys that will have very long
lives}. Births, $B$, are proportioned to the pyramid using the ``radix-1''
deaths distribution, $d_x$; e.g. $e_1$ is incremented by $d_1 B$, and so forth for all
ages, adding a new layer whose total over $e_x$ equals $B$. In this way births
increment most heavily around the modal age at death, typically very high in the
pyramid, between 60 and 80, depending on the year and population. Some are
unfortunate and decrement out of the pyramid in the same year as they are
incremented ($e_0$).

It is perhaps intuitive to see that the stable structure of the $e_x$ population
is determined primarily by the $d_x$ distribution and the rate of growth of the
population. Indeed, upon transforming fertility rates to the earlier-presented
$e_x$-SFR, one is just a few short steps away from a full Lotka-type renewal
model, namely (for females):

\begin{align}
1 &= \int _{y = 0}^\infty e^{-ry} d_{y'}^F \left(\;\int _{b = 0}^\infty
F_b^{F-F} \dd b\right)  \dd y \\
&= \frac{e_x\mathrm{-TFR}^F}{1 + \varsigma} \int _{y = 0}^\infty e^{-ry}
d_{y'}^F \dd y
\end{align}
, where $e_x\mathrm{-TFR}^F$ is female $e_x$-TFR, $\varsigma$ is the sex
ratio at birth, $y$ indexes remaining years of life, and $y'$ indexes age from
lifetable deaths distribution, $d_x$. In this case $y' = y$. Likewise for males:

\begin{align}
1 &= \frac{\varsigma \cdot e_x\mathrm{-TFR}^M}{1 + \varsigma} \int _{x =
0}^\infty e^{-rx} d_{x'}^M \dd x
\end{align}






