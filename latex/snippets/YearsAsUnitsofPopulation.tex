
\subsubsection{Years as the units of population}
Recall the concept and method of \textit{reproduction of years lived} proposed
by \citet{henry1965reflexions} and later replicated by \citet{cabre1990repro}
for Catalonia. Henry's method involves weighting a Net Reproduction Rate by
the ratio of daughters' to mothers' life expectancies:

\begin{equation}
R_{0}^{\ast} = R_{0} \frac{e_0^{daughters}}{e_0^{mothers}}
\end{equation}

This method essentially adjusts birth counts to account for changes (increases)
in mean generation length. $R_{0}^{\ast}$ is analagous to thinking of populations 
and their renovation in terms of total exposure, rather than as a given census cross-section. This
method can be calculated analagously for males, and can be operationalized 
in a variety of ways, depending on the available data. Ideally, 
$R_{0}$ is calculated for a cohort, $e_0^{daughters}$ and $e_0^{mothers}$ are
age generation-specific, and their ratio is weighted inside the calculation of
$R_0$ according to CTFR. This ideal case is not practical for the present
data because the cohort life expectancy of recent generations in unknown
\footnote{Even the oldest generation of mothers in the first year of data
presented here, $1969-50-1 = 1919$ , is not extinct. Thus all requisite cohort
life expectancies are unknown.}, and because the data window used is narrow to
cover the reproductive range of any cohort, thus cohort ASFR is not fully known.

One could reduce fertility data requirements by using period data, i.e.
use period instead of cohort ASFR and period $e_0$ instead of cohort $e_0$, both
for mothers and daughters:

\begin{equation}
R_{0}^{\ast P}(t) = \int _{\alpha = 0} ^{\beta} p_{a}^{f}(t) f_{a}^{f}(t)
\frac{e_{0}^f(t)}{e_{0}^f(t-a) \dd a}
\end{equation}

\textit{Periodizing} the formula in this way gains practicality at the cost of
potential distortion. The series of period $R_0^\ast$ for US and Spanish males
and females for the range of years treated here would look something like
Figure~\ref{fig:R0perHenry}:

\begin{figure}[ht!]
        \centering  
          \caption{Male and Female $e_0$-weighted Net Reproduction Rates,
          1969-2009, USA and Spain}
           % figure produced in
           % /R/ObservedVsExpectedBivariateBirthDistribution.R
           \makebox[\textwidth]{\includegraphics{Figures/R0perHenry}}
          %\includegraphics{Figures/ObservedvsExpectedBxy}
          \label{fig:R0perHenry}
\end{figure}

The trends in these series essentially transmit the same information
present in the sex-specific $R_0$ series shown in Figure~\ref{fig:NRRseries},
except, where period $e_0$-weighting has its greatest effect in the first years
of the series. The effect of this method is to change the units of population
from persons to years. 