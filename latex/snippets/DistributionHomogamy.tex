
The age combination of the male and female fertility schedules from
any given year varies greatly from the distribution that would be expected if
age and mother and age of father were selected randomly according to the two
single-sex distributions.

The expected cross-classified age distribution $\textbf{E}B(x,y)$ is defined as:

\begin{equation}
\textbf{E}\left[B_{xy}\right] = \frac{B_x B_y}{\int _{x = \alpha} ^\beta \int
_{y = \alpha} ^\beta B_{xy} \; \dd x \;\dd y}
\end{equation}
where $x$ stands for age of father and $y$ stands for age of mother.

\begin{figure}[ht!]
        \centering  
          \caption{Observed versus expected bivariate age distribution of
          parents, 1970, USA}
           % figure produced in
           % /R/ObservedVsExpectedBivariateBirthDistribution.R
           \makebox[\textwidth]{\includegraphics{Figures/ObservedvsExpectedBxy}}
          %\includegraphics{Figures/ObservedvsExpectedBxy}
          \label{fig:US1970obsexp}
\end{figure}


Visual inspection of surfaces of the observed and expected birth counts in
Figure~\ref{fig:US1970obsexp} confirms they are indeed quite different: The
observed surface shows a stronger homogamy-hypergamy pattern than the expected surface. How similar are the
observed and association-free $B_{xy}$ distributions from each other? One way to
judge the departure from randomness of this distribution is to calculate a
simple difference coefficient, $\theta$, namely:

\begin{equation}
\theta = 1 - \int \;\int min(f_1, f_2)
\end{equation}
,where $f_1$ is $B_{xy}$ and $f_2$ is $\textbf{E}B_{xy}$, both scaled to sum to
1. $\theta$ is constrained to fall between 0 and 1, where 1 indicates that the
two distributions are separate and 0 indicates identical distributions. In 1970 USA,
$\theta$ was equal to $0.47$, a value which can be understood to stand as the
degree of residual preference.

Note that that age-preference is an imprecise label for the variety
of preferences that may actually lead to observed age-combination biases. For
instance, preferences may reflect a third variable (e.g. socioeconomic
in nature) that covaries with age, so as to give the appearance of age
preferences. Furthermore, as \citet{bergstrom1994sweden}
demonstrate, pair matching may just as easily occur as a function of individual
preferences for event (mating, marriage) timing coupled with relative
availability, which follows partly from cohort size. This is consistent with
\citet{bhrolchain2001flexibility}, who concludes that age preferences for
mates are highly adaptive to availability conditions.

Despite this ambiguity in mechanisms behind age combination patterns, one can
create a rough index of the strength of hypergamy or homogamy, based on the
matrices represented in Figure~\ref{fig:US1970obsexp}. Giving equal reproductive
bounds to the birth count matrix $B_{xy}$ makes a square matrix, from which we
can separate the upper and lower triangles. Here, the lower triangle, $L$,
of $B_{xy}$ contains births due to age-hypergamous (father's age > mother's age)
parents and the upper triangle $U$ contains births due to age-hypogamous parents. Thus, a simple
measure of total hypergamy, $\widehat{H}$, can be taken as a ratio of the total
births in $L$ versus $U$, or in shorthand $\frac{B_{x>y}}{B_{x<y}}$, excluding
single-age exact homogamy on the matrix diagonal.

\begin{equation}
\widehat{H} = \frac{\sum L}{\sum U} 
\end{equation}

In this case, the $\widehat{H}$ will be calculated for the observed and expected
birth matrices. US data from 1970 yields and observed $\widehat{H}$ of $7.37$
versus an expected $\textbf{E}\widehat{H}$ of $1.75$. The later value is
possibly much higher than one would suspect, given that the $\textbf{E}B_{xy}$
is purged of association. It is due, as mentioned above to differences in the
shape and span of male and female single-sex fertility. For reference, I
will call this structural or latent hypergamy, as opposed to the residual, or
excess hypergamy, which is the ratio of observed (total) hypergamy to
structural hypergamy, in this case $4.21$ times higher
than structural hypergamy. While these types of values do not enter, per se, 
into any of the thus-far proposed two-sex solutions, they characterize the 
population in a basic way, and aid in understanding macro-level patterns. 

Let us then calculate two times series, one for total difference,
Figure~\ref{fig:Theta}, and another for our three measures of hypergamy,
Figure~\ref{fig:HypergamyStrength}:

\begin{figure}[!ht]
  \centering
    \caption{Departure from association-free bivariate distribution. USA,
    1969-2010 and Spain, 1975-2009}
     % figure produced in
     % /R/ObservedVsExpectedBivariateBirthDistribution.R
     \includegraphics{Figures/TotalVariationObsvsExpectedUSES}
     \label{fig:Theta}
\end{figure}

\begin{figure}[!ht]
  \centering
    % figure produced in
    % /R/ObservedVsExpectedBivariateBirthDistribution.R
    \caption{Strength of Hypergamy, $\frac{B_{x>y}}{B_{x<y}}$, total, structural
    and excess. USA, 1969-2010 and Spain, 1975-2009}
    \includegraphics{Figures/StrengthHypergamy}      
    \label{fig:HypergamyStrength}
\end{figure}










