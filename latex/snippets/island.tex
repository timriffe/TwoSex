
In illustrating the two-sex problem, it is convenient to work in the vacuum of a 
closed population. To this end it is rhetorically tempting to imagine a small
population of humans stranded on an island, and the reproductive social
conditions that would result from experiments in changing the age-sex
distribution of persons of reproductive age. In this philosophical bubble, it is
at once easy to imagine that changing the relative abundance and scarcity of
potential mates would indeed affect measured fertility levels, but at the same
time fall back to a position of female dominance \footnote{a term coined by
\citet{goodman1953population}}, noting that theoretically females are the
rate-limiting sex in human reproduction \citep{wood1994dynamics}. This later
point underpins the physiological tautology that when females are already very scarce
\footnote{this is incidentally a constant condition in polygynous socieities.}, changing the
relative numbers of males will not affect fertility. Female dominance of this
type among humans would apparently only arise in populations with extremely low
sex ratios, and we do not in general know a priori that female dominance is a 
constant or common atrribute of the proximate determinants of human sexual
reproduction. This has long been observed, as
\citet[p. 1]{kuczynski1935measurement} colorfully states:
\begin{citation}
The full effect of fecundity would be realized if all females, throughout their
entire child-bearing period, had sexual intercourse with procreative men and did
nothing to prevent conception nor to procure abortion. Since those conditions
are never and nowhere fulfilled, fertility always and everywhere lags behind
fecundity.
\end{citation}

In short, human females as a whole are never at or near saturation fertility
levels. Further, strict or serial monogamy is the norm in most human societies.
Other factors, such as lactation and nuptiality also intervene, as John
Bongaarts \citep{bongaarts1978framework, bongaarts1982fertility, bongaarts1983fertility} 
has many times clearly decomposed and illustrated. 

Thus in present-day western populations, although female fecundity determines
maximum potential fertility, a variety of constraints hold observed fertility at
much lower levels, around 10 - 30\% of its potential. Holding these intervening
factors constant, our imaginary island population experiments could indeed yield
some insights into the interaction between the sexes- the effects of abundance and scarcity of
potential mates on observed fertility. 

A first experiment on this island, or rather set of islands as of population
petri dishes in a laboratory (for we would need control populations too), would
simply record the response of female and male fertility under a set of sex
ratios. Do high sex-ratios inflate fertility, leaving few unmated females? Do
sub-unity sex-ratios deflate fertility or does sexual behavior change to keep 
fertility constant in this case? Is the magnitude of change in fertility (if
any) the same given proportionally equal excesses (deficiencies) of males or
females? This later test would address the frequently-used but to my knowledge thus-far unstated
assumption of commutativity in two-sex solutions, such as the harmonic mean
\citep{schoen1981harmonic}, or that of iterative proportional fitting
\citep{mc1975models}. Any sort of mean function that
fits into the stolarsky framework treats males and females essentially the same. This could be
mathematically convenient, but is an empirical question.

In the same way, one could check other frequently-used axiomatic assumptions
\citep{mcfarland1972comparison}, such as homogeneity (i.e. search for
population scaling effects), monotonicity (i.e. consistent direction of
response to changes in sex ratios), and so forth. Neither of these later two
assumptions is in my view necessary.

The results of said set of experiments would ideally be encapsulated in a simple
functional form. Indeed most of the literature on the two-sex problem is about
this functional form. While there have been some notable attempts to empirically
determine the best two-sex function \citep{keyfitz1972mathematics,
alho2000competing}, none of these efforts have led to a firm conclusion. This is
so partly because it is very difficult to bring data to bear on the problem,
because human sex ratios in reproductive ages are tyically not far from unity. 
Since most two-sex (fertility or marriage) mean functions are close to
identical when sex ratios (and therefore rate ratios) are close to unity, it is
 difficult to come to firm conclusions. 
 
 \citet{alho2000competing} looked for a kind of natural experiment in historical
  Finnish data, samely a swift jump in births in 1941.

