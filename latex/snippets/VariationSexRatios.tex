
The main problem with studying two-sex fertility/mating functions in human
populations is that sex ratios in reproductive ages tend to vary little from
unity. This is only the case if we take the large aggregate populations as the
subject of study. Various factors can cause effective sex ratios to diverge
greatly. Unevenly distributed contact opportunities between the sexes is the
most obvious and straightforward such skewing factor. Contact opportunities can
be a function of cultural constraints, such as norms, habits, institutions and
social network distributions; structural constraints, such as uneven spatial
distributions due to group size, migration, various kinds of cloistering and
differential mortality. Individual preferences can exggerate these factors even 
more. As a result, mating markets cannot be considered homogenous mixtures, and 
a $1:1$ sex ratio in reproductive ages will typically not reflect practical,
\textit{in-market} sex ratios. Such distortions can of course be dampened by
further migration, social networks, settling for less-than-ideal mates.

The practical problem for classical aggregate demography is that there are an
infinite number of ways that one could combine male and female vital rates
(fertility, namely) into an interactive two-sex rate-schedule. Where male and
female rates differ, unless one sex is perfectly dominant, the best estimate of
the true rate is expected to obtain an intermediate value

