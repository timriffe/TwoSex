\label{sec:exLotka2linear}
It has been noted that divergence between the sexes, in terms of predicted birth
counts, is often dampened when projected using rates and populations that are
structured according to remaining years. This does not, however, mean that the
problem of the sexes is in this context negligible. Instead, the problem has
only become slightly more tractable. I consider the problem more tractable
because in decreasing the magnitude of discrepancy between male and female
rates, the trade-offs inherent in the various two-sex solutions offered in the
literature also become smaller. This section will introduce two kinds of two-sex
extensions of the $e_x$-structured population model introduced in
Section~\ref{sec:exstructuredrenewal}:
\begin{enumerate}
  \item A linear extension, assuming fixed weights for males and female,
  similar in design to the age-structured two-sex model found in
  \citet{goodman1967age}.
  \item A non-linear extension, following the example in
  \citet{gupta1978alternative}.
  \item another non-linear extension? might need to break out sections\ldots
\end{enumerate}

\subsubsection{A linear two-sex extension to the $e_x$-structured population
model} 

\citet{goodman1967age} offers a suite of
formulas to determine the stable age-sex composition of a population taking into
account the vital rates of both sexes, assuming one can assign a relative weight
(summing to 1) to male and female fertility. In this case there are two
trade-offs: 1) one must (arbitrarily) choose weights, and 2) these weights are
constant. The fact of having constant weights keeps the solution linear
(interaction-free), but less realistic. The final result is bracketed by the
cases of male and female dominance, but the gap between these two extremes 
also measures the demgographer's subjective leeway, which we would like to
minimize. Both of these drawbacks may be reduced in the case of $e_x$-structured
populations, since: 
\begin{enumerate}
  \item $e_x$-structured populations have a more stable structure
than age-structured populations. [add section or appendix to prove this]
\item Mate-selection with respect to remaining years of
life is nearly random in $e_x$-structured populations (see
Section~\ref{sec:exobsexpected}).
\item The difference between male and female dominance (in terms of projected
birth counts) is often reduced, thereby limiting of the impact of the
demographer's ``dominance caprice'' on results (See
Section~\ref{sec:exdivergence}).
\end{enumerate}
Points (1) and (2) reduce (but do
not eliminate) the necessity of sex-interactions in a model. By this it is meant
that the proportional difference in results from one choice of model weights
over another is simply diminished. This being so, the comparative advantage of a
more sophisticated or realistic model is to some degree diminished. 

\paragraph{Formal definition of the Lotka-type renewal equation:}
\label{sec:2sexlinearmain}
As mentioned, choose some weight, $\sigma$, between 0 and 1 to apply to male
rates, where the female weight is defined as $1 - \sigma$. When $\sigma = 1$
there is perfect male dominance, and when $\sigma = 0$ there is perfect female
dominance. Of course, births to girls are subject to female mortality and births
to boys are subject to male mortality. As with
Equation~\eqref{eq:exLotkafemales}, this mortality enters in the equation by way
of the $d_x$ distribution used to distribute births over life expectancies. The
final renewal formula is defined as follows:

\begin{equation}
\label{eq:lineartwosexrenewal}
\begin{split}
1 = \frac{1 - \sigma}{2} 
            \int _{y'=0}^\infty \int _{a'=y'}^\infty e^{-ra'}
                      d_{a'}^F \left(f_{y'}^{F-F} + f_{y'}^{F-M}\right) \dd a'
                      \dd y' \\ + \;\frac{\sigma}{2}
            \int _{y=0}^\infty \int _{a=y}^\infty e^{-ra}
                      d_{a}^M  \left(f_{y}^{M-M} + f_{y}^{M-F}\right)\dd a \dd y
\end{split}
\end{equation}
, where $a'$, $y'$, $a$ and $y$ index female age, female remaining years, male
age and male remaining years, respectively. Fertility superscripts identify sex of
progentitor followed by sex of offspring, and $d_x$ must accord with the sex of
offspring. Such specific rates are chosen because data that would permit
empirical studies of the two-sex problem are typically sufficiently rich to allow 
for cross-tabulations by age of both parents as well as sex of birth. 
Therefore, Equation~\eqref{eq:lineartwosexrenewal} assumes that rates are
available by sex of progenitor, birth (4 combinations) and age (to be transformed to remaining years), 
and no additional variable is required for the sex ratio at birth. Indeed,
Equation~\ref{eq:lineartwosexrenewal} does not require such specific rates,
since rates of reach progenitor sex are simply summed, but sex-sex-specific
rates will be needed downstream for the calculation of other stable quantities,
so it is advisable to treat them as inputs from the start.

Weights, $\sigma$ and $1-\sigma$ are divided by 2 because
total births are counted twice in total (males and females from males \&
males and females from females). One could just as easily optimize to a sum of
2 on the left-hand side rather than discount weights.

The linear two-sex $r$, $r^\upsilon$, extracted from~\eqref{eq:lineartwosexrenewal} 
is \textit{not} guaranteed to be bounded by the $e_x$-structured $r^f$ and $r^m$,
and indeed $r^f$ and $r^m$ may not be recovered by setting $\sigma$ to 0 or 1,
respectively. This is so because the model includes births of both sexes to
progenitors of each sex, which changes the age-specific fertility curves
somewhat. That is to say, manipulation of $\sigma$ is insufficient to make the
one-sex model a degenerate case of the present model. $\sigma$ can only be
understood as indicative of the balance of dominance in fertility rates between
the sexes. The later choice would both reduce the complexity of
Equation~\ref{eq:lineartwosexrenewal} and guarantee exact bounds of $r^f$ and $r^m$ 
when $\sigma$ is set to 0 and 1, respectively. This author does not recognize 
the theoretical or practical merits of the single-sex modelling choice, as it 
is not the case that males are responsible for the birth of boys and females 
for the birth of girls\footnote{\citet{pollard1948measurement} took this idea
even further by swapping sexes: The fertility functions in this paper are based on the births of boys to mothers and girls to fathers, i.e. $M-F$ and $F-M$ fertility. This is parsimonious 
in terms of getting quick results that are guaranteed to fall within reasonable bounds, but is less
intuitively appealing}. This stance couples with the author's choice to not
include an explicit, let alone constant, variable for the sex ratio at birth.

It must be
noted that the value of $r^\upsilon$ is dependant upon the choice of $\sigma$, 
and that no guidelines are provided for choosing a good value of $\sigma$. 
This ambiguity also exists in the age-structured variant of the present model. 
For $e_x$-structured models, it has been claimed that sex-divergence is lesser than is the case for
age-structured models. Recall that this was the case for predictions of birth
counts, and not for the growth parameter, $r^\upsilon$. The
difference between the $e_x$-structured $r^f$ and $r^m$ is not necessarily lesser than is the case for
the age-structured $r^f$ and $r^m$. This will be discussed further along with
empirical results for the two populations considered in this dissertation.

\paragraph{An iterative approach to find $r^\upsilon$:}
\label{sec:exrenewalit2}
Steps to practically solve Equation~\eqref{eq:lineartwosexrenewal} for $r$ are
similar to those presented for the one-sex case in
Section~\ref{sec:exrenewalit}
\begin{enumerate}
  \item Derive a first rough estimate of the both-sex mean remaining years of
  life at reproduction, $\widehat{T^\upsilon}$, akin to Lotka's mean generation time,
  $T$. If one assumes a growth rate of $0$, then a good-enough guess will be:

\begin{equation}
\widehat{T^\upsilon} = \frac{\splitfrac{
   \big((1 - \sigma)  \int _{y'=0}^\infty \int_{a'=y'}^\infty 
       a' d_{a'}^F \left( f_{y'}^{F-F} + f_{y'}^{F-M} \right) \dd a'\dd y'}{ + 
   \sigma \int_{y=0}^\infty \int _{a=y}^\infty a d_{a}^M  \left( f_{y}^{M-M}+
   f_{y}^{M-F} \right) \dd a \dd y \big)}}{\splitfrac{\big( (1 - \sigma) 
   \int_{y'=0}^\infty \int_{a'=y'}^\infty d_{a'}^F \left( f_{y'}^{F-F} +
   f_{y'}^{F-M} \right) \dd a'\dd y'}{ +\sigma \int _{y=0}^\infty
   \int_{a=y}^\infty d_{a}^M \left( f_{y}^{M-M} + f_{y}^{M-F} \right) \dd a \dd
   y \big)}}
\end{equation}

  \item A first rough estimate of the net reproduction rate, $\widehat{R_0}$ (assuming
  $r=0$) is given by:

 \begin{equation}
 \begin{split}
 \widehat{R_0} = \frac{(1 - \sigma)}{2}  \int _{y'=0}^\infty \int_{a'=y'}^\infty 
                d_{a'}^F \left(f_{y'}^{F-F} + f_{y'}^{F-M}\right) \dd a'\dd y'
                \\ + \frac{\sigma}{2}  \int _{y=0}^\infty \int _{a=y}^\infty 
               d_{a}^M  \left(f_{y}^{M-M}+ f_{y}^{M-F}\right) \dd a \dd y
 \end{split}
 \end{equation}
  \item A first rough estimate of $r$, $r^0$, is given by:
   \begin{equation}
   r^0 = \frac{ln(\widehat{R_0})}{\widehat{T^\upsilon}}
   \end{equation}
  \item plug $r^0$ into Equation~\ref{eq:lineartwosexrenewal} to calculate a
  residual, $\delta^0$
  \item use $\delta^0$ and $\widehat{T^\upsilon}$ to calibrate the estimate of $r$
  using:
  \begin{equation}
  r^{1} = r^0 + \frac{\delta^0}{\widehat{T^\upsilon} - \frac{\delta^0}{r^0}}
  \end{equation}
  \item repeat step (3) to to derive a new $\delta^i$, then step (4) to refine
  $r^i$, until converging on a stable $r$ after some 30 iterations,
  depending on the degree of precision desired. ($\widehat{T^\upsilon}$ is not updated
  in this process).
\end{enumerate}

\paragraph{The two-sex linear $e_x$-structured model applied to data:}

This procedure has been applied to the data from the US and Spain with $\sigma
= [0,.5,1]$, which correspond to the cases of female-dominace, an
intermediate value, and male-dominance, and can be seen in
Figure~\ref{fig:rupsilonlinear2sex}\footnote{The data in
    Figure~\ref{fig:rupsilonlinear2sex} are available in
    Tables~\ref{tab:ex2linRepES}~and~\ref{tab:ex2linRepUS} of
    Appendix~\ref{appendix:ex2sexlinear}, along with the stable parameters
    $R_0^\upsilon$ and $T^\upsilon$.}

\begin{figure}[!ht]
  \centering
    \caption{Two-sex linear intrinsic growth rate, $r^\upsilon$, according to
    renewal Equation~\eqref{fig:rupsilonlinear2sex}, with $\sigma = [0, .5, 1]$ US 
    and Spain, 1969-2009}
     % figure produced in /R/ExLotka2Sex.R
     \includegraphics{Figures/exLotka2sexlinear}
     \label{fig:rupsilonlinear2sex}
\end{figure}

Patterns accord with trends generally known from the age-classified $r^f$ and
$r^m$, but values of $r^\upsilon$ are higher than the
age-classified intrinsic growth rates in all of the years studied. In all years
tested here, $r^\upsilon$ was indeed bounded by the $e_x$-structured $r^f$ and $r^m$. There have been some notable crossovers in
which was the greater $r^\upsilon$ derived from the border cases of $\sigma = 0$
and  $\sigma = 1$. Notably, in the US male dominance produced a higher
$r^\upsilon$ than female dominance from the start of observation until
1985, after which time there was a crossover that has persisted. Spain has
undergone two such crossovers, with female dominace producing a higher
$r^\upsilon$ than male dominace until 1980, and again recently starting in 2007.
Note that the spread between the border cases of male and female dominance is
tighter than between the single-sex cases. This observation should put the
demographer at ease in deciding the degree of dominance to assign via $\sigma$,
and makes it particularly easy to accept an intermediate value such as $.5$.
There have been series of years in both the US and Spain where swapping dominace
would have made virtually no difference in results or conclusions!

\paragraph{Other reproduction parameters:}
\label{sec:2sexlinearother}
Once two-sex linear $r^\upsilon$ has been found for the given $\sigma$, one may
proceed to find the corrsponding both-sex $T^\upsilon$ using:
\begin{equation}
T^\upsilon = \frac{\splitfrac{
   \big((1 - \sigma)  \int _{y'=0}^\infty \int_{a'=y'}^\infty 
       a'e^{-ra'} d_{a'}^F \left( f_{y'}^{F-F} + f_{y'}^{F-M} \right) \dd a'\dd
       y'}{ + \sigma \int_{y=0}^\infty \int _{a=y}^\infty a e^{-ra}d_{a}^M 
       \left( f_{y}^{M-M}+ f_{y}^{M-F} \right) \dd a \dd y \big)}}{\splitfrac{\big( (1 - \sigma) 
   \int_{y'=0}^\infty \int_{a'=y'}^\infty e^{-ra'} d_{a'}^F \left( f_{y'}^{F-F}
   + f_{y'}^{F-M} \right) \dd a'\dd y'}{ +\sigma \int _{y=0}^\infty
   \int_{a=y}^\infty e^{-ra} d_{a}^M \left( f_{y}^{M-M} + f_{y}^{M-F} \right)
   \dd a \dd y \big)}}
\end{equation}

Follow Equation~\eqref{eq:R0fromTy} to derive $R_0^\upsilon$ from $r^\upsilon$
and $T^\upsilon$. 

Arriving at sex-specific stable parameters from the two-sex parameters is
somewhat tricky. One might first estimate the stable total births to boys,
$\widehat{B^M}$ as:
\begin{equation}
\begin{split}
\label{eq:bboys0}
\widehat{B^M} = \frac{(1-\sigma)}{2} \int _{y'=0}^\infty \int _{a'=y'}^\infty
e^{-r^\upsilon a'} d_{a'}^F f_{y'}^{F-M} \dd a' \dd y' \\
+ \frac{\sigma}{2} \int _{y=0}^\infty \int _{a=y}^\infty e^{-r^\upsilon a}
d_{a}^M f_{y}^{M-M} \dd a \dd y
\end{split}
\end{equation}
, and a first estimate of births to girls,$\widehat{B^F}$, by:
\begin{equation}
\begin{split}
\label{eq:bgirls0}
\widehat{B^F} = \frac{(1-\sigma)}{2} \int _{y'=0}^\infty \int _{a'=y'}^\infty
e^{-r^\upsilon a'} d_{a'}^F f_{y'}^{F-F} \dd a' \dd y' \\
+ \frac{\sigma}{2} \int _{y=0}^\infty \int _{a=y}^\infty e^{-r^\upsilon a}
d_{a}^M f_{y}^{M-F} \dd a \dd y
\end{split}
\end{equation}
, but one notices that the stable population exposures, captured in
``$e^{-r^\upsilon a'} d_{a'}^F$'' and ``$e^{-r^\upsilon a}d_{a}^M$'' do not
adequately reflect the fact that males and females are born into the populations with a particular ratio, $\hat{S}$,
which is approximately (but not exactly) $\frac{\widehat{B^M}}{\widehat{B^F}}$ from
Equations~\eqref{eq:bboys0}~and~\eqref{eq:bgirls0}. This initial ratio is thus
an approximation of $S$. $\hat{S}$ can then be entered into the equations,
replacing the number 2 in the denominator of $\sigma$, as follows:
\begin{equation}
\begin{split}
\label{eq:bboys0}
B^M = (1-\sigma) \frac{1}{1+\hat{S}} \int _{y'=0}^\infty \int
_{a'=y'}^\infty e^{-r^\upsilon a'} d_{a'}^F f_{y'}^{F-M} \dd a' \dd y' \\
+ \sigma \frac{\hat{S}}{1+\hat{S}} \int _{y=0}^\infty \int _{a=y}^\infty
e^{-r^\upsilon a} d_{a}^M f_{y}^{M-M} \dd a \dd y
\end{split}
\end{equation}
and analagously for girls:
\begin{equation}
\begin{split}
\label{eq:bgirls0}
B^F = (1-\sigma)\frac{1}{1+\hat{S}}\int _{y'=0}^\infty \int _{a'=y'}^\infty
e^{-r^\upsilon a'} d_{a'}^F f_{y'}^{F-F} \dd a' \dd y' \\
+ \sigma \frac{\hat{S}}{1+\hat{S}}\int _{y=0}^\infty \int _{a=y}^\infty
e^{-r^\upsilon a} d_{a}^M f_{y}^{M-F} \dd a \dd y
\end{split}
\end{equation}
, giving the stable sex ratio at birth, $S^\upsilon$, as:
\begin{equation}
S^\upsilon = \frac{B^M}{B^F}
\end{equation}
$S$ only differs from $\hat{S}$ in the \nth{6} digit, and after the single
iteration just described obtains a constant value\footnote{The author offers
no proof of this single-iteration stability, but it was consistently observed
in numerical tests. There is very little ``wiggle-room'' for the sex ratio at
birth, despite the fact that the model allows $S$ to vary by $e_x$ of mother and
$e_x$ of father. Indeed, there are consistent sex-specific patterns in the sex ratio
at birth, both by mothers' and of fathers' $e_x$, but the
variariation in these is insufficient to move the stable value in continued
recursions of the above-described iteration.}. $S$ allows us to calculate the
remaining stable parameters. The both-sex stable birth rate, $b^\upsilon$ is given by:
\begin{equation}
b^\upsilon = \frac{1}{
            \splitfrac{\big(\frac{S^\upsilon}{1+S^\upsilon} \int _{y'=0}^\infty
            \int _{a'=y'}^\infty e^{-r^\upsilon a'} d_{a'}^F \dd a' \dd y'}{ + 
             \frac{1}{1+S^\upsilon} \int _{y=0}^\infty \int _{a=y}^\infty
             e^{-r^\upsilon a} d_{a}^M \dd a \dd y\big)}  }                   
\end{equation}
, which can be used to derive the stable $e_x$-structure of males and females,
$c_y^\upsilon$ and $c_{y'}^\upsilon$, respectively:

\begin{equation}
c_{y'}^\upsilon = b^\upsilon \frac{1}{1+S^\upsilon} \int _{a'=y'}^\infty
e^{-ra'} d_{a'}^F \dd a'
\end{equation}
, and
\begin{equation}
c_{y}^\upsilon = b^\upsilon \frac{S^\upsilon}{1+S^\upsilon} \int _{a=y}^\infty
e^{-ra} d_{a}^M \dd a'
\end{equation}
, where:
\begin{equation}
1 = \int c_{y'}^\upsilon \dd y' + \int c_{y}^\upsilon \dd y
\end{equation}
, the sex ratio in any given age, $S_y$, is:
\begin{equation}
S_y^\upsilon = \frac{c_{y}^\upsilon}{c_{y'}^\upsilon}
\end{equation}
, and the the overall sex ratio,$S^T$ will be:
\begin{equation}
S^T= \frac{\int c_{y}^\upsilon \dd y}{\int c_{y'}^\upsilon \dd y'}
\end{equation}

\paragraph{Reflections on the $e_x$-structured linear two-sex model:}
I posit that there exists a formal identity to relate the various results
(e.g. $r^f{y'}$ to $r^{\upsilon (0)}$), just as \citet[pp. 56]{coale1972growth} relates 
the age-structured $r^m$ and $r^f$, but this fruit will be left on the tree
for the time being.

Most important, as is visible in Figure~\ref{fig:rupsilonlinear2sex}, there is
simply very little spread in growth rates between the positions of extreme
dominance. One intuitively wishes to see a non-linear two-sex model that
accounts for interactions between both sexes and remaining years of life, just
as one wishes, in an age-structured model to allow for fluid interactions
between sex and age. In such a model, the laws of supply and demand would move
$\sigma$ according to the relative weight of male and female exposure. However,
the distance between male and female dominance represents around twice the
maximum difference in $r$ that one would observe upon applying the more
sophisticated model. This statement assumes 1) that the interactive model is
bounded by the dominant cases presented here, and 2) that one is comparing with
the case of $\sigma = 0.5$, a prudent choice. 

As a secondary point, notice also that the present linear model holds rates
constant with respect to remaining life expectancy, but \textit{not} with
respect to age. From year to year the population structure with
respect to remaining life expectancy changes, as does the underlying age
structure. One could re-derive age-specific fertility rates from the
$e_x$-specific fertility rates used here, and would note that since the
weighting variable has changed with time, so too would the weighted sum of
the $e_x$-specific rates inherent in any age-specific rate. This observation
heeds \citet{stolnitz1949recent}, who point out several ways in which
fertility rates are indeed simply weighted sums of even more specific weights.
Prior to the formulation of the present model we have pointed out another
dimension in which age (parity-race-class) -specific rates are weighted sums, and we have exploited
that, short of holding very cross-classified rates constant, one observes
greater stability over time with $e_x$-classified rates. Holding
$e_x$-classified rates constant will force underlying age-specific rates to fold
and adapt with each passing year (albeit not much). Forcing age to adjust in
accord with constant $e_x$-specific rates appears to this author to be just as
palatable as forcing $e_x$-specific rates to change under the constraint of
constant age-specific rates-- perhaps moreso. This judegment is passed on having
compared the observed volatility in the two kinds of specific rates and deciding
$e_x$-specific rates are more reconciliable with the stable population
assumption of fixed rates. This difference is not necessarily large, and may in
any case be an accident of history, as we have not pondered upon why it is that
$e_x$-specific rates would hold more constant over time than age-specific rates.
Part of this may owe to inadequancies in the method used to redistribute
age-classified data to $e_x$-classified data, as the method is new, and has not
undergone scrutiny beyond this very dissertation.



\subsection{}