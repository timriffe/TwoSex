
It has been noted that divergence between the sexes, in terms of predicted birth
counts, is often dampened when projected using rates and populations that are
structured according to remaining years. This does not, however, mean that the
problem of the sexes is in this context negligible. Instead, the problem has
only become slightly more tractable. I consider the problem more tractable
because in decreasing the magnitude of discrepancy between male and female
rates, the trade-offs inherent in the various two-sex solutions offered in the
literature also become smaller. This section will introduce two kinds of two-sex
extensions of the $e_x$-structured population model introduced in
Section~\ref{sec:exstructuredrenewal}:
\begin{enumerate}
  \item A linear extension, assuming fixed weights for males and female,
  inspired by \citet{goodman1967age}. 
\end{enumerate}


\subsubsection{A linear two-sex extension to the $e_x$-structured population
model} 

\citet{goodman1967age} offers a suite of
formulas to determine the stable age-sex composition of a population taking into
account the vital rates of both sexes, assuming one can assign a relative weight
(summing to 1) to male and female fertility. In this case there are two
trade-offs: 1) one must (arbitrarily) choose weights, and 2) these weights are
constant. The fact of having constant weights keeps the solution linear
(interaction-free), but less realistic. The final result is bracketed by the
cases of male and female dominance, but the gap between these two extremes 
also measures the demgographer's subjective leeway, which we would like the
minimize. Both of these drawbacks may be reduced in the case of $e_x$-structured
populations, since: 
\begin{enumerate}
  \item $e_x$-structured populations have a more stable structure
than age-structured populations. [add section or appendix to prove this]
\item Mate-selection with respect to remaining years of
life is nearly random in $e_x$-structured populations (see
Section~\ref{sec:exobsexpected}).
\item The difference between male and female dominance (in terms of projected
birth counts) is often reduced, thereby limiting of the impact of the
demographer's ``dominance caprice'' on results (See
Section~\ref{sec:exdivergence}).
\end{enumerate}
Points (1) and (2) reduce (but do
not eliminate) the necessity of sex-interactions in a model. By this it is meant
that the proportional difference in results from one choice of model weights
over another is simply diminished. This being so, the comparative advantage of a
more sophisticated or realistic model is to some degree diminished. 

\paragraph{Formal definition of the Lotka-type renewal equation}



Data that would permit empirical studies of the two-sex problem are
typically sufficiently rich to allow for cross-tabulations by age of both
parents as well as sex of birth. Therefore, formulas in continuation assume that
rates are available by sex of progenitor, birth (4 combinations) and age (to be
transformed to remaining years)

\paragraph{Construction of the corresponding projection matrix}

