
It has been noted that divergence between the sexes, in terms of predicted birth
counts, is often dampened when projected using rates and populations that are
structured according to remaining years. This does not, however, mean that the
problem of the sexes is in this context negligible. Instead, the problem has
only become slightly more tractable. I consider the problem more tractable
because in decreasing the magnitude of discrepancy between male and female
rates, the trade-offs inherent in the various two-sex solutions offered in the
literature also become smaller. This section will introduce two kinds of two-sex
extensions of the $e_x$-structured population model introduced in
Section~\ref{sec:exstructuredrenewal}:
\begin{enumerate}
  \item A linear extension, assuming fixed weights for males and female,
  inspired by \citet{goodman1967age}. 
  \item A non-linear extension, following the example in
  \citet{gupta1978alternative}.
  \item another non-linear extension? might need to break out sections\ldots
\end{enumerate}

\subsubsection{A linear two-sex extension to the $e_x$-structured population
model} 

\citet{goodman1967age} offers a suite of
formulas to determine the stable age-sex composition of a population taking into
account the vital rates of both sexes, assuming one can assign a relative weight
(summing to 1) to male and female fertility. In this case there are two
trade-offs: 1) one must (arbitrarily) choose weights, and 2) these weights are
constant. The fact of having constant weights keeps the solution linear
(interaction-free), but less realistic. The final result is bracketed by the
cases of male and female dominance, but the gap between these two extremes 
also measures the demgographer's subjective leeway, which we would like the
minimize. Both of these drawbacks may be reduced in the case of $e_x$-structured
populations, since: 
\begin{enumerate}
  \item $e_x$-structured populations have a more stable structure
than age-structured populations. [add section or appendix to prove this]
\item Mate-selection with respect to remaining years of
life is nearly random in $e_x$-structured populations (see
Section~\ref{sec:exobsexpected}).
\item The difference between male and female dominance (in terms of projected
birth counts) is often reduced, thereby limiting of the impact of the
demographer's ``dominance caprice'' on results (See
Section~\ref{sec:exdivergence}).
\end{enumerate}
Points (1) and (2) reduce (but do
not eliminate) the necessity of sex-interactions in a model. By this it is meant
that the proportional difference in results from one choice of model weights
over another is simply diminished. This being so, the comparative advantage of a
more sophisticated or realistic model is to some degree diminished. 

\paragraph{Formal definition of the Lotka-type renewal equation:}

As mentioned, choose some weight, $\sigma$, between 0 and 1 to apply to male
rates, where the female weight is defined as $1 - \sigma$. When $\sigma = 1$
there is perfect male dominance, and when $\sigma = 0$ there is perfect female
dominance. Of course, births to girls are subject to female mortality and births
to boys are subject to male mortality. As with
Equation~\eqref{eq:exLotkafemales}, this mortality enters in the equation by way
of the $d_x$ distribution used to distribute births over life expectancies. The
final renewal formula takes is defined as such:

\begin{equation}
\label{eq:lineartwosexrenewal}
\begin{split}
2 = (1 - \sigma) \left(\;\;
            \int _{y'=0}^\infty \int _{a'=y'}^\infty e^{-ra'}
                      d_{a'}^F f_{y'}^{F-F} \dd a' \dd y' + 
            \int _{y'=0}^\infty \int _{a=y'}^\infty e^{-ra}
                      d_{a}^M f_{y'}^{F-M} \dd a \dd y'\right) + \\ 
     \sigma \left(\;\;
            \int _{y=0}^\infty \int _{a'=y}^\infty e^{-ra'}
                      d_{a'}^F f_{y}^{M-F} \dd a' \dd y + 
            \int _{y=0}^\infty \int _{a=y}^\infty e^{-ra}
                      d_{a}^M f_{y}^{M-M} \dd a \dd y \right)
\end{split}
\end{equation}

Where $a'$, $y'$, $a$ and $y$ index female age, female remaining years, male age
and male remaining years, respectively. Fertility superscripts identify sex of
progentitor followed by sex of offspring, and $d_x$ must accord with the sex of
offspring. Such specific rates are chosen because data that would permit
empirical studies of the two-sex problem are typically sufficiently rich to 
allow for cross-tabulations by age of both parents as well as sex of birth.
Therefore, Equation~\eqref{eq:lineartwosexrenewal} assumes that rates are
available by sex of progenitor, birth (4 combinations) and age (to be transformed to remaining years), 
and no additional variable is required for the sex ratio at birth. 

\paragraph{An iterative approach to find $r$:}
\label{sec:exrenewalit2}
Steps to practically solve Equation~\eqref{eq:lineartwosexrenewal} for $r$ are
similar to those presented for the one-sex case in
Section~\ref{sec:exrenewalit}
\begin{enumerate}
  \item Derive a first rough estimate of the both-sex mean remaining years of
  life at reproduction, $\widehat{T^\upsilon}$, akin to Lotka's mean generation time,
  $T$. If one assumes a growth rate of $0$, then a good-enough guess will be:

\begin{equation}
\widehat{T^\upsilon} = \frac{\int _{y'=0}^\infty \int _{a'=y'}^\infty a'
d_{a'}^F f_{y'}^F \dd a' \dd y' + \int _{y=0}^\infty \int _{a=y}^\infty a
d_{a}^M f_{y}^M \dd a \dd y}{\int _{y'=0}^\infty \int _{a'=y'}^\infty
d_{a'}^F f_{y'}^F \dd a' \dd y' + \int _{y=0}^\infty \int _{a=y}^\infty
d_{a}^M f_{y}^M \dd a \dd y}
\end{equation}

  \item A first rough estimate of the net reproduction rate, $R_0$ is given by:
 \begin{equation}
 \begin{split}
 \frac{(1 - \sigma)}{2} \left(\;\;
            \int _{y'=0}^\infty \int _{a'=y'}^\infty 
                      d_{a'}^F f_{y'}^{F-F} \dd a' \dd y' + 
            \int _{y'=0}^\infty \int _{a=y'}^\infty 
                      d_{a}^M f_{y'}^{F-M} \dd a \dd y'\right) + \\ 
     \frac{\sigma}{2} \left(\;\;
            \int _{y=0}^\infty \int _{a'=y}^\infty 
                      d_{a'}^F f_{y}^{M-F} \dd a' \dd y + 
            \int _{y=0}^\infty \int _{a=y}^\infty 
                      d_{a}^M f_{y}^{M-M} \dd a \dd y \right)
 \end{split}
 \end{equation}
  \item A first rough estimate of $r$, $r^0$, is given by:
   \begin{equation}
   r^0 = \frac{ln(R_0)}{\widehat{T^\upsilon}}
   \end{equation}
  \item plug $r^0$ into Equation~\ref{eq:lineartwosexrenewal} to calculate a
  residual, $\delta^0$
  \item use $\delta^0$ and $\widehat{T^\upsilon}$ to calibrate the estimate of $r$
  using:
  \begin{equation}
  r^{1} = r^0 + \frac{\delta^0}{\widehat{T^\upsilon} - \frac{\delta^0}{r^0}}
  \end{equation}
  \item repeat step (3) to to derive a new $\delta^i$, then step (4) to refine
  $r^i$, until converging on a stable $r$ after some 30 iterations,
  depending on the degree of precision desired. ($\widehat{T^\upsilon}$ is not updated
  in this process).
\end{enumerate}

\paragraph{Construction of the corresponding projection matrix:}

