\label{sec:exLotka2linear}
It has been noted that divergence between the sexes, in terms of predicted birth
counts, is often dampened when projected using rates and populations that are
structured according to remaining years. This does not, however, mean that the
problem of the sexes is in this context negligible. Instead, the problem has
only become slightly more tractable. I consider the problem more tractable
because in decreasing the magnitude of discrepancy between male and female
rates, the trade-offs inherent in the various two-sex solutions offered in the
literature also become smaller. This section will introduce two kinds of two-sex
extensions of the $e_x$-structured population model introduced in
Section~\ref{sec:exstructuredrenewal}:
\begin{enumerate}
  \item A linear extension, assuming fixed weights for males and female,
  inspired by \citet{goodman1967age}. 
  \item A non-linear extension, following the example in
  \citet{gupta1978alternative}.
  \item another non-linear extension? might need to break out sections\ldots
\end{enumerate}

\subsubsection{A linear two-sex extension to the $e_x$-structured population
model} 

\citet{goodman1967age} offers a suite of
formulas to determine the stable age-sex composition of a population taking into
account the vital rates of both sexes, assuming one can assign a relative weight
(summing to 1) to male and female fertility. In this case there are two
trade-offs: 1) one must (arbitrarily) choose weights, and 2) these weights are
constant. The fact of having constant weights keeps the solution linear
(interaction-free), but less realistic. The final result is bracketed by the
cases of male and female dominance, but the gap between these two extremes 
also measures the demgographer's subjective leeway, which we would like the
minimize. Both of these drawbacks may be reduced in the case of $e_x$-structured
populations, since: 
\begin{enumerate}
  \item $e_x$-structured populations have a more stable structure
than age-structured populations. [add section or appendix to prove this]
\item Mate-selection with respect to remaining years of
life is nearly random in $e_x$-structured populations (see
Section~\ref{sec:exobsexpected}).
\item The difference between male and female dominance (in terms of projected
birth counts) is often reduced, thereby limiting of the impact of the
demographer's ``dominance caprice'' on results (See
Section~\ref{sec:exdivergence}).
\end{enumerate}
Points (1) and (2) reduce (but do
not eliminate) the necessity of sex-interactions in a model. By this it is meant
that the proportional difference in results from one choice of model weights
over another is simply diminished. This being so, the comparative advantage of a
more sophisticated or realistic model is to some degree diminished. 

\paragraph{Formal definition of the Lotka-type renewal equation:}

As mentioned, choose some weight, $\sigma$, between 0 and 1 to apply to male
rates, where the female weight is defined as $1 - \sigma$. When $\sigma = 1$
there is perfect male dominance, and when $\sigma = 0$ there is perfect female
dominance. Of course, births to girls are subject to female mortality and births
to boys are subject to male mortality. As with
Equation~\eqref{eq:exLotkafemales}, this mortality enters in the equation by way
of the $d_x$ distribution used to distribute births over life expectancies. The
final renewal formula takes is defined as such:

\begin{equation}
\label{eq:lineartwosexrenewal}
\begin{split}
1 = \frac{1 - \sigma}{2} \left(\;\;
            \int _{y'=0}^\infty \int _{a'=y'}^\infty e^{-ra'}
                      d_{a'}^F f_{y'}^{F-F} \dd a' \dd y' + 
            \int _{y'=0}^\infty \int _{a=y'}^\infty e^{-ra}
                      d_{a}^M f_{y'}^{F-M} \dd a \dd y'\right) + \\ 
     \frac{\sigma}{2} \left(\;\;
            \int _{y=0}^\infty \int _{a'=y}^\infty e^{-ra'}
                      d_{a'}^F f_{y}^{M-F} \dd a' \dd y + 
            \int _{y=0}^\infty \int _{a=y}^\infty e^{-ra}
                      d_{a}^M f_{y}^{M-M} \dd a \dd y \right)
\end{split}
\end{equation}
, where $a'$, $y'$, $a$ and $y$ index female age, female remaining years, male
age and male remaining years, respectively. Fertility superscripts identify sex of
progentitor followed by sex of offspring, and $d_x$ must accord with the sex of
offspring. Such specific rates are chosen because data that would permit
empirical studies of the two-sex problem are typically sufficiently rich to allow 
for cross-tabulations by age of both parents as well as sex of birth. 
Therefore, Equation~\eqref{eq:lineartwosexrenewal} assumes that rates are
available by sex of progenitor, birth (4 combinations) and age (to be transformed to remaining years), 
and no additional variable is required for the sex ratio at birth. 

Weights, $\sigma$ and $1-\sigma$ are divided by 2 because
total births are counted twice in total (males and females from males \&
males and females from females). One could just as easily optimize to a sum of
2 on the left-hand side rather than discount weights.

The linear two-sex $r$, $r^\upsilon$ given by~\eqref{eq:lineartwosexrenewal} is
\textit{not} guaranteed to be bounded by the $e_x$-structured $r^f$ and $r^m$,
and indeed $r^f$ and $r^m$ may not be recovered by setting $\sigma$ to 0 or 1,
respectively. This is so because the model includes births of both sexes to
progenitors of each sex, and the births of each sex are then subject to
sex-specific death distributions. That is to say, manipulation of $\sigma$ is
insufficient to make the one-sex model a degenerate case of the present model, save for the
exceptionally degenerate case where males and females have identical mortality.
Other authors in the past have been content to work with single-sex ($M-M$ and
$F-F$) fertility only, but this appears unncessesary and unintuitive.
\footnote{\citet{pollard1948measurement} took this idea even further by swapping
sexes: The fertility functions in this paper are based on the births of boys to mothers 
and girls to fathers, i.e. $M-F$ and $F-M$ fertility. This is parsimonious 
in terms of getting quick results that are guaranteed to fall within reasonable bounds, but is less
intuitively appealing}. The later choice would both reduce the complexity of
Equation~\ref{eq:lineartwosexrenewal} and guarantee exact bounds of $r^f$ and $r^m$ 
when $\sigma$ is set to 0 and 1, respectively. This author does not recognize 
the theoretical or practical merits of the single-sex modelling choice, as it 
is not the case that males are responsible for the birth of boys and females 
for the birth of girls. This stance couples with the author's choice to not
include an explicit, let alone constant, variable for the sex ratio at birth.

It must be
noted that the value of $r^\upsilon$ is dependant upon the choice of $\sigma$, and that no guidelines are provided for choosing a good value of $\sigma$. This ambiguity also exists in the age-structured variant of the present model. For $e_x$-structured models, it has been claimed that sex-divergence is lesser than is the case for
age-structured models. Recall that this was the case for predictions of birth
counts, and not for the growth parameter, $r^\upsilon$. The
difference between the $e_x$-structured $r^f$ and $r^m$ is not necessarily lesser than is the case for
the age-structured $r^f$ and $r^m$. This will be discussed further along with
empirical results for the two populations considered in this dissertation.

\paragraph{An iterative approach to find $r^\upsilon$:}
\label{sec:exrenewalit2}
Steps to practically solve Equation~\eqref{eq:lineartwosexrenewal} for $r$ are
similar to those presented for the one-sex case in
Section~\ref{sec:exrenewalit}
\begin{enumerate}
  \item Derive a first rough estimate of the both-sex mean remaining years of
  life at reproduction, $\widehat{T^\upsilon}$, akin to Lotka's mean generation time,
  $T$. If one assumes a growth rate of $0$, then a good-enough guess will be:

\begin{equation}
\widehat{T^\upsilon} = \frac{\int _{y'=0}^\infty \int _{a'=y'}^\infty y'
d_{a'}^F f_{y'}^F \dd a' \dd y' + \int _{y=0}^\infty \int _{a=y}^\infty y
d_{a}^M f_{y}^M \dd a \dd y}{\int _{y'=0}^\infty \int _{a'=y'}^\infty
d_{a'}^F f_{y'}^F \dd a' \dd y' + \int _{y=0}^\infty \int _{a=y}^\infty
d_{a}^M f_{y}^M \dd a \dd y}
\end{equation}

  \item A first rough estimate of the net reproduction rate, $R_0$ is given by:
 \begin{equation}
 \begin{split}
 \frac{(1 - \sigma)}{2} \left(\;\;
            \int _{y'=0}^\infty \int _{a'=y'}^\infty 
                      d_{a'}^F f_{y'}^{F-F} \dd a' \dd y' + 
            \int _{y'=0}^\infty \int _{a=y'}^\infty 
                      d_{a}^M f_{y'}^{F-M} \dd a \dd y'\right) + \\ 
     \frac{\sigma}{2} \left(\;\;
            \int _{y=0}^\infty \int _{a'=y}^\infty 
                      d_{a'}^F f_{y}^{M-F} \dd a' \dd y + 
            \int _{y=0}^\infty \int _{a=y}^\infty 
                      d_{a}^M f_{y}^{M-M} \dd a \dd y \right)
 \end{split}
 \end{equation}
  \item A first rough estimate of $r$, $r^0$, is given by:
   \begin{equation}
   r^0 = \frac{ln(R_0)}{\widehat{T^\upsilon}}
   \end{equation}
  \item plug $r^0$ into Equation~\ref{eq:lineartwosexrenewal} to calculate a
  residual, $\delta^0$
  \item use $\delta^0$ and $\widehat{T^\upsilon}$ to calibrate the estimate of $r$
  using:
  \begin{equation}
  r^{1} = r^0 + \frac{\delta^0}{\widehat{T^\upsilon} - \frac{\delta^0}{r^0}}
  \end{equation}
  \item repeat step (3) to to derive a new $\delta^i$, then step (4) to refine
  $r^i$, until converging on a stable $r$ after some 30 iterations,
  depending on the degree of precision desired. ($\widehat{T^\upsilon}$ is not updated
  in this process).
\end{enumerate}

This procedure has been applied to the data from the US and Spain with $\sigma
= [0,.5,1]$, which correspond to the cases of pseudo female-dominace, an
intermediate value, and pseudo male-dominance, and can be seen in
Figure~\ref{fig:rupsilonlinear2sex}

\begin{figure}[!ht]
  \centering
    \caption{Two-sex linear intrinsic growth rate, $r^\upsilon$, according to
    renewal Equation~\eqref{eq:lineartwosexrenewal}, with $\sigma = [0, .5, 1]$ US 
    and Spain, 1969-2009.}
     % figure produced in /R/ExLotka2Sex.R
     \includegraphics{Figures/exLotka2sexlinear}
     \label{fig:rupsilonlinear2sex}
\end{figure}

Patterns accord with trends generally known from the age-classified $r^f$ and
$r^m$, but values of $r^\upsilon$ are shown to be higher than in any year

\paragraph{Other reproduction parameters:}

Once two-sex linear $r^\upsilon$ has been found for the given $\sigma$, one may
proceed to find the corrsponding both-sex $T^\upsilon$, using the same strategy as in
Equation~\eqref{eq:Ty}, but with Equation~\eqref{eq:lineartwosexrenewal} in the
numerator and denominator. In this case, simply insert $y$ and $y'$,
respectively within the four integrals of the numerator\footnote{Writing out the
equation for $T^\upsilon$ would introduce unnecessary clutter, ocuppying twice the space as
Equation~\eqref{eq:lineartwosexrenewal}\ldots}. Follow
Equation~\eqref{eq:R0fromTy} to derive $R_0^\upsilon$ from $r^\upsilon$ and
$T^\upsilon$.

The both-sex birth rate, $b^\upsilon$ is given by:
\begin{equation}
b^\upsilon = \frac{1}{
            \int _{y'=0}^\infty \int _{a'=y'}^\infty e^{-r^\upsilon a'}
                      d_{a'}^F \dd a' \dd y' + 
            \int _{y=0}^\infty \int _{a=y}^\infty e^{-r^\upsilon a}
                      d_{a}^M \dd a \dd y}                     
\end{equation}

The female stable age structure is given by:

\begin{equation}
c_{y'} = b^\upsilon \int _{a'=y'}^\infty e^{-ra'} d_{a'}^F \dd a'
\end{equation}
, and the male stable age structure is given similarly.



\paragraph{Construction of the corresponding projection matrix:}

