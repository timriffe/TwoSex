
This characteristic of human populations, sexual dimorphism,
is manifest throughout demography. In observed populations,
fluctuations in vital rates are constantly underway, and can either magnify or
diminish differences between the sexes. In theoretical population models,
dimorphism is typically limited to fertility and mortality rates. Fertility 
is a two-sex interactive phenomenon, but in practice 
population models, such as projections, do not often treat it as
such, instead opting for female dominance in fertility rates. Mortality is
always modelled separately for both sexes, and can be reasonably thought of 
as independent for each sex. 

% not sure
Even if male and female
fertility schedules were in agreement, mortality differences would also
cause divergence between male and female single-sex models. 

% sex ratio assumptions mainly relevant for female dominant projections,
% not divergence
The sex ratio at
birth (SRB) varies from year to year, as well as over age, but it is usually
treated as a global variable, uniform over age and time. For the USA this is 
a harmless assumption, but for Spain, 
assumptions of a constant SRB would have been far off the mark. After
(hypothetically) forcing agreement between male and female fertility and
morality, fluctuations in the SRB would also lead to long run divergence.



Observed age-specific vital rates for Spain and the United States serve 
well to illustrate the peculiarities of the problem.  %

% ASFR for 1975, both countries
\begin{figure}[ht!]
        \centering  
          \caption{Male and Female Age-Specific Fertility Rates, 1975, USA and
          Spain}
           % figure produced in
           % /R/ObservedVsExpectedBivariateBirthDistribution.R
           \makebox[\textwidth]{\includegraphics{Figures/ASFR1975}}
          %\includegraphics{Figures/ObservedvsExpectedBxy}
          \label{fig:ASFR1975}
\end{figure}


% TFR 1969- 2010
\begin{figure}[ht!]
        \centering  
          \caption{Male and Female Total Fertility Rates, 1969-2009, USA and
          Spain}
           % figure produced in
           % /R/ObservedVsExpectedBivariateBirthDistribution.R
           \makebox[\textwidth]{\includegraphics{Figures/TFR}}
          %\includegraphics{Figures/ObservedvsExpectedBxy}
          \label{fig:TFRseries}
\end{figure}
% NRR single sex? US SRB?

\begin{figure}[ht!]
        \centering  
          \caption{Male and Female Net Reproduction Rates, 1969-2009, USA and
          Spain}
           % figure produced in
           % /R/ObservedVsExpectedBivariateBirthDistribution.R
           \makebox[\textwidth]{\includegraphics{Figures/R0mf}}
          %\includegraphics{Figures/ObservedvsExpectedBxy}
          \label{fig:NRRseries}
\end{figure}

% vs net TFR

% Can't assume constant SRB

% mortality divergence: compare e0, rates, inverse pyramid?

Sexual dimorphism in mortality is of primary significance to human reproduction.
Parents must survive in order to parent, and children must survive in order to
become parents. This later element, survival until reproductive ages, enters
directly into canoncial indicators such as the Net Reproduction Rate. Parental
survival does not, to the knowledge of this author, enter into indicators of
population reproductivity, except in the less-iluminating sense that parents
must survive in order to progress birth parities. This section will briefly present 
some novel methods and indicators for weighing mortality into measures of
reproductivity.