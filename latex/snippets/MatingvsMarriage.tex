
\citet{stolnitz1949recent} point out that 

\begin{citation}
nuptiality patterns
must be explicitly taken into acount in analyzing the progress of the age structure of the
population from it current to its stable form. Otherwise, the assumptio of
constant age-specific fertility rates would, in general, imply changes in
marital composition, marital fertility, or both-- results hardly consonant with
an analytical approach predicated on the assumption of ``fixed'' fertility
conditions.
\end{citation}

In present-day populations, mating really is the relevant concept here. There is
of course further heterogeneity within the concept of mating, as it is in
practice a stage-driven process, and can terminate in different situations of
\textit{matedness}, such as the well-known behavioral divide between marriage
and cohabitation. Marriage is in this sense a mere proxy for
\texit{mated-pairs} or \textit{childbearing unions}, and its efficiency as a
proxy has both decreased over time and varies between populations. To account exclusively for marriage transitions
in multistate demographic models was once advantageous. I
advocate either accounting separately for the state transitions of cohabitation
and nuptiality or ignoring differences between cohabitors and married-couples
and inluding both in the same \textit{mated-pair} state so as to more fully
capture the population of reproducers. Either of these refinements would still
essentially be proxies, but I expect them to serve better than marriage, per se.

