Stolnitz and Ryder

Purging demographic indicators of distortion from population structure.

Synthetic indices (TFR, lifetable quantities) were originally designed to purge
measures of interference from population structure.
\citet{kuczynski1932fertility, kuczynski1935measurement} pointed out several
instances of misleading conclusions that would reult from judging the growth
potential of a population on crude rates alone. \citep{stolnitz1949recent}
point out that the practice of using synthetic rates gained interpretative
convenience through the development of stable population theory
\citep{sharpe1911problem, lotka1922stability}, where the so-called Net
Reproduction Rate (NRR, or $R_0$) is doubly representative of the lifetime
average number of daughters per woman, as well as the population growth
multiplier per mean generation time. Further, NRR and the TFR contained within
it, can be calculated with data from a single period.

\citet{stolnitz1949recent} summarize the shortcomings of common synthetic
indices such as TFR in terms of their failure to fully remove further population structure effects. Age
specific rates namely control for age-hetorogeneity in a population, but not
other kinds of relevant population heterogeneity, such as parity. A given
age-specific fertility rate can be thought of, they explain, as a weighted sum
of the rates pertinent to each disaggregated population category within it,
where weights are the exposures specific to each category. A crude fertilty rate
is in this sense, a multidimentionally weighted sum, and a true indicator of
\textit{behavior} will be independent of such population weights. Extra
structure purging is tyically acheived by further dissagregation, producing
separate fertility indices for each parity or marital status or duration, for
instance. \citet[p. 120]{stolnitz1949recent} state that

\begin{citation}
[c]onsequently, the assumption of fixed future age-specific fertility rates is
tantamount to assuming variations in age-parity-specific rates.
\end{citation}






 Age-specific rates are namely also subject to further

Thus NRR and the TFR component contained
within make simultaneous reference to individual behaviour and aggregate
populaion change

Parsimonious demographic indices with
substantive interpretations are of course those that find footing beyond the
discipline of demography, and so 


 wherein the sum-product of a single-sex survival function and single-sex fertility function (net reproduction), simultaneously represents the