
%* all figs here produced in IllustrateDivergence.R

Divergence of single-sex population models is problem of practical significance
for demographers, and it stems from the fact that vital rates almost always
differ between the sexes, except in rare coincidences when rates cross over
time. This characteristic of human populations, sexual dimorphism,
is manifests itself in all concerns of demography. In observed populations,
fluctuations in sexual dimorphism are constantly underway. In theoretical
population models, dimorphism is typically limited to fertility and mortality
rates and the sex ratio at birth. Fertility is a two-sex interactive phenomenon,
but in practice population models, such as projections, do not often treat it as
such. Mortality is always modelled separately for both sexes, and can
be reasonably thought of as independent for each sex. If if male and female
rate schedules were identical, mortality differences would also
cause divergence between male and female single-sex models. The sex ratio at
birth (SRB) varies from year to year, but is usually treated as a global variable, 
uniform over age and time. For the USA this is a harmless assumption, but for Spain, 
assumptions of a constant SRB have been very wrong. Even if male and female fertility and morality schedules
were identical, Observed age-specific vital rates for Spain and the United
States serve well to illustrate the peculiarities of the problem.  %

% ASFR for 1975, both countries
\begin{figure}[ht!]
        \centering  
          \caption{Male and Female Age-Specific Fertility Rates, 1975, USA and
          Spain}
           % figure produced in
           % /R/ObservedVsExpectedBivariateBirthDistribution.R
           \makebox[\textwidth]{\includegraphics{Figures/ASFR1975}}
          %\includegraphics{Figures/ObservedvsExpectedBxy}
          \label{fig:ASFR1975}
\end{figure}


% TFR 1969- 2010
\begin{figure}[ht!]
        \centering  
          \caption{Male and Female Total Fertility Rates, 1969-2009, USA and
          Spain}
           % figure produced in
           % /R/ObservedVsExpectedBivariateBirthDistribution.R
           \makebox[\textwidth]{\includegraphics{Figures/TFR}}
          %\includegraphics{Figures/ObservedvsExpectedBxy}
          \label{fig:TFRseries}
\end{figure}
% NRR single sex? US SRB?

\begin{figure}[ht!]
        \centering  
          \caption{Male and Female Net Reproduction Rates, 1969-2009, USA and
          Spain}
           % figure produced in
           % /R/ObservedVsExpectedBivariateBirthDistribution.R
           \makebox[\textwidth]{\includegraphics{Figures/R0mf}}
          %\includegraphics{Figures/ObservedvsExpectedBxy}
          \label{fig:NRRseries}
\end{figure}

% vs net TFR

% Can't assume constant SRB

% mortality divergence: compare e0, rates, inverse pyramid?

Sexual dimorphism in mortality is of primary significance to human reproduction.
Parents must survive in order to parent, and children must survive in order to
become parents. This later element, survival until reproductive ages, enters
directly into canoncial indicators such as the Net Reproduction Rate. Parental
survival does not, to the knowledge of this author, enter into indicators of
population reproductivity, except in the less-iluminating sense that parents
must survive in order to progress birth parities. This section will briefly present 
some novel methods and indicators for weighing mortality into measures of
reproductivity.


\subsubsection{Years as units of population}
Recall the concept and method of \textit{reproduction of years lived} proposed
by \citet{henry1965reflexions} and later replicated by \citet{cabre1990repro}
for Catalonia. Henry's method involves weighting a Net Reproduction Rate by
the ratio of daughters' to mothers' life expectancies:

\begin{equation}
R_{0}^{\ast} = R_{0} \frac{e_0^{daughters}}{e_0^{mothers}}
\end{equation}

This method essentially adjusts birth counts to account for changes (increases)
in mean generation length. $R_{0}^{\ast}$ is analagous to thinking of populations 
and their renovation in terms of total exposure, rather than as a given census cross-section. This
method can be calculated analagously for males, and can be operationalized 
in a variety of ways, depending on the available data. Ideally, 
$R_{0}$ is calculated for a cohort, $e_0^{daughters}$ and $e_0^{mothers}$ are
age generation-specific, and their ratio is weighted inside the calculation of
$R_0$ according to CTFR. This ideal case is not practical for the present
data because the cohort life expectancy of recent generations in unknown
\footnote{Even the oldest generation of mothers in the first year of data
presented here, $1969-50-1 = 1919$ , is not extinct. Thus all requisite cohort
life expectancies are unknown.}, and because the data window used is narrow to
cover the reproductive range of any cohort, thus cohort ASFR is not fully known.

One could reduce fertility data requirements by using period data, i.e.
use period instead of cohort ASFR and period $e_0$ instead of cohort $e_0$, both
for mothers and daughters:

\begin{equation}
R_{0}^{\ast P}(t) = \int _{\alpha = 0} ^{\beta} p_{a}^{f}(t) f_{a}^{f}(t)
\frac{e_{0}(t)}{e_{0}(t-a) \dd a}
\end{equation}

\textit{Periodizing} the formula in this way gains practicality at the cost of
potential distortion. The series of period $R_0^\ast$ for US and Spanish males
and females for the range of years treated here would look something like
Figure~\ref{fig:R0perHenry}:

\begin{figure}[ht!]
        \centering  
          \caption{Male and Female $e_0$-weighted Net Reproduction Rates,
          1969-2009, USA and Spain}
           % figure produced in
           % /R/ObservedVsExpectedBivariateBirthDistribution.R
           \makebox[\textwidth]{\includegraphics{Figures/R0perHenry}}
          %\includegraphics{Figures/ObservedvsExpectedBxy}
          \label{fig:R0perHenry}
\end{figure}

The trends in these series essentially transmit the same information
present in the sex-specific $R_0$ series shown in Figure~\ref{fig:NRRseries},
except, where period $e_0$-weighting has its greatest effect in the first years
of the series. The effect of this method is to change the units of population
from persons to years. 

\subsubsection{Redistributing population by remaining years}

Now recall the observations of \citet{sanderson2005average}, more directly
relating to the popuation pyramid. These authors note that despite ageing in a
population, the mean \textit{remaining} years to be lived may increase at the
same time. This is due to improvements in mortality offsetting increases in the
mean age of a population. I now present a method to exactly redistribute
population counts (or exposures) according to period remaining life
expectancies, rather than according to age per se.

From the lifetable of a given population, sex-specific and calculated with a
radix of 1. Note the $d_x$ now sums to 1, and is therefore a proper density
function. $d_x$ can be thought of as the probability of dying in any given age
from the perspective of a 0-year-old, according to the given year's mortality
experience. It follows that the observed population of age 0 can be redistribted
according to $d_x$ and interpreted either as the expected death counts
in each year $t+x$, or as the distribution of persons currently-aged 0
reclassified according to remaining life expectancy. This can be done similarly
for age 1, by ignoring the mortality experience of age 0, and rescaling $d_x$ to
sum to 1, or generally:

\begin{equation}
\mathbf{E}(D_t) = \int _{a = 0} ^{\infty} P_a \frac{d_a}{\int _{b = a} ^{\infty}
d_b\, \dd b} \;\dd a
\end{equation}
where $P_a$ is the population of age $a$, $d_a$ is the
lifetable density function and $\mathbf{E}(D_t)$ is the expected number of
deaths in each future year $t$, also understood as a vector of the current 
population, redistributed into categories of remaining life expectancy. The key
observation is that we can sidestep the $e_x$ column of the lifetable, since
$e_x$ is essentially a weighted mean of $d_x$. The resulting population pyramid
is heterogeneous with respect to age within any given level of remaining life
expectancy, and looks like
Figures~\ref{fig:exPyrUS} and \ref{fig:exPyrES}\footnote{The idea to
redistribute the population pyramid in this way is due to a conversation with 
John MacInnes, and appears in \citep{MacInnes2013pop} (unpublished) using a
different method.} for the years 1975 and 2009. As a helpful pointer, note
that the population at the base of the pyramid is expected to decrement
within the \textit{next year}, thus the vertical axis can also can also be
thought of as year $t+x$, although $e_x$ more clearly identifies the pyramid
with year $t$ mortality conditions. In other words, the pyramid should not be
taken out of context as a forecast. Note that this pyramid represents the
exact same population: Underlying males sum to the correct total on the left and
females sum to the correct total on the right. Both pyramids have been rescaled to sum to 100, in 
order to more comparably represent population structure. \footnote{Incidentally,
a time series of remaining life expectancy pyramids for any given Western country will show 
 incredible stability over time, which is remarkable in light of
 observed ageing in the observed population pyramid. The simple interpretation
 of this kind of pyramid adds to its utility, and this author believes that $e_x$-specific
 population structure, and other here-unmentioned indicators that can be derived
 from this method sould make up a valuable new component to the contemporary
 demographer's toolbox, as well as help inform current population debates.}

\begin{figure}
        \centering
        \begin{subfigure}
                \centering
                \caption{US years-lived by $e_x$, 1975 and 2009}
                \includegraphics[scale = .8]{Figures/exPyramidUS}
                \label{fig:exPyrUS}
        \end{subfigure}
        \begin{subfigure}
                \centering
                \caption{Spain years-lived by $e_x$, 1975 and 2009}
                \includegraphics[scale = .8]{Figures/exPyramidES}
               
                \label{fig:exPyrES}
        \end{subfigure}
\end{figure}

We now apply this same redistribution technique in order to
calculate male and female $e_x$-specific fertility rates ($e_x$-SFR). For any
rate, the numerator and denominator require a common referent, thus both births and
exposures are redistributed according to year $t$ mortality conditions. Such a
rate cannot be directly compared with a typical age-specific rate, since the
time scales are different, but we can indeed apply some familiar tools in order
to analyze this new curve. Figure~\ref{fig:eSFR2009} shows an example $e_x$-SFR
from 2009, for both the US and Spain.

\begin{figure}[ht!]
        \centering  
          \caption{Male and Female $e_x$-specific fertility rates, 2009, USA and
          Spain}
           % figure produced in
           % /R/ObservedVsExpectedBivariateBirthDistribution.R
           \makebox[\textwidth]{\includegraphics{Figures/eSFR2009}}
          %\includegraphics{Figures/ObservedvsExpectedBxy}
          \label{fig:eSFR2009}
\end{figure}

The assumption of homogenous mortality is particularly consequential in the case
of fertility, where health selection is self-evident, but not easily measureable.
It is thus to be expected that the left tails in Figure~\ref{fig:eSFR2009} are
too thick. In contemporary Western populations, female $e_x$-SFR curves will be
further to the right than male curves for two reasons 1) female mortality is
almost universally lower than male mortality at any given age, thus associating
births at a given age with higher age-specific remaining life expectancies; 2)
female fertility is more tightly concentrated over young ages, partly due to
upper bound represented by menopause, and partl due to prevailing hypergamy. 

As with ASFR, $e_x$-SFR has changed its level and undergone a \textit{rightward}
displacement over time, although the interpretation of this is entirely
different.






