
%* all figs here produced in IllustrateDivergence.R

Divergence of single-sex population models is problem of practical significance
for demographers, and it stems from the fact that vital rates almost always
differ between the sexes, except in rare coincidences when rates cross over
time. This characteristic of human populations, sexual dimorphism,
is manifest in all aspects of demography. In observed populations, fluctuations
in sexual dimorphism are constatly underway, while in theoretical population
models, dimorphism is typcically limited to fertility, mortality and the sex
ratio at birth. For convenience, this author classifies these concepts as
two-sex interactive, sex-specific, and a global population characteristic, 
respectively. Observed age-specific vital rates for Spain and the United States
serve well to illustrate the peculiarities of the problem.  %

% ASFR for 1975, both countries
\begin{figure}[ht!]
        \centering  
          \caption{Male and Female Age-Specific Fertility Rates, 1975, USA and
          Spain}
           % figure produced in
           % /R/ObservedVsExpectedBivariateBirthDistribution.R
           \makebox[\textwidth]{\includegraphics{Figures/ASFR1975}}
          %\includegraphics{Figures/ObservedvsExpectedBxy}
          \label{fig:ASFR1975}
\end{figure}


% TFR 1969- 2010
\begin{figure}[ht!]
        \centering  
          \caption{Male and Female Total Fertility Rates, 1969-2009, USA and
          Spain}
           % figure produced in
           % /R/ObservedVsExpectedBivariateBirthDistribution.R
           \makebox[\textwidth]{\includegraphics{Figures/TFR}}
          %\includegraphics{Figures/ObservedvsExpectedBxy}
          \label{fig:TFRseries}
\end{figure}
% NRR single sex? US SRB?


% vs net TFR

% Can't assume constant SRB

% mortality divergence: compare e0, rates, inverse pyramid?


 









