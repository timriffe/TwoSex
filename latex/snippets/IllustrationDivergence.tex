
%* all figs here produced in IllustrateDivergence.R

Divergence of single-sex population models is problem of practical significance
for demographers, and it stems from the fact that vital rates almost always
differ between the sexes, except in rare coincidences when rates cross over
time. This characteristic of human populations, sexual dimorphism,
is manifests itself in all concerns of demography. In observed populations,
fluctuations in sexual dimorphism are constantly underway. In theoretical
population models, dimorphism is typically limited to fertility and mortality
rates and the sex ratio at birth. Fertility is a two-sex interactive phenomenon,
but in practice population models, such as projections, do not often treat it as
such. Mortality is always modelled separately for both sexes, and can
be reasonably thought of as independent for each sex. If if male and female
rate schedules were identical, mortality differences would also
cause divergence between male and female single-sex models. The sex ratio at
birth (SRB) varies from year to year, but is usually treated as a global variable, 
uniform over age and time. For the USA this is a harmless assumption, but for Spain, 
assumptions of a constant SRB have been very wrong. Even if male and female fertility and morality schedules
were identical, Observed age-specific vital rates for Spain and the United
States serve well to illustrate the peculiarities of the problem.  %

% ASFR for 1975, both countries
\begin{figure}[ht!]
        \centering  
          \caption{Male and Female Age-Specific Fertility Rates, 1975, USA and
          Spain}
           % figure produced in
           % /R/ObservedVsExpectedBivariateBirthDistribution.R
           \makebox[\textwidth]{\includegraphics{Figures/ASFR1975}}
          %\includegraphics{Figures/ObservedvsExpectedBxy}
          \label{fig:ASFR1975}
\end{figure}


% TFR 1969- 2010
\begin{figure}[ht!]
        \centering  
          \caption{Male and Female Total Fertility Rates, 1969-2009, USA and
          Spain}
           % figure produced in
           % /R/ObservedVsExpectedBivariateBirthDistribution.R
           \makebox[\textwidth]{\includegraphics{Figures/TFR}}
          %\includegraphics{Figures/ObservedvsExpectedBxy}
          \label{fig:TFRseries}
\end{figure}
% NRR single sex? US SRB?

\begin{figure}[ht!]
        \centering  
          \caption{Male and Female Net Reproduction Rates, 1969-2009, USA and
          Spain}
           % figure produced in
           % /R/ObservedVsExpectedBivariateBirthDistribution.R
           \makebox[\textwidth]{\includegraphics{Figures/R0mf}}
          %\includegraphics{Figures/ObservedvsExpectedBxy}
          \label{fig:NRRseries}
\end{figure}

% vs net TFR

% Can't assume constant SRB

% mortality divergence: compare e0, rates, inverse pyramid?


 









