
%* all figs here produced in IllustrateDivergence.R
Divergence of single-sex population models is a problem of practical
significance for demographers, and it stems from the fact that vital rates almost always
differ between the sexes, except in rare coincidences when rates cross over
time. Divergence in this sense refers to the exponentially increasing
disagreement between single-sex male and female models that unfolds when
projected toward infinity. This results from differing intrinsic growth rates,
$r$, that follow from Lotka's fundamental equation \citep{sharpe1911problem}:

\begin{equation}
\label{eq:lotkaeq}
1 = \int _0 ^\infty e^{-ra}p_a m_a \dd a 
\end{equation}
where $p_a$ are age-specific survival probabilities, $m_a$ are age-specific
single-sex fertility probabilities, and $r$ is the growth parameter to be
estimated. Thus, $m_a$ may either be the fertilty of girls born to mothers or of
boys born to fathers. \citet{yellin1977comparison} prove that this is to be
expected, as agreement between male and female versions of
 Equation~\eqref{eq:lotkaeq} implies an overdetermined
system. At the root of the problem is that the total numbers of births predicted
by the equations ought to, but never do, agree, aside from in the jump-off year
from which rates are initially derived, which is a tautology. In any instance
where single-sex $r$ estimates differ, forward projection will result in sex ratios 
that either grow toward infinity in the limit if $r^m
> r^f$ or decline to zero if $r^m < r^f$. If the gap between rates is large, this happens
quickly, if small, divergence is slower. This is in either case a modelling
inconvenience, and the crux of the two-sex problem. 

Single-sex intrinsic growth rates, $r^m$ and $r^f$, can be 
estimated from data. In looking at time series (see Figure~\ref{fig:rmf}) of 
 growth rates, observe that the sex-gap has varied over 
time, that the male rate is typcally higher than the female rate (aided greatly 
by the sex ratio at birth), and that there have been crossovers in the USA: 
the $r^f > r^m$ in 1994-1996, and again briefly in 2001. 

\begin{figure}[ht!]
        \centering  
          \caption{Male and Female intrinsic growth rates, Spain and US,
          1969-2009}
           % figure produced in
           % /R/rm_rf_divergence
           \makebox[\textwidth]{\includegraphics{Figures/rmf}}
          \label{fig:rmf}
\end{figure}

Perhaps even more curious are occassions when $r^m$ and $r^f$ have been on
opposite sides of zero, i.e. intrinsic growth and instrinsic decrease at the
same time. In the USA, this has happened many times in the period studied:
1972-1973, 1990, 2004-2005, and again recently in 2008. In Spain rates were 
briefly on opposites of zero in 1981-1982, in the middle of a period of sharp
decline in fertility. In all of these cases male growth rates were positive
while female growth rates were negative. Note that this does not mean that 
observed growth rates were of opposite sign, only that intrinsic
rates were. 

Differences in intrinsic growth rates underly divergence, but they
do not represent divergence in the projection, per se.
Figure~\ref{fig:rSRdoubling} playfully illustrates the divergence implied by the 
rates in each given year: Given the
male and female vital rates from each year, how many years would it take for one
sex to be double the size of the other, always using the year $t$ population as
the initial conditions?

\begin{figure}[ht!]
        \centering  
          \caption{$log(\mathrm{Years})$ until one sex is twice the size as the
          other, given separate single-sex projections using annual vital rates and initial
          conditions, Spain and US.}
           \quad
           % /R/rm_rf_divergence
           \makebox[\textwidth]{\includegraphics{Figures/rSRdoubling}}
          \label{fig:rSRdoubling}
\end{figure}

Clearly the run of years in the United States where $r^f$ and $r^m$ were very
close (approx 1994-2001) imply such slow rates of divergence that we could, as a
matter of accident, safely ignore the two sex problem in those years. These
tended to be the same years where the greater growth rate oscillated between
male and females. However, any acceptability threshold is a matter of
convenience and taste: presumably the demographer would like age-specific 
population estimates to be much closer to truth than \textit{half} or \textit{twice} the ideal value.
Dropping the badness threshold would of course decrease the waiting time until
it is met in any given year. These are practical questions. More
stringent are the demands of theoretical stable populations, where
sex consistency is very desirable. Not a single year of data presented here
meets the requirements of a consistent stable population, and even if this were
to be observed, it would be coincidentally rather than essentially so. Let us
for now conclude that the divergence of single sex models is demonstrated.











