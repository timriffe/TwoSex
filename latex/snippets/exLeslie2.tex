
The formal relations presented in
Sections~\ref{sec:2sexlinearmain}~and~\ref{sec:2sexlinearother} establish
coherence, and some merits have been presented, but these equations are perhaps
less relevant in the discrete world of applied demography. The model contianed
in Equation~\ref{eq:lineartwosexrenewal} is best replicated with a projection,
similar in concept to that offered for the single-sex $e_x$-structured case in
Section~\ref{sec:ex1sexleslie}. The two-sex linear projection matrix combines
the projection of each sex jointly in a single instrument, the construction of 
which is more involved than the single-sex case: Four times more involved to be precise. 

Assuming $n$ $e_x$-classes each for males and females, the dimensions of the
present matrix will be $2n \times 2n$, where male and female
$e_x$-classified population vectors by be joined, for instance end-to-end in a
single population vector. The convention used in this description will place
males, ordered by remaining years in positions $1:n$ of the vector $\textbf{p}$
and females ordered by remaing years in positions $(n+1):(2n)$ of $\textbf{p}$.
This being so, the projection matrix $\textbf{Y}$ must conform with these
locations of males and females, locating survival and fertility appropriately.

$\textbf{Y}$ is divided into four main blocks. The top left block is
nearly identical to the male single-sex case, and the bottom left block is
nearly identical to the female single-sex case. Both of these two submatrices
contain survival (all 1s) in the superdiagonal. Fertility is analagous, but
weighted according to $\sigma$. The lower left submatrix contains $M-F$
fertility weighted by $\sigma$ and distributed over female death probabilities,
$d_a^f$, and the upper right matrix contains $F-M$ fertility, weighted by
$1-\sigma$ and distributed according to male death probabilities.

% need to adjust spacing big-time!

\begin{figure}[h!]
 \begin{adjustwidth}{-1.5in}{-.5in}
\centering
\caption*{Example: A full two-sex remaining years ($e_y$)-structured projection
matrix, $\textbf{Y}$} 
\tiny{
$\textbf{Y} = \bordermatrix{
  {e_y }    \vspace{.6em} & 0_t^M                                             & 1_t^M                                       & 2_t^M                                    & 3_t^M   & 0_t^F                                                & 1_t^F                                          & 2_t^F                                       & 3_t^F  \cr 
  0_{t+1}^M \vspace{.6em} & \sigma  \lambda^M\tfrac{f_0^{M-M}d_0^M}{2}        &
  \sigma \lambda^Mf_1^{M-M}d_0^M + 1          & \sigma \lambda^Mf_2^{M-M}d_0^M           &  0      & (1-\sigma)\lambda^M\tfrac{f_0^{F-M}d_0^M}{2}         & (1-\sigma)\lambda^Mf_1^{F-M}d_0^M             & (1-\sigma\lambda^Mf_2^{F-M}d_0^M &  0     \cr 1_{t+1}^M \vspace{.6em} & \sigma \tfrac{f_0^{M-M}d_1^M}{2}                  & \sigma f_1^{M-M}d_1^M                       & \sigma f_2^{M-M}d_1^M + 1                &  0      & (1-\sigma) \tfrac{f_0^{F-M}d_1^M}{2}                 & (1-\sigma)f_1^{F-M}d_1^M                      & (1-\sigma)f_2^{F-M}d_1^M                    &  0     \cr 
  2_{t+1}^M \vspace{.6em} & \sigma \tfrac{f_0^{M-M}d_2^M}{2}                  & \sigma f_1^{M-M}d_2^M                       & \sigma f_2^{M-M}d_2^M                    &  1      & (1-\sigma) \tfrac{f_0^{F-M}d_2^M}{2}                 & (1-\sigma)f_1^{F-M}d_2^M                      & (1-\sigma)f_2^{F-M}d_2^M                    &  0     \cr 
  3_{t+1}^M \vspace{.6em} & \sigma \tfrac{f_0^{M-M}d_3^M}{2}                  & \sigma f_1^{M-M}d_3^M                       & \sigma f_2^{M-M}d_3^M                    &  0      & (1-\sigma) \tfrac{f_0^{F-M}d_3^M}{2}                 & (1-\sigma)f_1^{F-M}d_3^M                      & (1-\sigma)f_2^{F-M}d_3^M                    &  0     \cr 
  0_{t+1}^F \vspace{.6em} & \sigma  \lambda^F\tfrac{f_0^{M-F}d_0^F}{2}        & \sigma \lambda^Ff_1^{M-F}d_0^F              & \sigma \lambda^Ff_2^{M-F}d_0^F           &  0      & (1-\sigma) \lambda^F\tfrac{f_0^{F-F}d_0^F}{2}        & (1-\sigma)\lambda^Ff_1^{F-F}d_0^F +1          & (1-\sigma)\lambda^Ff_2^{F-F}d_0^F           &  0     \cr 
  1_{t+1}^F \vspace{.6em} & \sigma \tfrac{f_0^{M-F}d_1^F}{2}                  & \sigma f_1^{M-F}d_1^F                       & \sigma f_2^{M-F}d_1^F                    &  0      & (1-\sigma) \tfrac{f_0^{F-F}d_1^F}{2}                 & (1-\sigma)f_1^{F-F}d_1^F                      & (1-\sigma)f_2^{F-F}d_1^F +1                 &  0     \cr 
  2_{t+1}^F \vspace{.6em} & \sigma \tfrac{f_0^{M-F}d_2^F}{2}                  & \sigma f_1^{M-F}d_2^F                       & \sigma f_2^{M-F}d_2^F                    &  0      & (1-\sigma) \tfrac{f_0^{F-F}d_2^F}{2}                 & (1-\sigma)f_1^{F-F}d_2^F                      & (1-\sigma)f_2^{F-F}d_2^F                    &  1     \cr 
  3_{t+1}^F \vspace{.6em} & \sigma \tfrac{f_0^{M-F}d_3^F}{2}                  & \sigma f_1^{M-F}d_3^F                       & \sigma f_2^{M-F}d_3^F                    &  0      & (1-\sigma) \tfrac{f_0^{F-F}d_3^F}{2}                 & (1-\sigma)f_1^{F-F}d_3^F                      & (1-\sigma)f_2^{F-F}d_3^F                    &  0 
 }$}
  \end{adjustwidth}
\end{figure}


