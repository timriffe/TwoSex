
Now recall the observations of \citet{sanderson2005average}, more directly
relating to the popuation pyramid. These authors note that despite ageing in a
population, the mean \textit{remaining} years to be lived may increase at the
same time. This is due to improvements in mortality offsetting increases in the
mean age of a population. I now present a method to exactly redistribute
population counts (or exposures) according to period remaining life
expectancies, rather than according to age per se.

From the lifetable of a given population, sex-specific and calculated with a
radix of 1. Note the $d_x$ now sums to 1, and is therefore a proper density
function. $d_x$ can be thought of as the probability of dying in any given age
from the perspective of a 0-year-old, according to the given year's mortality
experience. It follows that the observed population of age 0 can be redistribted
according to $d_x$ and interpreted either as the expected death counts
in each year $t+x$, or as the distribution of persons currently-aged 0
reclassified according to remaining life expectancy. This can be done similarly
for age 1, by ignoring the mortality experience of age 0, and rescaling $d_x$ to
sum to 1, or generally:

\begin{equation}
\mathbf{E}(D_t) = \int _{a = 0} ^{\infty} P_a \frac{d_a}{\int _{b = a} ^{\infty}
d_b\, \dd b} \;\dd a
\end{equation}
where $P_a$ is the population of age $a$, $d_a$ is the
lifetable density function and $\mathbf{E}(D_t)$ is the expected number of
deaths in each future year $t$, also understood as a vector of the current 
population, redistributed into categories of remaining life expectancy. The key
observation is that we can sidestep the $e_x$ column of the lifetable, since
$e_x$ is essentially a weighted mean of $d_x$. The resulting population pyramid
is heterogeneous with respect to age within any given level of remaining life
expectancy, and looks like
Figures~\ref{fig:exPyrUS} and \ref{fig:exPyrES}\footnote{The idea to
redistribute the population pyramid in this way is due to a conversation with 
John MacInnes, and appears in \citep{MacInnes2013pop} (unpublished) using a
different method.} for the years 1975 and 2009. As a helpful pointer, note
that the population at the base of the pyramid is expected to decrement
within the \textit{next year}, thus the vertical axis can also can also be
thought of as year $t+x$, although $e_x$ more clearly identifies the pyramid
with year $t$ mortality conditions. In other words, the pyramid should not be
taken out of context as a forecast. Note that this pyramid represents the
exact same population: Underlying males sum to the correct total on the left and
females sum to the correct total on the right. Both pyramids have been rescaled to sum to 100, in 
order to more comparably represent population structure. \footnote{Incidentally,
a time series of remaining life expectancy pyramids for any given Western country will show 
 incredible stability over time, which is remarkable in light of
 observed ageing in the observed population pyramid. The simple interpretation
 of this kind of pyramid adds to its utility, and this author believes that $e_x$-specific
 population structure, and other here-unmentioned indicators that can be derived
 from this method sould make up a valuable new component to the contemporary
 demographer's toolbox, as well as help inform current population debates.}

\begin{figure}
        \centering
        \begin{subfigure}
                \centering
                \caption{US years-lived by $e_x$, 1975 and 2009}
                \includegraphics[scale = .8]{Figures/exPyramidUS}
                \label{fig:exPyrUS}
        \end{subfigure}
        \begin{subfigure}
                \centering
                \caption{Spain years-lived by $e_x$, 1975 and 2009}
                \includegraphics[scale = .8]{Figures/exPyramidES}
               
                \label{fig:exPyrES}
        \end{subfigure}
\end{figure}

We now apply this same redistribution technique in order to
calculate male and female $e_x$-specific fertility rates ($e_x$-SFR). For any
rate, the numerator and denominator require a common referent, thus both births and
exposures are redistributed according to year $t$ mortality conditions. Such a
rate cannot be directly compared with a typical age-specific rate, since the
time scales are different, but we can indeed apply some familiar tools in order
to analyze this new curve. Figure~\ref{fig:eSFR2009} shows an example $e_x$-SFR
from 2009, for both the US and Spain.

\begin{figure}[ht!]
        \centering  
          \caption{Male and Female $e_x$-specific fertility rates, 2009, USA and
          Spain}
           % figure produced in
           % /R/ObservedVsExpectedBivariateBirthDistribution.R
           \makebox[\textwidth]{\includegraphics{Figures/eSFR2009}}
          %\includegraphics{Figures/ObservedvsExpectedBxy}
          \label{fig:eSFR2009}
\end{figure}

The assumption of homogenous mortality is particularly consequential in the case
of fertility, where health selection is self-evident, but not easily measureable.
It is thus to be expected that the left tails in Figure~\ref{fig:eSFR2009} are
too thick. In contemporary Western populations, female $e_x$-SFR curves will be
further to the right than male curves for two reasons 1) female mortality is
almost universally lower than male mortality at any given age, thus associating
births at a given age with higher remaining life expectancies; 2)
female fertility is more tightly concentrated over young ages, partly due to
upper bound represented by menopause, and partly due to prevailing hypergamy. 

As with ASFR, $e_x$-SFR has changed its level and undergone a \textit{rightward}
displacement over time, although the interpretation of this is entirely
different. Observed fertility postponement should shift $e_x$-SFR unfavorably
(left) to higher mortality levels, however mortality improvements have in many
instances tended to offset this effect, acting to move the curve to higher
remaining life expectancies. This evolution in rates can, as with ASFR, be
represented with a standard surface, or be summarized with an indicator akin to
TFR, which we here call $e_x$-TFR.










