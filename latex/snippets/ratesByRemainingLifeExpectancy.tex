
Recall the observations of \citet{sanderson2005average}, more directly
relating to the popuation pyramid. These authors note that despite ageing in a
population, the mean \textit{remaining} years to be lived may increase. This is due 
to improvements in mortality offsetting (or more than offsetting) increases in
the mean age of a population. In general, a population looked at from the
perspective of life expectancy looks different, and yields complimentary
information to one looked at from the typical perspective of age. I will first
present a method to exactly redistribute population counts (or exposures) according to 
period remaining years of life, rather than according to age per se. A
reexamination of recent fertility patterns according to remaining years of life
will follow.

\subsubsection{Redistributing age-classified data by remaining years of life}
The steps required to carry out the present data transformation are conceptually
simple, and easy to implement once understood. From a given
population and year extract the $d_x$ column from the corresponding lifetable of
radix ($l_0$) equal to 1\footnote{If the lifetable was calculcated with a
different radix, then simply divide the $d_x$ column by $l_0$}. Note that in
this case the $d_x$ column sums to 1, and is therefore a proper density function. 
$d_x$ can now be thought of as the probability of dying in any given age from the
 perspective of a 0-year-old, according to the given year's mortality experience. 
 It follows that the observed population of age 0 can be redistribted according to
  $d_x$ and interpreted either as the expected death counts in each future year
$t+x$, or more intuitively as the distribution of persons currently-aged 0 according 
to remaining life expectancy. This can be done similarly for age 1, by ignoring the 
mortality experience of age 0, and rescaling $d_x$ to
sum to 1, or more generally redistributing each age and then summing to
$e_x$-specific totals:

\begin{equation}
\label{eq:dxredist}
\mathbf{E}(D_t) = \int _{a = 0} ^{\infty} P_a \frac{d_a}{\int _{b = a} ^{\infty}
d_b\, \dd b} \;\dd a
\end{equation}
where $P_a$ is the population of age $a$, $d_a$ is the
lifetable density function and $\mathbf{E}(D_t)$ is the expected number of
deaths in each future year $t$, also understood as a vector of the current 
population, redistributed into categories of remaining life expectancy. 

The function of this formula is not totally original, as \citet{vaupel2009life}
required a similar identity in the proof for the statement that:
\begin{quotation}
if an individual is chosen at random from a stationary population with a positive force
of mortality at all ages, then the probability the individual is one who has lived $a$ years
equals the probability the individual is one who has that number of years left to live.
\end{quotation}
, which takes this shape:
\begin{equation}
\label{eq:vaupelredist}
f(n | a) = \mu (a+n) \frac{l(a+n}{l(a)}
\end{equation}
, where $f(n | a) $ is the probability of dying $n$ years in the future given
survival to age $a$, and $\mu$ is the force of mortality.
Formula~\eqref{eq:vaupelredist} would require an
extra integral over age, in order to come to the same
overall pdf, and then needs to be multiplied into
age-classified populations. Formula~\eqref{eq:dxredist} is more convenient when
discretized\footnote{Formula~\eqref{eq:dxredist} is more convenient due 1) to
lifetable close-out issues and 2) because only one column from the lifetable is
required instead of 3 columns ($\mu_x$, $l_x$, $L_x$) in
Equation~\eqref{eq:vaupelredist} }, although both are equally valid. The use of either
 in the way presented in this section is to this author's knowledge novel.

\subsubsection{Population structure by remaining years of life}
The key
observation is that we can sidestep the $e_x$ column of the lifetable, since
$e_x$ is essentially a weighted mean of $d_x$. The resulting population pyramid
is heterogeneous with respect to age within any given level of remaining life
expectancy, and looks like
Figures~\ref{fig:exPyrUS} and \ref{fig:exPyrES}\footnote{The idea to
redistribute the population pyramid in this way is due to a conversation with 
John MacInnes, and appears in \citep{MacInnes2013pop} (unpublished) using a
different method.} for the years 1975 and 2009 in the US and Spain. As a helpful
pointer, note that the population at the base of the pyramid is expected to decrement
within the \textit{next year}, thus the vertical axis can also can also be
thought of as year $t+x$, although $e_x$ more clearly identifies the pyramid
with year $t$ mortality conditions. The pyramid should not be
taken out of context as a forecast. Note that this pyramid represents the exact same
population as an age-classified pyramid: Underlying males sum to the correct total on 
the left and females sum to the correct total on the right. Only the definition of age has
changed; instead of counting forward from birth we count \textit{age} in reverse
starting from death. For individuals, this feat would be impossible, but given
 the information contained in a period lifetable,  one can to great utility 
 redistribute population aggregates accoridng to $e_x$\footnote{To undertake
 the same but assumping future mortality changes (improvements) one might
 better undertake a fertility-free cohort component projection and collect the
 deaths from each future year $t+x$ until extinction. This possibility is not
 treated in the present dissertation.}. Both pyramids have been rescaled
 to sum to 100, in order to more comparably represent population structure\footnote{Incidentally, 
 a time series of remaining life expectancy pyramids for any given Western country will show 
 incredible stability over time, which is remarkable in light of observed ageing in the observed
  population pyramid. The simple interpretation of this kind of pyramid adds to its utility, and
   this author 
believes that $e_x$-specific population structure, and other here-unmentioned indicators 
that can be derived from this method sould make up a valuable new component to the contemporary
 demographer's toolbox, as well as help inform current population debates.}.

\begin{figure}
        \centering
        \begin{subfigure}
                \centering
                \caption{US years-lived by $e_x$, 1975 and 2009}
                \includegraphics[scale = .8]{Figures/exPyramidUS}
                \label{fig:exPyrUS}
        \end{subfigure}
        \begin{subfigure}
                \centering
                \caption{Spain years-lived by $e_x$, 1975 and 2009}
                \includegraphics[scale = .8]{Figures/exPyramidES}
               
                \label{fig:exPyrES}
        \end{subfigure}
\end{figure}

\subsubsection{Fertility rates by remaining years of life}
The technique presented in Equation~\ref{eq:dxredist} and illustrated in
Figures~\ref{fig:exPyrUS}~and~\ref{fig:exPyrES} can indeed be used to reclassify
any age-distributed data, assuming that the appropriate lifetable is available.
We now apply this redistribution technique in order to calculate male and 
female $e_x$-specific fertility rates ($e_x$-SFR). For any rate, the numerator 
and denominator require a common referent, thus both births and exposures are 
redistributed according to year $t$ mortality conditions. That
is to say, we take the extra step of moving the age-specific vector
of birth counts (by mothers' or fathers' age) into $e_x$-specific birth
vectors before dividing into  $e_x$-specific exposures. Such a rate cannot be
directly compared with a typical age-specific rate, since the time scales are different, but we 
can indeed apply some familiar tools in order to analyze this new curve.

\paragraph{The $e_x$-pattern of fertility}

The $e_x$-pattern of fertility is distinct from the age-pattern of fertility. 
In contemporary Western populations, female $e_x$-SFR curves will be
further to the right than male curves for two reasons: 1) Female mortality is
almost universally lower than male mortality at (and beyond) any given age,
thus associating births at a given age with higher remaining life expectancies; 2)
female fertility is more tightly concentrated over young ages, partly due to the
upper bound defined by menopause, and partly due to prevailing hypergamy.
Figure~\ref{fig:eSFR2009} shows an example $e_x$-SFR from 2009, for both the US and Spain.

\begin{figure}[ht!]
        \centering  
          \caption{Male and Female $e_x$-specific fertility rates, 2009, USA and
          Spain}
           % figure produced in
           % /R/Parents_ex.R
           \includegraphics{Figures/eSFR2009}
          \label{fig:eSFR2009}
\end{figure}

One may question whether the curves shown in Figure~\ref{fig:eSFR2009} properly
represent rates. This author argues that the same definition of events in the
numerator and exposures in the denominator has been applied, only the
structuring variable has changed from \textit{time since birth} to \textit{time
until death} (of progenitor here). In this way, age-classified and
$e_x$-classified populations have structure in the same sense. As with any
demographic variable, we may wish to analyze the intensity of demographic
phenomena removed of the distorting effects of population structure.
Working with event-exposure rates are just one way of doing so, simple
decomposition is another, and indeed such rates and decompositions are possible
in the aggregate both with respect to age and with respect to
$e_x$.

This is, in the best case, a rough calculation, for several reasons. The
assumption of homogenous mortality is particularly consequential in the case of 
fertility, where health selection is self-evident, but not easily measureable.
It is for this reason to be expected that the left tails in
Figure~\ref{fig:eSFR2009} are too thick. 

Furthermore, exposure is taken from the \textit{entire} population, not merely
the populaton within reproductive ages. The rates could be thusly recalculated,
for instance using female ages $13-50$ and male age $15-65$, and would look like 
Figure~\ref{fig:eSFR2009limits}, in some isntances a more reasonable if less
intelligible result\footnote{Rate surfaces based on $e_x$-specific fertility
data are calculated under a variety of reproductive spans in
Appendix~\ref{Appendix:reprospans}}.

\begin{figure}[ht!]
        \centering  
          \caption{Male and Female $e_x$-specific fertility rates, 2009, USA and
          Spain, with exposures redistributed using only female ages $13-50$ and
          male ages $15-65$}
           % figure produced in
           % /R/Parents_ex.R
          \includegraphics{Figures/eSFR2009limits}
          \label{fig:eSFR2009limits}
\end{figure}

Comparing Figures~\ref{fig:eSFR2009} and~\ref{fig:eSFR2009limits} reminds of the
comments of \citet{gupta1978alternative} and \citet{mitra1976effect} on the difficulty of
defining an \textit{effective} population for use in exposures. Clearly, persons
outside the reproductive age range will conventionally be excluded from
exposures. Other kinds of risk heterogeneity are known to exist, such as age
patterns in fecundability, contraceptive use and sexual intercourse, that are
unaccounted for in standard fertility measures. 

With no claim of superiority over the more
restrictive exposures used for Figure~\ref{fig:eSFR2009limits}, we will proceed
in this section by using exposures derived from all ages. One could weakly
defend this choice by noting that we are attempting to measure the period
reproductivity of an \textit{entire} population, not just part of it. The
reproductive span was an outcome of evolution, varies greatly between
individuals and populations, and is mutable, both due to ongoing
population-level genetic and hormonal changes and direct human intervention. We
will for the time being, be content to work with the cruder $e_x$-SFR, and note
that this rate, as any other is amenable to further disaggregation and
decomponsition.

\begin{figure}
        \centering
        \begin{subfigure}
                \centering
                \caption{Male and Female $e_x$-SFR surfaces, 1969-2009, USA}
                \includegraphics[scale = .8]{Figures/eSFRsurfacesUS}
                \label{fig:exSFRsurfUS}
        \end{subfigure}
        \begin{subfigure}
                \centering
                \caption{Male and Female $e_x$-SFR surfaces, 1975-2009, Spain}
                \includegraphics[scale = .8]{Figures/eSFRsurfacesES} 
                \label{fig:exSFRsurfES}
        \end{subfigure}
\end{figure}

As is visible in Figures~\ref{fig:exSFRsurfUS}~and~\ref{fig:exSFRsurfES}, 
$e_x$-SFR has changed its level and undergone a gradual displacement over 
time toward higher $e_x$ levels, an altogether propitious development
with respect to human altriciality. The interpretation of this displacement is
entirely different from that of postponement in ASFR. Observed fertility 
postponement should shift $e_x$-SFR unfavorably to higher mortality 
levels (lower $e_x$ levels), however mortality improvements have tended to 
offset this effect, acting to move the curve to higher
remaining life expectancies. This evolution in rates can, as with ASFR, also be
summarized with an indicator akin to TFR, which we here call $e_x$-TFR, seen in
Figure~\ref{fig:exTFR}.

\begin{figure}[ht!]
        \centering  
          \caption{Male and Female $e_x$-total fertility rates, Spain
          and USA, 1969-2009}
           % figure produced in
           % /R/Parents_ex.R
           \includegraphics{Figures/exTFR}
          \label{fig:exTFR}
\end{figure}

Canonical TFR can conveniently be imagined as the total number of
offspring that that an average female (male) will have in a lifetime assuming
perfect survival and constant fertility rates of the present year.
Since a lifetime measured in age counting from birth is the same length as a
lifetime measured in age counting backward from death, $e_x$-TFR in fact has the
same intrepretation. Why is this? Age-classified rates are of course
heterogenous within age with respect to remaining life expectancy, and here we have produced
an synthetic index based on the reverse idea. The age-classified
distribution of births and populations are quite different (there being an age
pattern to fertility rates). $e_x$-reclassifying these data not only changes the
center of gravity of numerator and denominator distributions, but asymmetrically
shifts underlying schedules, uniquely reshaping the pattern of
fertility. Summing over $e_x$-rates will almost yield a different total, our
synthetic $e_x$-TFR. 

Figures~\ref{fig:exSFRsurfES}~and~\ref{fig:exTFR} are reproduced according to
various definitions of the reproductive span in
Appendix~\ref{Appendix:reprospans}. Rates are shown to be sensitive to the
choice of reproductive span. For the remainder of this dissertation, we ignore
issues of age boundaries in the reproductive span for simplicity and
consistency, although this issue deserves further attention if the
$e_x$-perspective is deemed of merit.

\paragraph{Bivariate birth distribution by remaining years of life}

First, note that the observed bivariate $e_x$-distribution of birth counts is
very nearly identical to the expected distribution\footnote{The expected
distribution is defined as in Equation~\eqref{eq:expected}}.
Figure~\ref{fig:US1970obsexpex} compares these two distributions for birth counts in the USA in 1970 (compare with Figure~\ref{fig:US1970obsexp}). 

\begin{figure}[ht!]
        \centering  
          \caption{Observed versus expected bivariate
          distribution of birth counts by $e_x$-distribution of parents, 1970,
          USA}
           % figure produced in
           % /R/Parents_exCross.R
           \includegraphics{Figures/ObservedvsExpectedBexey}
          \label{fig:US1970obsexpex}
\end{figure}

It is
difficult to see any difference between the two surfaces in
Figure~\ref{fig:US1970obsexpex}, however we can measure the degree of
separation, $\theta$\footnote{See Equation~\eqref{eq:coefdiff}. Recall that 0 signifies
perfect overlap and 1 signifies perfect separation between the two
distributions}, just as for age-classified births (Compare with
Figure~\ref{fig:Theta}). One provisionally concludes that $e_x$-matching of
parents, at least with this level of approximation, appears to be very close to random.

\begin{figure}[ht!]
        \centering  
          \caption{Departure from association-free bivariate distribution of
          birth counts cross-classified by $e_x$ of mother and father. USA,
          1969-2010 and Spain, 1975-2009}
           % figure produced in
           % /R/Parents_exCross.R
           \includegraphics{Figures/TotalVariationObsvsExpectedexUSES}
          \label{fig:TotalVarobsexpex}
\end{figure}

Since the bivariate distribution by mothers' and fathers' $e_x$ is so close
to random, one could very closely replicate the full cross-classified matrix 
given only the two marginal
$e_x$ birth distributions by applying Equation~\eqref{eq:expected}. 




