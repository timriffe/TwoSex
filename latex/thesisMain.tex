
% lengthy preamble is aside. That's where packages are loaded as needed, new environments
% defined, etc
\documentclass[reqno,12pt,oneside,a4paper]{report} % right-side equation numbering, 12 point font, print one-sided 
%\documentclass[reqno,12pt,twoside,openright]{report} % right-side equation numbering, 12 point font, print two-sided, Chapters start on odd pages. Rackham only accepts one-sided, so this is for personal printings.

\usepackage{uab}         % Use UAB thesis style file, in folder with main file
\usepackage{aas_macros}  % To allow the reading of ADS journal references in the bibliography
\usepackage[intlimits]{amsmath} % Puts the limits of integrals on top and bottom
\usepackage{amsxtra}     % Use various AMS packages
\usepackage{amsthm}
\usepackage{amssymb}
\usepackage{mathtools}
\usepackage{amsfonts}
\usepackage{graphicx}    % Add some packages for figures. Read epslatex.pdf on ctan.tug.org
\usepackage{rotating}
\usepackage{color}
\usepackage{epsfig}
\usepackage{subfigure}   % To make subfigures. Read subfigure.pdf on ctan.tug.org
\usepackage{verbatim}
\usepackage{natbib}      % for bibtex
\usepackage{caption}     % to suppress caption numbering at times  
\usepackage{lscape}      % for landscape diagram
\usepackage[super]{nth}
\usepackage{textcomp}    % in text pretty 1/2
\usepackage{chngpage}
\usepackage{trivfloat}
\trivfloat{matrix}
\floatstyle{plaintop}
    \restylefloat{matrix}
\AtBeginDocument{\numberwithin{matrix}{section}}
\usepackage{placeins}
\usepackage{bookmark} % for dividing chapters into parts
% ---------------------------------------------------------------------
\usepackage[draft]{pdfpages}    % allows inclusion of multipage pdfs:
% change to [final] when ready to compile complete
% ---------------------------------------------------------------------  
% some custom citation aliases
\defcitealias{rackham1947trans}{Aristotle, \textit{Nichomachean
Ethics}, Book II, Chapter 6, Sections 4-5.}
%\defcitealias{bas09}{BCBS, 2009}

\usepackage[printonlyused]{acronym} % For the List of Abbreviations. Read acronym.pdf on ctan.tug.org
\usepackage{setspace}    % Allows you to specify the line spacing
\doublespacing           % \onehalfspacing for 1.5 spacing, \doublespacing for 2.0 spacing.
\newcommand{\sun}{\ensuremath{\odot}} % sun symbol is \sun
%\usepackage{leftidx}     % for better left subscripts
\usepackage{fixltx2e}     % in-text subscripting, no math
%\usepackage[OT1]{fontenc}
\usepackage{bm}
%%%%%%%%%%%%%%%%%%%%%%%%%%%%%%%%%%%%%%%%%%%%%%%%%%%%%%%%%%%%%%%%%%%%%%%%%%%%%%%

% Various theorem environments. All of the following have the same numbering
% system as theorem.

\theoremstyle{plain}
\newtheorem{theorem}{Theorem}
\newtheorem{prop}[theorem]{Proposition}
\newtheorem{corollary}[theorem]{Corollary}
\newtheorem{lemma}[theorem]{Lemma}
\newtheorem{question}[theorem]{Question}
\newtheorem{conjecture}[theorem]{Conjecture}
\newtheorem{assumption}[theorem]{Assumption}

\theoremstyle{definition}
\newtheorem{definition}[theorem]{Definition}
\newtheorem{notation}[theorem]{Notation}
\newtheorem{condition}[theorem]{Condition}
\newtheorem{example}[theorem]{Example}
\newtheorem{introduction}[theorem]{Introduction}

\theoremstyle{remark}
\newtheorem{remark}[theorem]{Remark}
%%%%%%%%%%%%%%%%%%%%%%%%%%%%%%%%%%%%%%%%%%%%%%%%%%%%%%%%%%%%%%%%%%%%%%%%%%%%%%%

\numberwithin{theorem}{chapter}     % Numbers theorems "x.y" where x
                                    % is the section number, y is the
                                    % theorem number

%\renewcommand{\thetheorem}{\arabic{chapter}.\arabic{theorem}}

%\makeatletter                      % This sequence of commands will
%\let\c@equation\c@theorem          % incorporate equation numbering
%\makeatother                       % into the theorem numbering scheme

%\renewcommand{\theenumi}{(\roman{enumi})}

%%%%%%%%%%%%%%%%%%%%%%%%%%%%%%%%%%%%%%%%%%%%%%%%%%%%%%%%%%%%%%%%%%%%%%%%%%%%%%
% cache appears to be the problem with displaying results
%\usepackage{Sweave}
%\SweaveOpts{cache=FALSE,tidy=TRUE}
%\usepackage{tikz}

% makes code more compact, changes indentation of code chunks
%\DefineVerbatimEnvironment{Sinput}{Verbatim} {xleftmargin=2em}
%\DefineVerbatimEnvironment{Soutput}{Verbatim}{xleftmargin=2em}
%\DefineVerbatimEnvironment{Scode}{Verbatim}{xleftmargin=2em}
%\fvset{listparameters={\setlength{\topsep}{0pt}}}
%\renewenvironment{Schunk}{\vspace{\topsep}}{\vspace{\topsep}}

%%%%%%%%%%%%%%%%%%%%%%%%%%%%%%%%%%%%%%%%%%%%%%%%%%%%%%%%%%%%%%%%%%%%%%%%%%%%%%%
% If printing two-sided, this makes sure that any blank page at the 
% end of a chapter will not have a page number. 
\makeatletter
\def\cleardoublepage{\clearpage\if@twoside \ifodd\c@page\else
\hbox{}
\thispagestyle{empty}
\newpage
\if@twocolumn\hbox{}\newpage\fi\fi\fi}
\makeatother 

%%%%%%%%%%%%%%%%%%%%%%%%%%%%%%%%%%%%%%%%%%%%%%%%%%%%%%%%%%%%%%%%%%%%%%%%%%%%%%

%This command creates a box marked ``To Do'' around text.
%To use type \todo{  insert text here  }.

\newcommand{\todo}[1]{\vspace{5 mm}\par \noindent
\marginpar{\textsc{To Do}}
\framebox{\begin{minipage}[c]{0.95 \textwidth}
\tt\begin{center} #1 \end{center}\end{minipage}}\vspace{5 mm}\par}

%%%%%%%%%%%%%%%%%%%%%%%%%%%%%%%%%%%%%%%%%%%%%%%%%%%%%%%%%%%%%%%%%%%%%%%%%%%%%%%
% for the d in integrals
\newcommand{\dd}{\; \mathrm{d}}



%%%%%%%%%%%%%%%%%%%%%%%%%%%%%%%%%%%%%%%%%%%%%%%%%%%%%%%%%%%%%%%%%%%%%%%%%%%%%%%
\begin{document}

\bibliographystyle{agu04}    % Set the bibliography style. agu04, plain, alpha, etc.

% Title page 
\titlepage{The Two-Sex Problem in Populations Structured by Remaining
Years of Life}{Timothy L. M. Riffe}{Doctorat en Demografia,} {Departament de
Geografia,\\Facultat de Filosofia i Lletres,}{Juny 2013} {Dr. Daniel Devolder\\
Dr. Trifon Missov\\ Dr. Mystery 3rd demographer}

% Begin the front matter as required by Rackham dissertation guidelines
\initializefrontsections

% Optional Frontispiece
\frontispiece{
\begin{adjustwidth}{-1in}{2in}
\begin{figure}
\raggedright
\begin{subfigure}
  \centering
  \includegraphics[width=3.5in]{Figures/ES_DxHist}
\end{subfigure}%
~
\begin{subfigure}
  \centering
  \includegraphics[width=3.5in]{Figures/US_DxHist}
\end{subfigure}
\end{figure}
\end{adjustwidth}
}

% Optional, but recommended, Copyright page
\copyrightpage{Timothy L. M. Riffe}

% Page numbering. If you don't include a frontispiece or copyright page, 
% you'll need to change this for two-sided printing.
\makeatletter
\if@twoside \setcounter{page}{4} \else \setcounter{page}{1} \fi
\makeatother
 
% Optional Dedication page
%\dedicationpage{For all the people}

% Optional Acknowledgements page
\startacknowledgementspage
% 
This dissertation would not have been possible without the continued support of
the Centre d'Estudis Demogr\`{a}fics, which was my academic home for the five
years from 2008 until 2012. Thanks are owed to many of the faculty and
staff of this wonderful institution. Most especially I thank my director since
the start, Albert Esteve, who gave me all the space, encouragement and support I
needed during my doctoral studies-- You kept me pointed in the right
direction. Thank you to Anna Cabr\'{e} for welcoming me to the CED,
for first imparting me with the fundamentals of classical demography, and for
having been a constant source of wisdom and advice. Thanks to I\~{n}aki
Permanyer for helping me with some methodological issues and overall critique
toward the end of this thesis. Thank you Daniel Devolder for always being
available to assist, for letting me pilfer your library, and for always
understanding what it was I was trying to say. Thanks to my CED peers for your
commraderie and empathy. Thank you Soco for keeping me legal and in
order.

Thank you to Vladimir Canudas Romo, for your encouragement and
empowerment. I would never have undertaken a formal demographic topic were it
not for you, and I would have never landed in Berkeley were it not for
you. Thanks to Alexia F\"{u}rnkranz-Prskawetz for your addictive enthusiasm for
models. Thank you Adrien Remund and Felix R\"{o}ssger for keeping me on my
toes-- more and more I think, thanks to you guys, that there is plenty of
low-hanging fruit left out there to pick.

Thank you John Wilmoth for bringing me to Berkeley, for your generosity, trust
and continuous positive feedback. Thanks to Magali Barbieri for your
encouragement and empathy, to Carl Boe for being available 24 hours to help with
methodological issues of any scale and relevance to the task at hand. Thank you
Robert Chung for sharing your experience, for your dedication of time
and mental resources to the problems that I have made, and for your goodwill in
general. Thank you to the HMD as a whole for providing me with an at once demanding,
didactic, trusting and friendly team and environment.

There are also acts of random kindness that possibly happen in all academic
communities, but which we may as well recognize here with names: Griffith Feeney
scanned and send me his PhD thesis amidst an intercontinental move-- wow,
thanks! Ron Lesthaeghe also dug up a copy of, photocopied, and sent me via snail mail a copy
of Stanlye Wijewickrema's thesis at no cost-- wow, thanks! Robert Chung, you
also deserve mention in the good samaritan section for helping me
``just-because''.

Thank you to my mom and dad for your faith in me and for your always genuine
curiosity about what it is I do. You've invested heavily in my education, and never once
faltered in light of my academic caprice. This has been a source of emotional
stability in light of my otherwise seemingly haphazard bet-taking, which finally
appears to be paying off.

Thank you Ainhoa for always believing in me, for telling me so, and for keeping
me pointed straight. This dissertation would never have been finished without
you!

My stay in the CED was financed primarily by a
fellowship from the Ag\`{e}ncia de Gesti\'{o} d'Ajuts Universitaris i de Recerca
of Catalunya, and for that I owe a debt of gratitude to my metahost, Catalonia,
and to Soco and Albert for help with the entailed paperwork, which I
would have never otherwise been able to manage. Other sources of funding
included an in-house stipend from the CED in 2008, a mobility scholarship 
from the Ministerio de Educaci\'{o}n of Spain, in principle supported 
part of my stay in Lund for the EDSD; a supplemental stipend from the
MPIDR during the start of the EDSD in Rostock in 2009; and especially the
WORLDFAM grant from Albert Esteve, which funded all of my conference
participation during the course of my PhD studies, as well as summer courses at
the ICPSR in 2009. The final phase of my financial support, post PhD fellowship,
came from my current employer, the Department of Demography at the University 
of California, Berkeley where I currently work for the Human Mortality Database
project, but where I also received constant encouragement to finish my PhD and
sizeable chunks of on-the-clock time alotted to work exclusively on this
dissertation. More than any government or institution, my mom
and dad are responsible for the material investment in my education that got me this far,
including occassional injections of support throughout the doctoral process.







\label{Acknowledgements}

% Optional Preface page
\startprefacepage
Demographers study population structure and demographic flows. In order
to assess the magnitude and intensity of demographic phenomena in real-world
populations, one must first remove potential distortions introduced by
population structure -- but population structure is itself an outcome
of demographic phenomena. Here, \textit{demographic phenomena} refer to natality
and mortality, and \textit{population structure} refers to classifying
information such as time and sex. There are other structuring variables whose effects we would also
like to purge if the measurement of demographic phenomena 
were our primary objective, and there are other kinds of phenomena that would
also need to be measured if the analysis of structure were the primary
objective. Such description is of secondary importance in this dissertation. Our
objective is to study an abstraction of population processes, namely the 
renewal model for closed populations structured by sex and time.

That we are concerned with the role of both sexes in the modeling of population
renewal should be no surprise, as humans reproduce sexually. That it is a
challenge for models to incorporate information from both sexes in the modeling
of fertility (marriage, reproduction) has been firmly established since
\citet{karmel1947relations}. This task is challenging because models must
produce a single result, a fertility outcome, from two information sources
(males and females), which when handled apart produce incongruous results. 
There is no obviously correct way to achieve this balancing act, although a
large number of suggestions have been made. We typically call these suggestions
``solutions,'' but they are not solutions in the sense of a solution to a math problem. A
solution in the context of the present problem means simply that a reasonable
result is produced in accordance with a predefined set of modeling objectives
decided upon either by the demographer or by consensus. The problem has not been
(and may never be) solved in the sense of a necessary and best model. Instead,
solutions are weighed in terms of fulfilling desirable properties versus theoretical 
or practical parsimony.

In this dissertation, we deal with only two population subgroups: males and
females, each of which is structured by some notion of time. The modeling
problems that emerge when dealing simultaneously with the two sexes may be
conceived of as a minimal case of the much larger practical problems 
presented by modeling populations subdivided into $N$
groups. Applied demographers often simultaneously project
populations divided into regions, races, educational groups, and a potentially
large number of other categorical distinctions. The modeling
challenges presented by multigroup plurality may in this sense be thought of as
the general problem, within which the two-sex problem is just a particular
instance that must be dealt with under particular constraints. While the two-sex
problem may in a sense be reduced to the notion of the $N$-group problem, the
sexes interact in a way, and reproduction is constrained in a way, that does 
not pertain to other kinds of population subgroups -- There is namely no
``race-ratio'' or ``state-ratio'' at birth akin to the sex ratio at birth.

Later in the
present work we will provide some measure of the magnitude of the two-sex problem, 
and it will be concluded that the
magnitude is large enough to be worth thoughtfully accounting for in
measurements and modeling. Were models to encompass even more groups, the potential
discrepancies entailed by simultaneous modeling would be even larger. That we may arrive at
insights from the more familiar two-sex case that apply to the $N$-group case is
a further motivation for thoughtful exploration of the problem at hand.
$N$-group generalizations will not be explored in the present work, though with
some additional work solutions discussed here may be extended in this
direction.

The balancing of the sexes in models without considering age is
much simpler both conceptually and in practice, as it is just a matter of
choosing some middle ground between males and females. Most of the literature on
the two-sex problem, and the properties that demographers deem desirable in
solutions, deals with the time-structuring variable, age. That modeling
decisions must be made with respect to both the interaction between sexes and
the interaction between ages makes the problem an altogether complex one. 

What is age but time passed since birth? Thus, age is \textit{time} with respect
to one of the demographic phenomena that we incorporate in models of population
renewal. The reason why demographers care about age is that all
demographic phenomena vary by age in known ways, and so in order to measure the
pure force of a demographic phenomenon one does well to take age into account --
the age patterns of demographic phenomena exhibit empirical regularities that lend themselves to
modeling \citep{coale1996development}. Age counts 
up from birth, starting at the beginning. We measure milestones and the lifecourse in 
terms of age; statistics are collected by age or year of birth, and age is in 
short \textit{known}. We do not know when we will die, but
this is also something that demographers think on. Namely, in properly
accounting for age (time since birth), we may faithfully approximate death
probabilities for each age, and therein know something about our probable time
of death. This later question is a subject of considerable interest!

Demographers, and especially actuaries, regularly think about, estimate, a
probable time of death for persons of a particular ages -- that were born in
particular years. Might we not also venture to take things a step further?
What if age were counted down to death instead of up
from birth? Literally, what might we learn about demographic phenomena and
population structure if beyond age (and due to the information we glean from
age) we were to structure populations by sex and remaining years of life? That
is a big job, and we will fail to complete it, instead laying out only the groundwork
for population renewal models wherein age is exchanged for remaining years of
life.

It is my stance that population renewal models ought to account for both sexes,
and for this reason roughly equal attention is given in this dissertation to the
two-sex problem -- a problem that never goes away -- and population structured
by remaining years of life -- a somewhat novel concept that must be hashed out
before again complicating things with the two-sex problem. I apologize for any lack of
rigor on both fronts. Namely, I neither reproduce formal proofs for the
properties of the solutions that I treat, nor do I provide proofs for the
(many) claims that I make. The filling of this gap is left for a later day --
either someone will do the favor of proving my claims right or wrong, or I
will find the time to learn to do so. Instead we are led in this dissertation
primarily by intuition, and I have placed a premium on the
data-grounded demonstration of the methods I propose. After all, might we not
wish to free these formal demographic musings from the vacuum and see what might
be learned? There is therefore the risk that some conceptual error or
miscalculation of mine -- and all errors and miscalculations herein are mine
alone -- will be a setback. This is my risk alone, but the possibility is
not that distressing. Rather, it is inherent to the business of charting new
territory, and this I have every intention of doing. If the maps I draw are no
good, the territory explored may still be good.

So it is that sex and \textit{time} are the structuring variables of
interest in this dissertation. Aside from comparing two-sex models (and often the male
and female one-sex models), we at times compare models that
specify age with models that specify time until death. We will learn that the
specification of time in models has a large impact on results, and it could
be said (with a pinch of jest, of course) that we herein expose a
fundamental \textit{two-age} problem in demography. This was of course not our goal from the
outset of the doctoral process. The narrative of how this dissertation came to
take the shape that it has is as follows.

The original objective for this dissertation was to hash out a survey of two-sex
``solutions'' and implement them in a standard and reproducible format while
applying each to contemporary datasets. I chose the topic after completing the
EDSD in the summer of 2010 in order to force myself to improve my formal
demographic skills, which I had only begun to develop in that program. That
is, I knew it would be difficult and possibly beyond my abilities. And so, I
began at the beginning, collecting all the materials I could locate on the 
two-sex problem, and reproducing methods in no particular order. After a few 
successful attempts (reproducing \citet{schoen1981harmonic},
\citet{mc1975models} and \citet{henry1972nuptiality}) on ad hoc acquired data, I
came to realize that all the methods in my scope will essentially require or
the same input data-- basic exposures, and births cross-tabulated by
sex and age of father and mother-- so I diverted attention to standardizing some
datasets to use throughout this dissertation-- The US and Spanish populations
for about the past four decades. I decided to discard or translate methods
dealing with the two-sex problem in marital transitions in the first place 
because Spain lacks good estimates of marital status exposures, 
and these would need to have been produced artisanally (laboriously). This
choice reduced the implementation workload, but the sex balancing strategies
from analytical family demography have still been taken into consideration where
possible.

When the mathematics or
presentation style in a given article were over my head, I typically took a few
steps back to some earlier or less complex method, or altogether went back to the basics 
in \citet{sharpe1911problem}, \citet{kuczynski1932fertility}, \citet{coale1972growth} 
or \citet{caswell2001matrix}. Some methods that were beyond my grasp in the 
begining \citep[e.g., ][]{mitra1978derivation, gupta1978alternative} were
finally understood and implemented later down the road. Others I still do not 
understand\cite[e.g., ][]{choo2006estimating}, despite having
reproducible code!

All along I had no vision or pretense of designing a new method, but I rather
na\"{\i}vely assumed that gradual familiarity with the tools at hand would lead
me to some minor tweak or meaningful critique of the existing palette of methods at
hand. For two years I did not produce anything novel and managed only to
reproduce a few branches of the above-mentioned survey of methods, and my
resolve waned. A spontaneous conversation with John MacInnes sparked
what was to become the second, but dominant, axis of this thesis, our
realignment of age in renewal models. John
mused about what a population pyramid would look like if it were drawn with
life expectancy on the vertical axis instead of age. We tried to imagine what
shape such a pyramid would assume, but were on the whole left guessing. I took a
stab at how one might go about transforming age-classified population counts to
remaining-years classified population counts, and came up with what is here
Equation~\ref{eq:dxredist}. Later I realized that the central component to that
formula, which says ``what is the probability of dying at age $x+n$ given
survival until age $x$,'' is fairly fundamental and already lying around in
formal demography, probably in various texts and minds -- I spotted it in the
wild in \citet{vaupel2009life}, and more prominently in
\citet{miller2001increasing}, but it's certainly well-known and/or easily
derivable. I have never seen this formula applied to redistribute population counts in the manner
suggested here, although the concept of remaining years until death has
certainly been considered. Miller refers to this
temporal concept as \textit{thanatological} age.\footnote{This phrase does not
appear in the cited paper, but Miller informs me that the phrase was
coined by Ken Wachter.}

Shortly thereafter, after looking at many such remaining-years pyramids and
coming to some exciting conclusions, I realized that one may just as well
restructure \textit{any} age-classified data in the same way. So I took a 
look at some thusly-restructured fertility rates, explored
some more, and spontaneously resolved to try to figure out what form the
fundamental Lotka equations would obtain if reworked to be based on
remaining-years classified data. From that moment I was self-obliged to 
bring \textit{this} family of
population models to bear upon the two-sex problem. Is the problem the same?
Will this transformation teach us anything? Does it make sense to simply
project remaining-years structured populations alongside age-structured
populations? That is what this dissertation is about.

As hinted before, it is the case that when one estimates population growth (or
some other interesting quantity) from a remaining-years classified population,
one arrives at a different result from that derived by the analogous
age-structured model. I do not undertake the worthy task of formalizing the
difference between age and remaining-years structured growth rates, as did
\citet{karmel1947relations} for the difference between male and female
reproduction rates (or \citet{coale1972growth}, put differently). As for this
new discrepancy, I have managed only the less glamorous feat of pointing out
that it exists, as did \citet{kuczynski1932fertility} for the male
and female cases. There is plenty of work left to do, and it is my hope herein
to stimulate discussion in the discipline of demography about whether structuring
aggregate demographic data (and the models derived from these) by remaining
years of life may contribute further insights into human population dynamics. At
times in this dissertation, I will posit how the nature of
remaining-years-structured populations is agreeable to stable population theory
and lends itself to population projections in general and to the sex balancing
undertaken in two-sex solutions in particular. Further, the strategies that
demographers have developed to patch the two-sex problem will provide us with
insights into the new discrepancy presented by our restructuring of age.

\label{Preface}

% Table of contents, list of figures, etc.
\tableofcontents      % Required
\listoffigures        % Required if there is more than one figure
\listoftables         % Required if there is more than one table
%\listofmaps          % Required if there is more than one map
%\listofappendices    % Required if there is more than one appendix
%\listofabbreviations % Optional. Abbreviations should be stored in a file named abbr.tex

% Inserts abstract from external Rnw file
\startabstractpage
{The Two-Sex Problem in Demography}{Timothy L. M. Riffe}{Chair: Dr. Albert Esteve}
% shoot for 150 words
%The most widely used measures and models of population formally assume
%fertility as a function of females only, 'female dominance'. Early formal
%demographers proposed presently used indicators, such as $R_{0}$ and TFR as
%approximations, and accepted the assumption of female dominance as a matter of
%convenience rather than conviction. The magnitude and direction of error in
%measures of fertility and reproductivity due to this practice are not always
%clear. Various adjustment procedures have been proposed in the past several
%decades, but there remains no consensus about the most desirable analytic
%method. This dissertation recapitulates the discussion,  assesses the proposals
%that have been made about the two-sex problem, makes methods available in
%reproducible programming code and comparatively applies this suite of methods
%to two contemporary populations, Spain and the USA. Extra sections make
%recommendations for two-sex adjustments in cohort component population
%projections.

One of the foremost problems in formal demography has been that of including
information from the vital rates for both sexes in models of population renewal
and growth, the so-called two-sex problem, which may be thought of as a subset
of the analytical problems entailed by multigroup population modelling. This
dissertation characterizes the two-sex problem by means of decomposing the vital 
rate components to the sex-gap between the male and female single-sex stable growth
rates. A suite of two-sex models for age-structured models from the
literature are presented. A new variety of age-structure, age based on remaining
years of life, is presented. Analogous models of population growth for the
single-sex and two-sex cases are developed for populations structured by
remaining years of life. It is found that populations structured by remaining
years of life produce less two-sex divergence than age-structured models,
thereby reducing some of the trade-offs inherent in two-sex modelling decisions.
In general, remaining-years-structured models are found to be more stable over 
time and closer to their ultimate stable forms than age-structured models. Models of
population growth based on remaining-years structure are found to diverge from
like-designed age-structured models, and this divergence is characterized in
terms of the two-sex problem.

\label{Abstract}

\startthechapters 
% inserts ch1 (Intro)
 %\SweaveOpts{prefix.string=Figures/intro}
 \chapter{Introduction}
 \label{chap:Intro}
 \todo{This is where I introduce everything in about 5 pages of simple,
easy-to-understand text.}


\section{Data}
All calculations in this dissertation, unless otherwise cited, are original 
and based on a small number of publicly available datasets that have
been modified and standardized according to a strict and simple protocol, as
described in following. Since the same small number of datasets
is used throughout this document, sources are not cited in situ, but rather
always refer to the same sources, as described here. Only two populations are
treated, Spain (ES) and the United States (US). Similar data for France was also
located, but was not included as it covered a shorted range of years. 
Since the data used in this work are so simple, calculations presented are
expected to be replicable for a variety of other populations, though not for
populations where births by age of father are not available.

\subsection{Birth counts}
Birth counts for Spain and the US were not available in tables of the format
required for this dissertation. For this reason, birth counts were tabulated from birth
register microdata publicly available as fixed-width text files from the 
Instituto Nacional de Estadistica (INE)\citep{MNPnacimientos} for Spain and the
National Center for Health Statistics (NCHS) (INE)\citep{NCHS19692009} for the US. For Spain, 
the years 1975-2009 are used and for the US, the years 1969-2009 are used
consistently throughout this dissertation. At the time of this writing, further
years are available, but not included. Earlier years for the US are also
available in earlier official publications, but these have not been digitized
for inclusion in this dissertation. Cross-tabulations for each year included age
of mother, age of father and sex of birth. Resident status was not used as a
selection criterion for births in either country.

In all cases for both countries, age of mother was
stated, but in some cases age of father was missing. Births with missing age of
father were redistributed proportionately over births to fathers of known age 
separately for each age of mother. In Spain births with unrecorded age of father 
tended to comprise less than 2\% of
all cases, and so we do not expect this procedure to affect
results, and no further sensitivity tests were performed. For the US, the
percent of all births where age of father was not recorded ranged between 7\%
and 18\%, as seen in Figure~\ref{fig:USmissingAge}. 

\begin{figure}[ht!]
        \centering  
          \caption{Proportion of births with age of father not recorded, US,
          1969-2009}
           % figure produced in /R/DataDiagnostics.R
           \includegraphics{Figures/USmissingAge}
          \label{fig:USmissingAge}
\end{figure}

For the US, the degree of missingness of fathers' age varies by age of
mother (not shown). For ages greater than 25, we do not expect this to affect
results in an important way. Averaged over all years, ages $<= 20$ all had missingness of more than
20\%; ages $<= 16$ had missingness of more than 40\%, and ages $<= 14$ had
missingness of more than 60\%. This may affect results if the age-pattern of
males of unrecorded age differs from that of males of recorded age in a
non-trivial way. This uncertainty enters into the male
age-pattern of fertility, as well as the bivariate
age distribution of births (age of mother by age of father) may affect results
for the US where these age-specific data are used.

For both countries, cross-tabulated
sex-specific birth counts were entered into matrices of standard 111$\times$111
dimensions, covering ages 0-110. Ages with no births simply contain zeros. Open
age groups from the original data were not redistributed over ages beyond the
bounds of the original microdata. Especially for young ages of fathers and the
upper ages of mothers, this will be visible in respective age
patterns, but the effect on overall results (TFR, growth rates) will be
trivial.

Where birth counts are not required differentiated by sex of birth, for
instance, we sum over sex. Birth counts by age of mother are always taken from
the column margin of the birth matrix, while age of father is the row margin.
This practice helps to minimize the number of data objects used.

\subsection{Exposures and mortality data}

All other data for the US and Spain were downloaded from the Human Mortality
Database (HMD)\citep{wilmoth2007methods}. These data include, most importantly,
population exposures\footnote{At the time of this writing, exposures from the
Human Fertility Database (\url{www.humanfertility.org}) may have been more
appropriate for certain age groups, but since we prefer to use all ages
$0-110+$, HMD exposures were utilized instead.} and population counts by age,
sex and year and the deaths distribution, $d_x$, from the sex-specific lifetables. $d_x$ informaion was 
always rescaled to sum to 1, which minimized rounding errors and simplified
programming. Other items drawn from the HMD but used less consistently
included, mortality hazards, $\mu_x$, survival curves, $l_x$ (also rescaled so that $l_0 = 1$), 
lifetable exposures, $L_x$,
life expectancies, $e_x$, and death counts by Lexis triangles. Each of these
items is used in single-age format, with ages $0-110+$. The open age group,
$110+$ is used as age 110 and is given no further treatment. The
respective uses of each of these items should be obvious from the context 
of the formulas being applied, and are stated explicitly in the text or in 
footnotes where the use may not be obvious.

HMD data itself has come from the respective official sources of these
two countries, and so will inherit whatever errors were present in the original
data prior to applying the HMD methods protocol. Most relevant for this
dissertation, Spanish intercensal population estimates, which are the basis of
HMD population estimates, have been subject to an uncommon smoothing
procedure over age by the INE. Where abrupt changes in cohort size occur, such
as the unusually large 1941 cohort, this procedure will have the effect of
decreasing the size of large cohorts and increasing the size of small neighboring cohorts. This is
highly undesirable for any demographic study and is apparently a legacy
practice that will soon cease\footnote{Thanks to Dr. Amand Blanes for bringing
this issue to my attention. The INE will likely release new retrospective
population estimates during the course of 2013, but these will come too late
for incorporation into the present dissertation.}. In this dissertation, this
distortion will be most noticeable in the calculation of event-exposure rates, 
wherein the numberator has not been subject to this exogenous smoothing, but 
the denominator has. It is unfortunately the case that alternative sources of 
population estimates for Spain are in worse condition. These effects will echo through all HMD mortality
estimates for Spain, as well as our own fertility calculations. 

\subsection{Empirical results in this dissertation}

Data-based results in this dissertation are with few exceptions displayed
graphically, rather than in the form of tables. Since the original data and
code used to produce results are all available, one could with minimal effort
and no guesswork derive the numbers represented in each figure. We prefer
graphical representation of results because this conveys larger amounts of
information in less space and is more intuitive for the reader. The reader
should understand that data are used primarily to illustrate the concepts under discussion, rather
than in search of some empirical truth. The two
above-mentioned caveats for the data used herein (missing fathers' age in the
US, and faulty population estimates for Spain) should be born in mind when
interpreting some figures, such as age-specific fertility curves. We do not
expect either of these two data drawbacks to affect summary results 
(e.g. growth rates, $r$) in a noticeable way, and we expect that any
\textit{broad} conclusions arrived at in following will be robust to these
original shortcomings. 

The user will also note that most results are derived deterministically.
Accounting for uncertainty in many of the results presented here would provide
the reader with more insight into particular kinds of results, such as projected
results or stable population structures occassionally displayed in figures.
Several of the methods to be presented in following are novel to the field of
demography, and so we may look upon the results dervied therefrom as test
results. The addition of stocasticity to these methods, if they are deemed of
worth, is left open as a branch for improvement. Here we only wish to point out
that the majority of figures will, for this reason, not contain confidence or
credibility bounds.




\part{The two-sex problem in age-structured populations}
\chapter{Measuring the Two-Sex Problem}
  
The purpose of the present chapter is to describe and quantify the two sex
problem, both as a whole and in terms of its constituent parts. Purely
mathematical treatments of the two-sex problem have often been content to 
prove (or point out that it has been proven) that males and females,
if modelled separately, will obtain different growth rates, which leads to
absurd and inacceptable results. Models that include both sexes must produce a
single growth rate if they are meant to relect observed human population renovation. This is
true in the same way that mathematical identities are true, and to point this
out, or reproduce one of the proofs of the two-sex problem, \textit{may} also have sufficed for the
present dissertation. Here the aim is to produce
intuition about the size and nature of the two-sex problem, and this will be
acheived by appealing to data. This intuition will tell us whether the 
problem is then trivial or worth accounting for in
population models. The conclusion will be that yes, it is usually worth our
while to account for the balance of sexes in projections and in models of human
population growth. In the scant instances where the two-sex problem would have
been trivial, the demographer incurs no penalty in accounting for it
nonetheless, and so it is advised to account for it.

The first task will be to measure the two sex problem. This will be done in
three ways: 1) By calculating intrinsic growth rates separately for the sexes.
The gap between male and female growth rates determines the ultimate speed of
divergence between the males and females; 2) By projecting each sex separately
in order to estimate how many years would need to pass before one sex grows to
twice the size of the other sex. If the answer is a few decades, then this is
grave indeed, and if it is a few millenia, then we might not worry about the
two-sex problem in modelling; 3) By simply comparing predictions of births using
male versus female rates. The size of discordance between predictions of total
birth counts also serves as a measuring stick.

Having illustrated the magnitude of the problem, we will explore the primary
causes for the two-sex problem, namely sex-differences (dimorphism) in the vital rates
that determine population growth. Specifically, these include fertility, the sex
ratio at birth, and mortality. We present time series of these phenomena and
briefly describe the main respects in which males and females differ, to the
extent that is relevant in understand the foundations of the two-sex problem. We
also illustrate how dimorphism has changed over time. The vital rates
used to estimate natural growth undergo changes, at times in different ways for
males and females. Outlining these changes makes clear that the nature and
composition of the two sex problem also changes over time.

The presentation of dimorphism is followed by an explicit decomposition of the
gap between male and female growth rates into components due to fertility,
mortality and the sex ratio at birth. This analytic exercise will tell us the
weight that each relevant element of the sexual dimorphism in vital rates has
had in the two-sex problem. We will see that the interplay between vital rates
in determining the size and direction of the sex-gap in intrinsic growth rates
is complex and inconsistent. Sex ratios consistently give males a head start in
growth rates in these two populations. This is offset slightly, but not entirely
by female advantages in survival. The size and direction and of the effect of
fertility has changed dramatically over time.

Finally, further analysis and speculation is offered in how age-interactions may
also affect the size and nature of the two-sex problem. This section is more
suggestive than definintive in nature. However, such considerations are relevant
to two-sex models to the extent that age-interactions are allowed for or
controlled for. It will be shown that bivariate age distributions are very far
from random, that these distributions change over time, and that the degree of
age-hypergamy in fertility has changed over time. This paints a more complex
picture of fertility change than is visible by merely looking at marginal
distributions of age-specific rates.










 
  \section{Magnitude of the two-sex problem}
    
This section seeks to expose the magnitude of the two-sex problem. This is
achieved in Section~\ref{sec:divlotkar} by measuring the gap between
male-specific and female-specific (canonical) intrinsic growth rates. Intrinsic growth rates are a
theoretical result -- an output of the application of stable population 
theory to data. If our treatment of the two-sex problem were limited to 
stable population theory, this would suffice. We will not, however, limit
ourselves to pointing out an inconsistency in an otherwise coherent and
self-contained set of mathematical abstractions. 

Applied demography is concerned with the more practical
business of population projections. Here too we briefly
expose the magnitude of the problem by summarizing results in two more tangible
ways: 1) Section~\ref{sec:ageSRdoubling} presents the results of carrying out
simultaneous projections of male and female single-sex populations to an 
arbitrary point of absurdity; 2) Section~\ref{sec:divbirth} displays the
results of the even simpler task of projecting births at fixed time intervals
and measuring the size of the discrepancy between male and female predictions.

In this way, we summarize the major discrepancy in terms of an exponential
growth parameter, a waiting time, and a relativized count.


    
    \subsection{Divergence}
      \label{sec:Divergence}
``Divergence'' here refers to two or more quantities growing farther and
farther apart with the passing of time -- quantities that have different
trajectories or speeds, branching, say. The quantities diverging in this way are
the male and female total populations, when modeled separately. Specifically, we refer to
the male and female stable populations, a product of vital rates,
theoretically removed from reference to real population counts. It is therefore
sufficient to speak of changes in the relative size of the male and female
populations, and further sufficient to speak of the
difference in the rate of change of these two populations, which is constant in
the limit. The intrinsic rate of increase in the Lotka model is $r$, and the
rate of separation between males and females can be captured in the difference
between the male and female rates, $r^m$ and $r^f$, respectively.

      
      \subsubsection{Exponential separation}
         \FloatBarrier
 \label{sec:divlotkar}
%* all figs here produced in IllustrateDivergence.R
As mentioned, divergence in this dissertation refers to the exponentially
increasing distance between single-sex male and female populations that unfolds
when they are simultaneously projected into the future -- or virtually projected
in the case of characteristic stable populations. The magnitude of separation increases
exponentially because males and females obtain different intrinsic 
growth rates, $r$, that are extracted from Lotka's fundamental equation
\citep{sharpe1911problem}:

\begin{equation}
\label{eq:lotkaeq}
1 = \int _0 ^\infty e^{-ra}p_a m_a \dd a 
\end{equation}
where $p_a$ are age-specific survival probabilities, $m_a$ are age-specific
single-sex fertility probabilities,\footnote{i.e., where $F_a^F$ is female
age-specific fertility,  $m_a = F_a^{F-F}$, which is female fertility calculated
using only daughters in the numerator, $F_a^{M-M}$ for males.} and $r$ is
the growth rate to be estimated.\footnote{In this dissertation, $r$ (and 
variations of $r$) are always estimated by using the (modified) strategy proposed 
by \citet{coale1957new}. Where modified, the new process is always described in full. In the present
case, we use Coale's version.} By ``single-sex'' it is meant that $m_a$ may
be specified either as the fertility of girls born to mothers or of boys born to
fathers. \citet{yellin1977comparison} prove that divergence is to be expected, as 
forced agreement between the male and female versions of Equation~\eqref{eq:lotkaeq}
 would imply an overdetermined system. In any instance
where single-sex $r$ estimates differ, projecting separately will result in sex
ratios that either grow toward infinity in the limit if $r^m
> r^f$ or decline to zero if $r^m < r^f$. If the gap between rates is large, this happens
quickly; if small, divergence is slower. This is in either case a modeling
problem of practical significance, and the crux of the two-sex problem. 

Single-sex intrinsic growth rates, $r^m$ and $r^f$, can be 
estimated from data. In looking at time series of 
 growth rates (see Figure~\ref{fig:rmf}), observe that the sex-gap has varied
 over time, that the male rate is typically higher than the female rate (aided greatly 
by the sex ratio at birth), and that there have been crossovers in the USA: 
$r^f > r^m$ in 1994-1996, and again briefly in 2001. 

\begin{figure}[ht!]
        \centering  
          \caption{Male and female intrinsic growth rates, Spain and US,
          1969-2009}
           % figure produced in
           % /R/rm_rf_divergence
           \includegraphics{Figures/rmf}
          \label{fig:rmf}
\end{figure}

Perhaps even more curious are occasions when $r^m$ and $r^f$ have been on
opposite sides of zero, i.e., exponential growth and exponential decay at the
same time. In the USA, this has happened many times in the period studied:
1972-1973, 1990, 2004-2005, and again recently in 2008. In Spain rates were 
briefly on opposites of zero in 1981-1982, in the middle of a period of sharp
decline in fertility. In all of these cases male growth rates were positive
while female growth rates were negative. Note that this does not mean that 
\textit{observed} year $t$ natural growth rates were of opposite signs, but
rather the intrinsic rate that characterizes the male and female stable
population models.
Figure~\ref{fig:rmfGap} again displays the information of interest, the size 
of the gap between $r^m$ and $r^f$ over time.

\begin{figure}[ht!]
        \centering  
          \caption{Gap between male and female intrinsic growth rates, Spain and
          US, 1969-2009}
           % figure produced in
           % /R/rm_rf_divergence
           \includegraphics{Figures/rmfGap}
          \label{fig:rmfGap}
\end{figure}
\label{par:coalermrf}
\citet[p. 57]{coale1972growth} points out that when $r^m > r^f$, as was
typically the case here, multiplying male exposures at each age by a factor equal to
$e^{(r^m - r^f)T^m}$, where $T^m$ is the male mean length of
generation,\footnote{where $T^m$ can be estimated as
$\frac{log(R_0^m)}{r^m}$} will bring $r^m$ in line with $r^f$. Alternatively,
$r^f$ can be aligned with $r^m$ by multiplying female exposures by a factor equal
 to $e^{-(r^m - r^f)T^f}$. This works in reverse when $r^f > r^m$.

 \FloatBarrier

      
      \subsubsection{Time until an unreasonable sex ratio}
        
Differences in intrinsic growth rates are the essence of divergence in
stable populations, but these do not necessarily represent divergence in
projections, per se. Figure~\ref{fig:rSRdoubling} gives a more intuitive idea of
the magnitude of divergence implied by the vital rates in each studied year. The following
 exercise is carried out: Given each year's male and
female vital rates, how many years would it take for the total population of one
sex to be double the size of the other, always using the year $t$ population as
the initial conditions?\footnote{These figures were determined using projections
based on the two single-sex Leslie matrices that characterize male and female
vital rates each year.}

\begin{figure}[ht!]
        \centering  
          \caption{$log(\mathrm{Years})$ until one sex is twice the size as the
          other, given separate single-sex projections using annual vital rates and initial
          conditions, Spain and US, 1969-2009}
           \quad
           % /R/rm_rf_divergence
           \makebox[\textwidth]{\includegraphics{Figures/rSRdoubling}}
          \label{fig:rSRdoubling}
\end{figure}

Clearly the run of years in the United States where $r^f$ and $r^m$ were very
close (approx 1994-2001) imply such slow rates of divergence that we could, as a
matter of accident, safely ignore the two sex problem in those years. These
tended to be the same years where the greater growth rate oscillated between
male and females. However, any acceptability threshold is a matter of
convenience and taste: presumably the demographer would like age-specific 
population estimates to be much closer to truth than \textit{half} or \textit{twice} the ideal value.
Dropping the badness threshold would of course decrease the waiting time until
it is met in any given year. These are practical questions. More
stringent are the demands of theoretical stable populations, where
sex consistency is very desirable. Not a single year of data presented here
meets the requirements of a consistent stable population, and even if this were
to be observed, it would be coincidentally rather than essentially so. 

      
      \subsubsection{Disagreement in predicted birth counts}
         \FloatBarrier
\label{sec:divbirth}
Aside from divergence in the characteristic growth rates of the single-sex
stable models, single-sex separation is amenable to observation in the everyday
practice of demography. At the root of the two-sex problem is that the total
numbers of births predicted by male and female rates ought to, but never do,
agree, aside from in the jump-off year from which rates are initially derived, 
which is a tautology. Let us therefore design the following practical excercise:
Given the fertility rates of the present year $t$ and known exposures for
future years, both separate for males and females, how many total births do we
predict in $n$ years, where $n$ is equal to $[1, 5, 10, 15]$ based on male
versus female inputs? Figure~\ref{fig:BirthCountDivergenceAge} displays the
results of this exercise, where the value plotted in the relative
difference between total births predicted by male
rates versus total births predicted by female rates, divided by
the average of the two predictions\footnote{$\frac{2(B^M - B^F)}{B^M + B^F}$}.

\begin{figure}[ht!]
        \centering  
          \caption{Relative difference (male - female) between predicted total
          birth counts in year $t+n$ based on year $t$ fertility rates and year $t+n$ exposures, US and Spain, 1969-2009.}
           % figure produced in
           % /R/BirthCountDivergenceAge.R
           \makebox[\textwidth]{\includegraphics{Figures/BirthCountDivergenceAge}}
          \label{fig:BirthCountDivergenceAge}
\end{figure}

Predicting births in year $t+1$ appears to entail a 1\% discrepancy in some
cases. In the first years for the US, the $t+15$ prediction (predicting
1984 births with 1969 rates) already entailed a 12\% relative difference
between the sexes ($B^M > B^F$), with separation between $t+15$ predictions
steadily falling over time. For Spain, $t+15$ predictions started (predicting 1990
births with 1975 rates) with little disagreement, but this has steadily grown to be as high
as 12\% in recent years.

Discrepancies illustrated here are net of observed secular changes in
fertility over time. That is to say, the relative differences in
Figure~\ref{fig:BirthCountDivergenceAge} are not prediction errors, but rather
the differences entailed betwen hypothetically choosing female or male
dominance. The short projection horizons tested here are well within the range
of horizons that demographers typically evaluate, and the magnitude of
discrepancy revealed here should give pause, even to the most ardent defender
of female dominance. The divergence of single sex
models has now been demonstrated for recent years in the US and Spain. Other
populations and years will show similar patterns, perhaps greater or lesser, as
the kind of divergence illustrated here is inherent in single-sex
population models.


      
  \section{Primary factors contributing to the two-sex problem}

    \subsection{Dimorphism}
       \FloatBarrier
Divergence between single-sex population models has been shown to be a problem
of both theoretical and practical significance for demographers, and it stems from
the fact that vital rates almost always differ between the sexes.
This characteristic of human populations, sexual dimorphism in vital rates,
is manifest in all subfields of demography. In following, we will use the term
sexual dimorphism, which enters into the present discussion via evolutionary
demography and biology\footnote{See, e.g., \citet{caswell1986two} for a paper
relevant to the present dissertation where the term \textit{dimorphism} is used
in the same way.}, to refer to sex differentiation, specifically with respect to demographic forces-- vital rates.In observed populations, fluctuations in vital rates are constantly underway, and can either magnify or 
diminish differences between single-sex intrinsic growth rates (or predicted births). 
In population models, dimorphism is relevant as it pertains to fertility and mortality 
rates, as well as the sex ratio at birth.

This section is exploratory and descriptive in nature. We seek here to
demonstrate 1) major differences between male and female rates and 2) the fact
that these gaps can and do change over time. We only touch upon rates that might
be relevant to the two-sex problem. The subsequent section~\ref{sec:dimorphASFR}
will quantify the contribution of the vital rates treated here to the size of the two-sex
problem.

      
      \subsubsection{Fertility rates}
         \FloatBarrier
 \label{sec:dimorphASFR}
 
It will later be seen that the effects of differential survival and the
sex ratio at birth on the maginitude of the two-sex problem are rather
consistent. This is not the case with fertility, which is, with respect
to the two-sex problem, volatile. To be explicit, fertility rates are in this
section (and previous sections) defined as births classified by age of
progenitor divided by person-years exposure classified by age of progenitor.
There are myriad ways to quantify fertility that demographers are well familiar
with. This section will only point out a few measures that are deemed by the
author to be relevant to the two-sex problem. Other factors that are known to
affect observed fertility, such as parity distributions, are not discussed. We
will biefly exposure differences between males and females as they pertain to
the maginitude and distribution of fertility rates. Magnitude is summarized in
terms of the total fertility rate (TFR), and much more attention is given to the
fertility distribution, which will be summarized by characterizing differences
in the age-pattern of male and female fertility, comparing the effective
age-bounds of male and female fertility, and creating a summary index of
distribution similarity between male and female fertility.

TFR is among the most well-known and understood demographic indicators, and
demographers have intuition about how it has developed in recent decades. These
two statements are more true for female TFR than for male TFR, though the study
of male fertility is said to be on the increase in recent years.

 % TFR 1969- 2010
\begin{figure}[ht!]
        \centering  
          \caption{Male and Female Total Fertility Rates, 1969-2009, USA and
          Spain}
           % figure produced in /R/IllustrateDivergence.R
           \includegraphics{Figures/TFR}
          \label{fig:TFRseries}
\end{figure}

Figure~\ref{fig:TFRseries} shows in parallel the trends in male and female TFR
in the years studied for Spain and the US. Note that
in the years of continuous decline, $TFR^M$ tended to be higher than $TFR^F$,
and in the years of gradual increase, $TFR^F$ tended to be higher than $TFR^M$. In the United States,
this crossover was observed around 1988, and in Spain around 1998. 

% ASFR for 1975, both countries
\begin{figure}[ht!]
        \centering  
          \caption{Male and Female Age-Specific Fertility Rates, 1975, USA and
          Spain}
           % figure produced in /R/IllustrateDivergence.R
           \includegraphics{Figures/ASFR1975}  
          \label{fig:ASFR1975}
\end{figure}

The distribution of fertility rates over age also differs between males and
females. Figure~\ref{fig:ASFR1975} displays ASFR in 1975 for both Spain and the
US. The distributions have moved over time, but some stylized observations will
pertain in any year. Namely, the steep increase in fertility rates over young
ages follows a similar pattern for males and females, but begins some 4-6 years
later for males than for females in these two populations. Peak male fertility
will be around 7 years later than peak female fertility, and this spread widens
over the ages in which fertility declines, creating a longer and fatter
right-side tail for male ASFR than for female ASFR. 

% ASFR bounds
\begin{figure}[ht!]
        \centering  
          \caption{Male and female fertility rate quantiles, 1969-2009, USA and
          Spain}
           % figure produced in /R/IllstrateDivergence.R
           \includegraphics{Figures/ASFRbounds}
          \label{fig:TFRboundsseries}
\end{figure}

The physiological bounds to fertility, menarche and menopause for
females -- spermarche and andropause for males -- are well known. These may be
considered semi-rigid bounds. One might also derive bounds based on the ages
where fertility crosses some decided-upon threshold\footnote{i.e. take a
strategy similar to that proposed in \citet{coale1971age} for choosing the
starting age of marriage.}. Figure~\ref{fig:TFRboundsseries} displays the
results of choosing lower and upper bounds as those ages that contain 99\% of
all fertility, along with the median age\footnote{In other words, quantiles are
taken from the ASFR distribution, not observed birth counts. Non-integer results are derived from
discrete single-age ASFR by taking quantiles from ASFR after linear
interpolation between single-age midpoints, all assumed to be mid-interval.}. 
These statistical bounds fall within the physiological bounds, necessarily. 

In general, we note that the central ages of fertility have tended to shift more
over time than the upper and lower statistical bounds, particularly swiftly for
both males and females in Spain in the 1990s, though the upper bound for
Spanish males increased in parallel to the median over the same period. The
statistical upper bound applied here has been increasing in recent years for
both US and Spanish females, and by 2009 was about a half year higher than in
1969. The upper bound for Spanish females decreased about 2 years from 1975
to 1995, and has since increased to be just half a year lower than in 1975. Over
the period studied, median ages of ASFR have increased by around 5 years for
males and females in both countries. It is particularly noteworthy that Spanish
male and female mdeian ages and upper bounds diverged for much of the period
examined, much moreso than for the US.

One way to judge the overall dissimilarity of these two distributions is to
calculate a simple difference coefficient, $\theta$, namely:

\begin{equation}
\label{eq:coefdiff}
\theta = 1 - \int \;\int min(f_1, f_2)
\end{equation}
,where $f_1$ is male ASFR and $f_2$ is female ASFR, both scaled to sum
to 1. $\theta$ is constrained to fall between 0 and 1, where 1 indicates that the
two distributions are separate and 0 indicates identical distributions.
Figure~\ref{fig:ASFRdissimilarity} displays the results of applying this
indicator to each year of data for the US and Spain. $\theta$ has followed
a wave pattern in both the US and Spain in the years studied here, though 
quite differently between the two countries. US male and female fertiltiy rate
distributions are on the whole more similar than Spanish males and females. The
US underwent overall divergence until around 1980, then rates converged until
around 2003, since which time they have slowly begun to diverge again. Spanish
rates converged until 1980, then began to diverge until the early 1990s, since
which time they have begun again to converge. If simplistic visual biases are to
be given any weight, and without consulting other sources of information, one
might presume that male and female rates in both countries will begin to diverge
again over the next decade. However, it is unknown at this time whether the
longer pattern in this indicator would indeed be sinusoidal\footnote{Births by
age of mother and father are indeed available for a further 3 or so decades
before the start of this series, but these have not been converted to data by
this author.}.

\begin{figure}[ht!]
        \centering  
          \caption{Dissimilarity between male and female ASFR, 1969-2009, USA
          and Spain}
           % figure produced in /R/IllstrateDivergence.R
           \includegraphics{Figures/ASFRdissimilarity}
          \label{fig:ASFRdissimilarity}
\end{figure}

To reiterate, Figures~\ref{fig:ASFRdissimilarity}~and~\ref{fig:TFRboundsseries}
say nothing of relative levels of fertility between males and females, but
rather of distributions. These marginal distributions, will exert influence on
two-sex divergence even if all other factors, including TFR, are equal between
males and females. This is because fertility will be weighted differently along
the sex-specific survival curves. In the decomposition of the sex-gap in
intrinsic growth rates to be presented in a later section, we do not, however,
differentiate between fertility levels and fertility distributions, per se. 

 \FloatBarrier
      
      \subsubsection{The sex-ratio at birth}
         \FloatBarrier
Clearly another major factor contributing to divergence between the single sex
male and female stable population models will be non-parous sex ratios at birth.
Since sex ratios at birth are typically greater than one, ceterus paribus, males
are given a greater $l_0$. To a certain extent, this advantage in $l_0$ is
offset by greater attrition due to excess male mortality. In this way, effective
sex ratios in reproductive ages can be ambiguously greater than or less than 1,
depending both on the sex ratio at birth and mortality conditions. The
single-sex Lotka Equation~\eqref{eq:lotkaeq} does not incorporate a third
variable for the sex ratio at birth, since we assume that rates can be
calculated separately by sex of birth. Equation~\eqref{eq:lotkaeq} could be
modified to incorporate such a variable, for instance, where $\varsigma$ is
$\tfrac{1}{1+SRB}$ for females and $\tfrac{SRB}{1+SRB}$ for males, and $m_a$
changes to either $f_a^F$ or $f_a^M$ to become either male or female ASFR. For
females, Equation~\ref{eq:lotkaeq} changes to:

\begin{equation}
\label{eq:lotkaeqSRB}
1 = \int _0 ^\infty e^{-ra}p_a^F \varsigma_a^F f_a^F \dd a 
\end{equation}
The male version is the same, with superscripts changed tp $^M$. In
Equation~\eqref{eq:lotkaeqSRB}, the sex ratio at birth is not assumed constant 
over age of mother or father, since SRB is known to decrease with age.
Figure~\ref{fig:SRB1975} demonstrates the age pattern (i.e. age of mother or
father) for the US and Spain in 1975.

\begin{figure}[ht!]
        \centering  
          \caption{Sex ratio at birth by age of progenitor, Spain
          and US, 1975}
           % figure produced in
           % /R/IllustrateDivergence.R
           \includegraphics{Figures/SRB1975}
          \label{fig:SRB1975}
\end{figure}

The age pattern to sex ratio at birth is susceptible to random
fluctuations. However, since the age-specific vector $\varsigma _a$ is summed
over age in~\eqref{eq:lotkaeqSRB}, these fluctuations are smoothed out, and in
fact, results will be identical to those from \eqref{eq:lotkaeq}. That there is
an age pattern to the sex ratio at birth makes evident that the total sex ratio 
at birth is nothing more than the birth-weighted
average of the age-specific sex ratios at birth. Since in any projection, or virtual projection
(as in the case of the stable population model), the initial and final population structure will differ, one should not blindly assume or force a constant SRB valid for both the initial and stable states
if more information is available\footnote{This later condition was the basis of
the two-sex stable population model presented in \citet{mitra1982alternative,mitra1978derivation,mitra1976effect}, and is in the
opinion of this author an unreasonable condition.}.

Aside from random fluctuations, especially evident in the oldest and youngest
ages, the age-pattern of SRB undergoes subtle changes over time. Further, there
are interactions in SRB by age of mother and age of father (these later two
also being marginal distributions). These are aspects that may also be
considered if models rely upon fertility rates cross-classified by age of mother
and father. Therefore, to the extent that there is a trend over time in the SRB
(see Figure~\ref{fig:SRByears}), part of this will owe to changes in the
age-patterns of fertility.

\begin{figure}[ht!]
        \centering  
          \caption{Sex ratio at birth, US, 1969-2009 and Spain,
          1975-2009}
           % figure produced in
           % /R/IllustrateDivergence.R
           \includegraphics{Figures/SRByear}
          \label{fig:SRByears}
\end{figure}

Note that there has been a general downward trend in the SRB in both Spain and
the United States in the period studied, that Spain has had a
higher\footnote{The difference between the US and Spain is also significant, not
shown.} SRB, peaking at over 1.09 in 1981\footnote{These extreme figures for
Spain agree with tabulations from other sources, such as the Human Mortality
Database.}, but falling ever since, first precipitously, then gradually. Since
the population of Spain is smaller, the series is much more volatile, but the trend is nonetheless clear in both countries. It is particularly relevant to note that
the assumption of a constant SRB of 1.05 in population projections in Spain
would have been and still would be very far from observed values, and would 
affect the resulting population structure. This is relevant not
just for two-sex models\footnote{Two-sex models are, however, especially advised
to take special care with the SRB.}, but also for standard female-dominant
projections, which treat males as a residual, splitting births based on some
assumption about the SRB.

This section is about dimorphism. The sex ratio at birth
falls in the domain of fertility, but is co-determined by unobserved mortality
(not treated here), as one of the determinants of the sex ratio at birth must be
sex-differentials in fetal mortality. This variety of dimorphism is especially relevant for the ultimate
sex structure of populations, since male and female survival cruves are subject 
to differing radices. For single-sex stable population models, the male growth
rate will necessarily be given an extra boost by SRB-inflated fertility
rates. This effect will be separated in the decomposition presented in
Section~\ref{sec:Decompr}.

      
      \subsubsection{Mortality}
        
Sexual dimorphism in mortality is of primary significance to human reproduction.
Parents must survive in order to parent, and children must survive in order to
become parents. This later element, survival until reproductive ages, enters
directly into canoncial indicators such as the Net Reproduction Rate. Parental
survival does not, to the knowledge of this author, enter into indicators of
population reproductivity, except in the less-iluminating
sense that parents must survive in order to progress birth parities. This section will briefly present 
some novel methods and indicators for weighing mortality into measures of
reproductivity.

\todo{$e_0$ differences}
\todo{$d_x$ overlap}
      %e_0 differences, dx overlap
    
    \subsection{Decomposition}
       \FloatBarrier
\label{sec:Decompr}
The main aspects of vital rates that contribute to the two-sex problem have by
now been illustrated, as has the maginitude of the problem, both in terms of
intrinsic growth rates and incongruous predictions of births. The primary
factors contributing to differences in $r$ have been indicated as mortality,
fertility, and the sex ratio at birth. This section takes the extra step of segmenting and
quantifying differences between the intrinsic growth rates $r^m$ and $r^f$ into
consituent parts for fertility, mortality, and the sex ratio at birth. Breaking
the components to the gap, a pure data exercise, enables us to
visualize how the two-sex problem (in terms of $r$) has evolved over time, and
lends to a better understanding of why we observe the gap in the first place.

The exercise carried out is as follows. Equation~\eqref{eq:lotkaeqSRB} has been
functionalized and applied to the US and Spanish data for males and females,
with $r$ estimated using the method of \citet{coale1957new}. The inputs to the
function are the mortality hazard, $\mu_a$, from which the survival function,
$p_a$, is derived internally using the Human Mortality Database Methods
Protocol \citep{wilmoth2007methods},\footnote{Indeed, it makes no difference how
mortality is specified, as the sum of the components that contribute to the
sex gap in $r$ will always be the same. The age distribution of the mortality
component of the decomposition will, however, depend on whether the
mortality input is specified as $\mu_x$, $d_x$, $l_x$, or directly as $L_x$ (the
discretized Lotka formula requires lifetable exposures, $L_x$, instead of the 
lifetable survival function, $l_x$). While we do not display the age pattern of
any of the decomposition components, decomposing based on $\mu_x$ would be the
most comparable in this instance, since the hazards in each age are independent
of other ages, which is not the case for $d_x$, $l_x$, or $L_x$.} ASFR, $f_a$,
and $\varsigma _a$ -- the proportion of fertility by age that is girls for
females, boys for males. Each of these inputs is separate for males and females, and thus
Equation~\eqref{eq:lotkaeqSRB} is evaluated twice, once for males and again for
females. Each evaluation will therefore produce estimates of the year $t$
instrinsic growth rates $r^m$ and $r^f$, and it is the gap between
these ($r^m$ - $r^f$) that we wish to decompose.

\begin{figure}[h!]
        \centering
        \begin{subfigure}
                \centering
                \caption{Components to difference in single-sex intrinsic growth
                rates ($r^m - r^f$), US, 1969-2009}
                \includegraphics[scale = .8]{Figures/DecomprUS}
                \label{fig:DecomprUS}
        \end{subfigure}
        \begin{subfigure}
                \centering
                \caption{Components to difference in single-sex intrinsic growth
                rates ($r^m - r^f$), Spain, 1975-2009}
                \includegraphics[scale = .8]{Figures/DecomprES}  
                \label{fig:DecomprES}
        \end{subfigure}
\end{figure}

The decomposition itself is performed using the
pseudo-continuous approximation outlined in \citet{horiuchi2008decomposition}. 
This method allows for arbitrary reduction of error in the decomposition, and
virtually arbitarary specification of the function itself (here our Equation
~\eqref{eq:lotkaeqSRB} but with $p_a$ a function of $\mu_a$) as well as the
number and variety of parameters the function assumes (here $\mu _a$, $f_a$, and
$\varsigma _a$). This is ideal for
the present case, since the functional form of the Lotka equation decomposed
here is \textit{somewhat} novel, and specification of a unique decomposition
formula would be potentially tedious. Output from the decomposition is given as
vectors of age-specific contributions from sex-differences in $\mu
_a$, $f_a$, and $\varsigma _a$ to the observed gap, $r^m$ - $r^f$. The values 
of these age-specific contributions to the observed gap may be either negative or
positive, but always sum to the observed gap, with a small arbitrary
error.\footnote{In the present case, we have ensured that the error of
decomposition is negligible and trivial. This is indeed computationally
intensive, but leaves no room for doubt in the interpretation of results.} We
do not explore the age-patterns to the contributions in $r^m$ - $r^f$, but
rather sum the age-vectors for each of the three components, yielding a total of
three components to the sex gap in $r$: one for mortality, another for
fertility, and a third for the sex ratio at birth itself. The exercise is
repeated for each year of data and summarized in Figures~\ref{fig:DecomprUS}
and~\ref{fig:DecomprES}.

Positive values in Figures~\ref{fig:DecomprUS} and~\ref{fig:DecomprES} reflect
component-specific contributions acting in the direction of $r^m > r^f$, while
negative values act in the direction of $r^m < r^f$. The sum of the three
components in each year is equal to the total observed gap. 

These results offer lessons. The sex ratio at birth, as
expected, consistently acts in favor of $r^m > r^f$. While this effect varies
subtly over time, decreasing on average in both countries, it is rather
consistent when compared to fertility and mortality. Just the reverse, and 
also as expected, mortality has consistently worked in
favor of $r^m < r^f$. This effect has tended to decline gradually over time in
both countries studied.\footnote{The author offers no prediction about whether
or not we will one day observe a crossover in the mortality component to working
in favor of $r^m > r^f$, but such an observation would indeed be consistent
with the direction of change observed over the period studied in both the US
and Spain.} 

The fertility component sheds more light on the observed gap than
the other two factors, as its direction of influence has been ambiguous, almost
sinusoidal in nature. One notes that in Spain, fertility
contributed to $r^m > r^f$ in the same years that the secular trend in fertility
dropped to its lowest levels (as measured, say, by the trend
in TFR in Figure~\ref{fig:TFRseries}). In the US, fertility contributed to $r^m
> r^f$ until 1987, and has worked in favor of females since then. The current
trend would predict a neutral effect of fertility in the US by around 2020.
Indeed male and female fertility rates are calculated on the basis of the same
total number of births, and thus differences in rates are due primarily to the
interaction between the fertility distribution and differences in
exposure\footnote{i.e., if one measures the \textit{level} of fertility in
terms of total births, necessarily shared between males and females.}. One notes
that the decomposition could in this way continue ad infinitum, since 
observed exposures are the result of past fertility, mortality and sex
 ratios at birth. Indeed, an interactive two-sex model would also have fertility rates
themselves as a function of exposures.

One further level of complexity may with little effort be added to the present
excercise, by splitting the $f_a$ (ASFR) into two components: one for the shape
over age $\rho _a$ and another for the overall level, $\tau$. In this case, 
$\rho _a$ is the fertility pdf and $\tau$ is TFR. $\tau \rho _a = f_a = ASFR$.
In this way, $\rho _a$ is understood as indicative of differences
between males and females in the distribution over age of fertility. This will
include effects from differences in the reproductive span as well as differences
in the mean and other parts of the distribution. $\tau$ (TFR) is now independent
of the shape of fertility and benchmarks the overall intensity of fertilty. We
then repeat the decomposition exercise, breaking the gap in $r$ into four
components. The sex ratio at birth and mortality effects will be identical to
the prior decomposition, and fertility will divide cleanly into the shape
component,$\rho _a$, and the level component, $\tau$. Results are displayed in a
similar fashion in Figures~\ref{fig:DecomprUS2}~and~\ref{fig:DecomprES2}.

Figures~\ref{fig:DecomprUS2}~and~\ref{fig:DecomprES2} demonstrate that fertility
effects are more complex than meets the eye. In both countries, the effects of
the shape of fertility and level of fertility were at times countervailing. The 
effect due to the shape of fertility was in several years of greater magnitude 
than that due to the level of fertility, especially for the Spanish
population -- though TFR, the overall level of fertility, tended to be more
determinant. One notes that most of the major changes in fertility in
Figure~\ref{fig:DecomprUS} were evidently due to TFR. For the Spanish
population, fertility effects were more evenly split between shape and level components,
though both have changed sign.
\FloatBarrier
\begin{figure}[h!]
        \centering
        \begin{subfigure}
                \centering
                \caption{Addittional decomposition into the components to
                difference in single-sex intrinsic growth rates ($r^m - r^f$), US, 1969-2009.}
                \includegraphics[scale = .8]{Figures/DecomprUS2}
                \label{fig:DecomprUS2}
        \end{subfigure}
        \begin{subfigure}
                \centering
                \caption{Addittional decomposition into the components to
                difference in single-sex intrinsic growth rates ($r^m - r^f$), Spain, 1975-2009.}
                \includegraphics[scale = .8]{Figures/DecomprES2}  
                \label{fig:DecomprES2}
        \end{subfigure}
\end{figure}

\FloatBarrier
From these trends several things should be clear:
\begin{itemize}
  \item There are factors that work in favor of $r^m
> r^f$ and vice versa, and others that are ambiguous.
  \item The balance of these factors is dynamic.
  \item The sign of the sex gap in $r$ is ambiguous.
  \item The often-observed male advantage in $r$
is not necessary, though males have a strong positive bias in the form of the
sex ratio at birth.
  \item Fertility is the most volatile of the three factors represented here,
  and it is the main factor that changes the sign of the gap.
  \item Part of the fertility effects is due to differences in the distribution
  of fertility over age and part is due to the overall level. Both of these
  components are also of ambiguous sign
  \item These two fertility components identified are potentially of similar
  magnitude and they do not necessarily change in sync.
\end{itemize}

This section should make clear why fertility (sometimes via marriage) functions
have been given the overwhelming amount of attention in discussions of the two-sex
problem. It is not consistently the case that fertility levels are
differentiated from fertility shapes, and this may perhaps be deserving of
attention. In any case, a two-sex model of population renewal must account for
(balance) these three factors in some way, so as to produce a consistent and 
unified account of population reproductivity.

One may rightly notice that we have not considered the interaction of age in our
current treatment of the sex gap in $r$. Given their inconsistent behavior,
fertility data are evidently in need of more exploration in this direction than
either mortality dimorphism or the sex ratio at birth. The following section 
provides an empirical summary and exploration of what kinds of age interactions
may be present in fertility data. The results to follow are intended to invite
reflection, and are not quantified in a further decomposition. 
 \FloatBarrier
    
  \section{Secondary factors contributing to the two-sex problem}
      \FloatBarrier
Three factors that virtually always require accounting for in two-sex models
have thus far been described and quantified for the two case-studies of Spain and
the United States: Fertility, mortality (survival) and the sex ratio at birth.
The degree to which these factors are pertinent also depends upon model 
specification. The previous decomposition exercise was based on a particular
model specification-- the most simple design that is consistent with
established stable population theory and that incorporates our factors of
interest. 

Many proposed two-sex models make assumptions about age mixing between
mates as well as inter-age competition for mates. Let us loosely 
label such modeling considerations under the umbrella concept of
age-heterogamy. The label is loose because the present
discussion does not deal with nuptiality, but rather directly with fertility.
The author will prefer to link the two concepts (fertility and nuptiality) via
the less binding concept of mating. Nuptiality, for this author, serves as a
statistical proxy for mating, and fertility is the result of presumed mating. No 
statistical analysis on the basis of
marriage data nor models that incorporate marriage as an intermediate state are
offered, per se, despite the fact that marriage and two-sex models have
been co-developed and for some are synonyms. To the extent that mating or
\textit{*gamy} enter into discussion in the paragraphs that follow, it is only
via inference from observed fertility patterns or as a rhetorical aid in interpreting observed fertility patterns.

Models may incorporate patterns of heterogamy along a
broad spectrum ranging from rigid, assuming a fixed age separation between mates--
 as in \citet{cabre1997tortulos}, \citet{karmel1947relations} or \citet{akers1967measuring}, 
 typically 2 or 3 years-- to flexible, which reaches its apogee in agent-based
modeling\footnote{The author claims this not because ABMs are more
sophisticated, but because aggregate-level patterns of mating in such models are
the result of potentially simple individual-level actions, which may not
necessarily follow an easily definable functional form or distribution.}.
Intermediate model varieties include those of \citet[e.g.]{gupta1972two} or
\citet{schoen1981harmonic}, which include either fixed matrices of age
combination distributions or a standard functional forms. Many model varieties
follow a similar strategy.  

The benefit to incorporating assumptions about age \textit{combinations} of
potential mates is that one need no longer assume that the marginal
distributions of male and female fertility are constant, but rather that
they adjust in some way to the relative abundance of mates in different
age-classes and/or to competition from other ages. Models can assure
that male and female marginal rates are in agreement to the extent that 
the same numbers of births are always predicted, but shift the
compromise (if any) between male and female rates to the less well-scrutinized
arena of age-age-specific rates. Note that in this case, the model still holds something
constant: either a particular age-combination pattern, an exposure-dependant
mean function between constant sex-specific age-age-rates, or some other governing rule that finds
compromise. Marginal fertility distributions under such models-- models that
incorporate feedback into rates from changing population stocks--, as the
weighted average of age-age-specific rates, may change over time, but still 
be consistent with the condition of constancy of stable populations.

Two-sex models that contain such feedback are capable of either approaching
stability in the same sense as single-sex models- at which time marginal
distributions indeed become constant, or entering into a fixed
cycle or a cycle that gradually diminishes with time\citep{chung1994cycles}.
This author conceives of fixed cycles as another form of stability, dynamic 
stability. The present thesis does investigate 
this issue, that of feedback cycles, further, nor does it attempt to
quantify the potential affects of the exploratory analysis of age-matching that
follows. It is hoped that the present section will provide
occasion for empirically-based reflection on the appropriateness of
constant age-heterogamy assumptions in two-sex models. We will see that patterns
of age heterogamy have at time undergone sharp changes, and at other times have
held constant.


    \subsection{Heterogamy}
       \FloatBarrier
The age combination of the male and female fertility schedules from
any given year varies greatly from the distribution that would be expected if
age of mother and age of father were selected randomly according to the two
single-sex distributions.

The expected cross-classified age distribution $\textbf{E}(B(a,a'))$, which
we would observe on average if age-mixing were random, is defined as:

\begin{equation}
\label{eq:expected}
\textbf{E}\left[B_{a,a'}\right] = \frac{B_a B_{a'}}{\int _{a = \alpha} ^\beta
\int _{a' \alpha} ^\beta B_{a,a'} \; \dd a \;\dd a'}
\end{equation}
where $a$ indexes age of father and $a'$ indexes age of mother.

\begin{figure}[ht!]
        \centering  
          \caption{Observed versus expected joint age distribution of
          parents, 1970, USA}
           % figure produced in
           % /R/ObservedVsExpectedBivariateBirthDistribution.R
           \makebox[\textwidth]{\includegraphics{Figures/ObservedvsExpectedBxy}}
          %\includegraphics{Figures/ObservedvsExpectedBxy}
          \label{fig:US1970obsexp}
\end{figure}

Visual inspection of surfaces of the observed and expected birth counts in
Figure~\ref{fig:US1970obsexp} confirms they are indeed quite different: the
observed surface (left) shows a stronger homogamy-hypergamy pattern than the
expected surface (right). How similar are the
observed and association-free $B_{a,a'}$ distributions to each other? Again, we
can use a dissimilarity index, and re-apply Equation~\ref{eq:coefdiff} to the
present data, where $f_1$ is $B_{a,a'}$ and $f_2$ is $\textbf{E}(B_{a,a'})$, both scaled 
to sum to 1. $\theta$ is constrained to fall between 0 and 1, where 1 indicates that the
two distributions are separate and 0 indicates identical distributions. In 1970 USA,
$\theta$ was equal to $0.47$, a value that could be understood to stand for the
degree of residual preference. Precisely, it is the proportion of these two
distributions that is not shared. 47\% is rather high -- it means that the 1970
heterogamy pattern is far from random. If we further decide that marginal
age-distributions are not to be taken for granted, then 47\% is a lower limit to
the departure from randomness.

Note that ``age-preference'' is an imprecise label for the variety
of preferences that may actually lead to observed age-combination biases. For
instance, preferences may reflect a third variable (e.g., socioeconomic
in nature) that covaries with age, so as to give
the appearance of age preferences. Furthermore, as \citet{bergstrom1994sweden}
demonstrate, pair matching may just as easily occur as a function of individual
preferences for event (mating, marriage) timing coupled with relative
availability, which follows partly from cohort size. This is consistent with the
argument that age preferences
for mates are highly adaptive in \citet{bhrolchain2001flexibility}. Indeed,
\citet{esteve2009long} conclude that observed heterogramy patterns in Spain have been codetermined by changing
age-preferences.

Despite this ambiguity in mechanisms behind age combination patterns, one can
create a rough index of the strength of hypergamy or homogamy, based on the
matrices represented in Figure~\ref{fig:US1970obsexp}. Giving equal reproductive
bounds to the birth count matrix $B_{a,a'}$ makes a square matrix, from which we
can separate the upper and lower triangles. Here, the lower triangle, $L$,
of $B_{a,a'}$ contains births due to age-hypergamous (father's age $>$ mother's
age) parents and the upper triangle $U$ contains births due to
age-hypogamous parents. Thus, a simple measure of total hypergamy, $\widehat{H}$, 
can be taken as a ratio of the total births in $L$ versus $U$, or in shorthand 
$\frac{B_{a>a'}}{B_{a<a'}}$, excluding single-age exact homogamy on the matrix
diagonal. This is the gender asymmetry ratio from \citet{esteve2009long}.

\begin{equation}
\widehat{H} = \frac{\sum L}{\sum U} 
\end{equation}

In this case, the $\widehat{H}$ will be calculated for the observed and expected
birth matrices. US data from 1970 yields an observed $\widehat{H}$ of $7.37$
versus an expected $\textbf{E}(\widehat{H})$ of $1.75$. That the later value is
greater than 1 may be surprising, given that the $\textbf{E}(B_{a,a'})$ is purged
of association. It is due, as mentioned above, to differences in the shape
and span of male and female single-sex fertility. For reference, I
will call this ``structural'' or ``latent'' hypergamy, as opposed to the
residual, or excess hypergamy, which is the ratio of observed (total) hypergamy to
structural hypergamy. For 1970 US data, excess hypergamy is $4.21$ times higher
than structural hypergamy. While these types of values do not enter, per se, 
into any of the thus-far mentioned two-sex models, they characterize the 
population in a basic way, and aid in understanding macro-level patterns. 

Let us then calculate two times-series, one for total difference,
Figure~\ref{fig:Theta}, \footnote{95\% confidence bands are produced
    using the method from Figure~\ref{fig:ASFRdissimilarity}} and another for
    our three measures of hypergamy, Figure~\ref{fig:HypergamyStrength}.
\begin{figure}[!ht]
  \centering
    \caption{Departure from association-free joint distribution. USA,
    1969-2010 and Spain, 1975-2009.}
     % figure produced in
     % /R/ObservedVsExpectedBivariateBirthDistribution.R
     \includegraphics{Figures/TotalVariationObsvsExpectedUSES}
     \label{fig:Theta}
\end{figure}
The joint age-distributions for both countries were far from being
association-free over the duration of the period studied. Since around
1979, Spain has undergone a roughly constant approach toward what would be the
expected distribution of births, random with respect to age of
partner. Since the decline in the departure from randomness in Spain 
may also be seen as closing a gap, one could just as
easily transform the data as such and view the secular change as one of an
\textit{accelerated} approach toward randomness.\footnote{i.e., One could see
the acceleration by taking the logit of the trend in $\theta$ shown.} The US
underwent a similar approach toward randomness from 1969 until around 1985,
since which time the trend has gradually moved upward. In recent years, the
departure from randomness in the US has been considerably higher than in Spain.

Developments with respect to our rough indicators of hypergamy have been more
consistent between the two countries, both of which have undergone nearly
monotonic declines in all three hypergamy indicators, save for the US since the
mid 1990s, which has held constant. The greatest drivers of the larger downward trend
have been declines in excess hypergamy: those more imaginably a result of
behavior and preference. In both countries, excess hypergamy is greater than
latent hypergamy, though it would appear that this observation may not hold forever. The author
 speculates that we may one day see a crossover, with latent hypergamy -- that
which is more or less a product of sex-differences in fertility distributions,
and which owes in part to evolved differences in the reproductive span --
obtaining a greater proportion of total hypergamy than excess hypergamy. In essence, the
downward trend for Spain confirms the observations of \citet{esteve2009long}
about the recent decrease in age hypergamy for Spain.\footnote{One difference,
however, is that \citet{esteve2009long} examines marriage patterns, while we
examine fertility patterns, though these two are expected to covary.}

\begin{figure}[!ht]
  \centering
    % figure produced in
    % /R/ObservedVsExpectedBivariateBirthDistribution.R
    \caption{Strength of hypergamy, $\frac{B_{x>y}}{B_{x<y}}$, total, structural
    and excess. USA, 1969-2010 and Spain, 1975-2009}
    \includegraphics{Figures/StrengthHypergamy}      
    \label{fig:HypergamyStrength}
\end{figure}

These trends, of substantive interest in their own right, will also be of
interest to the designer of two-sex reproductive models that incorporate
 assumptions about age-mixing. In order to avoid overly restrictive
 assumptions about male and female marginal fertility distributions, many
 model varieties make use of information about births cross-tabulated by
 ages of both parents, assuming that some aspect of this distribution
 (rather than the single-sex marginal distributions) is constant in time.
 This assumption will be valid only to the extent that joint age patterns in
 fertility rates are not codetermined by changing population structure and
 preferences. For this reason we have illustrated some aspects of the changes 
 observed in these underlying distributions over time. 
 
Models have been
 known to make all manner of age-related assumptions, from the simplicity of 
 fixed age-matching to sophisticated combinations of age-preferences 
 interacting with availability conditions. Even the latent hypergamy 
 indicator of Figure~\ref{fig:HypergamyStrength} does not contain information 
 about how much of observed change is due to preference, say, in the age at 
 childbearing, or to relations between males and females with respect to the
 timing of childbirth. Nonetheless, it should be clear that the joint distribution
with respect to age of progenitor is far from random and often in a state of
flux. This observation is a motivation behind certain non-linear
(population-dependent) extensions to two-sex solutions, as well as for separate
preference functions. In this dissertation, we do not explore solutions that
involve separate \textit{preference} functions, but in this section we have to a
certain extent shown why this modeling choice can be attractive -- Change is at
times large enough to be worth modeling in its own right.

      
  \section{Conclusions on the magnitude of the two-sex problem}
      \FloatBarrier

This chapter aimed to introduce a new variety of population structure--
thanatological age, or age counted as time until death. The basic steps needed
to carry out the transformation presented in Equation~\ref{eq:dxredist} were
indeed available to demographers via the work of
\citet{miller2001increasing}, \citet{lee2002approach}, \citet{vaupel2009life}
and possibly others unbeknownst to this author. Indeed the perspective as a
whole is widespread in demography, this concept of age as time until death
was already in existence, and the name given to it is of known coinage. Our only
addition was to integrate over age, arriving at a wholly 
redistributed population (count, rate) vector entailing loss of age
information.\footnote{If this has indeed been done before, then this author has
not seen it, and apologies will be issued} In a sense this is more of a loss than an addition,
as we throw away information in doing so. 

It has been with fresh eyes that we have investigated fertility rates in terms
of thanatological age. Recall the words of Coale:
``One of the characteristics of demographic research is a search for empirical regularities, particularly in the
age-schedules of rates of marriage, birth, death, and
migration'' \citep{coale1996development}. We may claim to have found
empirical regularities in the restructured fertility rates, although further
work is needed to verify this for a wider range of populations and years.
Indeed, these patterns appear to be parametrically tractable -- we mark this 
possibility with a cairn and continue onward. There is in any case no need to smooth data that
have been redistributed in this fashion! However, if it turns out that the
remaining-years pattern to fertility is more regular, and hence more
predicatable, than the age-pattern to fertility, one may find a model curve to
be of use. We have not given an explicit account of the regularity of
$e_y$-structured fertility over time, though a glance at the surfaces in
figures~\ref{fig:exSFRsurfUS}~and~\ref{fig:exSFRsurfES} invites speculation in
this direction.

We have also found that the cross-classified remaining-years births distribution
is regular to the extent that it greatly resembles its own association-free
distribution. Either this observation is indeed so, or it is the result of the
overly permissive assumption of homogeneity with respect to mortality in
procreating pair formation. Namely, if there is noticeable selection with
respect to mortality fitness in the mate search, which is rather imaginable
\citep{gangestad1993pathogen, roberts2008good}, the patterns seen in
Figure~\ref{fig:US1970obsexpex} will not reflect the true distribution. The
question is then how far off our homogeneity assumption is. This could be
answered by means of linked register data, as has often been done for countries
such as Sweden and Denmark. We may already surmise that procreating individuals
are on average more fit than non-procreating individuals. To the extent that
marriage is a proxy for mating, we already know that married males have a
mortality advantage over unmarried males, although
there are likely intervening factors \citep[see
e.g.,][]{rogers1995marriage, waite1995does}, and we do not know for sure whether
this is due to causation or selection \citep[see e.g.,][]{goldman1993marriage}.
For the present use, we do not necessarily care whether the differential might be due to
selection or causality. All this is to say that we ought to take in the
observations of cross-classified fertility with a grain of salt. Furthermore, as
is pointed out in the text, selection may disproportionately affect the tails 
in the lowest remaining years classes.\footnote{This possibility would be just
the opposite in high maternal mortality settings, which does not affect
our two populations.} The best we can do in this instance is imagine the
direction of bias, as has we have tried to make clear.

In general, we have seen that period fertility indices calculated using
remaining-years classified data track well to those calculated age-classified
data, but also that they return higher estimates of period fertility. At
first glance, one would expect the direction of difference to be negative 
rather than positive because we took the simplest possible course of
redistributing population counts (or exposures) from all age classes rather than
only from age classes in ages typically identified as reproductive ages. The
outcome is positive because the age-distributions of marginal birth counts and
population counts are different, but redistributed by the same lifetable death
distribution. We ought not expect differing initial distributions to wind up
proportionally in the same remaining years classes. Hence, the sum of these
ratios (events to exposures) over remaining years is also different. TFR, for
instance, is higher for remaining-years structured data than the age-equivalent
measure because on average there is less exposure per birth in the newly
redistributed data.

One may question whether there is any sense in allowing non-reproductive ages
into remaining-years exposures, and to this we have two responses. The first
comes by manner of analogy to age-specific fertility distributions, which
have tails that are longer than most people feel comfortable imagining, 
both for males and females. For males and
females there are so-called central ages of reproduction and there are
less-common ages of reproduction. Even for ages typically measured, say
45-49 for females, not all persons captured in these exposures are truly
exposed to the risk of fertility, and everyone knows this. Indeed a minority
of females in this age group are truly at risk of fertility. This does not imply
that the rate calculated is invalid, but rather that the rates are both low and
potentially subject to further decomposition. This is one reason why a large
portion of fertility studies deal with the measurement of the proximate determinants of fertility-- these
factors differentiate exposure for purposes of calculating more specific rates.
Age-specific rates are valid without such differentiation, as they help purge
our measurement of distortions from a certain degree of population structure. 

As \citet{stolnitz1949recent} so eloquently describe, demographic rates are
never fully purged of population structure. We may nonetheless \textit{de}structure in
any of myriad ways, and remaining-years structure is the way that we have shed
light upon in this dissertation. No claim is made about whether chronological or
thanatological age are the more efficient classifying variable for population
data, nor that one is more pure than the other. In sum, we have in this
dissertation provided yet another alternative structuring variable, and if one
questions the validity of a rate calculated on the basis of blended age groups,
one may ask: 1) how many persons are expected to die in $y$ years and 2) how
many births this year were to persons expected to die in $y$ years, and there we
have everything needed to calculate a rate. This is what we have done. Further
refinements are possible and are worthy of exploration.

The second response to the potential criticism about overly permissive
heterogeneity within remaining years classes for purposes of producing
fertility rates is that these rates nonetheless appear to obey a
certain degree of empirical regularity. The range of shapes possible for
fertility rates by remaining life expectancy, at least absent a mortality
crisis, is narrow. This we confidently claim, but the claim is in need of
further empirical support beyond this dissertation. The same observation may be made of
the underlying population structure by remaining years of life-- the range of 
shapes that we have observed this structure to obtain in Western countries, 
excluding mortality crises, is
narrow. These observations compel one to conclude that
remaining-years-specific fertility rates ($e$SFR) describe a substantive
underlying phenomena. If age-heterogeneity in remaining years classes 
were problematic in this sense, we would expect more erratic patterns in
$e$SFR over time and between populations, but this is not so. Instead, $e$SFR
invites comparison between populations and over time, and such comparisons are
decomposible into mortality and fertility components. We have not
taken this latter exploratory step, but this remains an exciting question, since
a stable $e$SFR pattern may hold over time even as fertility age patterns and
mortality levels change.

Most empirical findings arising from this thesis are based on data only
from the years 1969-2009 for the United States and 1975-2009 for
Spain\footnote{The primary exception is the relative \textit{stability} of
observed remaining-years-structured populations versus age-structured
populations, which we verified using the entire HMD.}. Evidently, any
novel findings produced therefrom, especially those dealing with fertility, 
are in need of verification from a wider array of
populations, so as to be placed into context. This is to say that there is ample opportunity to test
and refine the hypotheses produced here. Certainly $e$SFR will move over time,
but we suppose that this movement is meaningful and will aid our understanding
of population processes.

Net of the potential criticism of age-heterogeneity in remaining-years classes,
there are reasons to suspect that fertility timing and levels might also be
a function of remaining years of life. Although one's own age at death is
unknown, individuals do have a \textit{sense} of their own remaining life
expectancies, and this may condition fertility. This is not to say that people 
predict their own lifespans without bias. \citet{payne2013life}, for example,
concludes that the framing of the very questions used to asses subjective
remaining life expectancy in survey respondents can bias results greatly. This
source of bias has more to do with eliciting respondents' beliefs and
perceptions, although there are also biases in subjective predictions that vary
over age. The important thing is that people's subjective expectations of
remaining life \textit{correlate} positively with real remaining life
expectancies, and much evidence has been produced that would support
this \citep{hurd1995evaluation, mirowsky1999subjective, hurd2002predictive,
perozek2008using, delavande2011differential, post2012longevity}. These authors
tend to explain individuals' predictive power in terms of awareness of
their own particular genetic and environmental situation. Most people
understand the general mortality increase over age, and are able to
weigh these factors out to produce a rough idea of future death 
probabilities in certain age intervals. These findings have largely come from
health and retirement surveys of older persons beyond reproductive ages. Of
course, it is difficult to measure how well persons in \textit{reproductive}
ages are at predicting their own ages at death because panels would need to
run over a very long range of years in order to asses the accuracy of
predictions of the relatively young. 

Another particular variety of finding that lends support to our case are
a couple studies of risky behavior in youth. \citet{borowsky2009health} show
that risky behavior in teenagers increases with high perceived risks of early death, and these
behaviors include risky sexual behavior. \citet{wilson1997life} conclude that
high real mortality risks for youth due to \textit{perceived} factors such as 
accidents, suicide and homicide, predict reductions in the age of fertility, which at
least on the surface, is consistent with the idea of $e$SFR actually describing 
some sort of underlying pattern. 

None of these studies provide
direct support of our hypothesis that we ought expect that fertility behavior is
somehow a function of remaining life expectancy, but they do help our case.
The kind of data that would help shed direct light on this question would be a
panel study that includes questions about expected probabilities of survival
until (or death by) certain future ages, as well as the usual battery of 
fertility questions. This could be incorporated, for example, in future waves of
longitudinal youth surveys. To suit our ends directly, this survey would follow
up with actual ages of death. Evidently, much patience will be required to reap
any results in this direction, and there is much uncertainty at this preliminary
stage (we are only imagining such a survey at the moment) about whether results
would provide an estimate of true remaining-years fertility curves, and whether
foresight of one's own mortality moves this curve. Removing subjective
remaining life expectancy from our demands, one may reconstruct cohort $e$SFR on 
the basis of long-running population registers from countries such as Denmark or
Sweden.

Absent a well-rounded base of psychological, sociological and demographic
evidence in support of the belief that fertility patterns ought be a function of
remaining years of life (as well as a function of age), we rely on our own
set of evidence that simply show these patterns to be stable. Stable demographic
patterns are desirable, as they are useful for predicting. We encourage
the exploration of the predictive power of remaining-years-specific
demographic rates, just as economists have been keen to look for patterns in
retirement saving and investment as a function of perceived remaining years of
life.

We completed the chapter with some rather indefinite speculation about the
potential use of our $v_y$, Fisher's reproductive value rethought to correspond
to the thanatological perspective, in cognate fields of demography, or
evolutionary demography itself. One could just as well reframe $v_y$ in terms of
inter-age transfers rather than reproduction, as per \citet{lee2003rethinking},
although the perspective change will still pertain.

The purpose of this chapter was to define and explore the vital rates to be used
in the remained of this dissertation. We will proceed by defining a model of
population renovation akin to Lotka's renewal equation. From this model we will
extract and explore the intrinsic growth rate and some other stable
parameters that belong to the thanatological perspective. 

\FloatBarrier




% chapter is not really composed
\chapter{Modelling approaches to the two-sex problem}
   \FloatBarrier
Recall the observations of \citet{sanderson2005average}, more directly
relating to the popuation pyramid. These authors note that despite ageing in a
population, the mean \textit{remaining} years to be lived may increase. This is due 
to improvements in mortality offsetting (or more than offsetting) increases in
the mean age of a population. In general, a population looked at from the
perspective of life expectancy looks different from, behaves differently from,
and yields complimentary information to one looked at from the canonical
perspective of age. This observation is the point of departure for the
linear and non-linear two-sex models that I will introduce in this dissertation.
However, such steps are large, and require an involved demonstration of some
key differences between age clsasified and \textit{remaining years}-classified,
henceforth $e_x$-classified demographic data. Indeed a whole new one-sex
model must first be developed prior to re-delving into its two-sex extensions. 
I will first present a method
to exactly redistribute population counts (or exposures) according to period 
remaining years of life, rather than according to age per se. A reexamination 
of recent fertility patterns according to remaining years of life will follow.


  \section{Primary axioms}
     \FloatBarrier
 \label{sec:axioms}
It has been pointed out that the ideal functional form of a two-sex solution
cannot be empirically determined (Das Gupta, Schoen, Keyfitz, others). This is
because fertility is always undergoing secular changes, to the effect that
one cannot simply calibrate an ideal mean function (if a mean function were the
correct choice) net of outright both-sex fertility change. This we observe above
all with the Spanish data used in this dissertation: from 1975 until the mid
1990s fertility levels dropped so rapidly that in most cases the year $t+1$
birth count fell below that which would have been predicted by either of the
year $t$ male or female rates-- despite there having been a wide the gap between
male and female total fertility rates in those years. 

Even in less
extreme situations, where the year $t+1$ birth count is intermediate to that which would
have been predicted by the male and female year $t$ rates, one is unable to
separate the effects of relative changes in male versus female exposure from
simple changes in rates. That is to say, if there is some push and pull between
male and female rates, this cannot be measured if rates on the whole are either
rising or falling-- just as it is difficult to measure the net rising and
falling of rates when there is feedback and separation between male and female
rates. Even if one had a very large amount of data conformable to this problem
and an appropriate statistical technique so as to mete out these differences 
and estimate a function that could separate and capture the effects of our
imagined push and pull between male and female
rates\footnote{\citet{alho2000competing} come close to this ideal.}, it would be easy 
to suppose that this ideal function
might itself change according to certain conditions or certain periods.

This empirical obstacle has led demographers to devise a set of axioms,
necessary or desired characteristics, that the ideal two-sex fertility
(or marriage) function should abide by in order that it conform with our
expectations. Here we will enumerate all such axioms located in the literature 
before briefly discussing them in turn. Here, $M()$ is any function that
determines the both-sex rate using male, $P^m$, and female, $P^f$, exposures as
inputs. These exposures may be classified by some other variable, such as age, but
subscripts are ignored here unless pertinent.

\begin{description}
  \item[Availability:] $M(P^m, P^f) = 0$ if $P^m = 0$ or $P^f = 0$. Members of
  both sexes must be present in order for there to be a non-zero positive rate.
  \item[Homogeneity:] $kM(P^m, P^f) = M(kP^m, kP^f)$. Equal
  changes in the supply of males and females must lead to an equal change in the
  number of births (marriages).
  \item[Monotonicity:] for $k > 1$, $M(kP^m, P^f) \ge M(P^m, P^f)$ (and vice
  versa). If the supply of one sex increases while the other sex is held constant, the number of
  births (marriages) cannot decrease.
  \item[Symmetry:] for $P^m = P^f$, $M(kP^m, P^f) = M(P^m, kP^f)$. 
  \item[Competition:] if exposure in age $x$ for males is increased by some
  factor, but all other male and female ages are held constant, monotonicity
  applies to age $x$ of males, but rates for male ages $<x$ or $>x$ may only
  decrease or say the same. 
  \item[Subsitution:] The size of competition effects varies directly with
  age-proximity to $x$ among males. For instance, males of age 24 are closer
  substitutes for males of age 25 than are males of age 20.
  \item[Bracketing:] $M(P^m, P^f) > min(F^m, F^f)$ and $M(P^m, P^f) < max(F^m,
  F^f)$. The both-sex rate must be intermediate to the single-sex rates.
  \item[Proportionality in the extreme:] in situations of very extreme sex-ratio
  imbalance, changes in the amount of the minority sex should be reflected proportionately
  in the two-sex rate.
\end{description}
 
These axioms will now be briefly reflected upon in turn.

\paragraph{Availability:} This is the most elemental axiom, as it essentially
states a truism: If one sex is absent, there can be no reproduction in a species
that reproduces sexually. For the sake of philosophical completeness, we state
the following: 1) Assisted reproduction requires both sexes, so this is no
retort; 2) At present, technology that would negate this axiom, human
parthenogenesis, is not fully developed, although ther have been recent
advances\citep{revazova2007patient}. If and when technology would permit asexual
human reproduction, there will be legal hurdles, costs, and apoption lag. 
That is to say, potential anecdotes that would negate this axiom
will in any case not affect fertility rates in a significant way within the time
horizons that demographers currently project. This is not a tongue-in-cheek
observation, as technology in general is known to affect fertility in myriad
ways. For instance, in vitro fertilization and other forms of assisted
reproduction have had noticeable effects in the fertility and sex ratio at birth from particular
age groups.

\paragraph{Homogeneity:} This author finds the axiom of homogeneity to be on the
whole harmless, but not necessarily true. Homogeneity essentially states that
there are no scaling effects. It is easy to imagine that population size will
constrain or determine much of what happens within populations. This is
especially so when we think in terms of social organization, contact
opportunities and the countless other structural factors that may affect the
practice of mating and by extension fertility. Population size has been
given more attention in non-human ecology \citep{donalson1999population} than in
human demography, where considerations of population size have been primarily framed in terms of
carrying-capacities \citep[see e.g.][]{cohen1995human,hopfenberg2003human}. This
author is only aware of scaling in demographic process when studied as complex systems via
agent based modelling (ABM) \citep[e.g.][]{bruch2010scaling}. While ABMs have
been used to studying fertility and marriage \citep{billari2002wedding},
indirect scaling effects in such models have not been explicitly studied, not
have scaling effects been in introduced explicitly.

\paragraph{Monotonicity:} This may seem intuitive, but if viewed from a
sociological standpoint, it seems plausible that increased competition could actually lead to a
decrease in total births (marriages) via different mechansisms that we will
briefly hash out. Imagine a more complex model wherein individuals must
apportion time / effort / resources between mate search costs and competition. 
In the case of an increase in males while holding females
constant, increased competition between males in mate selection could
 scale non-linearly to-- and offset-- the standard predicted increase in matings
 that would result from increased male pressure on the market. In a different
 scenario, females faced with abundant potential mates may actually decrease their 
 search efforts and postpone the mate search until a later time,
 thereby acting to supress rates. Were this later effect present in the model,
 the effect of increases in one sex would be ambiguous, as it would depend on
 the relative forces of male pressure and female deprioritization. In yet
 another model scenario, a proportion of males faced with increased competition may 
 indeed cease to compete, and remove themselves from the market, thereby 
 decreasing pressure from the side of abundant males. Other similar effects may
 be dreamed up wherein the results of an increase in one sex only could be
 coplex and counterintuitive. 
 
 None of these complex model scenarios are particularly amenable to inclusion in
 a practical analytic model of mating / marriage / fertility markets. However,
 in indicating such potential countervailing forces-- all reasonable in the mind
 of this author-- one may at least question the necessity of holding
 monotonicity as \textit{axiomatic} in the sense of a functional necessity- a
 criterion by which the adequacy of a model may be judged. 

\paragraph{Symmetry:} It appears that symmetry, treated as an axiom, is also
inappropriate. Males and females differ not only with respect to vital rates,
but with respect to mate preferences and behavior \citep{buss1989sex}. There is
also evidence for variation in these differences by group size
\citep{fisman2006gender}, which plays into the previous axiom of monotonicity.
Clearly, if males and females have different preferences and also react
differently to differences in group size, we should expect different outcomes
from hypothetical sex-complementary compositional changes in the mating market.
For this reason we may also conclude that symmetry, though likely to be a
characteristic of the functional form assigned to the male-female
dependant fertility (marriage) function, ought not be given the status of an
axiomatic requirement for a good and proper model. That the functional forms
often used for marriage and fertlity often were symmetric with respect to the
sexes need not be drawback, but we ought not grant this characteristic post hoc
status as an axiom.

\paragraph{Competition:} It seems reasonable that, holding mate supply constant,
increases in matings in age $x$ either decrease or have no effect on ages
close to $x$ of the same sex. Some two-sex models have accounted for this axiom
\citep{henry1972nuptiality, mc1975models, choo2006estimating}, sometimes via
explicit preference functions \citep{parlett1972can, pollard1993interaction} but
many have not. These models are considerably more complex to implement than the 
alternatives. It is unclear to this author whether this axiom should be treated
as a requirement or a desirable property. 

\paragraph{Substitution:} In the case of inter-age (or inter-group) competition
for mates, it is intuitive that, since age can be thought of as
continuous, competition effects should vary inversely in magnitude as a function
of distance to the age that experienced a sudden change in effective
population. In the case that explicit preference functions are used, this axiom
is already dealt with, and \citet{choo2006estimating} also has this
characteristic.

\paragraph{Bracketing:} The interpretation of this axiom depends on context.
In the first instance, it states that the two-sex instrinsic growth rate, $r$,
must fall between the male and female single-sex intrinsic growth rates, $r^m$, $r^f$, 
respectively. Many authors have treated this axiom as a requirement
\citep{pollard1948measurement,
coale1972growth, gupta1976interactive, mitra1978derivation}, others have
argued otherwise \citep{gupta1973,schoen1981harmonic}, and indeed it has even
been proven an unreasonable condition \citep{yellin1977comparison}. This author
agrees that the single-sex growth parameters will not serve as two-sex bounds
because they are calculated in unreasonable isolation, namely, each constrained
by its own sex-specific fertility rates and without interaction between ages of
each sex. That is to say, in isolation the single-sex models may behave
strangely and not bound the true trajectory of the total population.

A second domain of bracketing could be in terms of the total births predicted by
males and females for year $t+1$ using the ASFR and sex-specific exposures. In
this case, the main difference is that the offspring of each sex is of both
sexes. In this case, bracketing appears a less troublesome condition, as we
essentially remove fertility sex-ratio constraints from the boundary
predictions. Absent secular change in birth rates or the age-pattern of fertility, 
we would expect one sex to ovestimate and the other to underetimate the birth count to be observed in
future years. 

\paragraph{Proportionality in the extreme:} In other words, at some point along
the continuum of potential sex ratios, the minority sex should experience
\textit{saturation}, in the sense that further increases in the majority will
not result in increased matings. In this same scenario, one may imagine that,
while still within the same extreme order of sex ratio magnitude, a unit
increase in effective population of the minority sex will lead to a unit
increase in predicted births (marriages). It is doubtful that this
situation would ever arise in a real projective scenario 

    
  \section{Models that maintain single-sex fertility}
  
     \subsection{Pollard 1948}
       \label{sec:pollardage}
\paragraph{Components to the model: }
Define two fertility functions, $F_{a'}^{F-M}$ and $F_a^{M-F}$, where the first
superscript indicates the sex of progenitor, the second superscript indicates
the sex of birth, $a$ indexes male age and $a'$ indexes female age. In
other words, the female births are determined by male rates and vice versa.
$p_a$ and $p_{a'}$ are the male and female probabilities of surviving to age 
$a$, $a'$. It will be convenient to combine these two items into \textit{net}
opposite-sex offspring functions:
\begin{align}
m_a &= p_aF_a^{M-F}\\
m_{a'} &= p_{a'}F_{a'}^{F-M}
\end{align}
Note that these function names are the same as in standard single-sex models,
but that sex of progenitor and offspring have been juxtaposed.
\paragraph{The renewal function(s): }
Given $m_a$ and $m_{a'}$, the renewal function to determine births in yeat $t$
is given by
\begin{align}
B^F &= \int_0^\infty B^M(t-a)m_a \dd a\\
B^M &= \int_0^\infty B^F(t-a)m_{a'} \dd a'
\end{align}
which can be converted such that sex of offspring and reference
progenitor are the same by moving back one generation and combining net
offspring functions:
\begin{align}
B^F &= \int_0^\infty \int_0^\infty B^F_{t-a-a'} m_am_{a'}\dd a \dd a'\\
B^M &= \int_0^\infty \int_0^\infty B^M_{t-a-a'} m_am_{a'}\dd a \dd a'
\end{align}
These later two functions combine into a single convenient renewal function
\begin{equation}
B^T = \int_0^\infty \int_0^\infty B_{t-a-a'}m_am_{a'}\dd a \dd a'
\end{equation}
Conveniently, all of these five renewal functions will converge to the same
ultimate intrinsic growth rate, $r$, which is the real root of the following equation: 
\begin{equation}
1 = \int_0^\infty \int_0^\infty e^{-(a+a')r}m_am_{a'}\dd a \dd a'
\end{equation}

In Pollard's model the sex ratio at birth and overall sex ratio
of the population are regulated by criss-crossed sex-specific fertility. Pollard
proves that the resulting estimate of $r$ will be intermediate to the male and female
single-sex intrinsic growth rates -- assuming a constant sex ratio at birth --
and the function remains linear. Further, the function has the advantage of being 
relatively easy to understand.
This author finds the method clever, but it has been rather ignored in the
literature because authors typically find the assumption of criss-crossed
fertility unrealistic. This seems like a fair criticism if the goal is to
faithfully reflect fertility dynamics. It would seem that Pollard's goal 
was to approximate the value of the two-sex growth rate while maintaining
a small set of model properties (bracketing, linearity, homogeneity
simplicity), but not to approximate true reproductive dynamics. Another drawback
is that the sex ratio at birth, if not assumed to be constant, depends on the
initial conditions. 

Given an optimized value of $r$, one can retrieve the stable age structure and
sex ratio at birth\footnote{Some advice is given in
\citet{pollard1948measurement} for arriving at the stable sex ratio, but it
would be easier to either just assume a sex ratio at birth or else iterate
forward to stability and derive it empirically.} and other stable parameters of
interest. Empirical results of Pollard's method will be compared later with
those of \citet{mitra1978derivation}.

     \subsection{Mitra 1978}
       \label{sec:mitraage}
\citet{mitra1978derivation}, like Pollard, also limited two-sex models to being
based on the building blocks of single-sex fertility. In this case, single-sex
fertility is conceived of as in the single-sex models, using male-male and female-female
fertility rates. Mitra aimed to produce a consistent
method to derive a two-sex intrinsic growth rate, $r$. Consistent here means 
that 1) a constant SRB is maintained in and along the trajectory to stability, 2) 
the essential \textit{shape} of fertility rates is held constant along the path 
to stability and 3) the stable $r$ is guaranteed to be bracketted by $r^m$ and $r^f$.

\paragraph{Components to the model: }
The model requires a fixed sex ratio at birth, $S$, although this need only
enter into formulas explicitly if one implements Mitra's formulas iteratively,
which we will not present here. The method proposed by \citet{mitra1978derivation} works
by assigning complementary scalar (uniform over age) weights, $\frac{1}{v_0}$ 
and $\frac{1}{1 - v_0}$, to male and female single-sex fertility rates, $F_a
^{M-M}$ and $F_{a'} ^{F-F}$. As elsewhere, the model requires the male
and female age-specific survival functions, $p_a^m$ and
$p_{a'}^f$, respectively. The initially weighted fertility rates are held
constant and placed into a unified two-sex Lotka unity equation in order to
determine $r$:

\begin{equation}
\label{eq:mitralotka}
1 = \int _{a=0} ^\infty e^{-r a} \frac{F_a ^{M-M}}{v_0} p_a^m \dd a + \int
_{a'=0} ^\infty \frac{F_{a'} ^{F-F}}{1 - v_0} p_{a'}^f \dd a'
\end{equation}
Upon determining the combined-sex growth rate, $r$, however, one must readjust
the weights, $v_\infty$, to correct the stable sex ratio at birth:
\begin{equation}
\label{eq:mitrastablev}
v_\infty = \int _0^\infty e^{-ra} \frac{F_a ^{M-M}}{v_0} p_a^m \dd a
\end{equation}

\paragraph{Estimating Mitra's $r$: } One can quickly converge upon a solution
to Equation~\eqref{eq:mitralotka} by modifying the method proposed in
\citet{coale1957new}\footnote{\citet{mitra1978derivation} alludes to this, but
does not get into any specifics.}:
\begin{enumerate}
  \item Calculate a trial estimate of $r$,
$\widehat{r}$ and a trial two-sex mean generation length
$\widehat{T}$. For trial values, one can use simple
assumptions, such as the arithmetic means of the single-sex Lotka parameters.
  \item Plug the trial $\widehat{r}^{(1)}$ into Equation~\eqref{eq:mitralotka}
  to calculate the residual, $\delta ^{(1)}$.
  \item Improve the estimate of $r^{i+1}$ using:
  \begin{equation}
  \widehat{r}^{(i+1)} = \widehat{r}^{(i)} + \frac{\delta^{(i)}}{\widehat{T} -
\frac{\delta ^{(i)}}{\widehat{r}^{(i)} }}
  \end{equation}
  \item Use the new improved estimate, $r^{(i+1)}$, to calculate a new residual,
  and repeat steps 2 and 3 until $\delta^{(i)}$ vanishes to zero.
\end{enumerate}

This method converges quickly and with greater precision than most
generic optimizers. Once $r$ is found, one takes the extra step of calculating
the stable weights, $v_\infty$ using Equation~\eqref{eq:mitrastablev}.

\paragraph{Summary of the method: } A characteristic of Mitra's model design is
that a given starting weight, $v_0$, will always result in a single, stable
$v_\infty$. Mitra's two-sex growth rate,
$r$, is unique for but depends upon the starting weights, $v_0$, and thus is not 
in general unique to a given set of vital rates, which is a drawback. Mitra
suggests that a good choice for $v_0$ would be the value that minimizes the 
departure from constancy for weighted single-sex fertility rates. This is an 
attractive choice because constant rates are of course the
basis of stability. Once a population attains stability, weights, and therefore 
rates, are constant. In practice, one chooses the $v_0$ that minimizes the sum
of the age-specific squared residuals (for males and females) between $F_a$ and 
$F_a \times \tfrac{v_0}{v^\ast}$.

\paragraph{Mitra's weights in the initial versus stable states: } If minimizing
the difference between starting and stable rates is the criterion for choosing $v_0$, 
then there is indeed a single stable $r$ that
corresponds to a given set of vital rates. We calculate Mitra's starting and
stable weights for the US and Spanish data and display them
in Figure~\ref{fig:Mitra1978v0vstar}.

\begin{figure}[ht!]
        \centering  
          \caption{Initial ($v_0$) and stable ($v$) weights according to the OLS
          criterion. US, 1969-2009, and Spain, 1975-2009 
          \citet{mitra1978derivation}}
           \quad
           % /R/Mitra1978.R
           \includegraphics{Figures/Mitra1978v0vstar}
          \label{fig:Mitra1978v0vstar}
\end{figure}

For Spain and the US throughout the period studied in this dissertation, both
$v_0$ and $v^\ast$ fell in the range $(.48,.6)$. $v_0$ was always
close to $.5$, entailing nearly equal weight for male and female rates.
The stable $v^\ast$ was consistently higher than $v_0$ and always higher than
$.5$, implying greater weights for males than females in stability. When $v > .5$, male rates weight
more than female rates, which was typically the case here, especially in the
limit, although this declined over the decades shown here. It is
tempting to interpret this result as contrary to the notion of female dominance,
which would intuit greater influence of females on overall fertility than
males. However, the interpretation of $v$ is unclear, and cannot necessarily in
this case be understood as direct evidence of male-leaning dominance.
\citet{mitra1978derivation} provides no guidance to interpret $v_0$, $v_\infty$,
less so a demographic meaning. 
\FloatBarrier
\paragraph{Critique of Mitra, 1978: } Initial and stable weights are attractive
for purposes of the OLS criterion and their potential for demographic interpretation, which has in any
case not been elaborated. This author considers this particular variety of
weights to be a superfluous byproduct of the model specification. Namely, $v_0$
and $v_\infty$ are needed only to maintain the SRB, and the SRB is
only problematic due to use of the single-sex fertility rates. Of course, males
are not exclusively responsible for the birth of boys and females are not 
responsible for the birth of girls -- the same critique applied to Pollard's
fertility rates applies here too. If the model were simply changed to allow for
the both-sex fertility of males and females, one could forego the
intricacies of balancing fertility and the SRB. As given, results are
sensitive to changes in the value of the SRB, and so this admits
unwelcome instability into the model. Further, since SRB varies by age,
changes in the age-sex-structure of the population ought to result in changes in
SRB, but Mitra's model forces a constant SRB. This decision reverberates
throughout the model specification.

If the model were to include the full fertility schedules of each sex (i.e.
births of both sexes to parents of each sex), then weights would not need to
vary between the initial and stable states. In this case, weights would only serve as a pure indicator of
dominance, as in \citet{goodman1967age}. The drawback, in this case would be
that the demographer is left with no endogenous criterion for choosing weights, save
perhaps for the relative size of male and female exposures \citep{mitra1976effect}. 
Furthermore, in either specification, males and females are treated on the same 
age scale, wherein the reproductive values of for example, 20-year old males and
females are directly combined to a single sum -- i.e., the model lacks age-sex
interactions and fertility schedules are rigid. 

We compare the results of this method with those from Pollard in
the following Section.

\FloatBarrier
       
     \subsection{Comparison of Mitra (1978) and Pollard (1948)}
       
Results from the models adhering to the notion of single-sex fertility,
cris-crossed or not, inevitably suffer from 

\begin{figure}[ht!]
        \centering  
          \caption{$r$ from Pollard (1948), Mitra (1978; OLS criterion),
          and single sex intrinsic growth rates. US, 1969-2009, and Spain, 1975-2009}
           % /R/Mitra1978.R
           \includegraphics{Figures/PollardMitrar}
          \label{fig:Mitra1978v0vstar}
\end{figure}













       
  \section{Models that use cross-classified fertility information}
     \subsection{Das Gupta 1978}
       \FloatBarrier
\citet{gupta1978alternative} states\footnote{and this fits nicely into the flow
of our own presentation.} ``The lesson we learn from the above is that our
starting point must not be the formulation of two equations, one for $B_M(t)$ and another for
$B_F(t)$, but of a single equation for $B(t)$ with the help of a bisexual
fertility function that can explain the occurrence of births of type $(a,a')$ in
terms of the availability of both males and females''.

Das Gupta introduced a series of proposals for two-sex reproduction models
throughout the decade of the 1970s \citep{gupta1972two, gupta1973us,
gupta1976interactive, gupta1978alternative}, of which we will present the last
one. To summarize how the model works, imagine we would like to determine a
unified two-sex fertility rate, $F_{a,a'}$. Here it is clear
what to put in the numerator, as births can be tabulated by the ages of both parents.
 We thus work to define the idea of two-sex exposure for each age-combination. Das Gupta's
suggestion is derive a series of probability density functions that apply to
each age of potential mother and each age of potential father from information
contained in the matrix of observed births. Define these age-specific pdfs for
males, $U_{a,a'}$, and for females, $V_{a,a'}$ as:

\begin{align}
U_{a,a'} &= \frac{B_{a,a'}}{\int B_{a,a'} \dd a'}\\
V_{a,a'} &= \frac{B_{a,a'}}{\int B_{a,a'} \dd a}
\end{align}
In discrete terms, one establishes two matrices, arranged according to our
standard in this dissertation with male age in rows and female age over columns.
The row marginal sums for $U_{a,a'}$ all equal 1 and the column marginal sums of
$V_{a,a'}$ all equal 1\footnote{both with the exception of ages with no
fertility, which are left as 0 if undefined.}. One then calculates Das Gupta's
approximation of bisexual exposure, $E_{a,a'}$, by redistributing male and
female age-specific exposure and summing for each combination of age:
\begin{equation}
E_{a,a'} = U_{a,a'}E_a + V_{a,a'}E_{a'}
\end{equation}
which is then used as the denominator to calculate $F_{a,a'}$:
\begin{equation}
F_{a,a'} = \frac{B_{a,a'}}{E_{a,a'}}
\end{equation}
which is assumed constant in the stable model. As elsewhere, define the
sex-specific radix-1 survival functions, $p_a$, and $p_{a'}$, and a sex ratio
at birth, $S$, from which we determine the proportion male at
birth, $\varsigma=\frac{S}{1+S}$. Then Das Gupta's two-sex renewal
function becomes:
\begin{equation}
B(t) = \int_{a=0}^\infty \int_{a'=0}^\infty \Big( \varsigma U_{a,a'} B(t-a) p_a
+ (1-\varsigma)V_{a,a'}B(t-a) p_{a'}\Big)F_{a,a'} \dd a \dd a'
\end{equation}
If $U_{a,a'}$, $V_{a,a'}$, $\varsigma$ and $F_{a,a'}$ are assumed constant, then
as $t$ approaches infinity, the intrinsic rate of growth, $r$, will stabilize.
$r$ is estimated from the Lotka-type unit equation:
\begin{equation}
\label{eq:Guptaeq}
1 = \int_{a=0}^\infty \int_{a'=0}^\infty \Big( \varsigma U_{a,a'} e^{-ra} p_a
+ (1-\varsigma)V_{a,a'}e^{-ra'} p_{a'}\Big)F_{a,a'} \dd a \dd a'
\end{equation}
\paragraph{Estimating Das Gupta's $r$: } The value of $r$ that makes
Equation~\eqref{eq:Guptaeq} hold can be either optimized or found using an iterative 
process similar to that proposed by \citet{coale1957new}. We explain the latter method, as it
converges very fast:

\begin{enumerate}
  \item establish a starting value for $r$,
$r^{(0)}$ and a trial two-sex mean generation length
$\widehat{T}$. For both values, one may use simple
assumptions, such as the arithmetic means of the single sex Lotka parameters.
  \item Plug the trial $\widehat{r}^{(0)}$ into Equation~\eqref{eq:Guptaeq}
  to calculate a residual, $\delta ^{(1)}$.
  \item Improve the estimate of $r^{i+1}$ using:
  \begin{equation}
  \widehat{r}^{(i+1)} = \widehat{r}^{(i)} + \frac{\delta^{(i)}}{\widehat{T} -
\frac{\delta ^{(i)}}{\widehat{r}^{(i)} }}
  \end{equation}
  \item Use the new improved estimate, $r^{(i+1)}$ to calculate a new residual,
  and repeat steps 2 and 3 until $\delta^{(i)}$ vanishes to zero.
\end{enumerate}

\paragraph{Summary of the method: } \citet{gupta1978alternative} assumes that exposure to risk of
 age $a$ males is not evenly distributed over each age of potential female mate-
 i.e. that it is not random\footnote{As opposed to an earlier rendition of
 this method \citep{gupta1972two}}. Rather, the exposure to risk is partitioned
 over ages of potential mates according to the distribution present in a given 
 cross-classified birth matrix. In partitioning exposure in this way for each
 age of male and female, the cross-classified male and female risks are additive, and
 form the total exposure to risk. 
 
 It is attractive that this total exposure to
 risk sums to the total male and female exposures, but it is unclear whether the
 distribution should be based on cross-classified birth tabulations, which will
 likely be laden with structural artifacts. In other words, as relatively large
 cohorts pass through reproductive ages, they will tend to produce more births
 than neighboring cohorts-- even if the large cohorts also suffer lower rates.
 This will cause a spike along a particular age margin in the birth matrix,
 usually for both males and females of the larger cohort. This birth spike will
 be present in the exposure redistribution matrices, $U_{a,a'}$ and
 $V_{a,a'}$, and it will also remain evident in fertility rates, $F_{a,a'}$.
 This is problematic even in the first iteration of a projection, as the
 hypothetical large cohort will have moved up one age. This artifact will
become a characteristic of the stable population even as abrupt cohort size
differences vanish with time. The initial structural artifacts in the supposed
constant parameters thus enter into both exposures and rates. 

To a certain extent the present model also removes much of the anomolies that
result from single-sex fertility assumptions-- $m_{a,a'}$ is the ferility of
both sexes, and $\varsigma$ enters into Equation~\eqref{eq:Guptaeq} as a radix
weight for the male and female population structures. There is no dominance
parameter in this model, per se. Since the sex ratio is not use for the
splitting of births in the estimation of $r$, but rather for the weighting of exposure,
the model is not restricted to producing an estimate of the intrinsic growth 
rate that is bracketed by the male and female single-sex rates\footnote{Despite Das Gupta's effort in
explicitly producing a bracketed model in \citet{gupta1976interactive}.}. This
aspect is not mentioned explicitly in \citet{gupta1978alternative}, although
faith in the bracketing axiom was already waning by the time of its writing
\citep{yellin1977comparison}.
 
 To the extent that exposure within the model is a funciton of both males and
 females, this model may be said to be interactive. One may notice that since
 exposure is additive that the model will behave poorly in the absence of one
 potentially reproductive age-sex combination in the future (births for this
 age would not drop to 0 as they should). This possibility would not likely
 arise in practice, but it is still the most basic and necessary of
 commonly stated axioms. Further, the method is not fully age-interactive. AN
 increase in males (females) of one age will affect the fertility of all ages of
 females (males), but males have no effect on males and females have no effect
 on females.

\paragraph{The method applied to the US and Spanish data: } We estimate Das
Gupta's intrinsic growth rate for each year of the US and Spanish data. On the
whole, $r$ tracks the development of $r^m$ and $r^f$ over time, and it is
typically bracketed by them, but there are several years for both populations
where Das Gupta's $r$ was greater than either of the single-sex $r$ estimates.
These were all years in which the sex gap in $r$ was particularly narrow, and in
all cases Das Gupta's $r$ was the greater.

\begin{figure}[ht!]
        \centering  
          \caption{$r$ from Das Gupta (1978) and single sex intrinsic growth rates. US, 1969-2009, and Spain, 1975-2009}
           % /R/DasGupta.R
           \includegraphics{Figures/Gupta1978r}
          \label{fig:Gupta1978r}
\end{figure}

For purposes of prediction and ease of implementation, Das Gupta's model is
close to acceptable, though in following we will explore some models that are
somewhat more palatable and more widely studied, starting with models whose
two-sex fertility rates are derived from the harmonic (or other) mean of male
and female rates \citet{schoen1981harmonic}.

\FloatBarrier

     \subsection{Harmonic and other means}
       \FloatBarrier
\label{sec:ageharmonic}
\begin{singlespace}
\begin{quote}
Now of everything that is continuous and divisible, it is possible to take the larger 
part, or the smaller part, or an equal part, and these parts may be larger, smaller, 
and equal either with respect to the thing itself or relatively to us; the equal part
 being a mean between excess and deficiency. By the mean of the thing I denote a point 
 equally distant from either extreme, which is one and the same for everybody; by the 
 mean relative to us, that amount which is neither too much nor too little, and this 
 is not one and the same for everybody.
\citetalias{rackham1947trans}
\end{quote}
\end{singlespace}

The most instinctual two-sex fertility (marriage) solution is to symmetrically
(with respect to the sexes) utilize information from the vital rates of both
sexes. Mean functions have been compared in the past\citep[see
e.g.][]{keyfitz1972mathematics}, but rated in terms of utility with difficulty.
In terms of the axioms mentioned in Section~\ref{sec:axioms}--rather than
performance-- the harmonic mean function has fared the best amongst a variety of
means. Schoen \citep{schoen1978standardized, schoen1977two, schoen1981harmonic}
provided a rationale and derivation for using the harmonic mean in order to 
balance marriage rates. \citet{martcheva2001mathematics} found evidence of
poor performance for the harmonic mean in projective scenarios. The same
strategy can be used to balance fertility rates, which is what we will do here. The method requires as inputs a matrix of birth counts cross-tabulated by age of father, $a$, and age of mother $a'$ 
and male and female exposures classified by age. The harmonic mean
\begin{equation}
\label{eq:harmonic}
H(P_a^m, P_{a'}^f) = \frac{2 P_a^m P_{a'}^f}{P_a^m + P_{a'}^f}
\end{equation}
is applied to male and female exposures in order to find an intermediate
denominator from which to calculate rates, $F_{a,a'}^H$:
 \begin{equation}
 \label{eq:harmonicrate}
 F_{a,a'}^H = \frac{B_{a,a'}}{H(P_a^m, P_{a'}^f)}
 \end{equation}
which in the stable population is assumed constant in time rather than
assuming constant male and female rates separately. In order to estimate 
a birth count in some future year $t+n$, calculate the harmonic mean
of male and female exposures and multiply into the constant harmonic rate:
 \begin{equation}
 B(t+n) = \int \int F_{a,a'}^H H\Big(P_{a}^m(t+n), P_{a'}^f(t+n)\Big) \dd a \dd
 a'
 \end{equation}
which we can rewrite to make year $t$ births a function of past births in the
renewal equation:
 \begin{equation}
 B(t) = \int \int F_{a,a'}^H H\Big(\varsigma B(t-a)p_a^m, (1-\varsigma) B(t-a)
 p_{a'}^f\Big) \dd a
 \dd a'
 \end{equation}
where $p_a^m$ and $p_{a'}^f$ are the male and female probabilities of surviving
from birth until age $a$, $a'$, and $\varsigma$ is the proportion male of
births, here assumed constant over age and time, though this may be relaxed.
Rewriting in this way brings us to a stable population framework. \citet{schoen1977two} 
proposed his own rectangular stable population framework, which 
will not be treated here. As $t$ becomes large, the annual growth factor
approaches a constant value equal to $e^r$, which can be estimated from the
following Lotka-type unity function: 

\begin{equation}
\label{eq:lotkaH}
1 = \int _{a=0}^\infty \int _{a'=0}^\infty F_{a,a'} H\Big(\varsigma
e^{-ra}p_a^m, (1-\varsigma)e^{-ra'}p_{a'}^f\Big)\dd a' \dd a
\end{equation}
where $F_{a,a'}^H$ is the constant fertility rate to be applied to the harmonic
mean of male and female exposures, $p_a^m$ and $p_{a'}^f$ are the male
and female radix-1 survival functions. $\varsigma$ serves to make the
male and female radices sum to 1, and also accounts for the fact that males and
females have slightly different $l_0$ values. 

\paragraph{Estimating $r$: } The two-sex harmonic intrinsic growth rate, $r$ can
be estimated in two ways, either assuming $\varsigma$ constant from the start
(likely based on the initial data) and using a generic optimizer, or by modifying the iterative procedure
suggested by \citet{coale1957new}, which works best if one simultaneously
estimates $r$ and $\varsigma$ (i.e. allowing $\varsigma$ to adjust to the
population structure, as it is known to vary with age). Here we will describe
the practical steps involved in the latter.

\begin{enumerate}
  \item Calculate the constant harmonic fertility rates for male and female
  births separately, $F_{a,a'}^{mH}$ and $F_{a,a'}^{fH}$
  \item Make a first estimate of the stable sex ratio at birth, $\hat{S}$; the
  initial observed sex ratio at birth is a good choice. From $S^0$ we derive a
  first estimate of the proportion male of births, $\varsigma^0$ (where
  superscripts indicate the iteration):
  \begin{equation}
  \varsigma^0 = \frac{S^0}{S^0+1}
  \end{equation}
  \item Find a first rough estimate of the net reproduction rate,
  $\widehat{R_0}$, assuming a growth rate of 0 and using the both-sex
  harmonic fertility rate $F_{a,a'}^{H} = F_{a,a'}^{mH} + F_{a,a'}^{fH}$:
  \begin{equation}
  \label{R0guessschoen}
  \widehat{R_0} = \int_{a=o}^\infty \int_{a'=0}^\infty H(\varsigma^0 p_a^m,
  (1-\varsigma^0)p_{a'}^f) F_{a,a'}^{H} \dd a' \dd a
  \end{equation}
  \item Assume a reasonable both-sex mean generation time, $\widehat{T}$.
  Weighting $a$ and $a'$ into Equation~\eqref{R0guessschoen} and then dividing
  by $\widehat{R_0}$ yields a good estimate of this. Otherwise one may simply
  choose a reasonable age, such as 30, or some mean of the male and female
  single-sex mean ages at reproduction.
  \item Calculate an initial value of $r$, $r^0$ as:
  \begin{equation}
  r^0 = \frac{log(\widehat{R_0})}{\widehat{T}}
  \end{equation}
  \item Now that we have a starting value, $r^0$, calculate a residual,
  $\delta^0$, from equation~\eqref{eq:lotkaH}:
  \begin{equation}
  \delta^i = 1 - \int _{a=0}^\infty \int _{a'=0}^\infty H(\varsigma^i p_a^m
  e^{-r^ia}, (1-\varsigma^i)p_{a'}^fe^{-r^ia'}) F_{a,a'}^H \dd a' \dd a
  \end{equation}
  \item Use $\delta^i$ to improve the estimate of $r$, $r^{i+1}$:
  \begin{equation}
  r^{i+1} = r^i - \frac{\delta^i}{\widehat{T} - \frac{\delta^i}{r^i}}
  \end{equation}
  \item Use the improved estimate of $r$ to update $\varsigma$:
  \begin{align}
  S^{i+1} &= \frac{\int_{a=o}^\infty \int_{a'=0}^\infty H(\varsigma^i
  e^{-r^{i+1}a} p_a^m, (1-\varsigma^i)^i e^{-r^{i+1}a'}p_{a'}^f) F_{a,a'}^{mH} \dd a' \dd a
  }{\int_{a=o}^\infty \int_{a'=0}^\infty H(\varsigma^i e^{-r^{i+1}a}
  p_a^m, (1-\varsigma^i)^i e^{-r^{i+1}a'}p_{a'}^f) F_{a,a'}^{fH} \dd a' \dd a }
  \\
  \varsigma^{i+1} &= \frac{S^{i+1}}{S^{i+1}+1}
  \end{align}
  \item Plug the new $\varsigma$ and $r$ estimates into step 5, to estimate a
  new residual, $\delta$, repeating steps 6-8 until $\delta$ vanishes to 0.
  Typicaly around 20 iterations are needed in order to reduce $\delta$ to
  be less than double floating point machine tolerance.
\end{enumerate}

This iterative procedure simultaneously produces an estimate of the stable
sex ratio at birth $S$ and the both-sex intrinsic growth rate, $r$. Really,
there is little room for $S$ to move between the initial and stable states,
since boy and girl births are in essence produced by (the harmonic mean of) both
males and females in this procedure. $S$
will only vary from the initial sex ratio at birth to the extent that there is
both an age pattern to the sex ratio at birth and the male and female stable age
structures differ from the initial age structures. Estimating both parameters at
the same time does not present a practical problem in the present case, and the
procedure converges faster than if $S$ is left assumed at some constant value.

One could abandon the iterative $r$ estimation procedure outlined above
and perform a standard cohort component projection, for instance using a
two-sex Leslie matrix. In this case, the fertility component of the Leslie
matrix would need to be updated between each iteration using equation~\ref{eq:asfrH} for either
males or females. One cannot easily perform standard matrix analysis of this
Leslie matrix, however, as it is not static in the standard way.

\paragraph{Other stable quantities: } Once one has identified the stable $r$ and
$S$, one may move on to estimate other stable parameters of interest, such as the 
both-sex stable birth rate, $b$:

\begin{equation}
b = \frac{1}{\int_{a = 0}^\infty e^{-ra} \varsigma p_a^m \dd a + \int_{a' =
0}^\infty e^{-ra'} \varsigma p_{a'}^f \dd a'}
\end{equation}
which may be used to calculate the male and female stable age structures, $c_a$
and $c_{a'}$:

\begin{equation}
c_a =  \varsigma  e^{-ra} p_a^m
\end{equation}
and analagously for females, where
\begin{equation}
1 = \int c_a + \int c_{a'}
\end{equation}
and the total population sex ratio, $S^{tot}$ is the ratio of these:
\begin{equation}
S^{tot} = \frac{\int c_a}{\int c_{a'}}
\end{equation}

\paragraph{Summary of the harmonic mean method: } The stable system outline here
is not taken word-for-word from Schoen's advice, but it is consistent with the 
notion of a constant \textit{force of attraction},
$F_{a,a'}^H$, and non-linear balancing of fertility rates based on the harmonic
mean of male and female exposures. The method presented here is only partially
sensitive across all ages to changes in the exposure of a single age in one sex.
That is to say, an increase in males of age $a$ will increase observed fertility rates for all ages
of females that share rates with males of age $a$. Further, females with
higher rates, $F_{a,a'}^H$, will typically observe greater increases, though this
depends on the distribution within $F^H$ and on relative exposure levels.
Lacking from this implementation are decreases in rates for males whose ages are close
to $a$, so-called spillover effects\citep{choo2006estimating}. That is to say,
an increase in age $a$ males, will not affect rates of males age $a-n$ or $a+n$, 
despite the fact that the pool of potential mates, females over
all ages $a'$, is shared. One would expect, ceteris paribus, that males of
similar ages would experience a decrease in rates, since some proportion of the
female pool will have been redirected to the increased stock of age $a$ males.
Hence, the model lacks this sense of competition. All other axioms appear to be
satisfied, except for that of bracketing, which we also deem superfluous.
Further, the harmonic mean is biased toward the minority sex, which is also intuitive.
 As stated before, one cannot empirically establish (for
humans) the ideal functional form of the fertility (marriage) function.

One satisfying property of the present method is that the harmonic mean
rates do not respond rigidly to mismatched population sizes between males and
females, but rather the mean rate is sensitive to relative size of male and
female stocks. In this way, the function is more dynamic than a weighted mean,
or Das Gupta's method presented in the previous section. Indeed, if the
demographer is not satisfied with the elasticity of the harmonic mean, one may
change $H()$ for any mean function, such as a generalized mean. An infinite number of other means
will also have the same desirable properties as the harmonic mean, such as 
dropping to 0 in the absence of one sex. Most means with this property that have
names (harmonic, geometric, logorithmic, \ldots) will produce almost
indistinguishably similar results. All such mean solutions will be symmetric
(blind) with respect to the sexes, although one could easily include weights.

\paragraph{The method applied to the US and Spanish data: }
In addittion to the harmonic mean, we have produced estimates of $r$ using the
geometric and logorithmic means, as well as the minimum function.
Figure~\ref{fig:schoenr} shows only the results of the harmonic mean and minimum
functions, as the geometric and logorithmic $r$ estimates would not be visually
distinguishable from those of the harmonic mean. From this lesson, we confirm
that if one is to use a mean function as a 2-sex fertility (marriage) function,
it really makes little difference which mean function one chooses, as long as it
satisfies the availability condition. The minimum function yields the least
consistent results, sometimes greater than the harmonic mean, sometimes less
than the harmonic mean, sometimes bracketed by the single-sex $r$ values, and
sometimes not. We note that the minimum function deviates the greatest from the
single-sex $r$ values when the sex-gap is trivial, and in these instances it is
always higher. The harmonic mean series is here always bracketed by the
single-sex $r$ values, although this is not a necessary result.

\begin{figure}[ht!]
        \centering  
          \caption{$r$ according to harmonic mean and minimum fertility
          functions compared with single sex intrinsic growth rates. US,
          1969-2009, and Spain, 1975-2009}
           % /R/Schoen1981.R
           \includegraphics{Figures/HMager}
          \label{fig:schoenr}
\end{figure}

In terms of complexity of implementation, solutions based on mean functions are
marginally less demanding than the Das Gupta solution, but this is primarily
because mean functions are more readily understood. The mean solution is seen as
onceptually simpler, yet yielding similar results and with more desirable
properties than either of the preceeding solutions. In following, we will
present two iterative fertility functions that allow for competition between age-groups of the same sex.

\FloatBarrier

     

  \section{Iterative proportional fitting}
      \subsection{Iterative Proportion Fitting}

\textit{mc1975models} noted that a simple way to incorporate inter-age
competion (or at least inter-age sensitivity) in marriage count- balancing is
to iteratively rescale a known cross-tabulation of counts (say, from the
previous year) by the separately predicted male and female margins. That is to
say, if males are in matrix rows, one scales each row to sum to the predicted
male margin, then one scales each column to sum to the predicted female margin
(or vice versa, females then males). In rescaling rows to sum to the predicted male margin, 
followed by columns to sum to the predicted female margin, after just a few iterations the process
converges to a particular distribution. Cell counts thus shift between both male
and female ages from the original count matrix to the iteratively predicted
matrix, but stay close to the original distribution. The method satisfies the
thus-far most difficult axiom to incorportate, that of inter-age competition. 

However, the sums of the respective male and female predicted margins will of
course not agree-- After converging to a distribution, the total predicted count
will \textit{flip-flop} between the total male and female predictions, 
which will have differed. Further, \citet{Matthews2013} note that the
final distribution will depend upon whether one starts by scaling the male or
female margin. Both of these problems may be overcome, these authors suggest, 
by starting in parallel within each iteration with the male and female matrix
margins, followed by the other sex, producing two possible two-step scalings.
The starting matrix for the next iteration is taken as the average of these two
outcomes. Since the end result of each iteration is an average, the sum will be
intermediate to the divergent male and female marginal predictions, and the
biverate distribution will be indifferent to whether one started the iterative
process with males or females. The method could of course be further generalized
to take any mean of these two matrices, and not just an arithmetic mean. Results
will vary.

Neither of the above-mentioned studies used their respective iterative
adjustment procedures to predict birth counts, although the \citet{Matthews2013}
method is just a smaller part of a more complex model that includes fertility.
In this dissertation we will treat the \citet{Matthews2013} method only,
modifying it very slightly, so as to be based on a generalized mean (which
allows for a harmonic mean, for instance) as the basic for the male-female
intra-iteration averaging.




    
  \section{Other method families not investigated here}
     \subsection{General equilibrium methods}
     \subsection{Ordinary differential equations}
     \subsection{Agent-based models}


\part{Populations structured by remaining years of life}
\chapter{Switching the direction of age}
   \FloatBarrier
Recall the observations of \citet{sanderson2005average}, more directly
relating to the popuation pyramid. These authors note that despite ageing in a
population, the mean \textit{remaining} years to be lived may increase. This is due 
to improvements in mortality offsetting (or more than offsetting) increases in
the mean age of a population. In general, a population looked at from the
perspective of life expectancy looks different from, behaves differently from,
and yields complimentary information to one looked at from the canonical
perspective of age. This observation is the point of departure for the
linear and non-linear two-sex models that I will introduce in this dissertation.
However, such steps are large, and require an involved demonstration of some
key differences between age clsasified and \textit{remaining years}-classified,
henceforth $e_x$-classified demographic data. Indeed a whole new one-sex
model must first be developed prior to re-delving into its two-sex extensions. 
I will first present a method
to exactly redistribute population counts (or exposures) according to period 
remaining years of life, rather than according to age per se. A reexamination 
of recent fertility patterns according to remaining years of life will follow.

  
  \section{Transforming time since birth to time until death}
     \FloatBarrier

The steps required to carry out the present data transformation are conceptually
simple, and easy to implement once understood. From a given
population and year extract the $d_x$ column from the corresponding lifetable of
radix ($l_0$) equal to 1\footnote{If the lifetable was calculcated with a
different radix, then simply divide the $d_x$ column by $l_0$}. Note that in
this case the $d_x$ column sums to 1, and is therefore a proper density function. 
$d_x$ can now be thought of as the probability of dying in any given age from the
 perspective of a 0-year-old, according to the given year's mortality
 experience. It follows that the observed population of age 0 can be
 redistributed according to $d_x$ and interpreted either as the expected death
 counts in each future year
$t+x$, or more intuitively as the distribution of persons currently-aged 0 according 
to remaining life expectancy. This can be done similarly for age 1, by ignoring the 
mortality experience of age 0, and rescaling $d_x$ to
sum to 1, or more generally redistributing each age and then summing to
$e_x$-specific totals:

\begin{align}
\label{eq:dxredist}
P_{y} &= \int _{a = 0} ^{\infty} P_a \frac{d_{a + y}}{\int _{b
= a} ^{\infty} d_b\, \dd b} \;\dd a
\\
&= \mathbf{E}(D_{t+y}) \notag
\end{align}
where $P_a$ is the population of age $a$, $d_a$ is the
lifetable density function and $\mathbf{E}(D_{t+y})$ is the expected number of
deaths $y$ years after the present year $t$, also understood as a vector of the
current population, redistributed into categories of remaining life expectancy,
$P_{y}$, our newly reclassified data.

The function of this formula is not original, as
\citet{miller2001increasing} and \citet{vaupel2009life} made use of a similar
identity:
\begin{equation}
\label{eq:vaupelredist}
f(n | a) = \mu (a+n) \frac{l(a+n}{l(a)}
\end{equation}
where $f(n | a) $ is the probability of dying $n$ years in the future given
survival to age $a$, and $\mu$ is the force of mortality.
\citet{miller2001increasing} used the formula to look at death distributions of
particular ages in projecting health expenditures.
Formula~\eqref{eq:vaupelredist} can thus be used to weight age-classified data
as well. When then integrated over age for a given $n$ is equal to
Equation~\eqref{eq:dxredist}.

Formula~\eqref{eq:dxredist} is more convenient when 
discretized\footnote{Formula~\eqref{eq:dxredist} is more convenient due 1) to lifetable 
close-out issues and 2) because only one column from the lifetable is required instead 
of 3 columns ($\mu_x$, $l_x$, $L_x$) in
Equation~\eqref{eq:vaupelredist} }, although both are equally valid.
Equation~\ref{eq:dxredist} is equivalent to:

\begin{equation}
P_n = \int _{a=0} ^\infty P_a \mu_{a+n} \frac{l_{a+n}}{l_a} da
\end{equation}
where $n$ is treated as $y$ in \ref{eq:dxredist}.
The use of either in the way
presented in this section is to this author's knowledge novel.

   
  \section{Populations structured by remaining years}
     \FloatBarrier
\label{sec:expopstruct1}
The resulting
population structure from applying Formula~\eqref{eq:dxredist} to age-classified
population data is diachronous\footnote{heterogeneous with respect to age.} 
within any given level of remaining life expectancy, and 
looks like Figures~\ref{fig:exPyrUS} and \ref{fig:exPyrES}\footnote{The idea to
redistribute the population pyramid in this way is due to a conversation with 
John MacInnes, and appears in \citep{MacInnes2013pop} (unpublished) using a
different method.} for the years 1975 and 2009 in the US and
Spain\footnote{The unlabeled inside cover artwork is the same 2009 pyramid (in
green) but preceeded by history (grey) and continued with a deterministic
projection (blue) under specific assumptions (Spain left, US right, vertical
axes comparable, horizontal axes not comparable).}. As a helpful pointer, note that the
population at the base of the pyramid is expected to decrement within the \textit{next year}, thus the vertical axis can also can also be thought of as year $t+y$, although $e_y$ more clearly identifies the pyramid with year $t$ mortality conditions. The pyramid should not be taken out of context as a forecast. Note that this pyramid represents the exact same
population as an age-classified pyramid: Underlying males sum to the correct total on 
the left and females sum to the correct total on the right. Only the definition of age has
changed; instead of counting forward from birth we count \textit{age} in reverse
starting from death. For individuals, this feat would be impossible, but given
 the information contained in a period lifetable, one can to great utility 
 redistribute population aggregates according to $e_y$\footnote{To undertake
 the same but assuming future mortality changes (improvements) one might
 better undertake a fertility-free cohort component projection and collect the
 deaths from each future year $t+y$ until extinction. This possibility is not
 treated in the present dissertation.}. Both pyramids have been rescaled
 to sum to 100, in order to more comparably represent population structure.

\begin{figure}
        \centering
        \begin{subfigure}
                \centering
                \caption{US population by remaining years, 1975 and 2009}
                \includegraphics[scale = .8]{Figures/exPyramidUS}
                \label{fig:exPyrUS}
        \end{subfigure}
        \begin{subfigure}
                \centering
                \caption{Spain population by remaining years, 1975 and 2009}
                \includegraphics[scale = .8]{Figures/exPyramidES}
               
                \label{fig:exPyrES}
        \end{subfigure}
\end{figure}

A time series of remaining life expectancy pyramids for any given Western 
country (excluding war years and epidemics, and especially after the rapid fall
in infant mortality) will show incredible stability over time, which is
remarkable in light of ageing in the observed population pyramid. The simple 
interpretation of this kind of pyramid adds to its utility, and this author 
believes that $e_y$-specific population structure, and
indicators that can be derived from this method (not treated here) should 
make up a valuable new component to the contemporary demographer's toolbox, as well 
as help inform current population debates. 

It will suffice for the time being to
illustrate that for Spain and the US in the years trated in this dissertation,
the remaining-years-structured population pyramid has been many times more
stable. This we will demonstrate by again making use of the difference
coefficient, $\theta$ from Equation~\ref{eq:coefdiff}, where $f_1$ is the
population structure for year $t$ and $f_2$ is the population structure from
year $t+1$ (males and females, together, scaled to sum to 1). We iteratively
produce $\theta$, comparing year $t$ and $t+1$ for age-structured pyramids in
the first place and for $e_y$-structured pyramids on the other. Pyramids are in
general very stable, so the difference $\theta$ in both cases will nearly always
fall below 0.02. However, $e_y$-$\theta$ is consistently and considerably lower
than the age-$\theta$. It will suffice to take the ratio of the two indicators,
 $e_y$-$\theta$ divided by age-$\theta$ over the period of study for both
 countries, as seen in Figure~\ref{fig:PyramidStability}.

\begin{figure}
      \centering
      % Figure made in PyramidStability.R
      \caption{Relative stability of $e_y$-structured populations to
                age-structured populations, year $t$ vs $t+1$, ratio of
                $\theta$, Spain and US, 1969-2009} 
         \includegraphics{Figures/PyramidStabilityThetaRatio}
      \label{fig:PyramidStability} 
\end{figure}

In Figure~\ref{fig:PyramidStability}, a value of 1 would indicate that the two
ways of structuring population are equally stable between years $t$ and $t+1$;
values less than 1 indicate that the $e_y$-structured population is more stable.
For instance, .5 means that the $e_y$-structured population was twice as stable,
.2 means 5 times more stable, and so forth. In all years in this dissertation,
$e_y$-structuring acted to stabilize the population somewhat. As a heuristic,
runs of years with continuous and modest improvements in mortality will produce
the most stable $e_y$-structured pyramids. This measure of stability compounds
as well: that is to say, and $e_y$-structured population in year $t$
compared with that from year $t+10$ will be much more stable than the same
comparison for the standard population. This lesson will reap dividends
throughout the remainder of this dissertation; we will exploit this observation
without investigating much further into its causes.

\FloatBarrier
    \subsection{Uncertainty in this structure}
      \FloatBarrier
\label{sec:structuncertainty}
The reader may reasonably question whether the structure observed in
Figures~\ref{fig:exPyrUS}~and~\ref{fig:exPyrES} is as certain as its graphical
representation would imply-- There are namely no confidence intervals in the
plot, despite the fact that we've used the deaths \textit{distribution}, $d_x$,
to redistribute population counts, and $d_x$ is naturally subject to random
fluctuations. Aside from typical sources of error for population estimates, and
assuming that age in the first place is correctly recorded, we may wish to asses
how much the present population structure could suffer distortion from noise in
$d_x$. 

To shed light on this question, the following exercise has been
carried out: 1) for each age of original data 1000 random deviates were drawn
from the poisson distribution, using the observed death count as the parameter
$\lambda$. 2) Dividing these simulated death counts by observed exposures gives
simulated death rates $\mu_x$. 3) $d_x$ is derived from $\mu_x$ using the HMD
methods protocol. 4) The population is redistributed 1000 times according to the
1000 random $d_x$ disrtibutions. 5) From these simulated population structure,
the 0.025 and 0.975 quantiles are extracted from each class of remaining years.
This yields some rough 95\% uncertainty bands.

It happens that uncertainty due to randomness in $d_x$ is so minor in both the
US and Spanish populations that these confidence bands superimposed on
Figure~\ref{fig:exPyrUS} would not be visible to the naked eye. There is
nonetheless a pattern to uncertainty in $e_y$-structured populations, 
which we display in Figure~\ref{fig:PyramidUncertainty}. Confidence bands are
larger for the Spanish population than for the US population due to
the difference in population size. The pattern over remaing years is for
uncertainty to decrease from the lowest life expectancies until $e_y$ is around 40 or 50, and then to
increase geometrically until the highest life expectancies. The highest
uncertainty corresponds with the thinnest part of the pyramid, however, and the
lowest uncertainty falls around the mode of the pyramid. Uncertainty for males
and females is similar in low $e_y$, but tends to become greater for males as
$e_y$ increases geometrically.

\begin{figure}
      \centering
      % Figure made in PyramidStability.R
      \caption{Width of 95\% uncertainty bands as percentage for each
      remaining years category, Spain and US, 1975}
         \includegraphics{Figures/exPyramiduncertainty1975}
      \label{fig:PyramidUncertainty} 
\end{figure}

By far the greatest source of uncertainty in this
pseudo-projection will arise if the demographer decides to account for
improvements in mortality. This later uncertainty will arise not only from
random fluctuations, but also due to the projection assumptions used. This
variety of adjustment, adding improvement to the mix, will be discussed in
following.

\FloatBarrier
    \subsection{Accounting for improvement in mortality}
      \FloatBarrier
\label{sec:eximprov}
As we are dealing with a \textit{forward-looking}\citep{sanderson2007new}
transformation of structure, the reader will likely intuit that our redistribution 
method will err toward pessimism to the extent that
future improvements in mortality are ignored. To take account of future
mortality improvements is already a projections mindset, and we have made no
claim that the this structure is a projection- Rather it is a speedometer, to
use the analogy of \citet{coale1972growth}. In any case, one best accounts for
future mortality improvements by applying assumptions about the rate of
improvment in the age-specific motality hazard, $\mu_a$, rather than directly
manipulating $d_a$, which is our redistribution vector. To illustrate by
example, take the population of infants and the full present vector of $\mu_a$. It seems fair that
that $\mu_0$, or something close to it, will apply to these infants, yet by age
5, $\mu_5$ will likely be too high, and certainly by age 50
this same static $\mu_50$ will be too high. For the sake of simplicity, let us assume that the
rate of improvement, $\iota$, applies equally over all ages and future years,
and is equal to about 0.5\% per year (conservative for some ages, liberal
for others). We would prefer to have a multiplicative factor, so we define
$\iota = 1-.005 = 0.995$. Then we modify $\mu_a$ for these 0-year-olds, in
the following way

\begin{equation}
\mu_a^0 = \mu_a \prod _0 ^\infty \iota 
\end{equation}

where the superscript indicates that we have done this for persons of age 0. Now
one converts the $\mu _a^0$ to the new $d_a^0$ and redistributes the infant
population accordingly. For higher ages, one calculates $\mu_a^x$ as

\begin{equation}
\mu_a^x = \mu_a \prod _x ^\infty \iota 
\end{equation}
always with the first age $a$ equal to $x$.
Now one needs a direct relation of $\mu _a$ and $d_a$, and for this we use the
HMD methodology, namely assuming that the average age at death in each age
interval is 0.5 (except for age 0, which uses the Coale-Demeny rule of thumb),
deriving death probabilities, $q_a$, followed by the survival function, $l_a$,
and taking the product of these two to arrive at $d_a$\footnote{This is the
abbreviated version. See the HMD Methods Protocol for the full version, which
for this dissertation we have functionalized.}. This must be done for each age,
to create as many modified $\mu _a^x $ vectors as there are ages, and then one applies the same
Equation~\ref{eq:dxredist}, swapping the $\mu _a^x$ vector as one iterates over
ages. Alternatively, one could just take some future evolution of $\mu_a$ from a
standard projection method, such as the Lee-Carter method
\citep{lee1992modeling}, finding the mortality trajectory expected for each individual, converting
this to $d_a$, and then redistributing the population accordingly. 

Here we execute the simplest assumption, though there is ample room for
improvment in the method. The results of decreasing year 2009 mortality rates in each
successive year (iteration) by a multiplicative factor of 0.995 are displayed in
Figures~\ref{fig:exPyrUSimpr}~and~\ref{fig:exPyrESimpr}.

\begin{figure}
        \centering
        \begin{subfigure}
                \centering
                \caption{US population by remaining years under fixed
                mortality and constant improvement of 0.5\% per year.}
                \includegraphics[scale = .8]{Figures/exPyramidUSimpr}
                \label{fig:exPyrUSimpr}
        \end{subfigure}
        \begin{subfigure}
                \centering
                \caption{Spanish population by remaining years under fixed
                mortality and constant improvement of 0.5\% per year.}
                \includegraphics[scale = .8]{Figures/exPyramidESimpr}
                \label{fig:exPyrESimpr}
        \end{subfigure}
\end{figure}

Here, common practice would have been to start the rate of improvement stronger
and let is taper off with time, or to allow effects to work differently over
age, but the present exercise is meant to be illustrative. Here one appreciates
the slight maleability of population structure in light of potential future
mortality improvements. The impending mode, composed largely but not entirely of
baby-boomers, is absorbed to a certain extent, and the overall picture is even
more optimistic than the original (as one would expect) for both countries.

To incorporate uncertainty into this method, it is
recommended to allow variaton in $\mu_x$ per the strategy outlined in the previous
section~\ref{sec:structuncertainty}. At this time, we will depart from the
present line of development. For the rest of this dissertation we will deal only with static
period deaths distributions, and we leave the thoughtful incorporation of mortality 
improvements and addittional uncertainty into the method for future work.

\FloatBarrier
    \subsection{Potential indicators of ageing implied by this structure}
      \FloatBarrier

The reader will have been quick to notice that the present pyramid lends itself
to the calculation of some simple proportions akin to those offered by
\citet[e.g.][]{sanderson2005average}. For instance, one may calculate the
proportion of the pyramid with remaining life expectancy less than some
threshold, such as 15 (that commonly chosen by Sanderson and Scherbov). Where
$P^T$ is the total population and $P_y$ is the population with exact remaining
life expectancy $y$, we calculate the proportion of the population with 15 or
less remaining years of life expectancy, $P_{y \le 15}$, as
\begin{equation}
P_{y \le 15} = \frac{\int _{j=0}^{y} P_j \dd j}{P^T}
\end{equation}

The results from this calculation will differ from the method called the
``Prospective Old-Age Dependency Ratio''(POADR) for two reasons. 1) POADR is
calculated by indexing the lifetable $e_x$ to some reference
lifetable\footnote{This is explained nicely in \citet{sanderson2007new}}. 2)
we have lost the age information necessary to cut off the total population
below age 20, although this could be accounted for with little trouble. In
doing so, however, we would lose the deaths of all those below age 20 that
would happen within the next 15 years. 

To better compare with the notion of prospective age, look to the example given
in \citet{sanderson2007new}: The authors give the example of comparing
individuals from 1950 and 2000 who each have an average remaining life 
expectancy of 30. These two individuals will 
evidently have had different chronological ages. Imagine that in 1950 $e_x = 30$ 
occured at age 40, and in 2000 at age 50. Then using 1950 as our standard, we 
could say that the 50-year-old in 2000 has a prospective age of 40, in the sense
of ``50 is the new 40''. 

This basis for the method is quite different than that presented here because
indexing is done via the $e_x$ column of the lifetable, which is a weighted
average. Specifically, it is the weighted average of all possible remaining life
expectancies, where $d_x$ are the weights. Chronological age is in this way
never dispensed with, and it is hard to imagine how, for instance, this method
could do a good job of producing a fully structured population, such as that
seen in Figures~\ref{fig:exPyrUS}~and~\ref{fig:exPyrES}. Our method instead
combines ages via $d_x$ so as to arrive at clean breakpoints of $y$. We do not
offer a thorough comparison between such measures, although this is a ripe
avenue for further work.

Another measure of population ageing that falls out of our method is the
Proportion of Life Lived (PLL)\footnote{Again, this idea owes to a
particular conversation with John MacInnes, upon which I formalized the
concept.}. Take, for instance the same 50-year-old man in 2000 with a remaining life expectancy of
30. This person will in the end on average live 80 years, and so has completed
$\tfrac{50}{80} = 0.625$ or 62.5\% of his life. We can refine this a bit further
by using the deaths distribution at ages 50 and higher in 2000; some 50-year-old
males will die at age 50, others at age 51, and so on up until the highest age.
With respect to each potential age of death, we may then calculate a proportion
of life lived, and proceed to take a weighted average of these. Formally, we
calculate PLL for the whole population as

\begin{equation}
PLL = \frac{\int _{y = 0} ^\infty \int _{a = 0} ^\infty \frac{a}{a + y}
\left[ P_a
\frac{d_{a + y} }{ \int _{b = a} ^{\infty} d_b\, \dd b  }\right] \dd a \dd y }{
P^T }
\end{equation}
where $a$ indexes age and $y$ indexes remaining years of life. For the US and
Spain, the time trend of this indicator for our two populations and our
relatively narrow range of years is displayed in Figure~\ref{fig:PLL}
\begin{figure}
      \centering
      % Figure made in PyramidStability.R
      \caption{Population proportion of life lived, PLL, US, 1969-2009 and
      Spain, 1975-2009.}
         \includegraphics{Figures/PLL}
      \label{fig:PLL} 
\end{figure}

This trend will be seen to agree roughly with those presented by other
\textit{forward-looking} indicators of ageing. PLL, however, is particularly
intuitive, requiring no involved examples to explain to non-demographers, and
the trend seen here is particularly clear and consistent. One could
calculate PLL in like manner for particular ages, age-ranges, or remaining-years
classes. This indicator will surely yield much lower levels if improvements
are accounted for as discussed in Section~\ref{sec:eximprov}, and uncertainty
may be introduced using the monte-carlo strategy outlined in Section~\ref{sec:structuncertainty}

This dissertation offers no further discussion of the
potential ageing indicators implied by the present structuring of population.
This and the preceeding two sections on uncertainty and the incorporation of
mortality improvements into the present redistribution method have been intended
primarily to placate what the author considered to have been the most likely
initial doubts. As one sees, there is ample room for improvement in all methods
thus far presented. As has been stated, for the remainder of this dissertation,
we will work with the simplest deterministic assumption of fixed period rates.
In the following sections we extend the current redistribution technique to the
domain of fertility. Restructured fertility rates will form the basis of the new
population models to be presented later in this dissertation.
\FloatBarrier
  \section{Fertility rates structured by remaining years}
     \FloatBarrier
The technique presented in Equation~\ref{eq:dxredist} and illustrated in
Figures~\ref{fig:exPyrUS}~and~\ref{fig:exPyrES} can indeed be used to reclassify
any age-distributed data, assuming that the appropriate lifetable is available.
\todo{add to this}
    
    \subsection{Fertility by remaining years}
      The technique presented in Equation~\ref{eq:dxredist} and illustrated in
Figures~\ref{fig:exPyrUS}~and~\ref{fig:exPyrES} can indeed be used to reclassify
any age-distributed data, assuming that the appropriate lifetable is available.
We now apply this redistribution technique in order to calculate male and 
female $e_x$-specific fertility rates ($e$SFR). For any rate, the numerator 
and denominator require a common referent, thus both births and exposures are 
redistributed according to year $t$ mortality conditions. That
is to say, we take the extra step of moving the age-specific vector
of birth counts (by mothers' or fathers' age) into $e_x$-specific birth
vectors before dividing into $e_x$-specific exposures. Explicitly:

\begin{align}
\label{eq:eSFR}
F_{y} &= \frac{\int _{a = 0} ^{\infty} B_a \frac{d_{a + y}}{\int _{b = a)}
^{\infty} d_b\, \dd b} \;\dd a}{\int _{a = 0} ^{\infty} F_a \frac{d_{a + y}}{\int _{b = a} ^{\infty} d_b\, \dd b} \;\dd a} \\
 &= \frac{B_{y}}{E_{y}}
\end{align}
, where $y$ indexes remaining years of life and $a$ indexes age, $B_a$ are
age-clsasified births, and $E_a$ are age-classified exposures. Remaining years
of life-specific rates cannot be directly compared with a typical age-specific
rate, since the time scales are different, but we 
can indeed apply some familiar tools in order to analyze this new curve.

The $e_x$-pattern of fertility is distinct from the age-pattern of fertility. 
In contemporary Western populations, female $e$SFR curves will be
further to the right than male curves for two reasons: 1) Female mortality is
almost universally lower than male mortality at (and beyond) any given age,
thus associating births at a given age with higher remaining life expectancies; 2)
female fertility is more tightly concentrated over young ages, partly due to the
upper bound defined by menopause, and partly due to prevailing hypergamy.
Figure~\ref{fig:eSFR2009} shows an example $e$SFR from 2009, for both the US and Spain.

\begin{figure}[ht!]
        \centering  
          \caption{Male and Female $e_x$-specific fertility rates, 2009, USA and
          Spain}
           % figure produced in
           % /R/Parents_ex.R
           \includegraphics{Figures/eSFR2009}
          \label{fig:eSFR2009}
\end{figure}

One may question whether the curves shown in Figure~\ref{fig:eSFR2009} properly
represent rates. This author argues that the same definition of events in the
numerator and exposures in the denominator has been applied, only the
structuring variable has changed from \textit{time since birth} to \textit{time
until death} (of progenitor here). In this way, age-classified and
$e_x$-classified populations have structure in the same sense. As with any
demographic variable, we may wish to analyze the intensity of demographic
phenomena removed of the distorting effects of population structure.
Working with event-exposure rates are just one way of doing so, simple
decomposition is another, and indeed such rates and decompositions are possible
in the aggregate both with respect to age and with respect to
$e_x$.

This is, in the best case, a rough calculation, for several reasons. The
assumption of homogenous mortality is particularly consequential in the case of 
fertility, where health selection is self-evident, but not easily measureable.
It is for this reason to be expected that the left tails in
Figure~\ref{fig:eSFR2009} are too thick. 

Furthermore, exposure is taken from the \textit{entire} population, not merely
the populaton within reproductive ages. The rates could be thusly recalculated,
for instance using female ages $13-50$ and male age $15-65$, and would look like 
Figure~\ref{fig:eSFR2009limits}, in some isntances a more reasonable if less
intelligible result\footnote{Rate surfaces based on $e_x$-specific fertility
data are calculated under a variety of reproductive spans in
Appendix~\ref{Appendix:reprospans}}.

\begin{figure}[ht!]
        \centering  
          \caption{Male and Female $e_x$-specific fertility rates, 2009, USA and
          Spain, with exposures redistributed using only female ages $13-50$ and
          male ages $15-65$}
           % figure produced in
           % /R/Parents_ex.R
          \includegraphics{Figures/eSFR2009limits}
          \label{fig:eSFR2009limits}
\end{figure}

Comparing Figures~\ref{fig:eSFR2009} and~\ref{fig:eSFR2009limits} reminds of the
comments of \citet{gupta1978alternative} and \citet{mitra1976effect} on the difficulty of
defining an \textit{effective} population for use in exposures. Clearly, persons
outside the reproductive age range will conventionally be excluded from
exposures. Other kinds of risk heterogeneity are known to exist, such as age
patterns in fecundability, contraceptive use and sexual intercourse, that are
unaccounted for in standard fertility measures. 

With no claim of superiority over the more
restrictive exposures used for Figure~\ref{fig:eSFR2009limits}, we will proceed
in this section by using exposures derived from all ages. One could weakly
defend this choice by noting that we are attempting to measure the period
reproductivity of an \textit{entire} population, not just part of it. The
reproductive span was an outcome of evolution, varies greatly between
individuals and populations, and is mutable, both due to ongoing
population-level genetic, nutritional and hormonal changes and direct human
intervention. We will for the time being, be content to work with the cruder $e$SFR, and note
that this rate, as any other, is amenable to further disaggregation and
decomponsition.

\begin{figure}
        \centering
        \begin{subfigure}
                \centering
                \caption{Male and Female $e$SFR surfaces, 1969-2009, USA}
                \includegraphics[scale = .8]{Figures/eSFRsurfacesUS}
                \label{fig:exSFRsurfUS}
        \end{subfigure}
        \begin{subfigure}
                \centering
                \caption{Male and Female $e$SFR surfaces, 1975-2009, Spain}
                \includegraphics[scale = .8]{Figures/eSFRsurfacesES} 
                \label{fig:exSFRsurfES}
        \end{subfigure}
\end{figure}

As is visible in Figures~\ref{fig:exSFRsurfUS}~and~\ref{fig:exSFRsurfES}, 
$e_x$-SFR has changed its level and undergone a gradual displacement over 
time toward higher $e_x$ levels, an altogether propitious development
with respect to human altriciality. The interpretation of this displacement is
entirely different from that of postponement in ASFR. Observed fertility 
postponement should shift $e$SFR unfavorably to higher mortality 
levels (lower $e_x$ levels), however mortality improvements have tended to 
offset this effect, acting to move the curve to higher
remaining life expectancies. 
      
    \subsection{A synthetic rate based on remaining years}
       \FloatBarrier
 \label{sec:etfr}
This evolution in rates can, as with ASFR, also be
summarized with an indicator akin to TFR, which we here call $e$TFR
\begin{equation}
\label{eq:exTFR}
e\mathrm{TFR} = \int _{y=0}^\infty F_y \dd y
\end{equation}
where $y$ indexes remaining years of life. A time series of this indicator
for the period studied is displayed in Figure~\ref{fig:exTFR}.

\begin{figure}[ht!]
        \centering  
          \caption{Male and female $e_y$-total fertility rates versus standard
          TFR, USA, 1969-2009 and Spain, 1975-2009}
           % figure produced in /R/Parents_ex.R
           \includegraphics{Figures/exTFR}
          \label{fig:exTFR}
\end{figure}

Canonical TFR can conveniently be imagined as the total number of
offspring that an average female (male) will have in a lifetime assuming
no mortality and constant fertility rates of the present year.
Since a lifetime measured in age counting from birth is the same length as a
lifetime measured in age counting backward from death, $e$TFR in fact has the
same interpretation. Why is this? Age-classified rates are of course
heterogeneous within age with respect to remaining life expectancy, and here we
have produced a synthetic index based on the reverse idea. The age-classified
distributions of births and populations are quite different (there being an age
pattern to fertility rates). $e_y$-reclassifying these data not only changes the
center of gravity of numerator and denominator distributions, but asymmetrically
shifts underlying schedules, uniquely reshaping the pattern of
fertility. Summing over $e_y$-rates, however, yields a different total -- our
synthetic $e$TFR. 

Figures~\ref{fig:exSFRsurfES}, ~\ref{fig:exSFRsurfUS}, and~\ref{fig:exTFR} are
reproduced according to various definitions of the reproductive span in
Appendix~\ref{Appendix:reprospans2}. Rates are shown to be sensitive to the
choice of reproductive span. For the remainder of this dissertation, we ignore
issues of age boundaries in the reproductive span for simplicity and
consistency, although this issue deserves further attention if the
remaining-years perspective is deemed to have merit.

\FloatBarrier
    
    \subsection{Heterogamy by remaining years}
       \FloatBarrier
\label{sec:exobsexpected}
First, note that the observed joint $e_y$-distribution of birth counts is
very nearly identical to the expected distribution.\footnote{The expected
distribution is defined as in Equation~\eqref{eq:expected}, which assumes
cross-proportionality between the male and female marginal distributions.}
Figure~\ref{fig:US1970obsexpex} compares these two distributions for 
birth counts in the US in 1970 (compare with Figure~\ref{fig:US1970obsexp}). 

\begin{figure}[ht!]
        \centering  
          \caption{Observed and expected joint distribution of birth
          counts by remaining years of parents, 1970, USA}
           % figure produced in /R/Parents_exCross.R
           \includegraphics{Figures/ObservedvsExpectedBexey}
          \label{fig:US1970obsexpex}
\end{figure}

It is
difficult to see any difference between the two surfaces in
Figure~\ref{fig:US1970obsexpex}; however, we can measure the degree of
separation, $\theta$,\footnote{See Equation~\eqref{eq:coefdiff}. Recall that 0
signifies perfect overlap and 1 signifies perfect separation between the two
distributions} just as for age-classified births (compare with
Figure~\ref{fig:Theta}). One provisionally concludes that $e_y$-matching of
parents, at least with this level of approximation, appears to be very close to
random\footnote{Confidence bands used in Figure~\ref{fig:TotalVarobsexpex}, as
elsewhere in this dissertation for difference coefficients, represent the
central 95\% of randomly generated $\theta$ values using Monte Carlo
simulations. The present case differs from earlier simulated confidence bands in
that age-classified death counts and age cross-classified birth counts 
are first drawn from Poisson distributions, with
observed counts taken as $\lambda$. $\mu_a$ is then derived from the randomly
generated death counts using exposures from the HMD, and $d_a$ is derived from
$\mu_a$. The simulated $d_a$ is then used to redistribute the randomly generated
age cross-classified births distribution by remaining years of life, which is
then compared with its own expected distribution, producing the random $\theta$.}. 
When compared using the
Kolmogorov-Smirnov test, in fact, one cannot under even the most generous level
of significance conclude that these two observed distributions come from
different theoretical distributions.

\begin{figure}[ht!]
        \centering  
          \caption{Departure from association-free joint distribution of
          birth counts cross-classified by $e_y$ of mother and father. USA,
          1969-2010 and Spain, 1975-2009}
           % figure produced in /R/Parents_exCross.R
           \includegraphics{Figures/TotalVariationObsvsExpectedexUSES}
          \label{fig:TotalVarobsexpex}
\end{figure}

Since the joint distribution by mothers' and fathers' $e_y$ is so close
to random, one could very closely replicate the full cross-classified matrix 
given only the two marginal
$e_y$ birth distributions by applying Equation~\eqref{eq:expected}. 

\FloatBarrier
      
    \subsection{Divergence in predicted birth counts}
       \FloatBarrier
\label{sec:exdivergence}
It has been noted that the observed and expected distributions of births by
remaining years of life of mothers and fathers very closely resemble each other
(see Figure~\ref{fig:TotalVarobsexpex}), almost enough so that we could
approximate the observed distribution by a random distribution given only the
marginal distributions. In any case, the result would be a much closer fit to
observed data than would be the same excercise if undertaken with typical
age-classified data. 

Further, it has been noted that the population pyramid is
much more stable (in the sense of less year-to-year
distributional variability) when classified by remaining years
of life than when classified by age. This is so because the $e_y$-classified 
pyramid does not uniformly decrement in single-year steps, due
to well-known and apparently stable trends of year-on-year mortality improvement
that have thus far shown no signs of abatement. Intuitively, the central bulge
in an $e_y$-classified population pyramid does not plummet to the base of the 
pyramid at a rate of 1
year per year, but rather much more slowly and smoothly, always leaving a
tapered base to the pyramid (the population expected to decrement soon), as
mortality improvements also lead to new births being incremented to ever-higher
$e_y$ values. In this way, the $e_y$-pyramid, at least in populations that have
radically reduced infant and child mortality and have late-transition fertility 
levels,\footnote{These two characteristics typically co-occur
\citep{macinnes2009reproductive}, and both conditions hold for the US and Spain
in the years presented here.} tends to obtain a characteristic \textit{leaf}
shape.

Since the $e_y$ population distribution can be predicted
with nearly equal facility and precision as the age-distributed population in
year $t+1$, one may ask whether, given the relative stability of underlying
exposures for both males and females, $e_y$-specific fertility rates are 
also more stable than
age-specific fertility rates from year to year. If this is so, then predicting 
birth counts $n$ years hence separately for the sexes based
on year $t$ $e$SFR and year $t + n$ $e_y$-classified exposures has the
potential to entail lower disagreement in predicted birth counts derived
from male and female rates and exposures than does the age
anologue of this same exercise.

If results show that projected divergence in predicted birth counts, holding
single-sex fertility rates constant, is lesser for the $e_y$-classified data
than for age-classified data, then we can safely say that the two-sex problem
has been reduced in size, albeit not solved. In the case that the magnitude of
the problem has been reduced by this simple transformation, one further
concludes that whatever empirical or axiomatic drawbacks entailed by 
two-sex adjustment procedures presently in the literature will also be reduced,
thereby making the two-sex problem in practice less problematic. 

This exercise has been carried out for both the US and Spain with $n$ equal to
1, 5, 10, and 15. In summary, for the US (see Table~\ref{tab:USexDiscrepancy}),
the sex-discrepancy entailed by $e_y$-classified data is on the order of five
times smaller than for age-classified
data, a welcome improvement. Further, the
$e_y$-method for the US entails sex discrepancies that vary roughly around zero, 
whereas age-classfied data were
always positively biased in the period studied. For Spain (See
Table~\ref{tab:ESexDiscrepancy}), we notice no difference in the magnitude of 
discrepancy, but indeed in the sign of
discrepancy.

% table of mean relative difference over whole span.
% US
\begin{table}
\caption{Relative discrepancy between single-sex projected births $n$ years
hence using $e_y$- versus age-classified data US, 1969-2009}
\label{tab:USexDiscrepancy}
\centering
\makebox[0pt][c]{\parbox{1.1\textwidth}{%
    \begin{minipage}[b]{0.45\hsize}
    \centering
        \caption*{Mean Relative Discrepancy}
        % latex table generated in R 2.15.2 by xtable 1.7-0 package
% Wed Feb 27 17:14:41 2013
\begin{tabular}{l|cc}
  \hline
 & $e_x$ & Age \\ 
  \hline
1-year & -0.0002 & 0.0038 \\ 
  5-year & -0.0024 & 0.0202 \\ 
  10-year & -0.0073 & 0.0444 \\ 
  15-year & -0.0131 & 0.0705 \\ 
   \hline
\end{tabular}

    \end{minipage}
    \hfill
    \begin{minipage}[b]{0.55\hsize}
    \centering
        \caption*{Mean Absolute Relative Discrepancy}
        % latex table generated in R 2.15.2 by xtable 1.7-0 package
% Mon Mar  4 13:33:41 2013
\begin{tabular}{rr}
  \hline
$e_x$ & Age \\ 
  \hline
0.0022 & 0.0049 \\ 
  0.0072 & 0.0248 \\ 
  0.0106 & 0.0505 \\ 
  0.0145 & 0.0743 \\ 
   \hline
\end{tabular}

    \end{minipage}
}}
\end{table}

% Spain
\begin{table}
\caption{Relative discrepancy between single-sex projected births $n$ years
hence using $e_y$- versus age-classified data, Spain 1975-2009}
\label{tab:ESexDiscrepancy}
\centering
\makebox[0pt][c]{\parbox{1.1\textwidth}{%
    \begin{minipage}[b]{0.45\hsize}
    \centering
        \caption*{Mean Relative Discrepancy}
        % latex table generated in R 2.15.2 by xtable 1.7-0 package
% Wed Feb 27 17:14:41 2013
\begin{tabular}{l|cc}
  \hline
 & $e_x$ & Age \\ 
  \hline
1-year & -0.0029 & 0.0036 \\ 
  5-year & -0.0168 & 0.0193 \\ 
  10-year & -0.0403 & 0.0401 \\ 
  15-year & -0.0641 & 0.0632 \\ 
   \hline
\end{tabular}

    \end{minipage}
    \hfill
    \begin{minipage}[b]{0.55\hsize}
    \centering
        \caption*{Mean Absolute Relative Discrepancy}
        % latex table generated in R 2.15.2 by xtable 1.7-0 package
% Mon Mar  4 13:33:41 2013
\begin{tabular}{rr}
  \hline
$e_x$ & Age \\ 
  \hline
0.0048 & 0.0047 \\ 
  0.0204 & 0.0238 \\ 
  0.0419 & 0.0437 \\ 
  0.0641 & 0.0633 \\ 
   \hline
\end{tabular}

    \end{minipage}
}}
\end{table}



    \subsection{Fisher's reproductive value-- flipped}
      \FloatBarrier
\label{sec:fisher}
We wish to mention in passing that thanatologically structured fertility rates,
$F_y$, are just a short step away from a remaining-years version of
\textit{Fisher's} \citeyear{fisher1999genetical} reproductive value.\footnote{I
thank Robert Chung for suggesting that I think about this.} Note that $y$ indicates the
temporal distance from death, and that therefore no decrement occurs unless $y=0$. 
Hence, $F_y$ is already in a sense discounted for mortality. The
reproductive value, $v_y$, in this setting becomes
\begin{equation}
\label{eq:eyfisher}
v_y = \frac{\int _0 ^y F_y \dd y}{\int _0 ^\infty F_y \dd y} 
\end{equation}
In other words, $v_y$ is the proportion of reproduction that remains between
remaining years, $y$, and death, $y=0$. As with the age-specific
reproductive value, $v_x$, this value is the same in the present and stable
populations. The age-structured variant of this indicator has been of great 
value in the field of biology and in evolutionary theory, and one wonders
if the present definition might be of explanatory use. In the age-perspective,
we expect selective pressures on population to be greatest where the reproductive value,
$v_x$, is highest (equal to 1). Under the standard view, the ages
where $v_x = 1$ also are the ages where we observe mortality minima, as well as
negative senescence. Indeed, mortality is at its minimum right before the
onset of fertility (for females) -- a better bet for the species in terms of
reproductive output than for infants, though these have the same reproductive
value by the $v_x$. Under $v_y$, things are not so clear -- in the highest
values of remaining years, the indicator will obtain the value of 1, and it will
fall off less precipitously with the approach of death, never fully dropping to
0, because remaining-years classes are heterogeous with respect to age
and therefore even very low values of $y$ will contain some reproduction.
Females, of course, do come much closer to $v_0 = 0$ than males, due to
menopause. Figure~\ref{fig:vyUS1990} provides example male and female $v_x$ and
$v_y$ curves for the US in 1990.

 \begin{figure}
                \centering
                \caption{Age-specific and remaining-years specific reproductive
                values, US, 1990}
                \includegraphics{Figures/vyUS1990}
                \label{fig:vyUS1990}
\end{figure}

Indeed, the two-sex problem has been brought to bear on the concept of
reproductive values \citep{samuelson1977generalizing}. In brief, one may
reasonably ask, what is the reproductive value of a 25-year-old male if there
are no females around? Hence the need for a two-sex solution. In-depth
thinking on the consequences for reproductive values of the present age
transformation is beyond the scope of this dissertation, much less how two-sex
solutions may be of use to the definition of more comprehensive reproductive
values. We plant this seed and move on.

\FloatBarrier

  \section{Reflections on remaining-years-structured vital rates}
      \FloatBarrier

This chapter aimed to introduce a new variety of population structure--
thanatological age, or age counted as time until death. The basic steps needed
to carry out the transformation presented in Equation~\ref{eq:dxredist} were
indeed available to demographers via the work of
\citet{miller2001increasing}, \citet{lee2002approach}, \citet{vaupel2009life}
and possibly others unbeknownst to this author. Indeed the perspective as a
whole is widespread in demography, this concept of age as time until death
was already in existence, and the name given to it is of known coinage. Our only
addition was to integrate over age, arriving at a wholly 
redistributed population (count, rate) vector entailing loss of age
information.\footnote{If this has indeed been done before, then this author has
not seen it, and apologies will be issued} In a sense this is more of a loss than an addition,
as we throw away information in doing so. 

It has been with fresh eyes that we have investigated fertility rates in terms
of thanatological age. Recall the words of Coale:
``One of the characteristics of demographic research is a search for empirical regularities, particularly in the
age-schedules of rates of marriage, birth, death, and
migration'' \citep{coale1996development}. We may claim to have found
empirical regularities in the restructured fertility rates, although further
work is needed to verify this for a wider range of populations and years.
Indeed, these patterns appear to be parametrically tractable -- we mark this 
possibility with a cairn and continue onward. There is in any case no need to smooth data that
have been redistributed in this fashion! However, if it turns out that the
remaining-years pattern to fertility is more regular, and hence more
predicatable, than the age-pattern to fertility, one may find a model curve to
be of use. We have not given an explicit account of the regularity of
$e_y$-structured fertility over time, though a glance at the surfaces in
figures~\ref{fig:exSFRsurfUS}~and~\ref{fig:exSFRsurfES} invites speculation in
this direction.

We have also found that the cross-classified remaining-years births distribution
is regular to the extent that it greatly resembles its own association-free
distribution. Either this observation is indeed so, or it is the result of the
overly permissive assumption of homogeneity with respect to mortality in
procreating pair formation. Namely, if there is noticeable selection with
respect to mortality fitness in the mate search, which is rather imaginable
\citep{gangestad1993pathogen, roberts2008good}, the patterns seen in
Figure~\ref{fig:US1970obsexpex} will not reflect the true distribution. The
question is then how far off our homogeneity assumption is. This could be
answered by means of linked register data, as has often been done for countries
such as Sweden and Denmark. We may already surmise that procreating individuals
are on average more fit than non-procreating individuals. To the extent that
marriage is a proxy for mating, we already know that married males have a
mortality advantage over unmarried males, although
there are likely intervening factors \citep[see
e.g.,][]{rogers1995marriage, waite1995does}, and we do not know for sure whether
this is due to causation or selection \citep[see e.g.,][]{goldman1993marriage}.
For the present use, we do not necessarily care whether the differential might be due to
selection or causality. All this is to say that we ought to take in the
observations of cross-classified fertility with a grain of salt. Furthermore, as
is pointed out in the text, selection may disproportionately affect the tails 
in the lowest remaining years classes.\footnote{This possibility would be just
the opposite in high maternal mortality settings, which does not affect
our two populations.} The best we can do in this instance is imagine the
direction of bias, as has we have tried to make clear.

In general, we have seen that period fertility indices calculated using
remaining-years classified data track well to those calculated age-classified
data, but also that they return higher estimates of period fertility. At
first glance, one would expect the direction of difference to be negative 
rather than positive because we took the simplest possible course of
redistributing population counts (or exposures) from all age classes rather than
only from age classes in ages typically identified as reproductive ages. The
outcome is positive because the age-distributions of marginal birth counts and
population counts are different, but redistributed by the same lifetable death
distribution. We ought not expect differing initial distributions to wind up
proportionally in the same remaining years classes. Hence, the sum of these
ratios (events to exposures) over remaining years is also different. TFR, for
instance, is higher for remaining-years structured data than the age-equivalent
measure because on average there is less exposure per birth in the newly
redistributed data.

One may question whether there is any sense in allowing non-reproductive ages
into remaining-years exposures, and to this we have two responses. The first
comes by manner of analogy to age-specific fertility distributions, which
have tails that are longer than most people feel comfortable imagining, 
both for males and females. For males and
females there are so-called central ages of reproduction and there are
less-common ages of reproduction. Even for ages typically measured, say
45-49 for females, not all persons captured in these exposures are truly
exposed to the risk of fertility, and everyone knows this. Indeed a minority
of females in this age group are truly at risk of fertility. This does not imply
that the rate calculated is invalid, but rather that the rates are both low and
potentially subject to further decomposition. This is one reason why a large
portion of fertility studies deal with the measurement of the proximate determinants of fertility-- these
factors differentiate exposure for purposes of calculating more specific rates.
Age-specific rates are valid without such differentiation, as they help purge
our measurement of distortions from a certain degree of population structure. 

As \citet{stolnitz1949recent} so eloquently describe, demographic rates are
never fully purged of population structure. We may nonetheless \textit{de}structure in
any of myriad ways, and remaining-years structure is the way that we have shed
light upon in this dissertation. No claim is made about whether chronological or
thanatological age are the more efficient classifying variable for population
data, nor that one is more pure than the other. In sum, we have in this
dissertation provided yet another alternative structuring variable, and if one
questions the validity of a rate calculated on the basis of blended age groups,
one may ask: 1) how many persons are expected to die in $y$ years and 2) how
many births this year were to persons expected to die in $y$ years, and there we
have everything needed to calculate a rate. This is what we have done. Further
refinements are possible and are worthy of exploration.

The second response to the potential criticism about overly permissive
heterogeneity within remaining years classes for purposes of producing
fertility rates is that these rates nonetheless appear to obey a
certain degree of empirical regularity. The range of shapes possible for
fertility rates by remaining life expectancy, at least absent a mortality
crisis, is narrow. This we confidently claim, but the claim is in need of
further empirical support beyond this dissertation. The same observation may be made of
the underlying population structure by remaining years of life-- the range of 
shapes that we have observed this structure to obtain in Western countries, 
excluding mortality crises, is
narrow. These observations compel one to conclude that
remaining-years-specific fertility rates ($e$SFR) describe a substantive
underlying phenomena. If age-heterogeneity in remaining years classes 
were problematic in this sense, we would expect more erratic patterns in
$e$SFR over time and between populations, but this is not so. Instead, $e$SFR
invites comparison between populations and over time, and such comparisons are
decomposible into mortality and fertility components. We have not
taken this latter exploratory step, but this remains an exciting question, since
a stable $e$SFR pattern may hold over time even as fertility age patterns and
mortality levels change.

Most empirical findings arising from this thesis are based on data only
from the years 1969-2009 for the United States and 1975-2009 for
Spain\footnote{The primary exception is the relative \textit{stability} of
observed remaining-years-structured populations versus age-structured
populations, which we verified using the entire HMD.}. Evidently, any
novel findings produced therefrom, especially those dealing with fertility, 
are in need of verification from a wider array of
populations, so as to be placed into context. This is to say that there is ample opportunity to test
and refine the hypotheses produced here. Certainly $e$SFR will move over time,
but we suppose that this movement is meaningful and will aid our understanding
of population processes.

Net of the potential criticism of age-heterogeneity in remaining-years classes,
there are reasons to suspect that fertility timing and levels might also be
a function of remaining years of life. Although one's own age at death is
unknown, individuals do have a \textit{sense} of their own remaining life
expectancies, and this may condition fertility. This is not to say that people 
predict their own lifespans without bias. \citet{payne2013life}, for example,
concludes that the framing of the very questions used to asses subjective
remaining life expectancy in survey respondents can bias results greatly. This
source of bias has more to do with eliciting respondents' beliefs and
perceptions, although there are also biases in subjective predictions that vary
over age. The important thing is that people's subjective expectations of
remaining life \textit{correlate} positively with real remaining life
expectancies, and much evidence has been produced that would support
this \citep{hurd1995evaluation, mirowsky1999subjective, hurd2002predictive,
perozek2008using, delavande2011differential, post2012longevity}. These authors
tend to explain individuals' predictive power in terms of awareness of
their own particular genetic and environmental situation. Most people
understand the general mortality increase over age, and are able to
weigh these factors out to produce a rough idea of future death 
probabilities in certain age intervals. These findings have largely come from
health and retirement surveys of older persons beyond reproductive ages. Of
course, it is difficult to measure how well persons in \textit{reproductive}
ages are at predicting their own ages at death because panels would need to
run over a very long range of years in order to asses the accuracy of
predictions of the relatively young. 

Another particular variety of finding that lends support to our case are
a couple studies of risky behavior in youth. \citet{borowsky2009health} show
that risky behavior in teenagers increases with high perceived risks of early death, and these
behaviors include risky sexual behavior. \citet{wilson1997life} conclude that
high real mortality risks for youth due to \textit{perceived} factors such as 
accidents, suicide and homicide, predict reductions in the age of fertility, which at
least on the surface, is consistent with the idea of $e$SFR actually describing 
some sort of underlying pattern. 

None of these studies provide
direct support of our hypothesis that we ought expect that fertility behavior is
somehow a function of remaining life expectancy, but they do help our case.
The kind of data that would help shed direct light on this question would be a
panel study that includes questions about expected probabilities of survival
until (or death by) certain future ages, as well as the usual battery of 
fertility questions. This could be incorporated, for example, in future waves of
longitudinal youth surveys. To suit our ends directly, this survey would follow
up with actual ages of death. Evidently, much patience will be required to reap
any results in this direction, and there is much uncertainty at this preliminary
stage (we are only imagining such a survey at the moment) about whether results
would provide an estimate of true remaining-years fertility curves, and whether
foresight of one's own mortality moves this curve. Removing subjective
remaining life expectancy from our demands, one may reconstruct cohort $e$SFR on 
the basis of long-running population registers from countries such as Denmark or
Sweden.

Absent a well-rounded base of psychological, sociological and demographic
evidence in support of the belief that fertility patterns ought be a function of
remaining years of life (as well as a function of age), we rely on our own
set of evidence that simply show these patterns to be stable. Stable demographic
patterns are desirable, as they are useful for predicting. We encourage
the exploration of the predictive power of remaining-years-specific
demographic rates, just as economists have been keen to look for patterns in
retirement saving and investment as a function of perceived remaining years of
life.

We completed the chapter with some rather indefinite speculation about the
potential use of our $v_y$, Fisher's reproductive value rethought to correspond
to the thanatological perspective, in cognate fields of demography, or
evolutionary demography itself. One could just as well reframe $v_y$ in terms of
inter-age transfers rather than reproduction, as per \citet{lee2003rethinking},
although the perspective change will still pertain.

The purpose of this chapter was to define and explore the vital rates to be used
in the remained of this dissertation. We will proceed by defining a model of
population renovation akin to Lotka's renewal equation. From this model we will
extract and explore the intrinsic growth rate and some other stable
parameters that belong to the thanatological perspective. 

\FloatBarrier





\chapter{Renewal in single-sex populations structured by remaining years of
life}
     \FloatBarrier
Recall the observations of \citet{sanderson2005average}, more directly
relating to the popuation pyramid. These authors note that despite ageing in a
population, the mean \textit{remaining} years to be lived may increase. This is due 
to improvements in mortality offsetting (or more than offsetting) increases in
the mean age of a population. In general, a population looked at from the
perspective of life expectancy looks different from, behaves differently from,
and yields complimentary information to one looked at from the canonical
perspective of age. This observation is the point of departure for the
linear and non-linear two-sex models that I will introduce in this dissertation.
However, such steps are large, and require an involved demonstration of some
key differences between age clsasified and \textit{remaining years}-classified,
henceforth $e_x$-classified demographic data. Indeed a whole new one-sex
model must first be developed prior to re-delving into its two-sex extensions. 
I will first present a method
to exactly redistribute population counts (or exposures) according to period 
remaining years of life, rather than according to age per se. A reexamination 
of recent fertility patterns according to remaining years of life will follow.

    \section{Age-structured renewal}
       \FloatBarrier
The age-structured pyramid shifts upward by 1 year with
each passing year, with some decrement occurring in each age of life, such that
the essential shape, primarily the fruit of past fertility\footnote{Credit is
owed to Kirk Scott for first imparting me with this heuristic.}, takes two or
more generations to be erased from memory\footnote{Credit is
owed to Anna Cabre for first imparting me with this heuristic.}. Births from the
age structured population fall to the bottom of the pyramid, and are grouped 
together into a cohort. This cohort is heterogenous with respect to future 
age (year) at death, but is
homogenous with respect to the year of birth. We are familiar with
the way the age-structured population model unfolds, as it
reflects both our experience of life and history of demography.
    
    \section{Remaining-years-structured renewal}
    % this includes the diagram
       \FloatBarrier
\label{sec:exrenewal}
We begin by describing renewal in age-structured populations, using the
population pyramid as a mental image. The description might appear to be a
statement of the obvious, but it serves as a guide to the following description of
$e_y$-structured renewal, which is not at first glance intuitive. The
age-structured pyramid shifts upward by one year with each passing year, with 
some decrement occurring in each age of life, such that the essential shape, 
primarily the result of past fertility,\footnote{Thanks are owed to Kirk Scott 
for first imparting this heuristic to me.} takes several decades to be erased 
from memory.\footnote{Thanks are owed to Anna Cabr\'{e} for
first imparting this heuristic to me.} Births from the age structured population 
are produced by a wide range of ages in the population pyramid; these
are assigned to the bottom of the pyramid and are grouped together into a single
cohort. This cohort is heterogeneous with respect to future age (year) at death, 
but it is homogeneous with respect to year of birth. We are familiar with
the way the age-structured population model unfolds, as it reflects both our
experience of life and the history of demography. The key characteristics are to
note where on the pyramid increment and decrement occur, and the direction of
movement in the pyramid with each passing year.

The $e_y$ structured pyramid, on the other hand, (see Figures~\ref{fig:exPyrUS}
and \ref{fig:exPyrES}) shifts down by one year each year. There are no deaths,
except for in the bottommost layer, those whose $y = 0$. Those with a life
expectancy of 20 move the next year into 19, and so forth, experiencing
increments from newly added births, but no decrement to mortality. 
Each $e_y$ class is heterogeneous with 
respect to year of birth (age) but homogeneous with respect to remaining 
years of life, forming what could be called a \textit{death cohort}. Fertility
can arise from individuals with nearly any remaining life expectancy; the 
age-boundedness of fecundity belongs to the age
perspective of demography. Thus the entire pyramid produces 
offspring.\footnote{The only exception
to this statement is the very top of the $e_y$-pyramid, consisting only of
pre-menarchical girls and pre-semenarchical boys who will have very long
lives.} Total births, $B$, are proportioned to the pyramid using the ``radix-1''
deaths distribution, $d_x$; for example, $P_{e_1}$ is incremented by $d_1 \cdot
B$, and so forth for all ages, adding a new layer whose total over $y$ equals $B$. In 
this way births
increment most heavily around the modal age at death, typically very high in the
pyramid, depending on the year and population. Some are
unfortunate and decrement out of the pyramid in the same year as they are
incremented (births where $y = 0$). See Figure~\ref{fig:exrenewal} for a
schematic visualization of $e_y$-structured population renewal.

\begin{figure}[ht!]
\begin{adjustwidth}{-1in}{1in}
        \centering  
          \caption{Schematic diagram of the renewal process in a population
          structured by remaining years of life.}
           % figure produced in /R/exRenewalDiagram.R
           \includegraphics[scale=.95]{Figures/exRenovationDiagram.pdf}
          \label{fig:exrenewal}
          \end{adjustwidth}
\end{figure}

%The elemental formula on the right side of the Figure~\ref{fig:exrenewal}
%diagram says that remaining years-structure births ($B_{y,t}$) are calculated
%on the basis of remaining years-structured fertility rates ($F_{y,t}$) and
%exposures ($E_{y,t}$), summed to total births ($B_t$) and then distributed over
%the pyramid according to the deaths distribution ($d_y = d_x$).
\FloatBarrier
    
   % break sections out here for separate \input
    \section{The single-sex renewal equation}
       \FloatBarrier
Given the renewal process described above, it is perhaps now intuitive to see
that the stable structure of the $e_y$-structured population is determined
primarily by the deaths distribution and the rate of growth of the
population. Indeed, upon transforming fertility rates to the earlier-presented
$e$SFR, one is just a few short steps away from a full Lotka-type renewal
model, namely (for females):

\begin{align}
\label{eq:exLotkafemales}
1 &= \int _{y'=0}^\infty \int _{a'=y'}^\infty e^{-ra'} d_{a'}^F f_{y'}^{F-F} \dd
a' \dd y'
\end{align}
where $a'$ indexes female age, $y'$ indexes female remaining years of life,
$d_{a'}^F$ is the age-distribution of female deaths from the radix-1 period
lifetable, and $f_{y'}^{F-F}$ are exact female-female (mother-daughter)
fertility probablilities by remaining years of life ($e$SFR, see Equation~\eqref{eq:eSFR}). Likewise for males:

\begin{align}
1 &= \int _{y=0}^\infty \int _{a=y}^\infty e^{-ra} d_a^M f_y^{M-M} \dd a \dd y
\end{align}

Equation~\ref{eq:exLotkafemales} is indeed similar to the
original age-structured Lotka equation, introduced in
Equation~\eqref{eq:lotkaeq}. First, note that the survival function $p_a$
inside Equation~\ref{eq:lotkaeq} can also be expressed in terms of
$d_a$ (current livings are the sum of future deaths):

\begin{equation}
p_a = \int _{x = a} ^\infty d_x \dd x
\end{equation}
in which case Equation~\ref{eq:lotkaeq} can be rewritten as:

\begin{align}
\label{eq:lotkadx}
1 &= \int _{a=0}^\infty \int _{b = a}^\infty e^{-ra} d_b m_a \dd b \dd a
\end{align}

All we have changed in order to derive Equation~\ref{eq:exLotkafemales}
is to turn $l_a$ and $m_a$ sideways, so to speak, multiplying the two vectors
together where they coincide in terms of remaining years instead of in terms of age. This
transformation is a simple change of perspective. $r$ still applies to sucessive 
time steps, but in terms of remaining years of life, it must be applied incrementally 
over the newcomers to
each grouping of remaining years of life, i.e. over the time-layers of the
$e_y$-structured pyramid.

      
      \subsection{An iterative approach to find $r$}
         \FloatBarrier
\label{sec:exrenewalit}
\citet{coale1957new} offers a fast-converging iterative approach to estimate the
instrinsic growth rate for age-structured populations. For the $e_y$-structured
renewal equation, a similar approach may be designed, with some slight
modifications. The following steps can be followed to estimate $r$ from
Equation~\ref{eq:exLotkafemales}:

\begin{enumerate}
  \item Derive a first rough estimate of the mean remaining years of life at
  reproduction, $\widehat{T^y}$, akin to Lotka's mean generation time, $T$. If
  one assumes a growth rate of $0$, then a good guess will be:
\begin{equation}
\widehat{T^y} = \frac{\int _{y=0}^\infty \int _{a=y}^\infty y d_a f_y \dd a
\dd y}{\int _{y=0}^\infty \int _{a=y}^\infty d_a f_y \dd a \dd y}
\end{equation}
 \footnote{$\widehat{T^y}$ appears to range between 50 and 70, judging by the
 two populations studied in this dissertation. \textit{True} $T^y$ is around 10
 years lower, ranging from 40-50.}
  \item A first rough guess at the net reproduction rate, $R_0$ is given by:
 \begin{equation}
  R_0 = \int _{y=0}^\infty \int _{a=y}^\infty d_a f_y \dd a
\dd y
\end{equation}
  \item A first rough estimate of $r$, $r^0$, is given by:
   \begin{equation}
   r^0 = \frac{ln(R_0)}{\widehat{T^y}}
   \end{equation}
  \item Plug $r^0$ into Equation~\ref{eq:exLotkafemales} to calculate a
  residual, $\delta^0$
  \item Use $\delta^0$ and $\widehat{T^y}$ to calibrate the estimate of $r$
  using:
  \begin{equation}
  r^{1} = r^0 + \frac{\delta^0}{\widehat{T^y} - \frac{\delta^0}{r^0}}
  \end{equation}
  \item Repeat step (3) to to derive a new $\delta^i$, then step (4) to refine
  $r^i$, until converging on a stable $r$ after some 30 iterations,
  depending on the degree of precision desired ($\widehat{T^y}$ is not updated
  in this process).
\end{enumerate}

The above procedure is both faster and more precise than minimizing the absolute
residual of Equation~\ref{eq:exLotkafemales} using a generic
optimizer\footnote{Use of a Newton-Raphson optimizer with analytic objective
and gradient functions may prove even more efficient, but I have not tried
this, since the present routine is more than efficient enough for practical
purposes.}.

        
      \subsection{Other stable parameters}
         \FloatBarrier
A final calculation of $T^y$ is given by:

\begin{equation}
\label{eq:Ty}
 T^y =  \frac{\int _{y=0}^\infty \int _{a=y}^\infty y e^{-ra} d_a f_y \dd a
\dd y}{\int _{y=0}^\infty \int _{a=y}^\infty e^{-ra} d_a f_y \dd a \dd y}
\end{equation}
, using $r$ from the iterative procedure. The net reproduction rate, $R_0$ is
related by, e.g.:

\begin{equation}
\label{eq:R0fromTy}
R_0 = e^{r T^y}
\end{equation}

The birth rate, $b$, is given by:

\begin{equation}
b = \frac{1}{\int _{y=0}^\infty \int _{a=y}^\infty e^{-ra} d_a \dd a
\dd y}
\end{equation}

The stable age structure, $c$, where $c_y$ is the
proportion of the stable population with remaining years to live $y$, is given
by:

\begin{equation}
c_y = b \int _{a=y}^\infty e^{-ra} d_a \dd a
\end{equation}

Other possibly interesting stable parameters may be estimated by
similarly translating the various definitions in the glossary of
\citet{coale1972growth} to the present perspective. Before presenting 
results or extending the present one-sex renewal
formula to two-sex linear and non-linear situations, the heart of this 
thesis, we first describe the construction of the Leslie matrix that corresponds to the
present model.
      
    \section{The single-sex projection matrix}
       \FloatBarrier
\label{sec:ex1sexleslie}
This section explains the construction of the projection matrix that corresponds
to the one-sex $e_y$-structured population model presented above. The objective is
to offer a practical discrete implementation of the prior
formulas, which may aid the reader in understanding main differences with the
classic one-sex Lotka renewal model and be of practical use for projections.
Matrix-based projections, while not ubiquitous in the practice of demography,
are nonetheless widespread and of high analytic utility. While the species of
matrix presented here is indeed used in data exercises elsewhere in this
dissertation (notably Section~\ref{sex:doublingex}), its properties will not be
explored beyond the construction advise given in this section. It is hoped that
the present section will facilitate exploration of the present stable system for
the interested reader. The only computational requisite is a statistical
environment that supports matrix operations, such as
\texttt{R}\citep{Rcitation}\footnote{\texttt{R} is the language used behind the
scenes for all computations and Figure production in this dissertation} or
\texttt{matlab}\citep{MATLAB:2010}.

If the reader is not familar with the construction
of age-structured Leslie matrices, a brief description may be found in
Appendix~\ref{Appendix:Caswell}, which is essentially a paraphrase of the
detailed description offered in \cite{caswell2001matrix}. As with
age-structured Leslie matrices, $e_y$-structured projection matrices,
$\textbf{Y}$, are square and of dimension $n \times n$, where $n$ is the number
of remaining years classifications into which the population is divided. The matrix contains
elements for survival and elements for fertility. Unlike Leslie matrices,
$\textbf{Y}$ is not sparse, but is primarily populated with non-zero entries.

Recall the description of renewal in an $e_y$-structured population offered in
Section~\ref{sec:exrenewal} and illustrated in Figure~\ref{fig:exrenewal}. Of
interest is that mortality only occurs in the population class with zero
remaining years of life. $e_y$-class 1 in year $t$ moves to 0 in year $t+1$.
In this way, populations shift \textit{down} rather than up with each time iteration.
Thus, instead of in the subdiagonal, we place survival in the superdiagonal, and
indeed all survival values are 1, since there is no decrement, and the
upper-left corner contains no entry for survival. As in
Appendix~\ref{Appendix:Caswell}, we illustrate using a 6$\times$6 matrix. The
survival component of $\textbf{Y}$ is organized as in
Matrix~\ref{matrix:ex1sexsurvival}.

\begin{matrix}[h!]
\centering
\caption{Survival component of one-sex remaining years
($e_y$)-structured projection matrix, $\textbf{Y}$} 
\label{matrix:ex1sexsurvival}
$\bordermatrix{{e_y } & 0_t & 1_t & 2_t & 3_t & 4_t & 5_t\cr 
                0_{t+1} & 0    &  1   & 0    & 0    & 0    & 0   \cr
                1_{t+1} & 0    &  0   & 1    & 0    & 0    & 0   \cr 
                2_{t+1} & 0    &  0   & 0    & 1    & 0    & 0   \cr 
                3_{t+1} & 0    &  0   & 0    & 0    & 1    & 0   \cr 
                4_{t+1} & 0    &  0   & 0    & 0    & 0    & 1   \cr
                5_{t+1} & 0    &  0   & 0    & 0    & 0    & 0   }$
\end{matrix}

 Fertility inputs to the matrix are derived from $e$SFR and the lifetable $d_x$
 distribution, where $x$ indexes age, but is translated to $y$, remaining years
 of life. Recall that fertility in an $e_y$-structured population occurs in all
 but the highest remaining years classes. Say for our example, that fertility is
 observed in classes 0-4, whereas the final class has no fertility. Where $f_y$
 indicates the fertility probability for class $y$ in the year $t$ entering
 population (in the matrix columns). Each $f_y$ is then distributed according to
 $d_x$, indeed with no further translation, since the $d_x$ column refers to age
 0, as such. Thus the fertility entry in row $m$ and column $n$ of $\textbf{Y}$
 will be $f_n \cdot d_m$. We assume that those dying over the course of year
 $t$ (the first column) are exposed to fertility for \textonehalf ~of the
 year\footnote{One might be tempted to not allow for fertility at all for
 females dying in year $t$, but recall that fertility is measured in the moment of
 birth, and not conception.},
 and so discount the fertility entry accordingly. Further, infant mortality, 
 $f_y \cdot d_0$, located in the first row, must also be discounted, since part
 of the mortality will occur in the same year $t$ and the rest in year $t+1$. 
 The first row of fertility must be further discounted by a factor, $\lambda$ in 
 order to account for the fact that infant mortality is higher in the lower Lexis 
 triangle than in the upper, i.e.
 of those infants that die in the first year of life, a proportion equal to
 $\lambda$ do not make it to December \nth{31} of the calendar year in which
 they were born\footnote{$\lambda$ can be derived directly from death counts
 data classified by Lexis triangles. In the US, lambda has behaved similarly
 for males and females, falling steadily from around $0.9$ in 1969 to $0.86$
 around 1990, since which time it has steadily risen to around $0.87$. That is
 to say, lambda has varied, but not drastically. Likewise for Spain, $\lambda$
 fell from around $0.885$ in 1975 to $0.86$ in the mid 1990s, since which time it
 has risen another \textonehalf~\%. In Spain  $\lambda$ has been around
 \textonehalf\% higher for males than females. These numbers are just meant to give a feel
 for the ranges that $\lambda$ can be expected to receive. If the demographer
 does not have information to derive $\lambda$ directly, ad hoc semidirect
 methods may be used to assign a reasonable proportion. } . The
 fertility component of $\textbf{Y}$ is then composed as in Matrix~\ref{matrix:ex1sexfertility}.

\begin{matrix}[h!]
\centering
\caption{Fertility component of one-sex remaining years
($e_y$)-structured projection matrix, $\textbf{Y}$} 
\label{matrix:ex1sexfertility}
$\bordermatrix{
  {e_y } \vspace{.6em}&                0_t  & 1_t  & 2_t  & 3_t  & 4_t  & 5_t\cr 
   0_{t+1} \vspace{.6em}& (1-\lambda) \tfrac{f_0d_0}{2} & (1-\lambda) f_1d_0 & (1-\lambda)
   f_2d_0 & (1-\lambda) f_3d_0 & (1-\lambda) f_4d_0 & 0 \cr 
   1_{t+1} \vspace{.6em}& \tfrac{f_0d_1}{2} & f_1d_1 & f_2d_1 & f_3d_1 & f_4d_1
   & 0   \cr 2_{t+1} \vspace{.6em}& \tfrac{f_0d_2}{2} & f_1d_2 & f_2d_2 & f_3d_2 & f_4d_2
   & 0   \cr 3_{t+1} \vspace{.6em}& \tfrac{f_0d_3}{2} & f_1d_3 & f_2d_3 & f_3d_3 & f_4d_3
   & 0   \cr 4_{t+1} \vspace{.6em}& \tfrac{f_0d_4}{2} & f_1d_4 & f_2d_4 & f_3d_4 & f_4d_4
   & 0   \cr 5_{t+1} \vspace{.6em}& \tfrac{f_0d_5}{2} & f_1d_5 & f_2d_5 & f_3d_5 & f_4d_5
   & 0   }$
\end{matrix}

The survival and fertility components of $\textbf{Y}$ add together elementwise,
thus the full 6$\times$6 matrix is composed as in Matrix~\ref{matrix:ex1sex}.

\begin{matrix}[h!]
\centering
\caption{A full one-sex remaining years ($e_y$)-structured projection
matrix, $\textbf{Y}$} 
\label{matrix:ex1sex}
$\textbf{Y} = \bordermatrix{
  {e_y } \vspace{.6em} & 0_t  & 1_t  & 2_t  & 3_t  & 4_t  & 5_t\cr 
  0_{t+1} \vspace{.6em}&  (1-\lambda) \tfrac{f_0d_0}{2} & (1-\lambda) f_1d_0 + 1 &
  (1-\lambda) f_2d_0 & (1-\lambda) f_3d_0 & (1-\lambda) f_4d_0 & 0 \cr 
    1_{t+1} \vspace{.6em}& \tfrac{f_0d_1}{2} & f_1d_1 & f_2d_1 + 1 & f_3d_1 & f_4d_1 & 0 \cr 
    2_{t+1} \vspace{.6em}& \tfrac{f_0d_2}{2} & f_1d_2 & f_2d_2 & f_3d_2 + 1 & f_4d_2 & 0 \cr 
   3_{t+1} \vspace{.6em}& \tfrac{f_0d_3}{2} & f_1d_3 & f_2d_3 & f_3d_3 & f_4d_3 + 1 & 0 \cr 
   4_{t+1} \vspace{.6em}& \tfrac{f_0d_4}{2} & f_1d_4 & f_2d_4 & f_3d_4 & f_4d_4 & 1 \cr 
   5_{t+1} \vspace{.6em}& \tfrac{f_0d_5}{2} & f_1d_5 & f_2d_5 & f_3d_5 & f_4d_5 & 0 }$
\end{matrix}

Remaining years classes should ideally terminate at the highest value permitted
by data. For the data used in this dissertation, there are 111 total age
classes, which translate to 111 total remaining years classes (0-110+). In practice 
$\textbf{Y}$ becomes
as 111$\times$111 matrix, with most entries non-zero. Construction may appear
tedious for this reason . However, note that the bulk of fertility entries can
be derived as the outer (tensor) product $d_x \otimes f_y$, leaving only the 
first row and first column discounting followed by the addition of the survival
superdiagonal. In most statistical programming languages constructing $\textbf{Y}$ entails only
a couple more lines of code than constructing a Leslie matrix.

As with Leslie matrices, the above projection matrix may be manipulated using
generic matrix techniques in order to extract such information as the intrinsic
growth rate, or the stable $e_y$ structure. The former is the natural log of the
largest real eigenvalue, and the later is the real part of the eigenvector that
corresponds to the largest real eigenvalue, rescaled to sum to 1\footnote{see
\citet[pp 86-87]{caswell2001matrix}}.

 \FloatBarrier

    \section{The method applied to US and Spain}
       When applied to the data from the US and Spain, we see the trends
displayed in Figure~\ref{fig:rex1sex}. In all years studied, $e_x$-structured
$r$ has been higher than the age-structured $r$, though the overall patterns of
change have been very similar. In that case of US males, in no year studied
has the $e_x$-structured $r^m$ dropped below 0.

\begin{figure}[!ht]
  \centering
    \caption{One-sex intrinsic growth rates, $r^m$ and $r^f$, according to
    renewal Equation~\eqref{eq:exLotkafemales}, US and Spain, 1969-2009.}
     % figure produced in /R/ExLotka1Sex.R
     \includegraphics{Figures/exLotka1sex}
     \label{fig:rex1sex}
\end{figure}

    \section{On the stablity of remaining years structure}
       \FloatBarrier
Upon viewing a variety of $e_y$-classified population leaves,\footnote{When
structured by remaining years of life, population pyramids in contemporary
Western countries look like leaves -- even more so when cohorts are colored
separately within the figure. } one finds abundant anecdotal evidence for the
existence of a characteristic shape. It has been claimed in this dissertation 
that the range of shapes that might be observed for this variety of population 
structure is relatively narrow -- relative with respect to age-classified
pyramids. The author offers no mathematical proof that this is so, but it is evident 
that the deaths
distribution is the primary force behind the $e_y$-structure, and demographers
recognize a characteristic shape to both $d_x$ and the force of mortality from
which it is derived. These characteristics are negative senescent youth, a hump
from the teenage age until ages 30-40, followed by Gompertz mortality, which
probably tapers off to constant, albeit high, mortality in the oldest of
old ages \citep{horiuchi1998deceleration, vaupel1997trajectories}. The 
$e_y$-structured population will tend to conform then to the distribution derived from 
the characteristic shape of the force of mortality,
while the affect of fertility change will be to weight the deaths distribution,
as new generations are added to the population. When fertility is assumed
constant, as in the stable population, the deaths distribution, weighted by 
the growth rate, becomes the only determinant of the shape. 

This being so, we may venture to complement the original claim, that observed
$e_y$-structures tend not to vary far from their characteristic shape, by
comparing observed with stable structures. To do this, we use the same 
measure of distribution separation seen elsewhere in this dissertation (see
Equation~\eqref{eq:coefdiff}), the difference coefficient, $\theta$, which is
the complement to the proportional overlap between two distributions. This we show in
Figure~\ref{fig:exstablepyr},\footnote{Trends actually indicate 95\% confidence
regions, which in this case are quite narrow. We have allowed for
stochasticity in birth counts and death counts, as elsewhere in this
dissertation, but taken the growth rate, population counts, and original
exposures as given.} where we see that for the US and Spanish populations, the
observed and stable distributions for males and females obtained some 80-95\%
overlap over the period studied. Single-sex male populations tended to be closer
to their stable form.

\begin{figure}[ht!]
       \centering
       \caption{Distribution dissimilarity of $e_y$-structured populations in
       year $t$ and corresponding year $t$ stable distributions. US, 1969-2009
       and Spain, 1975-2009}
        \includegraphics{Figures/exPyramidPresentvsStableDivergence}
        \label{fig:exstablepyr}
\end{figure}

The dramatic fertility drop in Spain is likely to have caused the distance
from the present to the stable structure to increase via abrupt changes in the
growth rate, which will have noticeably moved the modal $e_y$-class. We
do not decompose changes in dissimilarity over time into fertility and
mortality components in this dissertation, though this would be an informative
exercise and is left for future work. The degree of separation
between observed and stable age-structured populations follows a similar
year-to-year pattern. For the Spanish and US populations in the period studied,
age-$\theta$ has always been higher than $e_y$-$\theta$, indicating greater
separation between the stable and observed structures.
Figure~\ref{fig:exPyramidthetaratio} displays the ratio of these two measures 
of separation. High values in this figure indicate that the $e_y$-structure was
 much closer to its stable form than the age-structure to its stable form. This
evidence is used in support of the claim that $e_y$-structures are \textit{more
stable} than age-structure. We now complement this evidence with other
perspectives on stability.

\begin{figure}[ht!]
       \centering
       \caption{Ratio of observed versus stable dissimilarity in $e_y$- and
       age-structured populations; US 1969-2009 and Spain 1975-2009}
        \includegraphics{Figures/exPyramidthetaratio}
        \label{fig:exPyramidthetaratio}
\end{figure}

The degree of distributional separation between the present and stable
structure is not the entire story -- it represents only the starting and
theoretical stable states, but says nothing about the changes in structure that
would unfold in the process of convergence toward stability. The path to
stability may entail abrupt oscillations that last a few generations, or it may
proceed quickly and smoothly. We can measure such things as the speed at which
convergence occurs or the magnitude of the oscillations undergone in 
population structure along the path to stability. 

\begin{figure}[ht!]
       \centering
       \caption{Damping ratios. Age-classified versus $e_y$-classified
       trajectories. US, 1969-2009 and Spain, 1975-2009}
        \includegraphics{Figures/Damping}
        \label{fig:damping}
\end{figure}

Figure~\ref{fig:damping} displays the so-called damping ratio from the
respective projection matrices, which gives an indicator of the speed of
convergence. Superscripts in this figure indicate sex\footnote{These results
were derived by eigenvector analysis of the respective male and female, age-structured and
$e_y$-structured projection matrices using statistical tools from the
\texttt{popbio} package \citep{popbio2007} in the
\texttt{R} programming language \citep{Rcitation}. The \texttt{popbio} package
is primarily based on \citet{caswell2001matrix}.}. The damping ratio is the 
ratio of the largest to the second-largest real eigenvalue from the projection 
matrix\citep[p101]{caswell2001matrix}. Higher values indicate faster 
convergence, while lower values indicate likely-slower convergence. One notes 
that females here tend to undergo faster convergence than
males by this indicator, though this difference has been more consistent and
more marked in the US than for Spain. The US population would also have had a
theoretically faster journey to stability than the Spanish population, save for
the year range 1975-1985. The lengthening of the likely duration to stability in
Spain will have owed to the rapid decline in fertility that quickly changed the
shape of the stable structure, while the observed population structure
changed only slowly over the same period. This couples with the information
from Figure~\ref{fig:exstablepyr}, where we saw a dramatic increase in
dissimilarity between the observed and stable populations for Spain. Of interest
in the present discussion is that $e_y$-structured populations, with great difference, 
are seen here to converge faster than age-structured populations. With this we 
have another piece of evidence to support the claim that $e_y$-structured populations are more stable than
age-structured populations: $e_y$-structured populations have a shorter
trip to the stable structure.

This information we complement further by measuring the total departure from
stability from the initial to stable states, as proposed by
\citet{cohen1979cumulative}. The method works by projecting a given starting
population (the year $t$ population) forward a large number of years. For each
year $t+n$ of the projection, we measure the distributional difference from
the stable structure ($c_a$, or $c_y$) using the difference coefficient from
Equation~\ref{eq:coefdiff} (having scaled the year $t+n$ population and the
stable structure to each sum to 1), and integrate these differences over time.
Explicitly, and in discrete form, since this exercise is best varried out with
projection matrices, define the $e_y$-structured projection matrix,
$\textbf{Y}$, the year $t$ $e_y$-classified population vector
$\textbf{p}_y$, and the stable population vector, $\textbf{c}_y$

\begin{equation}
\label{eq:totaloscillation}
\mathrm{Total~Oscillation} = \sum _{t=0} ^\infty 1 - \sum _{y=0} ^\omega
min \left( \frac{\boldsymbol{Y} \boldsymbol{p}_{y,t}}{\sum \boldsymbol{Y}
\boldsymbol{p}_{y,t}}, \boldsymbol{c}_y \right)
\end{equation}
where
\begin{equation}
\boldsymbol{p}_{y,t+1} = \boldsymbol{Y} \boldsymbol{p}_{y,t}
\end{equation}

The population vector $\textbf{p}_{y,t}$ changes in each iteration based on the
projection matrix. Eventually the age structure stabilizes, after which time the central sum will
equal 0. This is in essence a measure of the total absolute departure from the
stable structure from the initial population until the stable population,
Cohen's \textit{D2} \citep{caswell2001matrix}. The process works the same way
for age-classified data, changing the subscript to $a$. The results of applying
Equation~\eqref{eq:totaloscillation} to the Spanish and US data are displayed
in Figure~\ref{fig:exCohenD2}. Larger values of this indicator signify larger
oscillations, which take longer to diminish to 0. One could
simplistically understand this as a measure of the difficulty, or friction,
along the path to stability.

Results are mostly consistent with previous indicators shown in this section --
$e_y$-structured populations oscillate less in the process of converging. This
is because the oscillations are smaller, which is because the distributional
overlap is greater, producing smaller waves in structure that disappear faster
and more smoothly. Curiously, females have a larger total oscillation than
males, save for the start and end of the Spanish age-classified series. This is
curious because, according to the damping ratio, females should approach
stability faster. On the whole, there has been a downward trend in this
indicator for the US population, and the trend in the Spanish population
coincides from the trend in overall departure from the stable form, as seen in
Figure~\ref{fig:exstablepyr}. The peaks for Spain in Figure~\ref{fig:exCohenD2}
also correspond with dips in the Figure~\ref{fig:damping} damping ratio, as
expected.

\begin{figure}[ht!]
       \centering
       \caption{Total oscillation along the path to
       stability. Age-classified versus $e_y$-classified trajectories. US,
       1969-2009 and Spain, 1975-2009}
        \includegraphics{Figures/CohenD2}
        \label{fig:exCohenD2}
\end{figure}

We have presented evidence in support of the statement that $e_y$-structured
populations are more stable than age-structured populations. There is some
risk that the evidence presented here has been accidental rather
than essential in nature. Namely, the range of years presented here for these
two populations may have coincidentally fallen at a point in time where
conditions were such as to make $e_y$-structure appear more stable. No formal
proof is offered that would support the claim that these observations were
necessarily so. We do, however make one final syllogistic appeal. It has been
noted that, as a simple heuristic, the shape of the remaining-years structured
population is in the first place determined by the age-pattern of mortality,
whereas the shape of the age-structured pyramid is in the first place
determined by temporal changes in fertility levels. Of mortality and
fertility, the shape of mortality will tend to be the more invariant of the two
(small populations excluded). The shape of mortality is less transient than is
fertility. The shape of mortality is less conditioned by perception, preference, 
culture, and planning than is the shape of fertility. This latter statement will
not hold all of the time, but it will hold most of the time, and particularly it
will have held in most Western populations in the past 50-or-so years. If one
accepts that mortality is in this sense more stable than fertility, one might
readily admit that the kind of results presented throughout the present section
were also to be expected. 

So it is that, in the absence of a formal proof, we
will be content to operate under the assumption that population structured by 
probabilistic time until death rather than recorded time since birth is less
volatile and closer to stable than typically observed age-structured
populations. These results are distinct from and complementary to our data
exercise from Section~\ref{sec:expopstruct1} which came to the same conclusion by examining
the distributional difference between population cross-sections lagged over a
series of years for the 46 populations of the HMD at the time of this
writing. We offer further speculation, but do not assume, that $e_y$-population structure is in some way the more essential of the two.

It may be noted that to the stable $e_y$-structure there corresponds a unique
age-structure, yet we have offered no formula to \textit{undo} the
age-transformed population back to its original age-structure. If stock 
is taken in the $r$ estimates produced in the $e_y$-structured model, 
then one can in like manner walk back to the
survival function and calculate the supposed age-structure, $c_a$:
\begin{equation}
c_a = \frac{e^{-ra}l_a}{\int _0 ^\infty e^{-ra}l_a \dd a}
\end{equation}
This indeed can be retrieved from the $c_y$ structure when noting that $l_a$ is
just the sum of future deaths, $d_a$ ($l_a = \int _a ^\infty d_a \dd a$), the
very building blocks of $c_y$, which we never fully dispensed with. In noting
this, one also realizes that to the stable age-structure there corresponds a
unique and stable $e_y$-structure (so long as vital rates in this instance come
from the age-perspective), in which case one simply inserts the age-derived intrinsic
growth rate into Equation~\eqref{eq:eybrate}~followed by~\eqref{eq:cy}. These
\textit{corresponding} stable structures are not explored further, and this
author is uncertain as to how to rectify the disagreements in structure that
result from the derivation of differing intrinsic growth rates.

 \FloatBarrier

      
    \section{Time until an unreasonable sex ratio}
       \FloatBarrier
 \label{sex:doublingex}
The basic projection matrix has been described for the single-sex
$e_y$-structured model. This tool permits us to repeat the illustrative
exercise from Section~\ref{sec:ageSRdoubling} wherein male and female populations are
projected separately and in parallel until such time as one sex outnumbers the
other sex by a factor of two. Long waiting times indicate less divergence, short
waiting times strong divergence. This exercise is close to being just another
viewpoint on the intrinsic growth rate, except that initial conditions are
expected not stable, and may therefore influence results. Again,
human sex ratios of 2 or \textonehalf~ are simply absurd, and this species of
indicator merely serves to compare.

In Figure~\ref{fig:exSRdoubling}, results from the age-structured model (dashed
lines) are compared with those of the $e_y$-structured model (solid lines).
These results were arrived at using the respective Leslie-matrices. Recall that
higher values indicate less or slower divergence, by this definition. For some 
years in both Spain and the US, the single-sex $e_y$-structured models were less 
divergent, and in other years the single-sex age-structured models were less divergent.
 For the age-structured models, very long waiting times are associated
with crossovers in $r$-- $r^m$ and $r^f$ have undergone no such crossovers for
the $e_y$-structured model in either Spain or the US, as was seen in
Figure~\ref{fig:rex1sex}. The rate of divergence for the $e_y$-structured models
was for this reason, relatively consistent over the range of years studied. 

\begin{figure}[ht!]
        \centering  
          \caption{$log(\mathrm{Years})$ until one sex is twice the size as the
          other, given separate single-sex projections using annual vital rates and initial
          conditions, $e_y$-structured model vs age-structured model. US,
          1969-2009 and Spain, 1975-2009}
           % figure produced in /R/exLeslie.R
           \includegraphics{Figures/ExrSRdoubling}
          \label{fig:exSRdoubling}
\end{figure}

The pace of divergence will be determined in the long run by the sex-gap in $r$.
As we saw for the age-structured model, the sex-gap in $r$ owes to various vital
rate components, which were revealed in a decomposition in
Section~\ref{sec:Decompr}. Likewise, the sex-gap in the $e_y$-structured model
is not the whole story, and it will be better understood if we examine the role
of each vital rate in determining its magnitude.

 \FloatBarrier
      
    \section{Decomposition of the sex gap in $r$}
      


\begin{figure}
        \centering
        \begin{subfigure}
                \centering
                \caption{Components to difference in single-sex intrinsic growth
                rates ($r^m - r^f$) when population is structured by remaining
                years, US, 1969-2009}
                \includegraphics[scale = .8]{Figures/DecomprExUS}
                \label{fig:exDecomprUS}
        \end{subfigure}
        \begin{subfigure}
                \centering
                \caption{Components to difference in single-sex intrinsic growth
                rates ($r^m - r^f$) when population is structured by remaining
                years, Spain, 1975-2009}
                \includegraphics[scale = .8]{Figures/DecomprExES}
               
                \label{fig:exDecomprES}
        \end{subfigure}
\end{figure}
      
    \section{Reflections on the single-sex findings}
      \FloatBarrier

This chapter aimed to introduce a new variety of population structure--
thanatological age, or age counted as time until death. The basic steps needed
to carry out the transformation presented in Equation~\ref{eq:dxredist} were
indeed available to demographers via the work of
\citet{miller2001increasing}, \citet{lee2002approach}, \citet{vaupel2009life}
and possibly others unbeknownst to this author. Indeed the perspective as a
whole is widespread in demography, this concept of age as time until death
was already in existence, and the name given to it is of known coinage. Our only
addition was to integrate over age, arriving at a wholly 
redistributed population (count, rate) vector entailing loss of age
information.\footnote{If this has indeed been done before, then this author has
not seen it, and apologies will be issued} In a sense this is more of a loss than an addition,
as we throw away information in doing so. 

It has been with fresh eyes that we have investigated fertility rates in terms
of thanatological age. Recall the words of Coale:
``One of the characteristics of demographic research is a search for empirical regularities, particularly in the
age-schedules of rates of marriage, birth, death, and
migration'' \citep{coale1996development}. We may claim to have found
empirical regularities in the restructured fertility rates, although further
work is needed to verify this for a wider range of populations and years.
Indeed, these patterns appear to be parametrically tractable -- we mark this 
possibility with a cairn and continue onward. There is in any case no need to smooth data that
have been redistributed in this fashion! However, if it turns out that the
remaining-years pattern to fertility is more regular, and hence more
predicatable, than the age-pattern to fertility, one may find a model curve to
be of use. We have not given an explicit account of the regularity of
$e_y$-structured fertility over time, though a glance at the surfaces in
figures~\ref{fig:exSFRsurfUS}~and~\ref{fig:exSFRsurfES} invites speculation in
this direction.

We have also found that the cross-classified remaining-years births distribution
is regular to the extent that it greatly resembles its own association-free
distribution. Either this observation is indeed so, or it is the result of the
overly permissive assumption of homogeneity with respect to mortality in
procreating pair formation. Namely, if there is noticeable selection with
respect to mortality fitness in the mate search, which is rather imaginable
\citep{gangestad1993pathogen, roberts2008good}, the patterns seen in
Figure~\ref{fig:US1970obsexpex} will not reflect the true distribution. The
question is then how far off our homogeneity assumption is. This could be
answered by means of linked register data, as has often been done for countries
such as Sweden and Denmark. We may already surmise that procreating individuals
are on average more fit than non-procreating individuals. To the extent that
marriage is a proxy for mating, we already know that married males have a
mortality advantage over unmarried males, although
there are likely intervening factors \citep[see
e.g.,][]{rogers1995marriage, waite1995does}, and we do not know for sure whether
this is due to causation or selection \citep[see e.g.,][]{goldman1993marriage}.
For the present use, we do not necessarily care whether the differential might be due to
selection or causality. All this is to say that we ought to take in the
observations of cross-classified fertility with a grain of salt. Furthermore, as
is pointed out in the text, selection may disproportionately affect the tails 
in the lowest remaining years classes.\footnote{This possibility would be just
the opposite in high maternal mortality settings, which does not affect
our two populations.} The best we can do in this instance is imagine the
direction of bias, as has we have tried to make clear.

In general, we have seen that period fertility indices calculated using
remaining-years classified data track well to those calculated age-classified
data, but also that they return higher estimates of period fertility. At
first glance, one would expect the direction of difference to be negative 
rather than positive because we took the simplest possible course of
redistributing population counts (or exposures) from all age classes rather than
only from age classes in ages typically identified as reproductive ages. The
outcome is positive because the age-distributions of marginal birth counts and
population counts are different, but redistributed by the same lifetable death
distribution. We ought not expect differing initial distributions to wind up
proportionally in the same remaining years classes. Hence, the sum of these
ratios (events to exposures) over remaining years is also different. TFR, for
instance, is higher for remaining-years structured data than the age-equivalent
measure because on average there is less exposure per birth in the newly
redistributed data.

One may question whether there is any sense in allowing non-reproductive ages
into remaining-years exposures, and to this we have two responses. The first
comes by manner of analogy to age-specific fertility distributions, which
have tails that are longer than most people feel comfortable imagining, 
both for males and females. For males and
females there are so-called central ages of reproduction and there are
less-common ages of reproduction. Even for ages typically measured, say
45-49 for females, not all persons captured in these exposures are truly
exposed to the risk of fertility, and everyone knows this. Indeed a minority
of females in this age group are truly at risk of fertility. This does not imply
that the rate calculated is invalid, but rather that the rates are both low and
potentially subject to further decomposition. This is one reason why a large
portion of fertility studies deal with the measurement of the proximate determinants of fertility-- these
factors differentiate exposure for purposes of calculating more specific rates.
Age-specific rates are valid without such differentiation, as they help purge
our measurement of distortions from a certain degree of population structure. 

As \citet{stolnitz1949recent} so eloquently describe, demographic rates are
never fully purged of population structure. We may nonetheless \textit{de}structure in
any of myriad ways, and remaining-years structure is the way that we have shed
light upon in this dissertation. No claim is made about whether chronological or
thanatological age are the more efficient classifying variable for population
data, nor that one is more pure than the other. In sum, we have in this
dissertation provided yet another alternative structuring variable, and if one
questions the validity of a rate calculated on the basis of blended age groups,
one may ask: 1) how many persons are expected to die in $y$ years and 2) how
many births this year were to persons expected to die in $y$ years, and there we
have everything needed to calculate a rate. This is what we have done. Further
refinements are possible and are worthy of exploration.

The second response to the potential criticism about overly permissive
heterogeneity within remaining years classes for purposes of producing
fertility rates is that these rates nonetheless appear to obey a
certain degree of empirical regularity. The range of shapes possible for
fertility rates by remaining life expectancy, at least absent a mortality
crisis, is narrow. This we confidently claim, but the claim is in need of
further empirical support beyond this dissertation. The same observation may be made of
the underlying population structure by remaining years of life-- the range of 
shapes that we have observed this structure to obtain in Western countries, 
excluding mortality crises, is
narrow. These observations compel one to conclude that
remaining-years-specific fertility rates ($e$SFR) describe a substantive
underlying phenomena. If age-heterogeneity in remaining years classes 
were problematic in this sense, we would expect more erratic patterns in
$e$SFR over time and between populations, but this is not so. Instead, $e$SFR
invites comparison between populations and over time, and such comparisons are
decomposible into mortality and fertility components. We have not
taken this latter exploratory step, but this remains an exciting question, since
a stable $e$SFR pattern may hold over time even as fertility age patterns and
mortality levels change.

Most empirical findings arising from this thesis are based on data only
from the years 1969-2009 for the United States and 1975-2009 for
Spain\footnote{The primary exception is the relative \textit{stability} of
observed remaining-years-structured populations versus age-structured
populations, which we verified using the entire HMD.}. Evidently, any
novel findings produced therefrom, especially those dealing with fertility, 
are in need of verification from a wider array of
populations, so as to be placed into context. This is to say that there is ample opportunity to test
and refine the hypotheses produced here. Certainly $e$SFR will move over time,
but we suppose that this movement is meaningful and will aid our understanding
of population processes.

Net of the potential criticism of age-heterogeneity in remaining-years classes,
there are reasons to suspect that fertility timing and levels might also be
a function of remaining years of life. Although one's own age at death is
unknown, individuals do have a \textit{sense} of their own remaining life
expectancies, and this may condition fertility. This is not to say that people 
predict their own lifespans without bias. \citet{payne2013life}, for example,
concludes that the framing of the very questions used to asses subjective
remaining life expectancy in survey respondents can bias results greatly. This
source of bias has more to do with eliciting respondents' beliefs and
perceptions, although there are also biases in subjective predictions that vary
over age. The important thing is that people's subjective expectations of
remaining life \textit{correlate} positively with real remaining life
expectancies, and much evidence has been produced that would support
this \citep{hurd1995evaluation, mirowsky1999subjective, hurd2002predictive,
perozek2008using, delavande2011differential, post2012longevity}. These authors
tend to explain individuals' predictive power in terms of awareness of
their own particular genetic and environmental situation. Most people
understand the general mortality increase over age, and are able to
weigh these factors out to produce a rough idea of future death 
probabilities in certain age intervals. These findings have largely come from
health and retirement surveys of older persons beyond reproductive ages. Of
course, it is difficult to measure how well persons in \textit{reproductive}
ages are at predicting their own ages at death because panels would need to
run over a very long range of years in order to asses the accuracy of
predictions of the relatively young. 

Another particular variety of finding that lends support to our case are
a couple studies of risky behavior in youth. \citet{borowsky2009health} show
that risky behavior in teenagers increases with high perceived risks of early death, and these
behaviors include risky sexual behavior. \citet{wilson1997life} conclude that
high real mortality risks for youth due to \textit{perceived} factors such as 
accidents, suicide and homicide, predict reductions in the age of fertility, which at
least on the surface, is consistent with the idea of $e$SFR actually describing 
some sort of underlying pattern. 

None of these studies provide
direct support of our hypothesis that we ought expect that fertility behavior is
somehow a function of remaining life expectancy, but they do help our case.
The kind of data that would help shed direct light on this question would be a
panel study that includes questions about expected probabilities of survival
until (or death by) certain future ages, as well as the usual battery of 
fertility questions. This could be incorporated, for example, in future waves of
longitudinal youth surveys. To suit our ends directly, this survey would follow
up with actual ages of death. Evidently, much patience will be required to reap
any results in this direction, and there is much uncertainty at this preliminary
stage (we are only imagining such a survey at the moment) about whether results
would provide an estimate of true remaining-years fertility curves, and whether
foresight of one's own mortality moves this curve. Removing subjective
remaining life expectancy from our demands, one may reconstruct cohort $e$SFR on 
the basis of long-running population registers from countries such as Denmark or
Sweden.

Absent a well-rounded base of psychological, sociological and demographic
evidence in support of the belief that fertility patterns ought be a function of
remaining years of life (as well as a function of age), we rely on our own
set of evidence that simply show these patterns to be stable. Stable demographic
patterns are desirable, as they are useful for predicting. We encourage
the exploration of the predictive power of remaining-years-specific
demographic rates, just as economists have been keen to look for patterns in
retirement saving and investment as a function of perceived remaining years of
life.

We completed the chapter with some rather indefinite speculation about the
potential use of our $v_y$, Fisher's reproductive value rethought to correspond
to the thanatological perspective, in cognate fields of demography, or
evolutionary demography itself. One could just as well reframe $v_y$ in terms of
inter-age transfers rather than reproduction, as per \citet{lee2003rethinking},
although the perspective change will still pertain.

The purpose of this chapter was to define and explore the vital rates to be used
in the remained of this dissertation. We will proceed by defining a model of
population renovation akin to Lotka's renewal equation. From this model we will
extract and explore the intrinsic growth rate and some other stable
parameters that belong to the thanatological perspective. 

\FloatBarrier




\chapter{Two-sex renewal in populations structured by remaining years}
   \FloatBarrier

It has been noted that divergence between the sexes, in terms of predicted birth
counts, is often dampened when projected using rates and populations that are
structured according to remaining years. This does not, however, mean that the
problem of the sexes is in this context negligible. Instead, the problem has
only become slightly more tractable. The author considers the problem more
tractable because in decreasing the magnitude of discrepancy between male and female
rates, the trade-offs inherent in the various two-sex solutions offered in the
literature also become smaller. This chapter will introduce some two-sex
extensions of the $e_y$-structured population model introduced in the earlier
Section~\ref{sec:exstructuredrenewal}:
\begin{enumerate}
  \item A dominance-weighted extension, assuming fixed weights for males and
  female, similar in design to the age-structured two-sex model found in
  \citet{goodman1967age}.
  \item An extension based on the generalized mean of the joint male-female
  exposures, based on that presented for age-structured populations in
  Section~\ref{sec:ageharmonic}.
  \item A two-sex extension proposed especially for the $e_y$-structured
  two-sex population, based on a constant departure from the association-free
  joint birth distribution. 
\end{enumerate}
  
  \section{A linear model of two-sex renewal}
     \FloatBarrier
 \label{sec:ex2sexdomweights}
\citet{goodman1967age} offers a suite of
formulas to determine the stable age-sex composition of a population taking into
account the vital rates of both sexes, assuming one can assign a relative weight
(summing to 1) to male and female fertility. This model was presented in
Section~\ref{sec:googmanage} for age-structured populations, and will now be
translated for the case of remaining-years structured populations. Recall that
this model entails two trade-offs: 1) one must (arbitrarily) choose dominance
weights, and 2) these weights are constant. The fact of having constant weights 
keeps the solution linear (interaction-free), but less realistic. The final result is bracketed by the
cases of male and female dominance, but the gap between these two extremes 
also measures the demographer's subjective leeway, which we would like to
minimize. Both of these drawbacks may be reduced in the case of $e_y$-structured
populations, since: 
\begin{enumerate}
  \item $e_y$-structured populations have a more stable (in terms of
  distributional variation from year-to-year) structure than age-structured
  populations.
  \item Mate-selection with respect to remaining years of
life is nearly random in $e_y$-structured populations (see
Section~\ref{sec:exobsexpected}).
  \item The difference between male and female dominance (in terms of projected
birth counts) is often reduced, thereby limiting of the impact of the
demographer's ``dominance caprice'' on results (See
Section~\ref{sec:exdivergence}).
\end{enumerate}
Points (1) and (2) reduce (but do
not eliminate) the necessity of sex-interactions in a model. By this it is meant
that the proportional difference in results from one choice of model weights
over another is simply diminished. This being so, the comparative advantage of a
more sophisticated or realistic model is to some degree diminished. Since the
weighting coefficients in this model do not change, we have taken the extra step
to design a projection matrix for this dominance model, and we put this to use
to study some of the transient properties of the present model, as well as
examine the resulting stable population structure.

    
    \subsection{The renewal equation}
       \FloatBarrier
\label{sec:2sexlinearmain}
As mentioned, choose some weight, $\sigma$, between 0 and 1 to apply to male
rates, where the female weight is defined as $1 - \sigma$. When $\sigma = 1$
there is perfect male dominance, and when $\sigma = 0$ there is perfect female
dominance. Of course, births to girls are subject to female mortality and births
to boys are subject to male mortality. As with
Equation~\eqref{eq:exLotkafemales}, this mortality enters in the equation by way
of the $d_x$ distribution used to distribute births over life expectancies. The
final renewal formula is defined as follows:

\begin{equation}
\label{eq:lineartwosexrenewal}
\begin{split}
1 = \frac{1 - \sigma}{2} 
            \int _{y'=0}^\infty \int _{a'=y'}^\infty e^{-ra'}
                      d_{a'}^F \left(f_{y'}^{F-F} + f_{y'}^{F-M}\right) \dd a'
                      \dd y' \\ + \;\frac{\sigma}{2}
            \int _{y=0}^\infty \int _{a=y}^\infty e^{-ra}
                      d_{a}^M  \left(f_{y}^{M-M} + f_{y}^{M-F}\right)\dd a \dd y
\end{split}
\end{equation}
, where $a'$, $y'$, $a$ and $y$ index female age, female remaining years, male
age and male remaining years, respectively. Fertility superscripts identify sex of
progentitor followed by sex of offspring, and $d_x$ must accord with the sex of
offspring. Such specific rates are chosen because data that would permit
empirical studies of the two-sex problem are typically sufficiently rich to allow 
for cross-tabulations by age of both parents as well as sex of birth. 
Therefore, Equation~\eqref{eq:lineartwosexrenewal} assumes that rates are
available by sex of progenitor, birth (4 combinations) and age (to be transformed to remaining years), 
and no additional variable is required for the sex ratio at birth. Indeed,
Equation~\ref{eq:lineartwosexrenewal} does not require such specific rates,
since rates of reach progenitor sex are simply summed, but sex-sex-specific
rates will be needed downstream for the calculation of other stable quantities,
so it is advisable to treat them as inputs from the start.

Weights, $\sigma$ and $1-\sigma$ are divided by 2 because
total births are counted twice in total (males and females from males \&
males and females from females). One could just as easily optimize to a sum of
2 on the left-hand side rather than discount weights.

The linear two-sex $r$, $r^\upsilon$, extracted from~\eqref{eq:lineartwosexrenewal} 
is \textit{not} guaranteed to be bounded by the $e_x$-structured $r^f$ and $r^m$,
and indeed $r^f$ and $r^m$ may not be recovered by setting $\sigma$ to 0 or 1,
respectively. This is so because the model includes births of both sexes to
progenitors of each sex, which changes the age-specific fertility curves
somewhat. That is to say, manipulation of $\sigma$ is insufficient to make the
one-sex model a degenerate case of the present model. $\sigma$ can only be
understood as indicative of the balance of dominance in fertility rates between
the sexes. The later choice would both reduce the complexity of
Equation~\ref{eq:lineartwosexrenewal} and guarantee exact bounds of $r^f$ and $r^m$ 
when $\sigma$ is set to 0 and 1, respectively. This author does not recognize 
the theoretical or practical merits of the single-sex modelling choice, as it 
is not the case that males are responsible for the birth of boys and females 
for the birth of girls\footnote{\citet{pollard1948measurement} took this idea
even further by swapping sexes: The fertility functions in this paper are based on the births of boys to mothers and girls to fathers, i.e. $M-F$ and $F-M$ fertility. This is parsimonious 
in terms of getting quick results that are guaranteed to fall within reasonable bounds, but is less
intuitively appealing}. This stance couples with the author's choice to not
include an explicit, let alone constant, variable for the sex ratio at birth.

It must be
noted that the value of $r^\upsilon$ is dependant upon the choice of $\sigma$, 
and that no guidelines are provided for choosing a good value of $\sigma$. 
This ambiguity also exists in the age-structured variant of the present model. 
For $e_x$-structured models, it has been claimed that sex-divergence is lesser than is the case for
age-structured models. Recall that this was the case for predictions of birth
counts, and not for the growth parameter, $r^\upsilon$. The
difference between the $e_x$-structured $r^f$ and $r^m$ is not necessarily lesser than is the case for
the age-structured $r^f$ and $r^m$. This will be discussed further along with
empirical results for the two populations considered in this dissertation.

      
    \subsection{An iterative approach to find $r$}
         \FloatBarrier
\label{sec:exrenewalit2}
Steps to practically solve Equation~\eqref{eq:lineartwosexrenewal} for $r$ are
similar to those presented for the one-sex case in
Section~\ref{sec:exrenewalit}
\begin{enumerate}
  \item Derive a first rough estimate of the both-sex mean remaining years of
  life at reproduction, $\widehat{T^\upsilon}$, akin to Lotka's mean generation time,
  $T$. If one assumes a growth rate of $0$, then a good-enough guess will be:

\begin{equation}
\widehat{T^\upsilon} = \frac{\splitfrac{
   \big((1 - \sigma)  \int _{y'=0}^\infty \int_{a'=y'}^\infty 
       y' d_{a'}^F \left( f_{y'}^{F-F} + f_{y'}^{F-M} \right) \dd a'\dd y'}{ + 
   \sigma \int_{y=0}^\infty \int _{a=y}^\infty y d_{a}^M  \left( f_{y}^{M-M}+
   f_{y}^{M-F} \right) \dd a \dd y \big)}}{\splitfrac{\big( (1 - \sigma) 
   \int_{y'=0}^\infty \int_{a'=y'}^\infty d_{a'}^F \left( f_{y'}^{F-F} +
   f_{y'}^{F-M} \right) \dd a'\dd y'}{ +\sigma \int _{y=0}^\infty
   \int_{a=y}^\infty d_{a}^M \left( f_{y}^{M-M} + f_{y}^{M-F} \right) \dd a \dd
   y \big)}}
\end{equation}

  \item A first rough estimate of the net reproduction rate, $\widehat{R_0}$ (assuming
  $r=0$) is given by:

 \begin{equation}
 \begin{split}
 \widehat{R_0} = \frac{(1 - \sigma)}{2}  \int _{y'=0}^\infty \int_{a'=y'}^\infty 
                d_{a'}^F \left(f_{y'}^{F-F} + f_{y'}^{F-M}\right) \dd a'\dd y'
                \\ + \frac{\sigma}{2}  \int _{y=0}^\infty \int _{a=y}^\infty 
               d_{a}^M  \left(f_{y}^{M-M}+ f_{y}^{M-F}\right) \dd a \dd y
 \end{split}
 \end{equation}
  \item A first rough estimate of $r$, $r^0$, is given by:
   \begin{equation}
   r^0 = \frac{ln(\widehat{R_0})}{\widehat{T^\upsilon}}
   \end{equation}
  \item Plug $r^0$ into Equation~\ref{eq:lineartwosexrenewal} to calculate a
  residual, $\delta^0$
  \item Use $\delta^0$ and $\widehat{T^\upsilon}$ to calibrate the estimate of $r$
  using:
  \begin{equation}
  r^{1} = r^0 + \frac{\delta^0}{\widehat{T^\upsilon} - \frac{\delta^0}{r^0}}
  \end{equation}
  \item Repeat step (3) to to derive a new $\delta^i$, then step (4) to refine
  $r^i$, until converging on a stable $r$ after some 30 iterations,
  depending on the degree of precision desired ($\widehat{T^\upsilon}$ is not updated
  in this process).
\end{enumerate}
        
    \subsection{Other stable parameters}
         \FloatBarrier
\label{sec:2sexlinearother}
Once two-sex linear $r$ and the stable proportion male of births, $\varsigma$,
have been found for the given $\sigma$, one may proceed to find the 
two-sex mean length of generation $T$ and stable $R_0$,
replacing the first guesses used in the iterative procedure described above.

We can derive the stable population sex ratio, $\bar{S}$:
\begin{equation}
\bar{S} = \frac{ \int_{y=0}^\infty \int_{a=y}^\infty \varsigma e^{-ra} d_{a}^M
\dd a \dd y}{\int_{y'=0}^\infty \int_{a'=y'}^\infty (1-\varsigma) e^{-ra'}
d_{a'}^F \dd a' \dd y'}
\end{equation}
The both-sex stable birth rate, $b$ is given by
\begin{equation}
b = \Bigg[
            \splitfrac{\big( \int _{y'=0}^\infty
            \int _{a'=y'}^\infty e^{-r a'} (1-\varsigma) d_{a'}^F \dd a' \dd
            y'}{ + \int _{y=0}^\infty \int _{a=y}^\infty
             e^{-r a} \varsigma d_{a}^M \dd a \dd y\big)} \Bigg] ^{-1}          
\end{equation}
which can be used to derive the stable $e_y$-structure of males and females,
$c_y$ and $c_{y'}$, respectively:

\begin{align}
c_{y'} = b (1-\varsigma) \int _{a'=y'}^\infty
e^{-ra'} d_{a'}^F \dd a' \notag \\
c_{y} = b \varsigma \int _{a=y}^\infty
e^{-ra} d_{a}^M \dd a'
\label{eq:stablecy}
\end{align}
where of course,
\begin{equation}
1 = \int c_{y'} \dd y' + \int c_{y} \dd y
\end{equation}

  
  \section{The linear two-sex projection matrix}
       \FloatBarrier
 \label{sec:ex2sxprojmat}
The formal relations presented in
Sections~\ref{sec:2sexlinearmain}~and~\ref{sec:2sexlinearother} establish
coherence, and some merits have been presented. This section offers tools
more relevant to the discrete practice of applied demography. The model
contained in Equation~\ref{eq:lineartwosexrenewal} is conformable to replication 
with a projection matrix, similar in concept to that offered for the 
single-sex $e_y$-structured
case in Section~\ref{sec:ex1sexleslie}. The two-sex linear projection matrix combines
the projection of each sex jointly in a single instrument, the construction of 
which is more involved than the single-sex case: Four times more involved to be precise. 

Assuming $n$ $e_y$-classes each for males and females, the dimensions of the
present matrix will be $2n \times 2n$, where male and female
$e_x$-classified population vectors by be joined, for instance end-to-end in a
single population vector. The convention used in this description will place
males, ordered by remaining years of life in positions $1:n$ of the vector
$\textbf{p}$ and females ordered by remaining years in positions $(n+1):(2n)$ of
$\textbf{p}$, i.e. end-to-end. This being so, the projection matrix $\textbf{Y}$
must conform with these locations of males and females, locating survival and 
fertility appropriately.

$\textbf{Y}$ is divided into four main blocks. The top left block is
nearly identical to the male single-sex case, and the bottom left block is
nearly identical to the female single-sex case. Both of these two submatrices
contain survival (all 1s) in the superdiagonal. Fertility is analogous, but must
be weighted according to $\sigma$ ($1-\sigma$). The lower left submatrix
contains $M-F$ fertility weighted by $\sigma$ and distributed over female death probabilities,
$d_a^f$, and the upper right matrix contains $F-M$ fertility, weighted by
$1-\sigma$ and distributed according to male death probabilities. As with the
single-sex projection matrix, fertility destined to die in the first year of
life must be further discounted by $\lambda$ to account for the part of infant
mortality that occurs before Dec. \nth{31} of the calendar birth year. In this
case $\lambda$ may optionally be entered separately for males and females.

% need to adjust spacing big-time!
\begin{landscape}
\begin{matrix}[h!]
\caption{A full two-sex remaining years ($e_y$)-structured
projection matrix, $\textbf{Y}$} 
\label{matrix:ex2sex}
 \begin{adjustwidth}{-.2in}{-.5in}
 \centering
\footnotesize{
$\bordermatrix{
  {e_y }    \vspace{1.2em} & 0_t^M                                             &
  1_t^M                                       & 2_t^M                                    & 3_t^M   & 0_t^F                                                & 1_t^F                                          & 2_t^F                                       & 3_t^F  \cr 0_{t+1}^M \vspace{1.2em} & \sigma  (1-\lambda^M)\tfrac{f_0^{M-M}d_0^M}{2}       
  & \sigma (1-\lambda^M)f_1^{M-M}d_0^M + 1          & \sigma (1-\lambda^M)f_2^{M-M}d_0^M           &  0      & (1-\sigma)(1-\lambda^M)\tfrac{f_0^{F-M}d_0^M}{2}         & (1-\sigma)(1-\lambda^M)f_1^{F-M}d_0^M             & (1-\sigma(1-\lambda^M)f_2^{F-M}d_0^M &  0     \cr 
  1_{t+1}^M \vspace{1.2em} & \sigma \tfrac{f_0^{M-M}d_1^M}{2}                  &
  \sigma f_1^{M-M}d_1^M                       & \sigma f_2^{M-M}d_1^M + 1                &  0      & (1-\sigma) \tfrac{f_0^{F-M}d_1^M}{2}                 & (1-\sigma)f_1^{F-M}d_1^M                      & (1-\sigma)f_2^{F-M}d_1^M                    &  0     \cr 2_{t+1}^M \vspace{1.2em} & \sigma \tfrac{f_0^{M-M}d_2^M}{2}                  &
  \sigma f_1^{M-M}d_2^M                       & \sigma f_2^{M-M}d_2^M                    &  1      & (1-\sigma) \tfrac{f_0^{F-M}d_2^M}{2}                 & (1-\sigma)f_1^{F-M}d_2^M                      & (1-\sigma)f_2^{F-M}d_2^M                    &  0     \cr 3_{t+1}^M \vspace{1.2em} & \sigma \tfrac{f_0^{M-M}d_3^M}{2}                  &
  \sigma f_1^{M-M}d_3^M                       & \sigma f_2^{M-M}d_3^M                    &  0      & (1-\sigma) \tfrac{f_0^{F-M}d_3^M}{2}                 & (1-\sigma)f_1^{F-M}d_3^M                      & (1-\sigma)f_2^{F-M}d_3^M                    &  0     \cr 0_{t+1}^F \vspace{1.2em} & \sigma (1-\lambda^F)\tfrac{f_0^{M-F}d_0^F}{2}     &
  \sigma (1-\lambda^F)f_1^{M-F}d_0^F          & \sigma (1-\lambda^F)f_2^{M-F}d_0^F       &  0      & (1-\sigma) (1-\lambda^F)\tfrac{f_0^{F-F}d_0^F}{2}    & (1-\sigma)(1-\lambda^F)f_1^{F-F}d_0^F +1          & (1-\sigma)(1-\lambda^F)f_2^{F-F}d_0^F           &  0     \cr 1_{t+1}^F \vspace{1.2em} & \sigma \tfrac{f_0^{M-F}d_1^F}{2}                  &
  \sigma f_1^{M-F}d_1^F                       & \sigma f_2^{M-F}d_1^F                    &  0      & (1-\sigma) \tfrac{f_0^{F-F}d_1^F}{2}                 & (1-\sigma)f_1^{F-F}d_1^F                      & (1-\sigma)f_2^{F-F}d_1^F +1                 &  0     \cr 2_{t+1}^F \vspace{1.2em} & \sigma \tfrac{f_0^{M-F}d_2^F}{2}                  &
  \sigma f_1^{M-F}d_2^F                       & \sigma f_2^{M-F}d_2^F                    &  0      & (1-\sigma) \tfrac{f_0^{F-F}d_2^F}{2}                 & (1-\sigma)f_1^{F-F}d_2^F                      & (1-\sigma)f_2^{F-F}d_2^F                    &  1     \cr 3_{t+1}^F \vspace{1.2em} & \sigma \tfrac{f_0^{M-F}d_3^F}{2}                  &
  \sigma f_1^{M-F}d_3^F                       & \sigma f_2^{M-F}d_3^F                    &  0      & (1-\sigma) \tfrac{f_0^{F-F}d_3^F}{2}                 & (1-\sigma)f_1^{F-F}d_3^F                      & (1-\sigma)f_2^{F-F}d_3^F                    &  0 }$}
  \end{adjustwidth}
\end{matrix}
\end{landscape}
Matrix~\ref{matrix:ex2sex} is a schematic representation of a two-sex
$e_y$-structured projection matrix. This example contains four classes of life
expectancy in order to economize space (as opposed to the six shown in
Matrix~\ref{matrix:ex1sex}). Such a matrix amenable to the data used in
this thesis would have final dimensions $222 \times 222$, since we work here
with 111 remaining life expectancy classes\footnote{The 111 $e_y$ classes
are namely derived from the 111 ages of $d_x$ provided by the HMD, ages
$0-110+$.}.


  
  \section{The method applied to US and Spain}
       \FloatBarrier
This procedure has been applied to the data from the US and Spain with $\sigma$
given the values of 0, 0.5, and 1, which correspond to the cases of
female-dominance, an intermediate value, and male-dominance, 
and can be seen in Figure~\ref{fig:rupsilonlinear2sex}. \footnote{The data in
Figure~\ref{fig:rupsilonlinear2sex} are available in Tables~\ref{tab:ex2linRepES}~and~\ref{tab:ex2linRepUS} of
    Appendix~\ref{appendix:ex2sexlinear}, along with the stable parameters
    $R_0$ and $T$.}

\begin{figure}[!ht]
  \centering
    \caption{Two-sex linear intrinsic growth rate, $r^\upsilon$, according to
    renewal Equation~\eqref{fig:rupsilonlinear2sex}, with $\sigma$ given the
    values 0, 0.5 and 1; US and Spain, 1969-2009}
     % figure produced in /R/ExLotka2Sex.R
     \includegraphics{Figures/exLotka2sexlinear}
     \label{fig:rupsilonlinear2sex}
\end{figure}

Patterns accord with trends generally known from the age-classified $r^f$ and
$r^m$, but values of $r$ are higher than the
age-classified intrinsic growth rates in all of the years studied. In all years
tested here, $r$ was indeed bounded by the $e_y$-structured $r^f$ and $r^m$.
We can confirm that our implementation is good in that the border cases
where $\sigma$ equals 0 or 1 produce the same results as the single-sex models.

\FloatBarrier

        
  \section{More on the stability of remaining-years population structure}
     \FloatBarrier
As with the single-sex case, one may measure the distributional distance between
the initial and stable conditions for two-sex $e_y$-structured populations. The
degree of separation, $\theta$, will be intermediate to those calculated for the 
single-sex cases, leaning closer to the male or
female indices depending on the value of $\sigma$ used to calculate the two-sex
stable population. Also as with the single-sex case, the damping
ratio may be calculated from the two-sex $e_y$-structured projection matrix 
presented in Section~\ref{sec:ex2sxprojmat}. Here the value is not
necessarily intermediate to the male and female single-sex cases, as seen in
Figure~\ref{fig:damping2}:

\begin{figure}[ht!]
        \centering  
          \caption{Damping ratios from two-sex $e_y$-structured projection
          matrices compared with single-sex values, Spain and US, 1969-2009}
           % figure produced in /R/exLeslie.R
           \includegraphics{Figures/Damping2}
          \label{fig:damping2}
\end{figure}
Note that in both cases the $\sigma$ used to calculate the two-sex matrices was
.5, in principle \textonehalf~informed by male vital rates and \textonehalf~
informed by female vital rates. For the US, as one might expect, the damping
ratio was intermediate to the single-sex male and female ratios. For
the Spanish population, however, the two-sex model is expected to stabilize
faster than either of the corresponding one-sex models. We speculate that this
will be in large part due to the explicit balancing of the male and female
populations by the sex ratio at birth, which is higher in Spain than in the US. 
In the two-sex model, the Spanish population moves forward as a whole rather 
than quickly diverging due to its high sex ratio. This may be a desirable
property.

Our other summary measure of transient dynamics, the total absolute
oscillation of population structure from the initial to stable states
\citep{cohen1979cumulative}, in this case tends to be intermediate to the male
and female values (see Figure~\ref{fig:cohend22sex}). One exception are the
years 1975-6 for the Spanish population, where total oscillation in his model
would have been higher than for either single-sex model. Recall that the damping
ratio for each year of data was higher (faster stability) for the two-sex case 
than either single-sex case. Only the $\sigma$ value of 0.5 was tested, but here
we see that other values of $\sigma$ also would not guarantee damping ratios or
total oscillations bracketed by the single-sex cases. That we see this in the
simple linear combination of male and female models might be a precursor to
observing that such measures for non-linear models will also not necessarily be
bracketed by the male and female single-sex cases.

\begin{figure}[ht!]
        \centering  
          \caption{Total oscillation along the path to
       stability. Two-sex ($\sigma = 0.5$) versus single-sex $e_y$-structured
       projection trajectories; US 1969-2009 and Spain 1975-2009}
           % figure produced in /R/exLeslie.R
           \includegraphics{Figures/CohenD22sex}
          \label{fig:cohend22sex}
\end{figure}
 \FloatBarrier






  
  \section{Reflections on the linear two-sex model}
       \FloatBarrier
I posit that there exists a formal identity to relate the various results
(e.g. $r^f{y'}$ to $r^{\upsilon (0)}$), just as \citet[pp. 56]{coale1972growth} relates 
the age-structured $r^m$ and $r^f$, but this fruit will be left on the tree
for the time being.

Most important, as is visible in Figure~\ref{fig:rupsilonlinear2sex}, there is
simply very little spread in growth rates between the positions of extreme
dominance. One intuitively wishes to see a non-linear two-sex model that
accounts for interactions between both sexes and remaining years of life, just
as one wishes, in an age-structured model to allow for fluid interactions
between sex and age. In such a model, the laws of supply and demand would move
$\sigma$ according to the relative weight of male and female exposure. However,
the distance between male and female dominance represents around twice the
maximum difference in $r$ that one would observe upon applying the more
sophisticated model. This statement assumes 1) that the interactive model is
bounded by the dominant cases presented here, and 2) that one is comparing with
the case of $\sigma = 0.5$, a prudent choice. 

As a secondary point, notice also that the present linear model holds rates
constant with respect to remaining life expectancy, but \textit{not} with
respect to age. From year to year the population structure with
respect to remaining life expectancy changes, as does the underlying age
structure. One could re-derive age-specific fertility rates from the
$e_x$-specific fertility rates used here, and would note that since the
weighting variable has changed with time, so too would the weighted sum of
the $e_x$-specific rates inherent in any age-specific rate. This observation
heeds \citet{stolnitz1949recent}, who point out several ways in which
fertility rates are indeed simply weighted sums of even more specific weights.
Prior to the formulation of the present model we have pointed out another
dimension in which age (parity-race-class) -specific rates are weighted sums, and we have exploited
that, short of holding very cross-classified rates constant, one observes
greater stability over time with $e_x$-classified rates. Holding
$e_x$-classified rates constant will force underlying age-specific rates to fold
and adapt with each passing year (albeit not much). Forcing age to adjust in
accord with constant $e_x$-specific rates appears to this author to be just as
palatable as forcing $e_x$-specific rates to change under the constraint of
constant age-specific rates-- perhaps moreso. This judegment is passed on having
compared the observed volatility in the two kinds of specific rates and deciding
$e_x$-specific rates are more reconciliable with the stable population
assumption of fixed rates. This difference is not necessarily large, and may in
any case be an accident of history, as we have not pondered upon why it is that
$e_x$-specific rates would hold more constant over time than age-specific rates.
Part of this may owe to inadequancies in the method used to redistribute
age-classified data to $e_x$-classified data, as the method is new, and has not
undergone scrutiny beyond this very dissertation.
  
  \section{Incorporating feedback into a two-sex remaining-years structured
  renewal models}

  \section{A generalized mean of male and female exposures}
\todo{Same as in Harmonic Schoen, but with a stolarsky mean instead of only
Harmonic}
       \FloatBarrier
Formulas are here couched in the harmonic
mean, but this may be generalized, given that we specify the mean itself as
a function. The harmonic mean function itself differs from
Equation~\eqref{eq:harmonic} only in its subscripts:
\begin{equation}
H(P_y^m, P_{y'}^f) = \frac{2 P_y^m P_{y'}^f}{P_y^m + P_{y'}^f}
\end{equation}
where, as elsewhere in this dissertation, $y$ and $y'$ index remaining years of
males and females. We begin the process by calculating a single joint fertility rate distribution,
later assumed constant:
 \begin{equation}
 F_{y,y'}^H = \frac{B_{y,y'}}{H(P_y^m, P_{y'}^f)}
 \end{equation}
again, only differing from Equation~\eqref{eq:harmonicrate} in the remaining
years subscripts. $F_{y,y'}^H$ is the primary model component. With this, we may
calculate the births for a given year:
 \begin{equation}
 \label{eq:Bharmonicex1}
 B(t) = \int \int F_{y,y'}^H H\Big(P_{y}^m(t), P_{y'}^f(t)\Big) \dd y \dd
 y'
 \end{equation}
The population count $P_y$ is, however, easily related to past births in a
roundabout way:
\begin{align}
P_y &= \int_{a=0}^\infty P_a \frac{d_{a+y}}{l_a} \dd a \notag \\
    &= \int_{a=0}^\infty \varsigma B(t-a)p_a \frac{d_{a+y}}{p_a} \dd a \notag \\
    &= \int_{a=0}^\infty \varsigma B(t-a)d_{a+y} \dd a
\end{align}
where $\varsigma$ is of course the proportion male and the survival function is
just the sum of future deaths: $p_a = \int _a^\infty d_a \dd a$. This identity
allows us to rewrite Equation~\eqref{eq:Bharmonicex1} in terms of past births:
 \begin{equation}
 \label{eq:Bharmonicex1}
 B(t) = \int \int F_{y,y'}^H H\Bigg(\varsigma \int _0^\infty B(t-a)d_{a+y}\dd
 a\;\;,\;\; (1-\varsigma) \int _0^\infty B(t-a')d_{a'+y'} \dd a'\Bigg) \dd y \dd
 y'
 \end{equation}
which when left to renew itself for many years on-end, will eventually attain a
constant rate of growth, $r$, in which case we may rewrite
Equation~\eqref{eq:Bharmonicex1} in entirely in terms of year $t$ births:
 \begin{equation}
 \label{eq:Bharmonicex1}
 B(t) = \int \int F_{y,y'}^H H\Bigg(\varsigma \int _0^\infty
 B(t)e^{-ra}d_{a+y}\dd a\;\;,\;\; (1-\varsigma) \int _0^\infty
 B(t)e^{-ra'}d_{a'+y'} \dd a'\Bigg) \dd y \dd y'
 \end{equation}
which lets us divide by $B(t)$ to arrive at our standard approachable unity
equation, which permits us to estimate the stable growth parameter, $r$:
 \begin{equation}
 \label{eq:exMeanUnity}
 1 = \int_0^\infty \int_0^\infty F_{y,y'}^H H\Bigg(\varsigma \int _0^\infty
 e^{-ra}d_{a+y}\dd a\;\;,\;\; (1-\varsigma) \int _0^\infty
 e^{-ra'}d_{a'+y'} \dd a'\Bigg) \dd y \dd y'
 \end{equation}

 \FloatBarrier

  \section{Adjustment using a constant ratio of observed to expected births}
      
The present section is motivated by the desire for a non-linear two-sex model of
$e_y$-structured population growth that takes advantage of the observation that
the observed bivariate distribution of births by remaining years of mothers 
and fathers, $B_{y,y'}$ is in our experience very close to the expected
distribution, taking the male and female marginals as given. We have noted 
that the overall distributional distance
between observed and expected counts is typically very small (see e.g.
Figure~\ref{fig:TotalVarobsexpex}), but we have not described any patterns that
may exist in the difference between these two distributions. There is indeed a
common pattern to the departure between the observed and expected distributions
of $e_y$ structured births \todo{show this pattern in Figure in appendix?}. The
method presented here will thus stay true to the stable population concept
 of fixed male and female $e_y$-specfic fertility rates, but will add a second
 fixed component, a constant \textit{ratio} between $B_{y,y'}$ and
 $\mathbb{E}(B_{y,y'})$, which be used as an adjustment intstrument, in effect
 providing flexibility in the male and female marginal rates, while forcing
 consistency, both in the total birth count and in the $e_y$-distribution of
 births. The method to be presented here will be shown to posses
 several desirable properties for two-sex models.
 
 




    \subsection{The renewal equation}
      
The present method works as follows: Take constant base male and female
$e_y$-specific fertility rates, $F_y$ and $F_{y'}$. Given these rates and a male
and female $e_y$-structured population, we can generate the male and female predictions of
birth counts. We know from Section~\ref{sec:exdivergence} that the male and
female total counts tend to differ by less than if we were to repeat the
same for age-structured populations. However, the two total counts will still
differ, and so cannot be taken directly as the marginal birth count distributions from
which to calculate the association-free joint distribution-- the denominator
in Equation~\eqref{eq:expected}. To generate the expected count matrix, we
therefore calculate the cross-product of the male and female predictions and
divide by a mean of the male and female total predictions as follows:

\begin{equation}
\label{eq:meanexp}
\widehat{\mathbb{E}}(B_{y,y'}) = \frac{\widehat{B_y} \widehat{B_{y'}}}{M(\int
\widehat{B_y} \dd y, \int \widehat{B_{y'}} \dd y')}
\end{equation}
where $\widehat{B_y}$ is calculated using the constant base rate for males,
$F_y$, applied to male exposures, $E_y$, and likewise for females. $M()$ is any
mean function. For flexibility, one could use a generalized mean, such as the
Stolarsky mean or Lehmer mean, for $M()$, or any specific mean function, such as
the harmonic mean, if desired. The choice of mean function in the denominator
will be seen to have a trivial effect on the ultimate estimate of the
intrinsic growth rate.

Next, we estimate a constant ratio, $R_{y,y'}$, between the
observed and expected counts, which we take simply as:

\begin{equation}
\label{eq:getR}
R_{y,y'} = \frac{B_{y,y'}}{\mathbb{E}(B_{y,y'})}
\end{equation}
from the year of departure.

Using $R_{y,y'}$, we adjust the estimated expected distribution,
$\widehat{\mathbb{E}}(B_{y,y'})$ element-wise, and then rescale to sum properly
to $\mathbb{E}(B)$, the chosen mean of the male and female marginal
predictions:

\begin{equation}
\label{eq:ratioadj}
\widehat{B_{y,y'}} = R_{y,y'}\widehat{\mathbb{E}}(B_{y,y'})\frac{\int \int
\widehat{\mathbb{E}}(B_{y,y'})}{\int \int
R_{y,y'}\widehat{\mathbb{E}}(B_{y,y'})}
\end{equation}

Let us call Equation~\eqref{eq:meanexp} the mean
expected function, $\mathbb{M}(\widehat{B_y}, \widehat{B_{y'}})$, and
Equation~\eqref{eq:ratioadj} the ratio adjustment function, 
$A(R_{y,y'},\mathbb{M}(\widehat{B_y}, \widehat{B_{y'}}))$.

The marginal predictions of birth counts, $\widehat{B_y}$ and
$\widehat{B_{y'}}$, in the stable population will be determined by
fixed fertility rates and population exposures, which are a function of the
deaths distribution and the growth rate, $r$, as in the other $e_y$-structured 
models presented in this dissertation. 

For instance, since $\widehat{B_y} = P_yF_y$, we can determine the year $t$
births as follows:
\begin{equation}
\label{eq:ex2sexCRunity1}
B(t) = \int_{y=0}^\infty \int_{y'=0}^\infty
A\left(R_{y,y'},\mathbb{M}\left(\;Py(t)Fy, P_{y'}(t)F_{y'} \right)\right) \dd y
\dd y'
\end{equation} 
Of course population by remaining years, $P_y$, is a function of $P_a$ and the
deaths distribution, $d_a$, and we know that $P_a$ is a function of past births
and survival probabilities, $P_a = \varsigma B(t-a)p_a$, (assuming constant
mortality and proportion male of births, $\varsigma$). So, we may rewrite
Equation~\eqref{eq:ex2sexCRunity1} in terms of past births:
\begin{equation}
\label{eq:ex2sexCRunity2}
\begin{split}
B(t) = \int_{y=0}^\infty \int_{y'=0}^\infty
A\left(R_{y,y'},\mathbb{M}\left(\;Fy \int_{a=0}^\infty\varsigma
B(t-a)d_{a+y}\dd a\;\;,\right. \right. \\\left.
\left.F_{y'}\int_{a'=0}^\infty(1-\varsigma) B(t-a')d_{a'+y'} \dd a'\right)\right) \dd y \dd y'
\end{split}
\end{equation}
since the $p_a$ cancels out $\int _a^\infty d_a \dd a$ in the denominator of
Equation~\eqref{eq:dxredist}. As one may suspect, if the hypothetical
population is left to evolve endogenously under constant vital rates, $d_a$
and $F_y$, eventually the size of each new cohort will be related to the size
of the previous cohort by a fixed and constant factor equal to $e^r$, where
$r$ is the two-sex intrinsic growth rate. In this case, we may rewrite
Equation~\eqref{eq:ex2sexCRunity2} in terms of year $t$ births:
\begin{equation}
\label{eq:ex2sexCRunity3}
\begin{split}
B(t) = \int_{y=0}^\infty \int_{y'=0}^\infty
A\left(R_{y,y'},\mathbb{M}\left(\;Fy \int_{a=0}^\infty\varsigma
B(t)e^{-ra}d_{a+y}\dd a\right. \right.\;\;, \\ \left.
\left.F_{y'}\int_{a'=0}^\infty(1-\varsigma) B(t)e^{-ra'}d_{a'+y'} \dd a'\right)\right) \dd y \dd y'
\end{split}
\end{equation}
Dividing out by $B(t)$ we arrive at the familiar Lotka unity-equation form,
which allows us to isolate and estimate $r$ as a function of vital rates in the
initial year:
\begin{equation}
\label{eq:ex2sexCRunity}
1 = \int_{y=0}^\infty \int_{y'=0}^\infty
A\left(R_{y,y'},\mathbb{M}\left(\;\int_{a=y}^\infty F_y \varsigma d_a
e^{-ra} \dd a, \int _{a'=y'}^\infty F_{y'} (1-\varsigma) d_{a'} e^{-ra'} \dd
a'\right)\right) \dd y
\dd y'
\end{equation} 
Fertility rates, $F_y$ and $F_{y'}$ are standard $e$SFR, including both sexes
of offspring, and $\varsigma$ is used to weight sex of
progenitor, not sex of offspring. As will be seen below, in order to fully estimate $r$, it is best
to estimate $r$ and $\varsigma$ together, since there is a pattern to
$\varsigma$ over $y$, and the population structure is expected to change
somewhat between the initial and stable states. 


\FloatBarrier

 
    \subsection{An iterative approach to find $r$}
      
 \FloatBarrier
\label{sec:ex2sexCRit}
Steps to practically solve Equation~\eqref{eq:ex2sexCRunity} for $r$ are
similar to those presented for the two-sex linear case in
Section~\ref{sec:exrenewalit2}. Namely, $r$ and the sex ratio at birth, $S$, are
estimated together in an iterative process, using parameter guesses as starting
values and then updating in each iteration. First derive as inputs the matrix
$R_{y,y'}$ using \eqref{eq:getR}, $e_y$-specific fertility vectors by
sex of progenitor and offspring, and the relevant $d_a$ vectors:
\begin{enumerate}
  \item Decide a starting value for $\hat{S}^0$, such as the initial observed SRB,
although 1.05 is a good enough guess. For Spain 1.07 might be more
reasonable. Use $\hat{S}^0$ to calculate $\varsigma^0$ using:
\begin{equation}
\label{eq:getvarsigma}
\varsigma^0 = \frac{\hat{S}^0}{1+\hat{S}^0}
\end{equation}
  \item A rough estimate of the net reproduction rate, $\widehat{R_0}$ (assuming
  $r=0$) is given by:
 \begin{equation}
 \label{eq:R0roughCR}
 \widehat{R_0} = \int_{y=0}^\infty \int_{y'=0}^\infty
A\left(R_{y,y'},\mathbb{M}\left(\;\int_{a=y}^\infty F_y \varsigma^0 d_a \dd a,
\int _{a'=y'}^\infty F_{y'} (1-\varsigma^0) d_{a'} \dd a'\right)\right)
 \end{equation}
  \item Weight $y$ and $y'$ into Equation~\eqref{eq:R0roughCR} and divide the
  new sum by $\widehat{R_0}$ to arrive at a first estimate of the mean
  generation time (in remaining years of life), $\widehat{T}$

  \item A good starting value $r$, $r^0$, is given by:
   \begin{equation}
   r^0 = \frac{ln(\widehat{R_0})}{\widehat{T}}
   \end{equation}
  \item Plug $r^i$ into Equation~\ref{eq:ex2sexCRunity} to calculate a
  residual, $\delta^i$
  \item Use $\delta^i$ and $\widehat{T}$ to calibrate the estimate of $r$
  using:
  \begin{equation}
  r^{i+1} = r^i + \frac{\delta^i}{\widehat{T} - \frac{\delta^i}{r^i}}
  \end{equation}
  \item Use the improved $r$ to re-estimate the sex ratio at birth, using
  sex-specific fertility rates, $F_y^{M-M}$ (father-son), $F_y^{M-F}$
  (father-daughter), $F_{y'}^{F-F}$ (mother-daughter) and $F_{y'}^{F-M}$
  (mother-son) fertility rates\footnote{This formula is ugly, but it is just
  Equation~\eqref{eq:ex2sexCRunity} twice; once with fertility rates for male
  births and again with fertility rates for female births.}:
  \begin{adjustwidth}{-1in}{0in}
  \begin{equation}
  S^{i+1} = \frac{\int_{y=0}^\infty \int_{y'=0}^\infty
A\left(R_{y,y'},\mathbb{M}\left(\;\int_{a=y}^\infty F_y^{M-M} \varsigma^i
d_ae^{-r^{i+1}a} \dd a, \int _{a'=y'}^\infty F_{y'}^{F-M} (1-\varsigma^i)
d_{a'}e^{-r^{i+1}a'} \dd a'\right)\right)}{\int_{y=0}^\infty \int_{y'=0}^\infty
A\left(R_{y,y'}, \mathbb{M}\left(\;\int_{a=y}^\infty F_y^{M-F} \varsigma^i d_a
e^{-r^{i+1}a} \dd a, \int _{a'=y'}^\infty F_{y'}^{F-F} (1-\varsigma^i)
d_{a'}e^{-r^{i+1}a'} \dd a'\right)\right)}
  \end{equation}
  \end{adjustwidth}
  and then update $\varsigma$ using: $\varsigma^{i+1} =
  \frac{S^{i+1}}{1+S^{i+1}}$ .
  \item With the updated $r$ and $\varsigma$, repeat steps 5-7 until $\delta$
  reduces to 0. Typically one achieves maximum double floating point precision
  in 5-20 iterations, though fewer iterations are required for
  most practical applications.
\end{enumerate}












      
    \subsection{Noteworthy properties of the model}
      
It has been mentioned that the two-sex model presented here is characterized by
a set of desirable properties. First, the ratio of observed to expected counts
taken from the initial year has a substantive meaning in that it summarizes
some manner of association that resembles preference or attraction, but which
cannot possibly be preference or attraction since individual lifespans are
unknown. For this reason, we take this as the metaphorical shadow of preference
after translation to the remaining years perspective. Second, this ratio adjustment 
method performs comparably with iterative proportional fitting when
predicting the distribution of year $t+1$ $e_y$-classified births. This will be illustrated
 empirically in following. Third, if one imagines the underlying age
 distribution within the $e_y$-classified data, the present method-- and indeed
all $e_y$-classified methods that use fertility rates or births crossed by the
remaining years of life of both parents-- accounts for
inter-age-competition on both sides of the exchange. This property is not
trivial, as no optimization is required in order to achieve this. 

\todo{rethink this. include estimates of r and esfr. compare stable esfr to IPF,
H}






\chapter{Reflections}
  \section{A New Problem?}
    \FloatBarrier

\begin{singlespace}
\begin{quote}
 If somebody scratches the spot where he has an itch, do we have to see 
 some progress? Isn't genuine scratching otherwise, or genuine itching 
itching? And can't this reaction to an irritation continue in the 
same way for a long time before a cure for the itching is discovered?
\citep{wittgenstein1984culture}
\end{quote}
\end{singlespace}

It should be clear enough that we have not \textit{solved} the two-sex problem
in the sense of laying the issue to rest once and for once all. With some
luck we will have succeeded in making the problem more tangible for some,
accessible to the extent that results have been made reproducible,Such a claim
would be misplaced and unjustified. Rather, in exploring a subset of



In the name of headway on the two-sex problem, the
present work took a deviation into much hitherto unexplored territory. Indeed,
demographers have known that this territory existed for some time, but have
only scratched its surface by way of approximate indicators. Certainly none had
ever explored the consequences for fertility of thanatological age, and none had
ever reframed demography's most enduring population models in terms of
thanatological age. As a result of this latter excercise, the two-sex problem
has not been solved, but rather been shown in a new light. Further, a new discrepancy 
has been uncovered, namely divergence between the age-structured and $e_y$-structured 
renewal processes, which is oftentimes much greater than divergence between 
the sexes per se. This author makes no claims about the legitimacy of 
the $e_y$-structured renewal model(s) presented here- only of their 
consistency and potential utility. 

As with the two-sex case, there are no
discrepancies per se in the year from which data were used to derive vital
rates, only in future projected years. As such, neither \textit{problem} deals with lived populations, but rather of
modelled populations. Why complicate the practice of population modelling with yet
another perspective? Might it not be the case that transforming a given
population to the $e_y$-perspective is fruitful, but not projecting one? At this
time, results fresh, and we may best wait and contemplate before coming to rash
conclusions. Upon the first illustrations of the two-sex problem (e.g.
\cite{kuczynski1932fertility}), the practice of demography centered itself on
the assumption of female dominance, whereas the science of demography took up
the hobby, still very much alive\footnote{At least one publication on the
topic have been released so far in 2013 at the time of this writing (\citet{Matthews2013})}




  \section{Summary of principle findings}
    
\subsection{Formal results}

\subsubsection{The basic transformation}
The basic transformation underlying all \textit{original} findings in this
dissertation is that of redistributing data classified by chronological age into
remaining years classes, or thanatological age, by means of the period
lifetable (see Equation~\eqref{eq:dxredist} on page~\pageref{eq:dxredist}). This
feat is note genuinely new, as all the parts to the requisite formula were 
already present in the literature, and indeed \citep{miller2001increasing} had
already used the basic component to this transformation in forecasting health
expenditures. The tiny step of integrating over age, so as to arrive at data
\textit{structured} by remaining years of life appears to be new. Further, the
extention of this principle to various kinds of age-structured data, especially
fertility data, but also population stocks in general, is also new.

The graphical device that falls out of this transformation,
akin to the age-structured population pyramid, the remaining years population
\textit{leaf}, is scattered throughout this dissertation. It is essentially the
same as a population pyramid and exhibits no particular innovation as far as
visual communication is concerned. The only difference for demographers will be
to imagine generations shifting downward with the passage of time instead of
upward. While the graphical device is not new, the content-- a picture of a
population reshuffled to transmit information about the death cohorts that would
unfold according to a given lifetable-- certainly is. The population leaf
transmits a much higher density of information than a line graph of a given time
trend in an ageing indicator. It is complementary to the population pyramid in
terms of understanding population ageing and has the potential the make


\subsubsection{Transformation of vital rates}
\subsubsection{The notion of thanatological renovation}
\subsubsection{Development of the single-sex model}
\subsubsection{Translation of two-sex perspectives}
\subsubsection{A new two-sex method for the thanatological perspective}

\subsection{Empirical findings}
All empirical findings arising from this thesis have been based only on data
from the years 1969-2009 for the United States and 1975-2009 for Spain.
Evidently, any novel findings produced therefrom are in need of verification
from a wider array of populations, so as to be placed into context. For the most
basic age-transformation, data are readily available from such
large-scale datasources as the Human Mortality
Database\footnote{\url{www.mortality.org/}}, the Human Lifetable
Database\footnote{\url{www.lifetable.de/}}, the Human Fertility
Database\footnote{\url{www.humanfertlity.org/}}, the World Health
Organization\footnote{\url{http://www.who.int/healthinfo/statistics/mortality/en/index.html}},
and the United Nations Population
Division\footnote{\url{http://esa.un.org/unpd/wpp/unpp/panel_population.htm}},
and other data sources. This is to say that there is ample opportunity to test
and refine many of the hypotheses or empirical findings presented here.
\subsubsection{Stability of thanatological population structure}
Populations structured by remaining years of life have been found 
\subsubsection{Population rejuvenation}
\subsubsection{Difference in estimated growth rates under two age definitions}

  \section{Consequences of principle findings}
    \todo{what are the theoretical, paradigm consequences}
  \section{Future research directions}
    \todo{talk about future lines of work that can be opened up}

\subsection{Refinements}

\subsubsection{The basic transformation}
\subsubsection{Heterogeneity}


\subsection{Extensions}

\subsection{Unexplored domains}



\startappendices

\todo{appendices should be apart from prior chapter. new chapter?}

 %\SweaveOpts{prefix.string=Figures/Appendix1}
 %\appendix{Proofs of Decomposition Formulas Used in Dissertation}
 \label{app:Appendix1}
 % 

\chapter{Fertility rates by remaining years of life under different assumed
reproductive spans}
\label{Appendix:reprospans}
One may rightly wish to see $e_x$-classified fertility rates calculated where
exposures in the denominator are taken only from ages within the known
reproductive span. For many, this will more closely represent the population exposed. Bounding the
original age-clsasified exposures introduces a second problem, namely that of
determining the which age-bounds to use for males and females. Results are
sensitive to the choice, especially when comparing males and females, since 1)
the male reproductive span is much longer than the female span, and 2) the
$e_x$-distributed population shows a greater and steadier sex-imbalance than the
age-classified population. As expected, results are sensitive to the choice of
bounds. In following, Figures~\ref{fig:exSFRsurfUS},~\ref{fig:exSFRsurfES}
and~\ref{fig:exTFR} are reproduced after first limiting original
age-classified exposures to certain reproductive bounds. These include:

\begin{itemize}
  \item ages 15-55 for both males and females (Section~\ref{sec:1555}).
  \item ages 13-49 for females and 15-64 for males (Section~\ref{sec:1364}).
  \item ages higher than the 1st and lower than the 99th
  quantiles of ASFR for males and females separately, with ASFR averaged over the entire period
  studied (Section~\ref{sec:quant}).
  \item ages higher than the 1st and lower than the
  99th quantiles of ASFR for each year for males and females separately. Only
  $e_x$-TFR is presented here (Section~\ref{sec:quantan}).
\end{itemize}

\pagebreak

\section{ages 15-55 for both males and females}
\label{sec:1555}
\begin{figure}[ht!]
        \centering
        \begin{subfigure}
                \centering
                \caption{Male and Female $e_x$-SFR surfaces, 1969-2009, USA,
                redistributing exposures only from ages 15-55}
                \includegraphics[scale = .8]{Figures/eSFRsurfacesUSlim1555}
                \label{fig:exSFRsurfUSlim15_55}
        \end{subfigure}
        \begin{subfigure}
                \centering
                \caption{Male and Female $e_x$-SFR surfaces, 1975-2009, Spain,
                redistributing exposures only from ages 15-55}
                \includegraphics[scale = .8]{Figures/eSFRsurfacesESlim1555} 
                \label{fig:exSFRsurfESlim15_55}
        \end{subfigure}
\end{figure}

\begin{figure}[ht!]
        \centering  
          \caption{Male and Female $e_x$-total fertility rates, Spain
          and USA, 1969-2009}
           % figure produced in
           % /R/Parents_ex.R
           \includegraphics{Figures/exTFRlim1555}
          \label{fig:exTFRlim15_55}
\end{figure}
\pagebreak

\section{ages 13-49 for females and 15-64 for males}
\label{sec:1364}
\begin{figure}[ht!]
        \centering
        \begin{subfigure}
                \centering
                \caption{Male and Female $e_x$-SFR surfaces, 1969-2009, USA,
                redistributing exposures only from ages 13-49 for females and 15-64 for males}
                \includegraphics[scale = .8]{Figures/eSFRsurfacesUSlim1364mixed}
                \label{fig:exSFRsurfUSlim1364}
        \end{subfigure}
        \begin{subfigure}
                \centering
                \caption{Male and Female $e_x$-SFR surfaces, 1975-2009, Spain,
                redistributing exposures only from ages 13-49 for females and 15-64 for males}
                \includegraphics[scale = .8]{Figures/eSFRsurfacesESlim1364mixed} 
                \label{fig:exSFRsurfESlim1364}
        \end{subfigure}
\end{figure}

\begin{figure}[ht!]
        \centering  
          \caption{Male and Female $e_x$-total fertility rates, Spain
          and USA, 1969-2009}
           % figure produced in
           % /R/Parents_ex.R
           \includegraphics{Figures/exTFRlim1364mixed}
          \label{fig:exTFRlim13_64}
\end{figure}

\pagebreak
\section{ages higher than the 1st and lower than the 99th quantiles of ASFR,
full period}
\label{sec:quant}
\begin{figure}[ht!]
        \centering
        \begin{subfigure}
                \centering
                \caption{Male and Female $e_x$-SFR surfaces, 1969-2009, USA,
                redistributing exposures only from the 1st-99th quantiles of
                ASFR over the full period}
                \includegraphics[scale = .8]{Figures/eSFRsurfacesUSlimBquant}
                \label{fig:exSFRsurfUSlimBquant}
        \end{subfigure}
        \begin{subfigure}
                \centering
                \caption{Male and Female $e_x$-SFR surfaces, 1975-2009, Spain,
                redistributing exposures only from the 1st-99th quantiles of
                ASFR over the full period}
                \includegraphics[scale = .8]{Figures/eSFRsurfacesESlimBquant} 
                \label{fig:exSFRsurfESlimBquant}
        \end{subfigure}
\end{figure}

\begin{figure}[ht!]
        \centering  
          \caption{Male and Female $e_x$-total fertility rates, Spain
          and USA, 1969-2009}
           % figure produced in
           % /R/Parents_ex.R
           \includegraphics{Figures/exTFRlimBquant}
          \label{fig:exTFRlimBquant}
\end{figure}
\pagebreak

\section{ages higher than the 1st and lower than the 99th quantiles of ASFR,
each year}
\label{sec:quantan}
In comparing Figures~\ref{fig:exTFRlimBquant}~and~\ref{fig:exTFRlimBquantyr},
one notes that flexibly changing the age bounds included in $e_x$-classified
exposures according to year-to-year changing ASFR quantiles does not make much
difference as compared to holding the same bounds over the entire period. If
the central 98\% of fertility moves over age with time, then year-to-year
flexbility may be desirable. These data do not undergo large enough changes in
these thresholds to justify this practice. Further, surfaces are best rendered
based upon constant bounds.

 \begin{figure}[ht!]
        \centering  
          \caption{Male and Female $e_x$-total fertility rates, Spain
          and USA, 1969-2009}
           % figure produced in
           % /R/Parents_ex.R
           \includegraphics{Figures/exTFRlimBquantyr}
          \label{fig:exTFRlimBquantyr}
\end{figure}

 \chapter{Projected divergence from $e_x$-classified data}
\label{Appendix:exdivergence}


 \chapter{Equation~\ref{eq:exLotkafemales} applied to US and Spanish data:
estimates of $r$, $T^y$ and $R_0$}

\begin{table}
\caption{Intrinsic growth rate, $r$, mean remaining years of life at
reproduction, $T^y$, and net reproduction, $R_0$, according to renewal
equation~\ref{eq:exLotkafemales}, US, 1969-2009.}
\label{tab:exRepUS}
\centering
\makebox[0pt][c]{\parbox{1.1\textwidth}{%
    \begin{minipage}[b]{0.45\hsize}
    \centering
        \caption*{Males}
        % latex table generated in R 2.15.2 by xtable 1.7-0 package
% Mon Mar  4 13:57:49 2013
\begin{tabular}{cccc}
  \hline
 & $r$ & $T^y$ & $R_0$ \\ 
  \hline
1969 & 0.0069 & 41.64 & 1.331 \\ 
  1970 & 0.0077 & 42.07 & 1.381 \\ 
  1971 & 0.0056 & 41.93 & 1.263 \\ 
  1972 & 0.0023 & 41.34 & 1.098 \\ 
  1973 & 0.0007 & 41.27 & 1.028 \\ 
  1974 & 0.0007 & 41.73 & 1.029 \\ 
  1975 & 0.0003 & 41.99 & 1.011 \\ 
  1976 & 0.0002 & 42.12 & 1.009 \\ 
  1977 & 0.0014 & 42.61 & 1.062 \\ 
  1978 & 0.0012 & 42.64 & 1.051 \\ 
  1979 & 0.0023 & 43.16 & 1.106 \\ 
  1980 & 0.0030 & 43.25 & 1.138 \\ 
  1981 & 0.0029 & 43.37 & 1.134 \\ 
  1982 & 0.0031 & 43.59 & 1.144 \\ 
  1983 & 0.0025 & 43.43 & 1.113 \\ 
  1984 & 0.0025 & 43.42 & 1.113 \\ 
  1985 & 0.0030 & 43.43 & 1.137 \\ 
  1986 & 0.0026 & 43.37 & 1.120 \\ 
  1987 & 0.0028 & 43.44 & 1.130 \\ 
  1988 & 0.0033 & 43.53 & 1.156 \\ 
  1989 & 0.0042 & 43.89 & 1.202 \\ 
  1990 & 0.0048 & 44.31 & 1.238 \\ 
  1991 & 0.0040 & 44.30 & 1.196 \\ 
  1992 & 0.0034 & 44.29 & 1.163 \\ 
  1993 & 0.0025 & 43.96 & 1.116 \\ 
  1994 & 0.0018 & 43.93 & 1.084 \\ 
  1995 & 0.0012 & 43.86 & 1.052 \\ 
  1996 & 0.0008 & 44.12 & 1.037 \\ 
  1997 & 0.0005 & 44.43 & 1.022 \\ 
  1998 & 0.0007 & 44.69 & 1.030 \\ 
  1999 & 0.0006 & 44.75 & 1.025 \\ 
  2000 & 0.0011 & 45.00 & 1.049 \\ 
  2001 & 0.0006 & 44.98 & 1.026 \\ 
  2002 & 0.0004 & 45.01 & 1.017 \\ 
  2003 & 0.0007 & 45.25 & 1.034 \\ 
  2004 & 0.0008 & 45.66 & 1.036 \\ 
  2005 & 0.0008 & 45.77 & 1.036 \\ 
  2006 & 0.0015 & 46.35 & 1.074 \\ 
  2007 & 0.0017 & 46.71 & 1.083 \\ 
  2008 & 0.0011 & 46.69 & 1.052 \\ 
  2009 & 0.0002 & 46.75 & 1.007 \\ 
   \hline
\end{tabular}

    \end{minipage}
    \hfill
    \begin{minipage}[b]{0.55\hsize}
    \centering
        \caption*{Females}
        % latex table generated in R 2.15.3 by xtable 1.7-1 package
% Mon Jun 24 05:22:18 2013
\begin{tabular}{cccc}
  \hline
 & $r$ & $T$ & $R_0$ \\ 
  \hline
1969 & 0.0050 & 50.61 & 1.289 \\ 
  1970 & 0.0058 & 51.07 & 1.346 \\ 
  1971 & 0.0038 & 50.91 & 1.211 \\ 
  1972 & 0.0004 & 50.37 & 1.018 \\ 
  1973 & -0.0013 & 50.26 & 0.936 \\ 
  1974 & -0.0015 & 50.71 & 0.929 \\ 
  1975 & -0.0019 & 51.09 & 0.908 \\ 
  1976 & -0.0020 & 51.18 & 0.904 \\ 
  1977 & -0.0007 & 51.73 & 0.966 \\ 
  1978 & -0.0010 & 51.75 & 0.952 \\ 
  1979 & 0.0003 & 52.34 & 1.017 \\ 
  1980 & 0.0010 & 52.30 & 1.055 \\ 
  1981 & 0.0010 & 52.41 & 1.052 \\ 
  1982 & 0.0012 & 52.63 & 1.067 \\ 
  1983 & 0.0006 & 52.36 & 1.031 \\ 
  1984 & 0.0007 & 52.35 & 1.037 \\ 
  1985 & 0.0013 & 52.39 & 1.068 \\ 
  1986 & 0.0010 & 52.34 & 1.056 \\ 
  1987 & 0.0013 & 52.38 & 1.072 \\ 
  1988 & 0.0020 & 52.45 & 1.109 \\ 
  1989 & 0.0029 & 52.90 & 1.167 \\ 
  1990 & 0.0037 & 53.23 & 1.216 \\ 
  1991 & 0.0031 & 53.22 & 1.177 \\ 
  1992 & 0.0023 & 53.15 & 1.131 \\ 
  1993 & 0.0014 & 52.70 & 1.077 \\ 
  1994 & 0.0008 & 52.60 & 1.041 \\ 
  1995 & 0.0000 & 52.39 & 1.001 \\ 
  1996 & -0.0003 & 52.40 & 0.985 \\ 
  1997 & -0.0007 & 52.45 & 0.966 \\ 
  1998 & -0.0004 & 52.57 & 0.978 \\ 
  1999 & -0.0006 & 52.46 & 0.971 \\ 
  2000 & 0.0000 & 52.61 & 1.002 \\ 
  2001 & -0.0004 & 52.54 & 0.980 \\ 
  2002 & -0.0006 & 52.52 & 0.967 \\ 
  2003 & -0.0003 & 52.66 & 0.986 \\ 
  2004 & -0.0002 & 53.04 & 0.988 \\ 
  2005 & -0.0002 & 53.11 & 0.989 \\ 
  2006 & 0.0006 & 53.65 & 1.033 \\ 
  2007 & 0.0009 & 53.98 & 1.048 \\ 
  2008 & 0.0002 & 53.84 & 1.009 \\ 
  2009 & -0.0009 & 53.83 & 0.955 \\ 
   \hline
\end{tabular}

    \end{minipage}
}}
\end{table}

\begin{table}
\caption{Intrinsic growth rate, $r$, mean remaining years of life at
reproduction, $T^y$, and net reproduction, $R_0$, according to renewal
equation~\ref{eq:exLotkafemales}, Spain, 1975-2009.}
\label{tab:exRepUS}
\centering
\makebox[0pt][c]{\parbox{1.1\textwidth}{%
    \begin{minipage}[b]{0.45\hsize}
    \centering
        \caption*{Males}
        % latex table generated in R 2.15.2 by xtable 1.7-0 package
% Mon Mar  4 14:11:51 2013
\begin{tabular}{cccc}
  \hline
 & $r$ & $T^y$ & $R_0$ \\ 
  \hline
1975 & 0.0095 & 42.14 & 1.492 \\ 
  1976 & 0.0095 & 42.61 & 1.499 \\ 
  1977 & 0.0083 & 42.90 & 1.429 \\ 
  1978 & 0.0071 & 42.94 & 1.359 \\ 
  1979 & 0.0051 & 43.04 & 1.244 \\ 
  1980 & 0.0034 & 43.20 & 1.160 \\ 
  1981 & 0.0013 & 42.91 & 1.058 \\ 
  1982 & 0.0002 & 43.06 & 1.008 \\ 
  1983 & -0.0020 & 42.47 & 0.919 \\ 
  1984 & -0.0028 & 42.44 & 0.889 \\ 
  1985 & -0.0041 & 42.09 & 0.841 \\ 
  1986 & -0.0053 & 42.12 & 0.799 \\ 
  1987 & -0.0062 & 42.06 & 0.771 \\ 
  1988 & -0.0069 & 41.86 & 0.750 \\ 
  1989 & -0.0077 & 41.66 & 0.726 \\ 
  1990 & -0.0082 & 41.48 & 0.710 \\ 
  1991 & -0.0086 & 41.32 & 0.700 \\ 
  1992 & -0.0086 & 41.42 & 0.700 \\ 
  1993 & -0.0094 & 41.18 & 0.680 \\ 
  1994 & -0.0105 & 40.94 & 0.652 \\ 
  1995 & -0.0110 & 40.65 & 0.639 \\ 
  1996 & -0.0111 & 40.58 & 0.636 \\ 
  1997 & -0.0107 & 41.04 & 0.646 \\ 
  1998 & -0.0110 & 41.03 & 0.638 \\ 
  1999 & -0.0102 & 41.10 & 0.658 \\ 
  2000 & -0.0090 & 41.70 & 0.687 \\ 
  2001 & -0.0089 & 41.94 & 0.689 \\ 
  2002 & -0.0084 & 42.14 & 0.701 \\ 
  2003 & -0.0075 & 42.30 & 0.727 \\ 
  2004 & -0.0070 & 42.79 & 0.740 \\ 
  2005 & -0.0069 & 42.80 & 0.743 \\ 
  2006 & -0.0063 & 43.43 & 0.762 \\ 
  2007 & -0.0062 & 43.46 & 0.763 \\ 
  2008 & -0.0050 & 44.00 & 0.801 \\ 
  2009 & -0.0063 & 43.89 & 0.759 \\ 
   \hline
\end{tabular}

    \end{minipage}
    \hfill
    \begin{minipage}[b]{0.55\hsize}
    \centering
        \caption*{Females}
        % latex table generated in R 2.15.2 by xtable 1.7-0 package
% Mon Mar  4 14:11:51 2013
\begin{tabular}{cccc}
  \hline
 & $r$ & $T^y$ & $R_0$ \\ 
  \hline
1975 & 0.0078 & 50.12 & 1.479 \\ 
  1976 & 0.0081 & 50.68 & 1.510 \\ 
  1977 & 0.0067 & 51.02 & 1.409 \\ 
  1978 & 0.0053 & 51.14 & 1.313 \\ 
  1979 & 0.0033 & 51.40 & 1.186 \\ 
  1980 & 0.0012 & 51.45 & 1.065 \\ 
  1981 & -0.0015 & 51.21 & 0.927 \\ 
  1982 & -0.0026 & 51.40 & 0.875 \\ 
  1983 & -0.0046 & 50.85 & 0.792 \\ 
  1984 & -0.0056 & 51.08 & 0.752 \\ 
  1985 & -0.0068 & 50.81 & 0.709 \\ 
  1986 & -0.0081 & 50.70 & 0.664 \\ 
  1987 & -0.0090 & 50.75 & 0.632 \\ 
  1988 & -0.0097 & 50.71 & 0.613 \\ 
  1989 & -0.0105 & 50.63 & 0.589 \\ 
  1990 & -0.0110 & 50.50 & 0.573 \\ 
  1991 & -0.0115 & 50.39 & 0.560 \\ 
  1992 & -0.0114 & 50.69 & 0.562 \\ 
  1993 & -0.0123 & 50.35 & 0.539 \\ 
  1994 & -0.0133 & 50.12 & 0.513 \\ 
  1995 & -0.0138 & 49.92 & 0.502 \\ 
  1996 & -0.0139 & 49.88 & 0.501 \\ 
  1997 & -0.0134 & 50.20 & 0.510 \\ 
  1998 & -0.0139 & 50.07 & 0.497 \\ 
  1999 & -0.0128 & 50.22 & 0.526 \\ 
  2000 & -0.0118 & 50.74 & 0.551 \\ 
  2001 & -0.0112 & 51.07 & 0.566 \\ 
  2002 & -0.0108 & 51.23 & 0.574 \\ 
  2003 & -0.0097 & 51.25 & 0.608 \\ 
  2004 & -0.0093 & 51.83 & 0.618 \\ 
  2005 & -0.0088 & 51.83 & 0.633 \\ 
  2006 & -0.0081 & 52.47 & 0.654 \\ 
  2007 & -0.0079 & 52.55 & 0.661 \\ 
  2008 & -0.0066 & 52.92 & 0.703 \\ 
  2009 & -0.0081 & 52.78 & 0.651 \\ 
   \hline
\end{tabular}

    \end{minipage}
}}
\end{table}


 \chapter{Equation~\ref{eq:lineartwosexrenewal} applied to US and Spanish data}
\label{appendix:ex2sexlinear}
This appendix provides numerical results from the application of the
estimation techniques described in Section~\ref{sec:exLotka2linear}. Estimates
of $r^\upsilon$, $T^\upsilon$ and $R_0^\upsilon$ under three values of $\sigma$.
Recall that $\sigma$ determines the relative dominance of each sex, where 0 is
female dominance, 1 is male dominant, and .5 is an intermediate level of
dominance (which is not necessarily an arithmetic mean). As stated in the
main text, the border cases where $\sigma = [0,1]$ do not revert to the
single-sex $e_x$-structured values presented in Section~\ref{sec:exstructuredrenewal} and
Appendix~\ref{appendix:exlotka1sex}.

\begin{table}
  \begin{adjustwidth}{-1in}{-.5in}
  \centering
    \caption{Two-sex linear intrinsic growth rate, $r^\upsilon$, mean remaining
years of life at reproduction, $T^\upsilon$, and net reproduction,
$R_0^\upsilon$, according to renewal Equation~\eqref{eq:lineartwosexrenewal},
with $\sigma = [0, .5, 1]$ US, 1969-2009.}
    \label{tab:ex2linRepUS}
        % latex table generated in R 2.15.3 by xtable 1.7-1 package
% Wed May  1 09:28:38 2013
\begin{tabular}{cccccccccc}
  \hline
 & $r^{\upsilon (\sigma = 0)}$ & $r^{\upsilon (\sigma = .5)}$ & $r^{\upsilon (\sigma = 1)}$ & $T^{\upsilon (\sigma = 0)}$ & $T^{\upsilon (\sigma = .5)}$ & $T^{\upsilon (\sigma = 1)}$ & $R_0^{\upsilon (\sigma = 0)}$ & $R_0^{\upsilon (\sigma = .5)}$ & $R_0^{\upsilon (\sigma = 1)}$ \\ 
  \hline
1969 & 0.0050 & 0.0060 & 0.0069 & 50.62 & 46.01 & 41.63 & 1.289 & 1.316 & 1.330 \\ 
  1970 & 0.0058 & 0.0068 & 0.0077 & 51.08 & 46.45 & 42.06 & 1.346 & 1.370 & 1.380 \\ 
  1971 & 0.0038 & 0.0047 & 0.0056 & 50.91 & 46.30 & 41.92 & 1.211 & 1.243 & 1.263 \\ 
  1972 & 0.0004 & 0.0013 & 0.0023 & 50.38 & 45.73 & 41.33 & 1.018 & 1.063 & 1.098 \\ 
  1973 & -0.0013 & -0.0003 & 0.0007 & 50.27 & 45.63 & 41.26 & 0.936 & 0.987 & 1.028 \\ 
  1974 & -0.0015 & -0.0003 & 0.0007 & 50.72 & 46.08 & 41.72 & 0.929 & 0.985 & 1.029 \\ 
  1975 & -0.0019 & -0.0008 & 0.0003 & 51.10 & 46.39 & 41.97 & 0.908 & 0.965 & 1.011 \\ 
  1976 & -0.0020 & -0.0008 & 0.0002 & 51.18 & 46.50 & 42.11 & 0.904 & 0.962 & 1.009 \\ 
  1977 & -0.0007 & 0.0004 & 0.0014 & 51.73 & 47.03 & 42.60 & 0.966 & 1.020 & 1.062 \\ 
  1978 & -0.0010 & 0.0002 & 0.0012 & 51.75 & 47.05 & 42.64 & 0.952 & 1.007 & 1.051 \\ 
  1979 & 0.0003 & 0.0014 & 0.0023 & 52.34 & 47.61 & 43.15 & 1.017 & 1.068 & 1.106 \\ 
  1980 & 0.0010 & 0.0021 & 0.0030 & 52.30 & 47.64 & 43.24 & 1.055 & 1.103 & 1.138 \\ 
  1981 & 0.0010 & 0.0020 & 0.0029 & 52.42 & 47.76 & 43.36 & 1.052 & 1.099 & 1.134 \\ 
  1982 & 0.0012 & 0.0022 & 0.0031 & 52.63 & 47.99 & 43.59 & 1.067 & 1.111 & 1.144 \\ 
  1983 & 0.0006 & 0.0016 & 0.0025 & 52.37 & 47.77 & 43.42 & 1.031 & 1.078 & 1.113 \\ 
  1984 & 0.0007 & 0.0016 & 0.0025 & 52.35 & 47.77 & 43.42 & 1.037 & 1.080 & 1.113 \\ 
  1985 & 0.0013 & 0.0021 & 0.0030 & 52.39 & 47.79 & 43.43 & 1.068 & 1.108 & 1.137 \\ 
  1986 & 0.0010 & 0.0019 & 0.0026 & 52.34 & 47.75 & 43.36 & 1.056 & 1.093 & 1.120 \\ 
  1987 & 0.0013 & 0.0021 & 0.0028 & 52.39 & 47.81 & 43.44 & 1.072 & 1.106 & 1.130 \\ 
  1988 & 0.0020 & 0.0027 & 0.0033 & 52.45 & 47.89 & 43.52 & 1.109 & 1.137 & 1.156 \\ 
  1989 & 0.0029 & 0.0036 & 0.0042 & 52.90 & 48.34 & 43.95 & 1.167 & 1.189 & 1.201 \\ 
  1990 & 0.0037 & 0.0043 & 0.0048 & 53.23 & 48.69 & 44.30 & 1.216 & 1.231 & 1.238 \\ 
  1991 & 0.0031 & 0.0036 & 0.0040 & 53.22 & 48.69 & 44.30 & 1.177 & 1.190 & 1.196 \\ 
  1992 & 0.0023 & 0.0029 & 0.0034 & 53.15 & 48.65 & 44.29 & 1.131 & 1.151 & 1.163 \\ 
  1993 & 0.0014 & 0.0020 & 0.0025 & 52.70 & 48.25 & 43.95 & 1.077 & 1.100 & 1.116 \\ 
  1994 & 0.0008 & 0.0013 & 0.0018 & 52.60 & 48.19 & 43.93 & 1.041 & 1.066 & 1.084 \\ 
  1995 & 0.0000 & 0.0006 & 0.0012 & 52.40 & 48.05 & 43.86 & 1.001 & 1.030 & 1.052 \\ 
  1996 & -0.0003 & 0.0003 & 0.0008 & 52.41 & 48.19 & 44.12 & 0.985 & 1.014 & 1.037 \\ 
  1997 & -0.0007 & -0.0001 & 0.0005 & 52.45 & 48.36 & 44.42 & 0.966 & 0.997 & 1.022 \\ 
  1998 & -0.0004 & 0.0001 & 0.0007 & 52.57 & 48.56 & 44.69 & 0.978 & 1.007 & 1.030 \\ 
  1999 & -0.0006 & 0.0000 & 0.0006 & 52.46 & 48.54 & 44.75 & 0.971 & 1.001 & 1.025 \\ 
  2000 & 0.0000 & 0.0006 & 0.0011 & 52.61 & 48.74 & 44.99 & 1.002 & 1.028 & 1.049 \\ 
  2001 & -0.0004 & 0.0001 & 0.0006 & 52.54 & 48.70 & 44.97 & 0.980 & 1.006 & 1.026 \\ 
  2002 & -0.0006 & -0.0001 & 0.0004 & 52.53 & 48.70 & 45.00 & 0.967 & 0.995 & 1.017 \\ 
  2003 & -0.0003 & 0.0003 & 0.0007 & 52.66 & 48.89 & 45.24 & 0.986 & 1.013 & 1.034 \\ 
  2004 & -0.0002 & 0.0003 & 0.0008 & 53.04 & 49.29 & 45.66 & 0.988 & 1.015 & 1.036 \\ 
  2005 & -0.0002 & 0.0003 & 0.0008 & 53.11 & 49.38 & 45.77 & 0.989 & 1.015 & 1.036 \\ 
  2006 & 0.0006 & 0.0011 & 0.0015 & 53.65 & 49.94 & 46.34 & 1.033 & 1.056 & 1.074 \\ 
  2007 & 0.0009 & 0.0013 & 0.0017 & 53.98 & 50.29 & 46.71 & 1.048 & 1.068 & 1.083 \\ 
  2008 & 0.0002 & 0.0006 & 0.0011 & 53.84 & 50.21 & 46.69 & 1.009 & 1.033 & 1.052 \\ 
  2009 & -0.0009 & -0.0003 & 0.0002 & 53.84 & 50.23 & 46.74 & 0.955 & 0.984 & 1.007 \\ 
   \hline
\end{tabular}

  \end{adjustwidth}
\end{table}

\begin{table}
  \begin{adjustwidth}{-1in}{-.5in}
    \centering
    \caption{Two-sex linear intrinsic growth rate, $r^\upsilon$, mean remaining
years of life at reproduction, $T^\upsilon$, and net reproduction,
$R_0^\upsilon$, according to renewal Equation~\eqref{eq:lineartwosexrenewal}, with
$\sigma = [0, .5, 1]$ Spain, 1975-2009.}
    \label{tab:ex2linRepES}
        % latex table generated in R 2.15.2 by xtable 1.7-0 package
% Wed Mar  6 13:00:01 2013
\begin{tabular}{cccccccccc}
  \hline
 & $r^{\upsilon (\sigma = 0)}$ & $r^{\upsilon (\sigma = .5)}$ & $r^{\upsilon (\sigma = 1)}$ & $T^{\upsilon (\sigma = 0)}$ & $T^{\upsilon (\sigma = .5)}$ & $T^{\upsilon (\sigma = 1)}$ & $R_0^{\upsilon (\sigma = 0)}$ & $R_0^{\upsilon (\sigma = .5)}$ & $R_0^{\upsilon (\sigma = 1)}$ \\ 
  \hline
1975 & 0.0085 & 0.0085 & 0.0084 & 49.91 & 46.04 & 42.16 & 1.531 & 1.476 & 1.426 \\ 
  1976 & 0.0087 & 0.0086 & 0.0085 & 50.43 & 46.56 & 42.68 & 1.550 & 1.493 & 1.440 \\ 
  1977 & 0.0072 & 0.0073 & 0.0074 & 50.76 & 46.85 & 42.97 & 1.443 & 1.409 & 1.374 \\ 
  1978 & 0.0058 & 0.0060 & 0.0062 & 50.86 & 46.91 & 43.01 & 1.342 & 1.327 & 1.307 \\ 
  1979 & 0.0035 & 0.0040 & 0.0044 & 51.08 & 47.07 & 43.16 & 1.194 & 1.207 & 1.209 \\ 
  1980 & 0.0014 & 0.0022 & 0.0027 & 51.15 & 47.14 & 43.30 & 1.075 & 1.107 & 1.126 \\ 
  1981 & -0.0013 & -0.0002 & 0.0006 & 50.91 & 46.84 & 43.01 & 0.938 & 0.990 & 1.025 \\ 
  1982 & -0.0025 & -0.0013 & -0.0004 & 51.09 & 46.99 & 43.16 & 0.879 & 0.939 & 0.982 \\ 
  1983 & -0.0048 & -0.0034 & -0.0023 & 50.52 & 46.41 & 42.63 & 0.785 & 0.854 & 0.906 \\ 
  1984 & -0.0058 & -0.0043 & -0.0031 & 50.71 & 46.48 & 42.61 & 0.743 & 0.820 & 0.877 \\ 
  1985 & -0.0073 & -0.0055 & -0.0042 & 50.42 & 46.15 & 42.29 & 0.694 & 0.775 & 0.836 \\ 
  1986 & -0.0086 & -0.0068 & -0.0054 & 50.32 & 46.10 & 42.32 & 0.648 & 0.732 & 0.797 \\ 
  1987 & -0.0096 & -0.0077 & -0.0062 & 50.34 & 46.07 & 42.29 & 0.616 & 0.703 & 0.771 \\ 
  1988 & -0.0104 & -0.0083 & -0.0067 & 50.26 & 45.93 & 42.13 & 0.592 & 0.682 & 0.753 \\ 
  1989 & -0.0114 & -0.0091 & -0.0074 & 50.17 & 45.77 & 41.94 & 0.566 & 0.659 & 0.732 \\ 
  1990 & -0.0120 & -0.0097 & -0.0079 & 50.04 & 45.60 & 41.76 & 0.549 & 0.643 & 0.718 \\ 
  1991 & -0.0125 & -0.0101 & -0.0083 & 49.92 & 45.46 & 41.62 & 0.537 & 0.633 & 0.708 \\ 
  1992 & -0.0124 & -0.0100 & -0.0082 & 50.19 & 45.64 & 41.74 & 0.536 & 0.633 & 0.710 \\ 
  1993 & -0.0133 & -0.0108 & -0.0090 & 49.88 & 45.36 & 41.49 & 0.516 & 0.612 & 0.688 \\ 
  1994 & -0.0144 & -0.0119 & -0.0100 & 49.63 & 45.10 & 41.26 & 0.489 & 0.585 & 0.662 \\ 
  1995 & -0.0150 & -0.0124 & -0.0105 & 49.42 & 44.84 & 40.98 & 0.477 & 0.574 & 0.651 \\ 
  1996 & -0.0151 & -0.0125 & -0.0106 & 49.38 & 44.79 & 40.91 & 0.476 & 0.572 & 0.649 \\ 
  1997 & -0.0145 & -0.0120 & -0.0102 & 49.75 & 45.20 & 41.33 & 0.487 & 0.581 & 0.656 \\ 
  1998 & -0.0149 & -0.0124 & -0.0106 & 49.67 & 45.13 & 41.26 & 0.478 & 0.571 & 0.646 \\ 
  1999 & -0.0138 & -0.0115 & -0.0097 & 49.80 & 45.26 & 41.35 & 0.503 & 0.595 & 0.668 \\ 
  2000 & -0.0126 & -0.0104 & -0.0088 & 50.35 & 45.85 & 41.93 & 0.532 & 0.621 & 0.692 \\ 
  2001 & -0.0121 & -0.0100 & -0.0085 & 50.65 & 46.15 & 42.20 & 0.542 & 0.629 & 0.698 \\ 
  2002 & -0.0116 & -0.0097 & -0.0082 & 50.83 & 46.33 & 42.37 & 0.554 & 0.639 & 0.707 \\ 
  2003 & -0.0104 & -0.0087 & -0.0074 & 50.89 & 46.47 & 42.51 & 0.589 & 0.668 & 0.731 \\ 
  2004 & -0.0099 & -0.0082 & -0.0070 & 51.48 & 47.00 & 42.98 & 0.602 & 0.680 & 0.741 \\ 
  2005 & -0.0094 & -0.0080 & -0.0068 & 51.49 & 47.04 & 43.00 & 0.615 & 0.688 & 0.746 \\ 
  2006 & -0.0086 & -0.0073 & -0.0062 & 52.13 & 47.69 & 43.61 & 0.639 & 0.707 & 0.762 \\ 
  2007 & -0.0084 & -0.0071 & -0.0062 & 52.21 & 47.76 & 43.65 & 0.646 & 0.711 & 0.763 \\ 
  2008 & -0.0069 & -0.0059 & -0.0052 & 52.63 & 48.25 & 44.15 & 0.694 & 0.751 & 0.796 \\ 
  2009 & -0.0084 & -0.0073 & -0.0064 & 52.50 & 48.11 & 44.02 & 0.643 & 0.705 & 0.755 \\ 
   \hline
\end{tabular}

  \end{adjustwidth}
\end{table}




 \chapter{Construction of the standard one-sex Leslie matrix}
\label{Appendix:Caswell}
The Leslie matrix is a tool used for age-structured cohort component population
projections. Here we present the elements that correspond to a simple one-sex
age-structured population. More details on each point presented here can be
found in \cite{caswell2001matrix}.

Say we have a population with $n$ age-classes. Call the vector of age-specific
population counts $\textbf{p}$



 
 \nocite{*} % cites everything in References.bib
\startbibliography
 \begin{singlespace} % Bibliography must be single spaced
  \bibliography{References}   % Use the BibTeX file ``References.bib''.
 \end{singlespace}

\end{document}
