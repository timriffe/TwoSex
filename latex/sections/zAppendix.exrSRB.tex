
\chapter{Results from remaining-years structured renewal models: $r$ and SRB.}
\label{appendix:exallrestimates}

This appendix provides numerical results from all remaining-years two-sex
methods applied to the US and Spanish populations for the years 1969-2009 and
1975-2009, respectively. The two results to be listed are the intrinsic growth
rate, $r$, and the intrinsic sex ratio at birth, $S$, which strays slightly from
the initial value of the sex ratio at birth due to changes in population
structure between the initial and stable states and our inclusion of an
age-pattern to the sex ratio at birth for males and females via sex-of-birth
specific fertility rates. These results are placed into four tables, first $r$
for the US (Table~\ref{tab:USexALL}), followed by $S$
(Table~\ref{tab:USexSRBALL}) for the US, then $r$ and $S$ for the Spanish
population (Tables~\ref{tab:ESexALL} and~\ref{tab:ESexSRBALL}). Throughout, we
use superscripts in the column headers to identify the model according to the following key:
\begin{description}
  \item[$r^m$] Section~\ref{sec:ex2sexequation} single-sex male.
  \item[$r^f$] Section~\ref{sec:ex2sexequation} single-sex female.
  \item[$r^{\sigma=1}$] Chapter~\ref{sec:ex2sexdomweights} weighted-dominance
  method with 100\% male information. This is identical to the single-sex male rate.
  \item[$r^{\sigma=0}$] Chapter~\ref{sec:ex2sexdomweights} weighted-dominance
  method with 100\% female information. This is identical to the single-sex female rate.
  \item[$r^{\sigma=0.5}$] Chapter~\ref{sec:ex2sexdomweights} weighted-dominance
  method with information split 50-50 between males and females.
  \item[$r^{HM}$] Chapter~\ref{sec:ex2sexschoen} mean method on the basis of
  the harmonic mean.
  \item[$r^{GM}$] Chapter~\ref{sec:ex2sexschoen} mean method on the basis of
  the geometric mean.
  \item[$r^{LM}$] Chapter~\ref{sec:ex2sexschoen} mean method on the basis of
  the logarithmic mean.
  \item[$r^{RADJ-HM}$] Chapter~\ref{sec:CRchap} ratio-adjustment method with
  male and female marginal birth predictions balanced by the harmonic mean prior to
  calculating the expected distribution, followed by the ratio-adjustment.
  \item[$r^{IPF-HM}$] Chapter~\ref{sec:ipfex} with male and female marginal
  birth predictions balanced by the harmonic mean prior to re-estimating rates
  using iterative proportional fitting.
\end{description}
The same superscripts are used for stable sex ratios as birth, where $S(t)$
simply refers to the observed sex ratio at birth for the given year. Results
with full digit precision are available by executing the accompanying \texttt{R}
code. Such precision should not give a false sense of exactitude, however, but
serves only for verification when reproducing results. These estimates were
arrived at by following the step-by-step instructions outlined in the text.
Notably, as mentioned in the text, the sex ratio at birth, $S$, does not vary
greatly between the initial and stable states, typically varying between
methods only in the \nth{5} digit. This should put the reader at ease if
questioning the stability of optimizing two parameters simultaneously. The
stable $S$ will only differ if there is both a pattern over remaining years
and a difference in remaining-years population structure in stability.

\begin{table}
  \begin{adjustwidth}{-1in}{-.5in}
  \centering
    \caption{Intrinsic growth rates, $r$, from remaining-years renewal models.
    US, 1969-2009.}
    \label{tab:USexALL}
        \small{% latex table generated in R 2.15.3 by xtable 1.7-1 package
% Thu May  9 15:38:01 2013
\begin{tabular}{ccccccccccc}
  \hline
 & $r^m$ & $r^f$ & $r^{(\sigma = 1)}$ & $r^{(\sigma = 0)}$ & $r^{(\sigma = 0.5)}$ & $r^{HM}$ & $r^{GM}$ & $r^{LM}$ & $r^{RAdj-HM}$ & $r^{IPF-HM}$ \\ 
  \hline
1969 & 0.0069 & 0.0050 & 0.0069 & 0.0050 & 0.0060 & 0.0059 & 0.0059 & 0.0059 & 0.0041 & 0.0059 \\ 
  1970 & 0.0077 & 0.0058 & 0.0077 & 0.0058 & 0.0068 & 0.0067 & 0.0067 & 0.0067 & 0.0046 & 0.0068 \\ 
  1971 & 0.0056 & 0.0038 & 0.0056 & 0.0038 & 0.0047 & 0.0046 & 0.0046 & 0.0046 & 0.0032 & 0.0047 \\ 
  1972 & 0.0023 & 0.0004 & 0.0023 & 0.0004 & 0.0013 & 0.0012 & 0.0012 & 0.0012 & 0.0009 & 0.0013 \\ 
  1973 & 0.0007 & -0.0013 & 0.0007 & -0.0013 & -0.0003 & -0.0005 & -0.0005 & -0.0005 & -0.0002 & -0.0003 \\ 
  1974 & 0.0007 & -0.0015 & 0.0007 & -0.0015 & -0.0003 & -0.0006 & -0.0005 & -0.0005 & -0.0002 & -0.0004 \\ 
  1975 & 0.0003 & -0.0019 & 0.0003 & -0.0019 & -0.0008 & -0.0010 & -0.0010 & -0.0010 & -0.0005 & -0.0008 \\ 
  1976 & 0.0002 & -0.0020 & 0.0002 & -0.0020 & -0.0008 & -0.0011 & -0.0010 & -0.0010 & -0.0006 & -0.0009 \\ 
  1977 & 0.0014 & -0.0007 & 0.0014 & -0.0007 & 0.0004 & 0.0002 & 0.0002 & 0.0002 & 0.0003 & 0.0004 \\ 
  1978 & 0.0012 & -0.0010 & 0.0012 & -0.0010 & 0.0002 & -0.0000 & -0.0000 & -0.0000 & 0.0001 & 0.0001 \\ 
  1979 & 0.0023 & 0.0003 & 0.0023 & 0.0003 & 0.0014 & 0.0013 & 0.0012 & 0.0012 & 0.0009 & 0.0013 \\ 
  1980 & 0.0030 & 0.0010 & 0.0030 & 0.0010 & 0.0021 & 0.0020 & 0.0019 & 0.0019 & 0.0014 & 0.0020 \\ 
  1981 & 0.0029 & 0.0010 & 0.0029 & 0.0010 & 0.0020 & 0.0019 & 0.0019 & 0.0018 & 0.0013 & 0.0020 \\ 
  1982 & 0.0031 & 0.0012 & 0.0031 & 0.0012 & 0.0022 & 0.0021 & 0.0021 & 0.0021 & 0.0014 & 0.0022 \\ 
  1983 & 0.0025 & 0.0006 & 0.0025 & 0.0006 & 0.0016 & 0.0015 & 0.0014 & 0.0014 & 0.0010 & 0.0015 \\ 
  1984 & 0.0025 & 0.0007 & 0.0025 & 0.0007 & 0.0016 & 0.0016 & 0.0015 & 0.0015 & 0.0011 & 0.0016 \\ 
  1985 & 0.0030 & 0.0013 & 0.0030 & 0.0013 & 0.0021 & 0.0021 & 0.0020 & 0.0020 & 0.0014 & 0.0021 \\ 
  1986 & 0.0026 & 0.0010 & 0.0026 & 0.0010 & 0.0019 & 0.0018 & 0.0018 & 0.0017 & 0.0012 & 0.0019 \\ 
  1987 & 0.0028 & 0.0013 & 0.0028 & 0.0013 & 0.0021 & 0.0021 & 0.0020 & 0.0020 & 0.0014 & 0.0021 \\ 
  1988 & 0.0033 & 0.0020 & 0.0033 & 0.0020 & 0.0027 & 0.0027 & 0.0026 & 0.0026 & 0.0018 & 0.0027 \\ 
  1989 & 0.0042 & 0.0029 & 0.0042 & 0.0029 & 0.0036 & 0.0036 & 0.0035 & 0.0035 & 0.0024 & 0.0036 \\ 
  1990 & 0.0048 & 0.0037 & 0.0048 & 0.0037 & 0.0043 & 0.0043 & 0.0042 & 0.0042 & 0.0028 & 0.0043 \\ 
  1991 & 0.0040 & 0.0031 & 0.0040 & 0.0031 & 0.0036 & 0.0036 & 0.0035 & 0.0035 & 0.0024 & 0.0036 \\ 
  1992 & 0.0034 & 0.0023 & 0.0034 & 0.0023 & 0.0029 & 0.0029 & 0.0028 & 0.0028 & 0.0019 & 0.0029 \\ 
  1993 & 0.0025 & 0.0014 & 0.0025 & 0.0014 & 0.0020 & 0.0020 & 0.0019 & 0.0019 & 0.0013 & 0.0020 \\ 
  1994 & 0.0018 & 0.0008 & 0.0018 & 0.0008 & 0.0013 & 0.0013 & 0.0012 & 0.0012 & 0.0009 & 0.0013 \\ 
  1995 & 0.0012 & 0.0000 & 0.0012 & 0.0000 & 0.0006 & 0.0006 & 0.0005 & 0.0005 & 0.0004 & 0.0006 \\ 
  1996 & 0.0008 & -0.0003 & 0.0008 & -0.0003 & 0.0003 & 0.0003 & 0.0002 & 0.0002 & 0.0002 & 0.0003 \\ 
  1997 & 0.0005 & -0.0007 & 0.0005 & -0.0007 & -0.0001 & -0.0001 & -0.0002 & -0.0002 & -0.0000 & -0.0001 \\ 
  1998 & 0.0007 & -0.0004 & 0.0007 & -0.0004 & 0.0001 & 0.0001 & 0.0000 & 0.0000 & 0.0001 & 0.0001 \\ 
  1999 & 0.0006 & -0.0006 & 0.0006 & -0.0006 & 0.0000 & -0.0000 & -0.0001 & -0.0001 & 0.0000 & 0.0000 \\ 
  2000 & 0.0011 & 0.0000 & 0.0011 & 0.0000 & 0.0006 & 0.0005 & 0.0005 & 0.0005 & 0.0004 & 0.0006 \\ 
  2001 & 0.0006 & -0.0004 & 0.0006 & -0.0004 & 0.0001 & 0.0001 & 0.0000 & 0.0000 & 0.0001 & 0.0001 \\ 
  2002 & 0.0004 & -0.0006 & 0.0004 & -0.0006 & -0.0001 & -0.0001 & -0.0002 & -0.0002 & -0.0001 & -0.0001 \\ 
  2003 & 0.0007 & -0.0003 & 0.0007 & -0.0003 & 0.0003 & 0.0002 & 0.0002 & 0.0002 & 0.0002 & 0.0003 \\ 
  2004 & 0.0008 & -0.0002 & 0.0008 & -0.0002 & 0.0003 & 0.0003 & 0.0002 & 0.0002 & 0.0002 & 0.0003 \\ 
  2005 & 0.0008 & -0.0002 & 0.0008 & -0.0002 & 0.0003 & 0.0003 & 0.0002 & 0.0002 & 0.0002 & 0.0003 \\ 
  2006 & 0.0015 & 0.0006 & 0.0015 & 0.0006 & 0.0011 & 0.0011 & 0.0010 & 0.0010 & 0.0007 & 0.0011 \\ 
  2007 & 0.0017 & 0.0009 & 0.0017 & 0.0009 & 0.0013 & 0.0013 & 0.0013 & 0.0012 & 0.0008 & 0.0013 \\ 
  2008 & 0.0011 & 0.0002 & 0.0011 & 0.0002 & 0.0006 & 0.0006 & 0.0006 & 0.0006 & 0.0004 & 0.0006 \\ 
  2009 & 0.0002 & -0.0009 & 0.0002 & -0.0009 & -0.0003 & -0.0004 & -0.0004 & -0.0004 & -0.0002 & -0.0003 \\ 
   \hline
\end{tabular}
}
  \end{adjustwidth}
\end{table}

\begin{table}
  \begin{adjustwidth}{-1in}{-.5in}
  \centering
    \caption{Stable sex ratio at birth, $S$, from remaining-years renewal
    models. US, 1969-2009.}
    \label{tab:USexSRBALL}
        \small{% latex table generated in R 2.15.3 by xtable 1.7-1 package
% Thu May  9 15:46:24 2013
\begin{tabular}{cccccccccc}
  \hline
 & $S(t)$ & $S^{(\sigma = 1)}$ & $S^{(\sigma = 0)}$ & $S^{(\sigma = 0.5)}$ & $S^{HM}$ & $S^{GM}$ & $S^{LM}$ & $S^{RAdj-HM}$ & $S^{IPF-HM}$ \\ 
  \hline
1969 & 1.0530 & 1.0529 & 1.0529 & 1.0529 & 1.0529 & 1.0529 & 1.0529 & 1.0526 & 1.0529 \\ 
  1970 & 1.0547 & 1.0546 & 1.0546 & 1.0546 & 1.0546 & 1.0546 & 1.0546 & 1.0543 & 1.0546 \\ 
  1971 & 1.0518 & 1.0517 & 1.0517 & 1.0517 & 1.0517 & 1.0517 & 1.0517 & 1.0516 & 1.0517 \\ 
  1972 & 1.0512 & 1.0510 & 1.0511 & 1.0510 & 1.0510 & 1.0510 & 1.0510 & 1.0510 & 1.0510 \\ 
  1973 & 1.0521 & 1.0519 & 1.0520 & 1.0520 & 1.0519 & 1.0519 & 1.0519 & 1.0520 & 1.0520 \\ 
  1974 & 1.0548 & 1.0547 & 1.0548 & 1.0547 & 1.0547 & 1.0547 & 1.0547 & 1.0547 & 1.0547 \\ 
  1975 & 1.0537 & 1.0534 & 1.0535 & 1.0534 & 1.0534 & 1.0534 & 1.0534 & 1.0535 & 1.0534 \\ 
  1976 & 1.0525 & 1.0525 & 1.0525 & 1.0525 & 1.0525 & 1.0525 & 1.0525 & 1.0525 & 1.0525 \\ 
  1977 & 1.0526 & 1.0525 & 1.0525 & 1.0525 & 1.0525 & 1.0525 & 1.0525 & 1.0525 & 1.0525 \\ 
  1978 & 1.0527 & 1.0526 & 1.0527 & 1.0526 & 1.0526 & 1.0526 & 1.0526 & 1.0526 & 1.0526 \\ 
  1979 & 1.0517 & 1.0516 & 1.0517 & 1.0516 & 1.0516 & 1.0516 & 1.0516 & 1.0516 & 1.0516 \\ 
  1980 & 1.0528 & 1.0527 & 1.0528 & 1.0527 & 1.0527 & 1.0527 & 1.0527 & 1.0527 & 1.0528 \\ 
  1981 & 1.0516 & 1.0515 & 1.0515 & 1.0515 & 1.0515 & 1.0515 & 1.0515 & 1.0514 & 1.0515 \\ 
  1982 & 1.0506 & 1.0506 & 1.0506 & 1.0506 & 1.0506 & 1.0506 & 1.0506 & 1.0506 & 1.0506 \\ 
  1983 & 1.0519 & 1.0519 & 1.0519 & 1.0519 & 1.0519 & 1.0519 & 1.0519 & 1.0519 & 1.0519 \\ 
  1984 & 1.0502 & 1.0502 & 1.0502 & 1.0502 & 1.0502 & 1.0502 & 1.0502 & 1.0502 & 1.0502 \\ 
  1985 & 1.0521 & 1.0520 & 1.0520 & 1.0520 & 1.0520 & 1.0520 & 1.0520 & 1.0520 & 1.0520 \\ 
  1986 & 1.0509 & 1.0508 & 1.0508 & 1.0508 & 1.0508 & 1.0508 & 1.0508 & 1.0508 & 1.0508 \\ 
  1987 & 1.0500 & 1.0499 & 1.0500 & 1.0499 & 1.0500 & 1.0499 & 1.0499 & 1.0499 & 1.0500 \\ 
  1988 & 1.0500 & 1.0499 & 1.0500 & 1.0500 & 1.0500 & 1.0500 & 1.0500 & 1.0500 & 1.0500 \\ 
  1989 & 1.0498 & 1.0498 & 1.0498 & 1.0498 & 1.0498 & 1.0498 & 1.0498 & 1.0497 & 1.0498 \\ 
  1990 & 1.0497 & 1.0497 & 1.0497 & 1.0497 & 1.0497 & 1.0497 & 1.0497 & 1.0497 & 1.0497 \\ 
  1991 & 1.0458 & 1.0458 & 1.0458 & 1.0458 & 1.0458 & 1.0458 & 1.0458 & 1.0458 & 1.0458 \\ 
  1992 & 1.0500 & 1.0500 & 1.0500 & 1.0500 & 1.0500 & 1.0500 & 1.0500 & 1.0500 & 1.0500 \\ 
  1993 & 1.0500 & 1.0499 & 1.0500 & 1.0499 & 1.0499 & 1.0499 & 1.0499 & 1.0499 & 1.0499 \\ 
  1994 & 1.0478 & 1.0478 & 1.0478 & 1.0478 & 1.0478 & 1.0478 & 1.0478 & 1.0478 & 1.0478 \\ 
  1995 & 1.0490 & 1.0490 & 1.0489 & 1.0490 & 1.0490 & 1.0490 & 1.0490 & 1.0489 & 1.0490 \\ 
  1996 & 1.0471 & 1.0470 & 1.0471 & 1.0471 & 1.0471 & 1.0471 & 1.0471 & 1.0470 & 1.0471 \\ 
  1997 & 1.0477 & 1.0476 & 1.0477 & 1.0476 & 1.0476 & 1.0476 & 1.0476 & 1.0476 & 1.0476 \\ 
  1998 & 1.0472 & 1.0471 & 1.0472 & 1.0472 & 1.0472 & 1.0472 & 1.0472 & 1.0472 & 1.0472 \\ 
  1999 & 1.0488 & 1.0488 & 1.0488 & 1.0488 & 1.0488 & 1.0488 & 1.0488 & 1.0488 & 1.0488 \\ 
  2000 & 1.0480 & 1.0480 & 1.0480 & 1.0480 & 1.0480 & 1.0480 & 1.0480 & 1.0480 & 1.0480 \\ 
  2001 & 1.0457 & 1.0456 & 1.0457 & 1.0456 & 1.0456 & 1.0456 & 1.0456 & 1.0456 & 1.0456 \\ 
  2002 & 1.0480 & 1.0479 & 1.0480 & 1.0479 & 1.0479 & 1.0480 & 1.0480 & 1.0480 & 1.0479 \\ 
  2003 & 1.0487 & 1.0487 & 1.0487 & 1.0487 & 1.0487 & 1.0487 & 1.0487 & 1.0487 & 1.0487 \\ 
  2004 & 1.0485 & 1.0485 & 1.0485 & 1.0485 & 1.0485 & 1.0485 & 1.0485 & 1.0485 & 1.0485 \\ 
  2005 & 1.0494 & 1.0493 & 1.0493 & 1.0493 & 1.0493 & 1.0493 & 1.0493 & 1.0493 & 1.0493 \\ 
  2006 & 1.0496 & 1.0495 & 1.0496 & 1.0495 & 1.0495 & 1.0495 & 1.0495 & 1.0495 & 1.0495 \\ 
  2007 & 1.0475 & 1.0475 & 1.0475 & 1.0475 & 1.0475 & 1.0475 & 1.0475 & 1.0475 & 1.0475 \\ 
  2008 & 1.0478 & 1.0478 & 1.0478 & 1.0478 & 1.0478 & 1.0478 & 1.0478 & 1.0478 & 1.0478 \\ 
  2009 & 1.0482 & 1.0481 & 1.0482 & 1.0481 & 1.0481 & 1.0481 & 1.0481 & 1.0481 & 1.0481 \\ 
   \hline
\end{tabular}
}
  \end{adjustwidth}
\end{table}

\begin{table}
  \begin{adjustwidth}{-1in}{-.5in}
  \centering
    \caption{Intrinsic growth rates, $r$, from remaining-years renewal models.
    Spain, 1975-2009.}
    \label{tab:ESexALL}
        \small{% latex table generated in R 2.15.3 by xtable 1.7-1 package
% Thu May  9 15:38:00 2013
\begin{tabular}{ccccccccccc}
  \hline
 & $r^m$ & $r^f$ & $r^{(\sigma = 1)}$ & $r^{(\sigma = 0)}$ & $r^{(\sigma = 0.5)}$ & $r^{HM}$ & $r^{GM}$ & $r^{LM}$ & $r^{RAdj-HM}$ & $r^{IPF-HM}$ \\ 
  \hline
1975 & 0.0095 & 0.0078 & 0.0095 & 0.0078 & 0.0087 & 0.0087 & 0.0087 & 0.0087 & 0.0063 & 0.0087 \\ 
  1976 & 0.0095 & 0.0081 & 0.0095 & 0.0081 & 0.0089 & 0.0088 & 0.0088 & 0.0088 & 0.0063 & 0.0088 \\ 
  1977 & 0.0083 & 0.0067 & 0.0083 & 0.0067 & 0.0076 & 0.0076 & 0.0075 & 0.0075 & 0.0053 & 0.0075 \\ 
  1978 & 0.0071 & 0.0053 & 0.0071 & 0.0053 & 0.0063 & 0.0063 & 0.0062 & 0.0062 & 0.0044 & 0.0063 \\ 
  1979 & 0.0051 & 0.0033 & 0.0051 & 0.0033 & 0.0042 & 0.0042 & 0.0042 & 0.0042 & 0.0029 & 0.0042 \\ 
  1980 & 0.0034 & 0.0012 & 0.0034 & 0.0012 & 0.0024 & 0.0023 & 0.0023 & 0.0023 & 0.0016 & 0.0024 \\ 
  1981 & 0.0013 & -0.0015 & 0.0013 & -0.0015 & 0.0000 & -0.0001 & -0.0001 & -0.0001 & 0.0000 & -0.0001 \\ 
  1982 & 0.0002 & -0.0026 & 0.0002 & -0.0026 & -0.0011 & -0.0013 & -0.0013 & -0.0013 & -0.0008 & -0.0012 \\ 
  1983 & -0.0020 & -0.0046 & -0.0020 & -0.0046 & -0.0032 & -0.0034 & -0.0034 & -0.0034 & -0.0022 & -0.0033 \\ 
  1984 & -0.0028 & -0.0056 & -0.0028 & -0.0056 & -0.0041 & -0.0044 & -0.0043 & -0.0043 & -0.0028 & -0.0041 \\ 
  1985 & -0.0041 & -0.0068 & -0.0041 & -0.0068 & -0.0053 & -0.0057 & -0.0056 & -0.0056 & -0.0037 & -0.0054 \\ 
  1986 & -0.0053 & -0.0081 & -0.0053 & -0.0081 & -0.0066 & -0.0070 & -0.0069 & -0.0069 & -0.0046 & -0.0067 \\ 
  1987 & -0.0062 & -0.0090 & -0.0062 & -0.0090 & -0.0075 & -0.0080 & -0.0079 & -0.0078 & -0.0052 & -0.0076 \\ 
  1988 & -0.0069 & -0.0097 & -0.0069 & -0.0097 & -0.0082 & -0.0087 & -0.0085 & -0.0085 & -0.0057 & -0.0082 \\ 
  1989 & -0.0077 & -0.0105 & -0.0077 & -0.0105 & -0.0090 & -0.0095 & -0.0094 & -0.0093 & -0.0063 & -0.0090 \\ 
  1990 & -0.0082 & -0.0110 & -0.0082 & -0.0110 & -0.0095 & -0.0101 & -0.0099 & -0.0099 & -0.0067 & -0.0096 \\ 
  1991 & -0.0086 & -0.0115 & -0.0086 & -0.0115 & -0.0100 & -0.0106 & -0.0104 & -0.0103 & -0.0070 & -0.0100 \\ 
  1992 & -0.0086 & -0.0114 & -0.0086 & -0.0114 & -0.0099 & -0.0105 & -0.0103 & -0.0102 & -0.0070 & -0.0100 \\ 
  1993 & -0.0094 & -0.0123 & -0.0094 & -0.0123 & -0.0107 & -0.0114 & -0.0112 & -0.0111 & -0.0077 & -0.0108 \\ 
  1994 & -0.0105 & -0.0133 & -0.0105 & -0.0133 & -0.0118 & -0.0125 & -0.0123 & -0.0122 & -0.0085 & -0.0119 \\ 
  1995 & -0.0110 & -0.0138 & -0.0110 & -0.0138 & -0.0123 & -0.0130 & -0.0128 & -0.0127 & -0.0090 & -0.0124 \\ 
  1996 & -0.0111 & -0.0139 & -0.0111 & -0.0139 & -0.0124 & -0.0131 & -0.0129 & -0.0128 & -0.0091 & -0.0125 \\ 
  1997 & -0.0107 & -0.0134 & -0.0107 & -0.0134 & -0.0119 & -0.0126 & -0.0124 & -0.0123 & -0.0087 & -0.0120 \\ 
  1998 & -0.0110 & -0.0139 & -0.0110 & -0.0139 & -0.0123 & -0.0130 & -0.0128 & -0.0127 & -0.0090 & -0.0124 \\ 
  1999 & -0.0102 & -0.0128 & -0.0102 & -0.0128 & -0.0114 & -0.0119 & -0.0117 & -0.0117 & -0.0083 & -0.0114 \\ 
  2000 & -0.0090 & -0.0118 & -0.0090 & -0.0118 & -0.0103 & -0.0108 & -0.0106 & -0.0106 & -0.0075 & -0.0104 \\ 
  2001 & -0.0089 & -0.0112 & -0.0089 & -0.0112 & -0.0099 & -0.0104 & -0.0102 & -0.0102 & -0.0072 & -0.0100 \\ 
  2002 & -0.0084 & -0.0108 & -0.0084 & -0.0108 & -0.0095 & -0.0099 & -0.0098 & -0.0098 & -0.0069 & -0.0096 \\ 
  2003 & -0.0075 & -0.0097 & -0.0075 & -0.0097 & -0.0086 & -0.0089 & -0.0088 & -0.0088 & -0.0061 & -0.0086 \\ 
  2004 & -0.0070 & -0.0093 & -0.0070 & -0.0093 & -0.0081 & -0.0084 & -0.0083 & -0.0083 & -0.0058 & -0.0081 \\ 
  2005 & -0.0069 & -0.0088 & -0.0069 & -0.0088 & -0.0078 & -0.0081 & -0.0081 & -0.0080 & -0.0056 & -0.0079 \\ 
  2006 & -0.0063 & -0.0081 & -0.0063 & -0.0081 & -0.0071 & -0.0073 & -0.0073 & -0.0073 & -0.0050 & -0.0071 \\ 
  2007 & -0.0062 & -0.0079 & -0.0062 & -0.0079 & -0.0070 & -0.0072 & -0.0072 & -0.0072 & -0.0049 & -0.0070 \\ 
  2008 & -0.0050 & -0.0066 & -0.0050 & -0.0066 & -0.0058 & -0.0059 & -0.0060 & -0.0060 & -0.0040 & -0.0058 \\ 
  2009 & -0.0063 & -0.0081 & -0.0063 & -0.0081 & -0.0071 & -0.0073 & -0.0073 & -0.0073 & -0.0050 & -0.0072 \\ 
   \hline
\end{tabular}
}
  \end{adjustwidth}
\end{table}

\begin{table}
  \begin{adjustwidth}{-1in}{-.5in}
  \centering
    \caption{Stable sex ratio at birth, $S$, from remaining-years renewal
    models. Spain, 1975-2009.}
    \label{tab:ESexSRBALL}
        \small{% latex table generated in R 2.15.3 by xtable 1.7-1 package
% Thu May  9 15:46:24 2013
\begin{tabular}{cccccccccc}
  \hline
 & $S(t)$ & $S^{(\sigma = 1)}$ & $S^{(\sigma = 0)}$ & $S^{(\sigma = 0.5)}$ & $S^{HM}$ & $S^{GM}$ & $S^{LM}$ & $S^{RAdj-HM}$ & $S^{IPF-HM}$ \\ 
  \hline
1975 & 1.0724 & 1.0724 & 1.0724 & 1.0724 & 1.0724 & 1.0724 & 1.0724 & 1.0717 & 1.0724 \\ 
  1976 & 1.0640 & 1.0640 & 1.0640 & 1.0640 & 1.0640 & 1.0640 & 1.0640 & 1.0635 & 1.0640 \\ 
  1977 & 1.0689 & 1.0688 & 1.0688 & 1.0688 & 1.0688 & 1.0688 & 1.0688 & 1.0684 & 1.0688 \\ 
  1978 & 1.0738 & 1.0737 & 1.0737 & 1.0737 & 1.0737 & 1.0737 & 1.0737 & 1.0733 & 1.0737 \\ 
  1979 & 1.0681 & 1.0679 & 1.0679 & 1.0679 & 1.0679 & 1.0679 & 1.0679 & 1.0677 & 1.0679 \\ 
  1980 & 1.0780 & 1.0779 & 1.0778 & 1.0779 & 1.0779 & 1.0779 & 1.0779 & 1.0778 & 1.0779 \\ 
  1981 & 1.0916 & 1.0915 & 1.0917 & 1.0916 & 1.0916 & 1.0916 & 1.0916 & 1.0916 & 1.0916 \\ 
  1982 & 1.0873 & 1.0871 & 1.0871 & 1.0871 & 1.0871 & 1.0871 & 1.0871 & 1.0871 & 1.0871 \\ 
  1983 & 1.0762 & 1.0762 & 1.0760 & 1.0761 & 1.0761 & 1.0761 & 1.0761 & 1.0762 & 1.0761 \\ 
  1984 & 1.0828 & 1.0830 & 1.0829 & 1.0830 & 1.0830 & 1.0830 & 1.0830 & 1.0829 & 1.0830 \\ 
  1985 & 1.0734 & 1.0733 & 1.0732 & 1.0732 & 1.0732 & 1.0732 & 1.0732 & 1.0734 & 1.0732 \\ 
  1986 & 1.0737 & 1.0735 & 1.0733 & 1.0734 & 1.0734 & 1.0734 & 1.0734 & 1.0736 & 1.0734 \\ 
  1987 & 1.0769 & 1.0770 & 1.0770 & 1.0770 & 1.0770 & 1.0770 & 1.0770 & 1.0770 & 1.0770 \\ 
  1988 & 1.0717 & 1.0720 & 1.0718 & 1.0719 & 1.0719 & 1.0719 & 1.0719 & 1.0717 & 1.0719 \\ 
  1989 & 1.0708 & 1.0705 & 1.0706 & 1.0706 & 1.0706 & 1.0706 & 1.0706 & 1.0708 & 1.0706 \\ 
  1990 & 1.0699 & 1.0695 & 1.0697 & 1.0696 & 1.0696 & 1.0696 & 1.0696 & 1.0700 & 1.0696 \\ 
  1991 & 1.0720 & 1.0721 & 1.0722 & 1.0722 & 1.0722 & 1.0722 & 1.0722 & 1.0720 & 1.0722 \\ 
  1992 & 1.0662 & 1.0661 & 1.0663 & 1.0662 & 1.0662 & 1.0662 & 1.0662 & 1.0662 & 1.0662 \\ 
  1993 & 1.0699 & 1.0701 & 1.0698 & 1.0699 & 1.0699 & 1.0699 & 1.0699 & 1.0698 & 1.0699 \\ 
  1994 & 1.0668 & 1.0666 & 1.0665 & 1.0666 & 1.0666 & 1.0666 & 1.0666 & 1.0667 & 1.0666 \\ 
  1995 & 1.0643 & 1.0641 & 1.0645 & 1.0642 & 1.0642 & 1.0642 & 1.0642 & 1.0643 & 1.0643 \\ 
  1996 & 1.0612 & 1.0608 & 1.0610 & 1.0609 & 1.0609 & 1.0609 & 1.0609 & 1.0613 & 1.0609 \\ 
  1997 & 1.0625 & 1.0626 & 1.0627 & 1.0627 & 1.0627 & 1.0627 & 1.0627 & 1.0625 & 1.0627 \\ 
  1998 & 1.0727 & 1.0720 & 1.0722 & 1.0721 & 1.0721 & 1.0721 & 1.0721 & 1.0729 & 1.0721 \\ 
  1999 & 1.0616 & 1.0612 & 1.0614 & 1.0613 & 1.0613 & 1.0613 & 1.0613 & 1.0618 & 1.0613 \\ 
  2000 & 1.0706 & 1.0704 & 1.0704 & 1.0704 & 1.0705 & 1.0704 & 1.0704 & 1.0707 & 1.0704 \\ 
  2001 & 1.0567 & 1.0564 & 1.0565 & 1.0564 & 1.0564 & 1.0564 & 1.0564 & 1.0568 & 1.0564 \\ 
  2002 & 1.0648 & 1.0646 & 1.0648 & 1.0647 & 1.0647 & 1.0647 & 1.0647 & 1.0650 & 1.0647 \\ 
  2003 & 1.0620 & 1.0617 & 1.0619 & 1.0618 & 1.0618 & 1.0618 & 1.0618 & 1.0622 & 1.0618 \\ 
  2004 & 1.0690 & 1.0688 & 1.0690 & 1.0689 & 1.0689 & 1.0689 & 1.0689 & 1.0692 & 1.0689 \\ 
  2005 & 1.0620 & 1.0620 & 1.0620 & 1.0620 & 1.0620 & 1.0620 & 1.0620 & 1.0623 & 1.0620 \\ 
  2006 & 1.0659 & 1.0655 & 1.0658 & 1.0656 & 1.0657 & 1.0657 & 1.0657 & 1.0659 & 1.0657 \\ 
  2007 & 1.0640 & 1.0638 & 1.0639 & 1.0639 & 1.0639 & 1.0639 & 1.0639 & 1.0639 & 1.0639 \\ 
  2008 & 1.0675 & 1.0674 & 1.0676 & 1.0675 & 1.0675 & 1.0675 & 1.0675 & 1.0676 & 1.0675 \\ 
  2009 & 1.0707 & 1.0706 & 1.0706 & 1.0706 & 1.0706 & 1.0706 & 1.0706 & 1.0708 & 1.0706 \\ 
   \hline
\end{tabular}
}
  \end{adjustwidth}
\end{table}












