\documentclass[reqno,12pt,twoside,a4paper]{report} % right-side equation
%numbering, 12 point font, print one-sided
%\documentclass[reqno,12pt,twoside,openright]{report} % right-side equation numbering, 12 point font, print two-sided, Chapters start on odd pages. Rackham only accepts one-sided, so this is for personal printings.

\usepackage{uab}         % Use UAB thesis style file, in folder with main file
\usepackage{aas_macros}  % To allow the reading of ADS journal references in the bibliography
\usepackage[intlimits]{amsmath} % Puts the limits of integrals on top and bottom
\usepackage{amsxtra}     % Use various AMS packages
\usepackage{amsthm}
\usepackage{amssymb}
\usepackage{mathtools}
\usepackage{amsfonts}
\usepackage{graphicx}    % Add some packages for figures. Read epslatex.pdf on ctan.tug.org
\usepackage{rotating}
\usepackage{color}
\usepackage{epsfig}
\usepackage{subfigure}   % To make subfigures. Read subfigure.pdf on ctan.tug.org
\usepackage{verbatim}
\usepackage{natbib}      % for bibtex
\usepackage{caption}     % to suppress caption numbering at times  
\usepackage{lscape}      % for landscape diagram
\usepackage[super]{nth}
\usepackage{textcomp}    % in text pretty 1/2
\usepackage{chngpage}
\usepackage{trivfloat}
\trivfloat{matrix}
\floatstyle{plaintop}
    \restylefloat{matrix}
\AtBeginDocument{\numberwithin{matrix}{section}}
\usepackage{placeins}
\usepackage{bookmark} % for dividing chapters into parts
% ---------------------------------------------------------------------
\usepackage[draft]{pdfpages}    % allows inclusion of multipage pdfs:
% change to [final] when ready to compile complete
% ---------------------------------------------------------------------  
% some custom citation aliases
\defcitealias{rackham1947trans}{Aristotle, \textit{Nichomachean
Ethics}, Book II, Chapter 6, Sections 4-5.}
\defcitealias{NCHS19692009}{\textit{NCHS} (1969-2009)}
%\defcitealias{bas09}{BCBS, 2009}i

\usepackage[printonlyused]{acronym} % For the List of Abbreviations. Read acronym.pdf on ctan.tug.org
\usepackage{setspace}    % Allows you to specify the line spacing
\doublespacing           % \onehalfspacing for 1.5 spacing, \doublespacing for 2.0 spacing.
\newcommand{\sun}{\ensuremath{\odot}} % sun symbol is \sun
%\usepackage{leftidx}     % for better left subscripts
\usepackage{fixltx2e}     % in-text subscripting, no math
%\usepackage[OT1]{fontenc}
\usepackage{bm}
%%%%%%%%%%%%%%%%%%%%%%%%%%%%%%%%%%%%%%%%%%%%%%%%%%%%%%%%%%%%%%%%%%%%%%%%%%%%%%%

% Various theorem environments. All of the following have the same numbering
% system as theorem.

\theoremstyle{plain}
\newtheorem{theorem}{Theorem}
\newtheorem{prop}[theorem]{Proposition}
\newtheorem{corollary}[theorem]{Corollary}
\newtheorem{lemma}[theorem]{Lemma}
\newtheorem{question}[theorem]{Question}
\newtheorem{conjecture}[theorem]{Conjecture}
\newtheorem{assumption}[theorem]{Assumption}

\theoremstyle{definition}
\newtheorem{definition}[theorem]{Definition}
\newtheorem{notation}[theorem]{Notation}
\newtheorem{condition}[theorem]{Condition}
\newtheorem{example}[theorem]{Example}
\newtheorem{introduction}[theorem]{Introduction}

\theoremstyle{remark}
\newtheorem{remark}[theorem]{Remark}
%%%%%%%%%%%%%%%%%%%%%%%%%%%%%%%%%%%%%%%%%%%%%%%%%%%%%%%%%%%%%%%%%%%%%%%%%%%%%%%

\numberwithin{theorem}{chapter}     % Numbers theorems "x.y" where x
                                    % is the section number, y is the
                                    % theorem number

%\renewcommand{\thetheorem}{\arabic{chapter}.\arabic{theorem}}

%\makeatletter                      % This sequence of commands will
%\let\c@equation\c@theorem          % incorporate equation numbering
%\makeatother                       % into the theorem numbering scheme

%\renewcommand{\theenumi}{(\roman{enumi})}

%%%%%%%%%%%%%%%%%%%%%%%%%%%%%%%%%%%%%%%%%%%%%%%%%%%%%%%%%%%%%%%%%%%%%%%%%%%%%%
% cache appears to be the problem with displaying results
%\usepackage{Sweave}
%\SweaveOpts{cache=FALSE,tidy=TRUE}
%\usepackage{tikz}

% makes code more compact, changes indentation of code chunks
%\DefineVerbatimEnvironment{Sinput}{Verbatim} {xleftmargin=2em}
%\DefineVerbatimEnvironment{Soutput}{Verbatim}{xleftmargin=2em}
%\DefineVerbatimEnvironment{Scode}{Verbatim}{xleftmargin=2em}
%\fvset{listparameters={\setlength{\topsep}{0pt}}}
%\renewenvironment{Schunk}{\vspace{\topsep}}{\vspace{\topsep}}

%%%%%%%%%%%%%%%%%%%%%%%%%%%%%%%%%%%%%%%%%%%%%%%%%%%%%%%%%%%%%%%%%%%%%%%%%%%%%%%
% If printing two-sided, this makes sure that any blank page at the 
% end of a chapter will not have a page number. 
\makeatletter
\def\cleardoublepage{\clearpage\if@twoside \ifodd\c@page\else
\hbox{}
\thispagestyle{empty}
\newpage
\if@twocolumn\hbox{}\newpage\fi\fi\fi}
\makeatother 

%%%%%%%%%%%%%%%%%%%%%%%%%%%%%%%%%%%%%%%%%%%%%%%%%%%%%%%%%%%%%%%%%%%%%%%%%%%%%%

%This command creates a box marked ``To Do'' around text.
%To use type \todo{  insert text here  }.

\newcommand{\todo}[1]{\vspace{5 mm}\par \noindent
\marginpar{\textsc{To Do}}
\framebox{\begin{minipage}[c]{0.95 \textwidth}
\tt\begin{center} #1 \end{center}\end{minipage}}\vspace{5 mm}\par}

%%%%%%%%%%%%%%%%%%%%%%%%%%%%%%%%%%%%%%%%%%%%%%%%%%%%%%%%%%%%%%%%%%%%%%%%%%%%%%%
% for the d in integrals
\newcommand{\dd}{\; \mathrm{d}}

% footnote numbering is continuous throughout.
% found tip here:
% http://tex.stackexchange.com/questions/10448/continuous-footnote-numbering
\usepackage{chngcntr}
\counterwithout{footnote}{chapter}