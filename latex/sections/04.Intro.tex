\todo{Intro in development!}

Before using the term at length, we offer a quick definition of the two-sex
problem:

\begin{singlespace}
\begin{quote}
The \textbf{two-sex problem}: Separate male and female predictions of events
will differ when 1) the events depend on both sexes, 2) the same events 
form the numerators of the rates of each sex that are used in the prediction and
3) the exposures used to calculate these same rates change from year to year
(time to time). We would like a single prediction based on information from the
demographic rates of each sex.
\end{quote}
\end{singlespace}

 Such events include heterosexual marriages and the vast majority of births.
 A manifestation of this problem in formal demography is that the male
 and female single-sex renewal models calculated from the data of any observed 
 population will diverge from one another, and therefore neither can represent
 the population as a whole. Models that account for the rates of both sexes
 so as to produce a single and consistent result or prediction are
 variously referred to throughout this dissertation as two-sex \textit{methods},
 two-sex \textit{solutions}, two-sex \textit{rate balancing}, two-sex 
 \textit{adjustments}, and so on with no strict distinction between terms.
 
 The two-sex problem in human demography has until this work either been defined 
 and studied for populations that are structured by sex only, for populations
structured by both sex and age, or for populations structured by age, sex and
marital status. The problem enjoys a long history in formal demography, and most
contemporary applied demographers likely have a rough idea of what the problem
is, but it is not at the forefront of the consciousness of most contemporary
practicing demographers. For this reason we dedicate a sizeable portion of 
this dissertation, Chapter~\ref{ch:Measuring}, to defining, describing and
decomposing the two-sex problem for populations structured by age and sex. The
primary (and likely original) contribution of this chapter is the decomposition
of the sex-gap in intrinsic growth rates into various vital rate components of 
the classic renewal model. This effort will convince the reader that the two-sex
problem is worth thinking about and accounting for, and it justifies a large
portion of this dissertation.

These activities are carried out formally in the context of the fundamental
Lotka equations for single-sex population renewal and empirically on the basis 
US and Spanish data from the past four decades. The choice to work with the two-sex problem in
renewal models is not an idle endeavor, as the practice of projecting
population-- the bread and butter of applied demographers-- is grounded in
models of population renewal. That is to say, in implementing a population 
projection, one always has a renewal model in mind, and all indices of
population reproductivity refer to a notion of renewal. The methods
described herein are applicable as-is to implementation in population
projections.

We suppose that part of the reason why practicing
demographers often do not account for the two-sex problem in projections (and 
elsewhere) is that formal demographers have been unable to supply a
consensus solution on how to deal with it. Of course, the lack of consensus on
demographic methods is no obstacle for demographers in other tasks-- there are 
different ways to calculate lifetables, different ways to estimate exposures,
different ways to smooth or graduate demographic schedules, and so forth, but
this does not stop demographers from. Unfortunately there is no
best and true method to account for the two-sex problem in renewal models or 
projections; there are only approximations. For this reason in
Chapter~\ref{sec:modellingapproaches} we present a suite of 
approaches applicable to the US and Spanish example data 
used thoughout this work. These methods are evaluated in situ. The selection of
two-sex methods examined is not exhaustive, but has been selected on
an ad hoc basis of convenience



 broad
comparison is done in the context of stable population renewal. Some of these methods will be modified from their original context.

 The two-sex problem has never been
explored for the case of populations structured by remaining years of life (thanatological age). This is necessarily so because Lotka's renewal model has never been translated to the case of populations structured by thanatological age, and this latter endeavor banks on the ability to structure a population by thanatological age in the first place.


\section{Data}
All calculations in this dissertation, unless otherwise cited, are original 
and based on a small number of publicly available datasets that have
been modified and standardized according to a strict and simple protocol, as
described in following. Since the same small number of datasets
is used throughout this document, sources are not cited in situ, but rather
always refer to the same sources, as described here. Only two populations are
treated, Spain (ES) and the United States (US). Similar data for France was also
located, but was not included as it covered a shorted range of years. 
Since the data used in this work are so simple, calculations presented are
expected to be replicable for a variety of other populations, though not for
populations where births by age of father are not available.

\subsection{Birth counts}
Birth counts for Spain and the US were not available in tables of the format
required for this dissertation. For this reason, birth counts were tabulated from birth
register microdata publicly available as fixed-width text files from the 
Instituto Nacional de Estadistica (INE)\citep{MNPnacimientos} for Spain and the
National Center for Health Statistics (NCHS) (INE)\citep{NCHS19692009} for the US. For Spain, 
the years 1975-2009 are used and for the US, the years 1969-2009 are used
consistently throughout this dissertation. At the time of this writing, further
years are available, but not included. Earlier years for the US are also
available in earlier official publications, but these have not been digitized
for inclusion in this dissertation. Cross-tabulations for each year included age
of mother, age of father and sex of birth. Resident status was not used as a
selection criterion for births in either country.

In all cases for both countries, age of mother was
stated, but in some cases age of father was missing. Births with missing age of
father were redistributed proportionately over births to fathers of known age 
separately for each age of mother. In Spain births with unrecorded age of father 
tended to comprise less than 2\% of
all cases, and so we do not expect this procedure to affect
results, and no further sensitivity tests were performed. For the US, the
percent of all births where age of father was not recorded ranged between 7\%
and 18\%, as seen in Figure~\ref{fig:USmissingAge}. 

\begin{figure}[ht!]
        \centering  
          \caption{Proportion of births with age of father not recorded, US,
          1969-2009}
           % figure produced in /R/DataDiagnostics.R
           \includegraphics{Figures/USmissingAge}
          \label{fig:USmissingAge}
\end{figure}

For the US, the degree of missingness of fathers' age varies by age of
mother (not shown). For ages greater than 25, we do not expect this to affect
results in an important way. Averaged over all years, ages $<= 20$ all had missingness of more than
20\%; ages $<= 16$ had missingness of more than 40\%, and ages $<= 14$ had
missingness of more than 60\%. This may affect results if the age-pattern of
males of unrecorded age differs from that of males of recorded age in a
non-trivial way. This uncertainty enters into the male
age-pattern of fertility, as well as the bivariate
age distribution of births (age of mother by age of father) may affect results
for the US where these age-specific data are used.

For both countries, cross-tabulated
sex-specific birth counts were entered into matrices of standard 111$\times$111
dimensions, covering ages 0-110. Ages with no births simply contain zeros. Open
age groups from the original data were not redistributed over ages beyond the
bounds of the original microdata. Especially for young ages of fathers and the
upper ages of mothers, this will be visible in respective age
patterns, but the effect on overall results (TFR, growth rates) will be
trivial.

Where birth counts are not required differentiated by sex of birth, for
instance, we sum over sex. Birth counts by age of mother are always taken from
the column margin of the birth matrix, while age of father is the row margin.
This practice helps to minimize the number of data objects used.

\subsection{Exposures and mortality data}

All other data for the US and Spain were downloaded from the Human Mortality
Database (HMD)\citep{wilmoth2007methods}. These data include, most importantly,
population exposures\footnote{At the time of this writing, exposures from the
Human Fertility Database (\url{www.humanfertility.org}) may have been more
appropriate for certain age groups, but since we prefer to use all ages
$0-110+$, HMD exposures were utilized instead.} and population counts by age,
sex and year and the deaths distribution, $d_x$, from the sex-specific lifetables. $d_x$ informaion was 
always rescaled to sum to 1, which minimized rounding errors and simplified
programming. Other items drawn from the HMD but used less consistently
included, mortality hazards, $\mu_x$, survival curves, $l_x$ (also rescaled so that $l_0 = 1$), 
lifetable exposures, $L_x$,
life expectancies, $e_x$, and death counts by Lexis triangles. Each of these
items is used in single-age format, with ages $0-110+$. The open age group,
$110+$ is used as age 110 and is given no further treatment. The
respective uses of each of these items should be obvious from the context 
of the formulas being applied, and are stated explicitly in the text or in 
footnotes where the use may not be obvious.

