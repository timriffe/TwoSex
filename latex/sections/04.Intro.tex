Before using the term at length, we offer a quick definition of the two-sex
problem:

\begin{singlespace}
\begin{quote}
The \textbf{two-sex problem}: Separate male and female predictions of events
will differ when 1) the events depend on both sexes, 2) the same events 
form the numerators of the rates of each sex that are used in the prediction,
and 3) the exposures used to calculate these same rates change from year to year
(time to time). We would like a single prediction based on information from the
demographic rates of each sex.
\end{quote}
\end{singlespace}

 Such events include births and heterosexual marriages.
 A manifestation of this problem in formal demography is that the male
 and female single-sex renewal models calculated from the data of any observed 
 population will diverge from one another, and therefore neither represents
 the population as a whole. Models that account for the rates of both sexes
 so as to produce a single and consistent result or prediction are
 variously referred to throughout this dissertation as two-sex \textit{methods},
 two-sex \textit{solutions}, two-sex \textit{rate balancing}, two-sex 
 \textit{adjustments}, and so on with no strict distinction between terms.
 
 The two-sex problem in human demography has until this work been defined 
 and studied either for populations that are structured by sex only, for
 populations structured by both sex and age, or for populations structured by
 sex, age, and marital status. The problem enjoys a long history in formal demography, and most
contemporary applied demographers likely have a rough idea of what the problem
is, but it is not at the forefront of the consciousness of most contemporary
practicing demographers. For this reason we dedicate a sizable portion of 
this dissertation, Part 1, to defining, describing,
and decomposing (Chapter~\ref{ch:Measuring}) the two-sex problem for populations
structured by age and sex, and to describing a set of methods developed to deal
with this problem (Chapter~\ref{sec:modelingapproaches}). The primary (and likely
original) contribution of Chapter~\ref{ch:Measuring} is the decomposition of the 
sex-gap in intrinsic growth rates into the various vital rate components of the 
classic renewal model. This effort will convince the reader that the two-sex
problem is worth thinking about and accounting for, and it justifies a large
portion of the remainder of this dissertation.

These activities are carried out formally in the context of the fundamental
Lotka equations for single-sex population renewal and empirically on the basis 
US and Spanish data from the past four decades. The choice to work with the two-sex problem in
renewal models is not an idle endeavor, as the practice of projecting
population -- the bread and butter of applied demographers -- is grounded in
models of population renewal. That is to say, in implementing a population 
projection, one always has a renewal model in mind, and all indices of
population reproductivity refer to a notion of renewal. The methods
described herein are applicable as-is to implementation in population
projections.

We suppose that part of the reason that practicing
demographers often do not account for the two-sex problem in projections (and 
elsewhere) is that formal demographers have been unable to supply a
consensus solution on how to deal with it. Of course, the lack of consensus on
demographic methods is no obstacle for demographers in other tasks -- there are 
different ways to calculate lifetables, estimate exposures, smooth or graduate
demographic schedules, and so forth, but this does not stop demographers from
doing these things. Unfortunately there is no
best and true method to account for the two-sex problem in renewal models or projections; 
there are only approximations, and this is likely to remain the case. For this
reason, in Chapter~\ref{sec:modelingapproaches} we present a suite of 
approaches applicable to the US and Spanish example data 
used throughout this work. These methods are evaluated in situ. The selection of
two-sex methods examined is not exhaustive, but has been selected on
an ad hoc basis of convenience to the author. It is hoped that the set of
two-sex model implementations provided here will be translatable and
transferible to population projection engines. In this way, the demographer may
thoughtfully select from among the two-sex methods conscientiously, just as
every maker of lifetables chooses a method to calculate $a_x$.

For each method separately, we begin by explaining the model of renewal. This
is followed by a step-by-step guide to estimating the intrinsic growth rate,
$r$, for the given model. All such estimation procedures have been developed by
modifying the fast-converging method of \citet{coale1957new} to the renewal
model at hand. At times other parameters of the stable population are also presented
on the basis of $r$. In all cases, some empirical results are derived for the
method in question and some model properties are discussed. Some of these
methods will be modified from their original context to suit the needs of this
dissertation.

The first two methods presented, those of \citet{pollard1948measurement} in
Section~\ref{sec:pollardage} and
\citet{mitra1978derivation} in Section~\ref{sec:mitraage}, serve more for context and curiosity than for
practical use. These two models namely do not distribute the fertility of each
sex of birth between both sexes of parents, which either introduces instability
or an unrealistic notion of renewal. Section~\ref{sec:googmanage} presents one
way of using \textit{dominance weights} to decide how to divide the information source 
for fertility prediction between males and females. We find this method
convenient, and so it is translated to the case of populations structured by
thanatological age later on. 

In the following we explore three models that make use of the full joint
distribution of fertility by the ages of both parents. The model of
\citet{gupta1978alternative} is presented in Section~\ref{sec:dasgupta}. This is
followed in Section~\ref{sec:ageharmonic} by a method to balance fertility
predictions by taking a mean of male and female exposures in each age combination, 
which we largely demonstrate on the
basis of the harmonic mean, although the mean could be of any kind. This method
is attractive for different reasons, and so will also be translated to the case
of remaining-years structure. Finally, Section~\ref{sec:IPF} presents an
iterative method to balance the birth counts predicted by males and females in
different age combinations, and this method is also implemented for the case of
remaining-years structured populations. Section~\ref{sec:other} very briefly
discusses a large number of other modeling approaches that have also been
taken, or that could be taken, with respect to two-sex population modeling, but
that are beyond the scope of this dissertation.

The two-sex problem has never been explored for the case of populations
structured by remaining years of life (thanatological age). This is necessarily 
so because Lotka's single-sex renewal model \citep{sharpe1911problem} has never
been translated to the case of populations structured by thanatological age, 
and this latter endeavor banks on structuring a population by 
thanatological age in the first place, which is apparently itself novel. This
last item, structuring populations by remaining years of life, is done by means
of a slight modification to formulas that already exist, and so most of the
novelty therein owes to framing the pre-existent desire for the remaining-years perspective in
terms of population structure. These will be the first tasks of Part 2
of this dissertation.

Section~\ref{sec:thetransformation} provides and relates the formulas to
translate age-structured data into remaining-years structured data. The most
basic demographic data amenable to such restructuring are population counts.
This we present in Section~\ref{sec:expopstruct1} for our two populations.
Many of the results that fall out of this activity (or consequences of it) are
of potential immediate utility, and so we briefly discuss some aspects of
thanatologically structured populations, such as uncertainty
(Section~\ref{sec:structuncertainty}), the incorporation of mortality
improvements in the method (Section~\ref{sec:eximprov}), and a couple of
the more obvious and direct measures of population \textit{aging}
(Section~\ref{sec:exageing}).

A further conceptual leap is necessary before the business of population
modeling may be undertaken atop this new population structure -- namely, the
observation that this method of population restructuring works just the same for any age-classified data. Of
relevance for our population modeling objective is the restructuring of
age-classified birth counts and thencefrom the production of fertility rates by
remaining years of life. Section~\ref{sec:exrates} explores these fertility
rates for the US and Spanish populations. The practice of population modeling
on the basis of such rates will be palatable only if a high degree of empirical
regularity is present in the data. This we confirm, revealing for the first
time the characteristic shape of thanatologically-structured fertility rates--
the thanatological analogue to ASFR, which we call $e$SFR. Such rates may be
presented in bulk in the form of a standard ``\textit{remaining years} $\times$
year'' demographic surface, but we also summarize the intensity of fertility
for a given year in Section~\ref{sec:etfr} by summing rates over remaining years of life to produce
the thanatological analogue to TFR, $e$TFR, which enjoys the
same basic interpretation as TFR yet sums to a different value. In
Section~\ref{sec:exdivergence} we demonstrate that male and female predictions
of future births based on thanatological rates will tend to differ by less than 
is the case for predictions made on the basis of age-structured rates. Lastly,
also as a brief diversion, we relate $e$TFR to a remaining-years reformulation
of Fisher's reproductive value.

Having demonstrated sufficient empirical regularity in the remaining-years
pattern to the necessary vital rates, we reconceive of the classic notion of
age-structured population renewal in Section~\ref{sec:exrenewal}. That
description is intended to be intuitive, and we hash out a diagram
(Figure~\ref{fig:exrenewal}) that should serve as a visual mnemonic for the 
concept of renewal in populations structured by remaining years of life. In
Section~\ref{sec:ex2sexequation} we undertake a broad project of defining the
single-sex renewal equation(s) for this perspective of population, including the
provision of an iterative method to optimize $r$
(Section~\ref{sec:exrenewalit}), formulas for the derivation of stable
population structure (Section~\ref{sec:ex1sexother}), and induction of the
projection matrix\footnote{This product is the analogue to the well-known Leslie
matrix.} that corresponds to the remaining-years perspective 
(Section~\ref{sec:ex1sexleslie}). After displaying some trends in
remaining-years $r$ for the US and Spain (Section~\ref{sec:trendsinrex1sex}),
 we explore the speed of divergence between the single-sex models in comparison with divergence for the age-structured
single-sex models (Section~\ref{sex:doublingex}). Finally, in
Section~\ref{sec:exdecomposer} we quantify the vital rate components to the
sex-gap in the thanatological $r$ using a flexible method of demographic
decomposition. This closes our treatment of single-sex renewal for populations
structured by remaining years of life, leaving much terrain unexplored.

Having developed a working single-sex model for the remaining-years perspective,
Part 3 of this dissertation translates and explores a set of the two-sex
methods from Chapter~\ref{sec:modelingapproaches} to thanatologically
structured populations. This is carried out in much the same sequence as was
done for age-structured two-sex models, first describing the fertility
balancing method and deriving the renewal model, second explaining step-by-step
instructions for estimating $r$ (and SRB it turns out) from given data, third
working out a selection of some other stable parameters, and
finally a demonstration of (some aspect of) the method on the basis of the US
and Spanish data used throughout this dissertation.

We begin the two-sex methods translation with the
dominance-weighted method of \citet{goodman1967age} in
Chapter~\ref{sec:ex2sexdomweights}. As this is the first two-sex renewal model
explored for remaining-years populations, we take some extra care to compare
results both with the remaining-years single-sex models and with age-structured models. Notably, we
define the two-sex dominance-weighted remaining-years structured
projection matrix (Section~\ref{sec:ex2sxprojmat}), and use this to explore
several aspects of the stable population structure
(Section~\ref{sec:ex2sexdomweightsstabstruct}). 

Second, we translate the
method based on a generalized mean of male and female
remaining-years specific exposures, also for the most part on the basis of the
harmonic mean (Chapter~\ref{sec:ex2sexschoen}), although results do not vary
much if other common (and reasonable) means are chosen instead. The values of
the intrinsic growth rate, $r$, that one derives with this method are very close
to those given by the dominance-weighted method when equal weight is given to
the male and female rates. For this reason we focus the empirical
demonstration of the method on other stable consequences that do not apply in
the latter case, most notably on the stable versus the initial $e$SFR.

Third, in Section~\ref{sec:ipfex} we translate the iterative matrix method
originally presented in Section~\ref{sec:IPF} to the two-sex remaining-years
structured population. This method indeed yields estimates of $r$ that are distinct from
any of the previous methods (for at least one of our two populations). We then
examine the initial versus stable fertility rates, and compare results with
those obtained from the generalized mean method.

Finally, in Chapter~\ref{sec:CRchap} we sketch out a novel two-sex adjustment
method for this population structure based on marginal male and female fertility
rates and a fixed ratio of observed birth counts to association-free expected
birth counts in the joint male-female distribution. The model is seen to give
acceptable results, but we conclude that it displays no particular advantage
over the generalized mean or IPF methods, and so we discard it in order to move
on to the IPF method.

That the methods and observations presented here are so out of the ordinary
invites one to reflect rather than to conclude, although in
Chapter~\ref{sec:reflections} we attempt both. Namely, we consider whether
there is anything to be gained by conceiving of reproduction in terms of 
remaining years of life, why it is that the
remaining-years structured family of models yields a different estimate of
population growth than does the age-structured family of models, and how our
experience with the two-sex problem might yield insight on this lack of
congruence. Further, we summarize any empirical findings of note, though these
are not central to our objectives. Finally we summarize theoretical
contributions from this work and suggest a broad and ambitious research agenda
to be grounded in it.

\section{Data}
All calculations in this dissertation, unless otherwise cited, are original 
and based on a small number of publicly available datasets that have
been modified and standardized according to a strict and simple protocol, as
described in following. Since the same small number of datasets
is used throughout this document, sources are not cited in situ, but rather
always refer to the same sources, as described here. Only two populations are
treated, Spain (ES) and the United States (US). Similar data for France was also
located, but was not included as it covered a shorted range of years. 
Since the data used in this work are so simple, calculations presented are
expected to be replicable for a variety of other populations, though not for
populations where births by age of father are not available.

\subsection{Birth counts}
Birth counts for Spain and the US were not available in tables of the format
required for this dissertation. For this reason, birth counts were tabulated from birth
register microdata publicly available as fixed-width text files from the 
Instituto Nacional de Estadistica (INE)\citep{MNPnacimientos} for Spain and the
National Center for Health Statistics (NCHS) (INE)\citep{NCHS19692009} for the US. For Spain, 
the years 1975-2009 are used and for the US, the years 1969-2009 are used
consistently throughout this dissertation. At the time of this writing, further
years are available, but not included. Earlier years for the US are also
available in earlier official publications, but these have not been digitized
for inclusion in this dissertation. Cross-tabulations for each year included age
of mother, age of father and sex of birth. Resident status was not used as a
selection criterion for births in either country.

In all cases for both countries, age of mother was
stated, but in some cases age of father was missing. Births with missing age of
father were redistributed proportionately over births to fathers of known age 
separately for each age of mother. In Spain births with unrecorded age of father 
tended to comprise less than 2\% of
all cases, and so we do not expect this procedure to affect
results, and no further sensitivity tests were performed. For the US, the
percent of all births where age of father was not recorded ranged between 7\%
and 18\%, as seen in Figure~\ref{fig:USmissingAge}. 

\begin{figure}[ht!]
        \centering  
          \caption{Proportion of births with age of father not recorded, US,
          1969-2009}
           % figure produced in /R/DataDiagnostics.R
           \includegraphics{Figures/USmissingAge}
          \label{fig:USmissingAge}
\end{figure}

For the US, the degree of missingness of fathers' age varies by age of
mother (not shown). For ages greater than 25, we do not expect this to affect
results in an important way. Averaged over all years, ages $<= 20$ all had missingness of more than
20\%; ages $<= 16$ had missingness of more than 40\%, and ages $<= 14$ had
missingness of more than 60\%. This may affect results if the age-pattern of
males of unrecorded age differs from that of males of recorded age in a
non-trivial way. This uncertainty enters into the male
age-pattern of fertility, as well as the bivariate
age distribution of births (age of mother by age of father) may affect results
for the US where these age-specific data are used.

For both countries, cross-tabulated
sex-specific birth counts were entered into matrices of standard 111$\times$111
dimensions, covering ages 0-110. Ages with no births simply contain zeros. Open
age groups from the original data were not redistributed over ages beyond the
bounds of the original microdata. Especially for young ages of fathers and the
upper ages of mothers, this will be visible in respective age
patterns, but the effect on overall results (TFR, growth rates) will be
trivial.

Where birth counts are not required differentiated by sex of birth, for
instance, we sum over sex. Birth counts by age of mother are always taken from
the column margin of the birth matrix, while age of father is the row margin.
This practice helps to minimize the number of data objects used.

\subsection{Exposures and mortality data}

All other data for the US and Spain were downloaded from the Human Mortality
Database (HMD)\citep{wilmoth2007methods}. These data include, most importantly,
population exposures\footnote{At the time of this writing, exposures from the
Human Fertility Database (\url{www.humanfertility.org}) may have been more
appropriate for certain age groups, but since we prefer to use all ages
$0-110+$, HMD exposures were utilized instead.} and population counts by age,
sex and year and the deaths distribution, $d_x$, from the sex-specific lifetables. $d_x$ informaion was 
always rescaled to sum to 1, which minimized rounding errors and simplified
programming. Other items drawn from the HMD but used less consistently
included, mortality hazards, $\mu_x$, survival curves, $l_x$ (also rescaled so that $l_0 = 1$), 
lifetable exposures, $L_x$,
life expectancies, $e_x$, and death counts by Lexis triangles. Each of these
items is used in single-age format, with ages $0-110+$. The open age group,
$110+$ is used as age 110 and is given no further treatment. The
respective uses of each of these items should be obvious from the context 
of the formulas being applied, and are stated explicitly in the text or in 
footnotes where the use may not be obvious.

HMD data itself has come from the respective official sources of these
two countries, and so will inherit whatever errors were present in the original
data prior to applying the HMD methods protocol. Most relevant for this
dissertation, Spanish intercensal population estimates, which are the basis of
HMD population estimates, have been subject to an uncommon smoothing
procedure over age by the INE. Where abrupt changes in cohort size occur, such
as the unusually large 1941 cohort, this procedure will have the effect of
decreasing the size of large cohorts and increasing the size of small neighboring cohorts. This is
highly undesirable for any demographic study and is apparently a legacy
practice that will soon cease\footnote{Thanks to Dr. Amand Blanes for bringing
this issue to my attention. The INE will likely release new retrospective
population estimates during the course of 2013, but these will come too late
for incorporation into the present dissertation.}. In this dissertation, this
distortion will be most noticeable in the calculation of event-exposure rates, 
wherein the numberator has not been subject to this exogenous smoothing, but 
the denominator has. It is unfortunately the case that alternative sources of 
population estimates for Spain are in worse condition. These effects will echo through all HMD mortality
estimates for Spain, as well as our own fertility calculations. 

\subsection{Empirical results in this dissertation}

Data-based results in this dissertation are with few exceptions displayed
graphically, rather than in the form of tables. Since the original data and
code used to produce results are all available, one could with minimal effort
and no guesswork derive the numbers represented in each figure. We prefer
graphical representation of results because this conveys larger amounts of
information in less space and is more intuitive for the reader. The reader
should understand that data are used primarily to illustrate the concepts under discussion, rather
than in search of some empirical truth. The two
above-mentioned caveats for the data used herein (missing fathers' age in the
US, and faulty population estimates for Spain) should be born in mind when
interpreting some figures, such as age-specific fertility curves. We do not
expect either of these two data drawbacks to affect summary results 
(e.g. growth rates, $r$) in a noticeable way, and we expect that any
\textit{broad} conclusions arrived at in following will be robust to these
original shortcomings. 

The user will also note that most results are derived deterministically.
Accounting for uncertainty in many of the results presented here would provide
the reader with more insight into particular kinds of results, such as projected
results or stable population structures occassionally displayed in figures.
Several of the methods to be presented in following are novel to the field of
demography, and so we may look upon the results dervied therefrom as test
results. The addition of stocasticity to these methods, if they are deemed of
worth, is left open as a branch for improvement. Here we only wish to point out
that the majority of figures will, for this reason, not contain confidence or
credibility bounds.


