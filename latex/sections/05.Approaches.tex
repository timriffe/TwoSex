% 
These are some opening words to the first non-introductory chapter. This chapter introduces an exhaustive suite of
 approaches to the two-sex problem, and works through the various analytic adjustments. Authors have tended to agree on a list of conditions for a valid sex-consistency adjustment: 1) homogeneity, meaning that if the numbers of males and females doubles, then so does the number of births (marriages) (I personally do not subscribe to this one), 2) only positive fertility (marriage) rates are allowed and, 3) if one sex is missing there is no fertilty (or marriages). Then there have been a few guidlines that have been more desirable than mathematically necessary, e.g., that the two-sex rate must be bracketed by the separate male and female one-sex rates- this has been shown more than once to be not necessarilyu true \citep{yellin1977comparison}

\begin{description}
\item[Alho]: competing risks
\item[Castillo-Chavez]: logistic, minimum and harmonic
\item[Caswell]: bifurcation, exstinction
\item[Choo-Siow]: matching with spillover, utility
\item[Chung]: cycles. (still acquiring article)
\item[Das Gupta]: stable pops, general approach
\item[Decker]: extension of Choo-Siow model
\item[Feeney]: (Diss) unknown contrib, in acquisition.
\item[Fredrickson]: random mating vs strict monogamy, implications
\item[Henry]: matrix decomp (panmictic circles)
\item[Hoppensteadt] general formula for 2-sex age structures differential equation.
\item[Inaba]: Cauchy problem loose differential
\item[Karmel]: deterministic model with fixed heterogamy (4 year)
\item[Kendall]: weighted mean
\item[Keyfitz]: comparison of means methods
\item[Kirschner]: general mixing models (context HIV)
\item[Kuczynski]: arithmetic average, idea that two-sex must fall in one sex interval
\item[Lotka]: analogy to Lotka-Volterra predator prey
\item[Martcheva]: Fredrickson-Hoppensteadt model, exponential
\item[Maxin]: including divorce
\item[McFarland]: iterative contingency table (1971 Diss, have article)
\item[Milner]: partial differential equation
\item[Mitra]: instrinsic rates, building from Das Gupta
\item[Pollak]: stability framework, birth matrix mating rule (BMMR) with persistant unions
\item[Pollard]: generalized harmonic
\item[Pruess]: question of stability and exponentiality
\item[Schoen]: harmonic
\item[Tennenbaum]: analogy to foraging model (2006 Diss)
\item[Thieme]: general solution to age-structured pop with subgroups (i.e. sexes with marstat)
\item[Waldstaetter]: trying to acquire still (1990 Diss)
\item[Yang and Minlner]: logistic
\item[Zacher]: also on topic of exponentiality (group with Pruess, Martcheva)
\end{description}

\section{Means and bracketing}

\begin{singlespace}
\begin{quote}
Now of everything that is continuous and divisible, it is possible to take the larger part, or the smaller part, or an equal part, and these parts may be larger, smaller, and equal either with respect to the thing itself or relatively to us; the equal part being a mean between excess and deficiency. By the mean of the thing I denote a point equally distant from either extreme, which is one and the same for everybody; by the mean relative to us, that amount which is neither too much nor too little, and this is not one and the same for everybody.
\citetalias{rackham1947trans}
\end{quote}
\end{singlespace}


It being the case that summary measures, such as $r$, $R_0$, $B$ or $M$ based on male or female rates will nearly always differ, it may be reasonable to suppose that the true rates, those descriptive of the whole population, lie somewhere between the one-sex linear rates. This, in keeping with \citet{yellin1977comparison}, we term bracketing, and to be clear I consider it a weak assumption. If the true rates are bracketed by the single-sex rates, then one way to estimate them might be to calulate one of a variety of potential means. If necessary, the mean rates can be rescaled back to each sex such in order that they produce the same summary measure, i.e. forcing consistency. This modeling decision is not as sharply defined as it might appear at first glance. Three major refinements must be made in order to decide how to apply the strategy of means by combining some of the following considerations:

\begin{itemize}
\item One must decide what to take the mean of, and how this relates age-sex-specific rates to the final summary measure, i.e. between top-down or bottom-up averaging.
\item There are several candidate varieties of means. Demographers have most often compared the Pythagorean means: arithmetic, geometric and harmonic. For the sake of thoroughness, we will also consider logorithmic, identric, hedonic, contraharmonic, arithmetic-geometric and root mean squares.
\item Males and females can either be given equal or unequal weight. For the later, weights must be derived from data.
\end{itemize}

Results will vary based on different combinations of these considerations, have different implications for model flexibility, and entail more or less reasonable assumptions, which will be discussed in following. 

\subsection{A mean of what?}
Take for instance births, $B$, which we calculate by multiplying age-specific fertility rates to population exposed to fertility and then summing:

\begin{align}
B = \sum _{x=\alpha} ^{\omega} f_x N_x
\end{align}

Alternatively, and more intuitive for program or spreadsheet implementation, one can express this in terms of vectors, where $\bm{f}$ is a vector of ASFR, $\bm{n}$ a vector of population exposures and $\bm{b}$ a vector of births by age of progenitor (male or female as the case may be). The above formula becomes:

\begin{align}
\bm{b} &= \bm{f}\cdot\bm{n} \notag \\ \notag \\
B &= \sum \bm{b}
\label{birthvec}
\end{align}

Clearly, in the data year from which we estimate rates, calculating $B$ from either male or female rates will necessarily produce the same number, but in later years (iterations) the births calculated by males, $B^m$, and by females, $B^f$, will differ. This is the discrepancy that we wish to remedy, such that the male and female rates produce the same amount of births, either in total or by age of mother and father.

\subsubsection{Top-down rescaling}
The simplest, but most rigid, manner of forcing consistency is to take a mean of the births estimated by males and females, $\bar{B}$, and use it to monotonically rescale the single-sex rates. The resclaed rates are then taken used to estimate births in year $t$ of the model, and this procedure is repeated at each model iteration, forcing consistency throughout. This is the method described by \citet{keilman1985nuptiality} for a (then) experimental projection model in the Netherlands, and which used the harmonic mean of total marriages, $M$, to rescale male and female marriage schedules. An intuitive moniker for this method is top-down rescaling. Where $\bm{f^{\star}}$ is the vector of rescaled ASFR:

\begin{align}
\bm{f^{m\star }} &= \bm{f^m} \left(\frac{\bar{B}}{B^m}\right) \notag \\ \notag \\
\bm{f^{f\star }} &= \bm{f^f} \left(\frac{\bar{B}}{B^f}\right)
\label{simplerescale}
\end{align}

In R code, equation \ref{simplerescale} looks something like that displayed below, when \texttt{fm}, \texttt{ff}, \texttt{nm} and \texttt{nf} are defined vectors containing male and female fertility rates and population exposures, respectively. Here, the arithmetic mean, \texttt{mean()}, is implemented as an example, though this can be switched out for other another mean function.

%<<echo=F,results=hide>>=
%# here we generate some fake variables, just for a brief code demonstration:
%set.seed(1)
%nm <- rev(sort((1000+1000*runif(70))*seq(from=1,to=.65,length.out=70)))
%nf <- rev(sort((1000+1000*runif(70))*seq(from=1,to=.75,length.out=70)))
%fm <- c(rep(0,15),sort(runif(10)),rev(sort(runif(25))),rep(0,20))
%ff <- c(rep(0,12),sort(runif(7)),rev(sort(runif(20))),rep(0,31))
%@

%\singlespacing
%<<verbatim=TRUE,results=hide>>=
%# Births predicted from males and females:
%bm 		<- fm*nm
%bf		<- ff*nf
%# arithmetic average of sums:
%bbar 	<- mean(c(sum(bm),sum(bf)))
%# rescale male and female fertility:
%fmstar 	<- fm*(bbar/sum(bm))
%ffstar	<- ff*(bbar/sum(bf))
%@
%\doublespacing

This method preserves all aspects of the fertility PDF for each sex. Consider the case where one sex, say females, experiences a disproportionate increase in the number of 20-24 year-olds and all other ages for males and females remain the same. This will cause the total of births predicted by females to increase, and so increase somewhat the \textit{mean} of births predicted by male and female rates. Uniform rescaling assumes that the excess females from this one age class will be mated evenly across the distribution of males, and the other age classes of females will be equally disadvantaged by the boom in 20-24 year-olds. One could reasonably expect ripple-effects in competition across the ages from such a sudden spike, but one would also expect neighboring age groups to be more affected than distant age groups. For this reason, top-down rescaling is considered rigid; the sex-specific fertility PDFs never change in accordance with shifting age-distributions of the sexes. In a sense, all ages are affected equally by adjustments. A positive aspect of this adjustment is that it will never produce a negative number, and it will always respect zeros for ages with no fertility.

\subsubsection{Age-specific rescaling}
Still preferable would be to allow adjusted age schedules, $f_x^{\star}$, to change flexibly by preserving some amount of the age-heterogamy pattern present in the population. That is to say the above mentioned excess in 20-24 year-old females should translate more directly to increased rates for similarly aged males, but have a much dampened affect on older males. It should also predjudice the marriage prospects of 15-19 and 25-29 year-old females more than that of older females. This desirable quality in model feedback consitutes an improvement, but is itself rather difficult to implement satisfactorily. 

The simplest approach for age-specific rate rescaling is to assume fixed heterogamy, i.e. all parents and/or spouses having an exact difference in age. This value is generally taken to be the mean age difference between spouses, e.g., from 2 to 5 years in whole numbers, depending on the population and year. This was an intermediate step in \citet{karmel1947relations}, assuming 4-year fixed age heterogamy before progressing to include all age combinations, and by \citet{cabre1997tortulos} to predict a marriage squeeze in Spain, assuming 3-year fixed heterogamy. For example, assuming 3-year age differences, under fixed age heterogamy, a sudden spike in 25-year-old males will increases marriages of 22 year-old females, but have no effect on distant ages. The problem is that spillover effects are ignored entirely, with neigboring male ages unpredjudiced and neighboring female ages receiving no extra pressure to marry. This method therefore only gives a good approximation of squeezes when changes are broad and gradual, or when the variance in age heterogamy is very low (which has yet to be observed). Furthermore, older ages would tend to be disqualified from consideration, since male fertility continues well beyond female menopause.

To retain fixed heterogamy but permit spillover effects, one could assign a moving age-window of potential spouses, assigning another window for ages giving the greatest competition and taking both into consideration for each single age. However, these windows would have to change by age and would also be unnecessarily rigid. Similarly, a weighted window could be used, with weights spanning the ages of all potential spouses and a different set of weights to take into account all potential competitor ages. In either case, it is unclear how one would apply these windows, weighted or not, simultaneously so as to resolve the issue of rate adjustments. If one knew how to apply moving windows, then in principle, one could maintain this as a given set of constraints, to be applied to changing stocks each year, each age of male and female having an inherent propensity to marry, but constrained by the market and relatively loose heterogamy parameters. However, the fixing of windows and/or weighting schemes would also be in a way accidents of prior heterogamy outcomes. Apparently no studies have undertaken any variant of the present ``moving window'' proposal, but instead leap to the next level of complexity.

The most thorough method, that which comes the closest to continuous rate distributions of potential mates, is to consider all age combinations of mates or spouses. Generally this is done by calculating a rate for each \textit{potential} mate combination in a particular year, producing two rate matrices, one for males and another for females. Predicted births (or marriages) for each age combination are calculated separately from the male and female rates, producing two more matrices what will be unique from one another in nearly all non-zero entries. A mean prediction is then calculated, using a selected mean function, and this is then used to adjust the male and female rates separately.

Symbolically, where $\bm{M}$ is a matrix of counts of births (or marriages) by age of male partner and female partner, $\bm{m_0}$ and $\bm{f_0}$ are vectors of male and female exposures the same year (the jump-off year) and whose lengths correspond with the row and column dimensions of $\bm{M}$, respectively, we derive male and female rate matrices, $\bm{W^m}$ and $\bm{W^f}$:

\begin{align}
\bm{W^m} &= diag(\bm{m_0}^{-1}) \times \bm{M} \notag \\ \notag \\
\bm{W^f} &= \bm{M} \times diag(\bm{f_0}^{-1})
\end{align}

This kind of matrix operation may appear exotic to most demographers and some explanation is in order. Recalling that male ages are in the rows of $\bm{M}$ and females ages in the columns, to derive male rates, one must divide \textit{row-wise} by the vector of male exposures and \textit{column-wise} by the vector of female exposures. This translates into matrix operations by taking the inverse of the (strictly non-zero positive) vectors of exposures and converting them into diagonal matrices. Multiplying from the left of $\bm{M}$ divides row-wise (males) and multiplying on the right divides column-wise (females). The resulting rate matrices, $\bm{W^m}$ and $\bm{W^f}$, are of the same dimensions as $\bm{M}$, are age-indexed inthe exact same way, and have a straightforward interpretation. For instance $\bm{W_{30,27}^m}$ is the fertility (or marriage) rate for 30 year-old males and 27-year old females with the exposure of 30 year-old males in the denominator, and $\bm{W_{30,27}^f}$ is the same, except the exposure of 27 year-old females in the denominator. The row margins of $\bm{W^m}$ are the familiar male ASFR and the column margins of $\bm{W^f}$ are female ASFR.

As above, multiplying these sex-specific rate matrices by the original sex-specific exposures (using analogous diagonal matrix trick) yields the same count matrix $\bm{M}$, as should be the case for the year from which data were taken. Changing the male and female exposures, as happens when iterating to the next year in a model, and repeating this procedure will produce two divergent matrices of $\bm{M}$. The strategy to force consistency is analogous to the above simpler case. First, derive the two divergent sex-specific count matrices for time $t$, $\bm{M_t^m}$ and $\bm{M_t^f}$. Then, take the element-wise mean of these two matrices to yield $\bm{M^{\star}}$, and use this to rescale the male and female rate matrices. 

\begin{align}
\bm{M_{t}^{m}} &= diag(\bm{m_t}) \times \bm{W^m} \notag \\ \notag \\
\bm{M_{t}^{f}} &= diag(\bm{f_t}) \times \bm{W^f} \\ \notag \\
\bm{\bar{M_{t}}} &= meanfun(\bm{M_{t}^{m}},\bm{M_{t}^{f}}) \\ \notag \\
\bm{W_t^{m\star}} &= \left(\bm{\bar{M_{t}}} \circ \frac{1}{\bm{M_{t}^{m}}}\right) \circ \bm{W^m} \notag \\ \notag \\
\bm{W_t^{f\star}} &= \left(\bm{\bar{M_{t}}} \circ \frac{1}{\bm{M_{t}^{f}}}\right) \circ \bm{W^f}
\end{align}

\noindent, where $meanfun$ is a general mean function, and can be switched out for any of the various means discussed in the next section. Above, $\circ$ stands for the Hadamard product of two matrices, i.e. the element-wise product, rather than the standard matrix product; and $\frac{1}{\bm{M_{t}}}$ is understood as $\frac{1}{\bm{M_{i,j,t}}}$, that is to say, the element-wise inverse of the matrix, \textit{not} the standard matrix inverse.

This produces two adjusted rate matrices, $\bm{W^{m\star}}$ and $\bm{W^{f\star}}$, which when multiplied into the corresponding exposures from year $t$ (using the diagonal matrix trick), separately yield the exact same count matrix, $\bm{M_{t}^{\star}}$. In this way, the rate matrices $\bm{W^m}$ and $\bm{W^f}$ can be maintained into indefinite future iterations, or assumptions may be applied as to how they change. These matrices are used as external standards. In the end, adjusted rate matrices will always be returned that produce consistent event counts, but these may be considerably different from the standards used, due to density dependent model feedback.

An R implementation of age-combination-specific consistency adjustments turns out to be much more straightforward than the above formulas would suggest. Specifically, R allows division of a matrix by a vector without prior conversion into a diagonal matrix. Omitting this step increases code legibility. A code sample to demonstrate this point, where \texttt{\%$\ast$\%} is the R operator for matrix multiplication:


%\singlespacing
%<<keep.source=TRUE>>=
%set.seed(1)
%# a random matrix:
%A <- matrix(runif(4),2)
%# a random vector with which to do row-division:
%b <- runif(2)
%# equality of row-wise division by vector (TRUE):
%all.equal((A/b),diag(1/b)%*%A)
%# likewise, for column division (TRUE):
%all.equal(t(t(A)/b),A%*%diag(1/b))
%@
%\doublespacing

\noindent, thus later code chunks will prefer the \texttt{A/b} formulation for row-wise division by a vector, as it is also computationally lighter. To demonstrate, assume we have matrix $\bm{M}$, tabulated from data, an example is given below.

In general, after tabulating the intitial matrix $\bm{M}$, it is good practice to smooth this along both dimensions in order to reduce the effects of stochasticity among less common age-combinations. Otherwise, random events from year zero will echo through the model. This results in fewer cells containing zeros and less noise on the two-dimensional perimeters. 

Arguments have been made that methods using data based on all age combinations from a given year still do not adequately account for inter-age competition in mating. The problem is that the rates derived as standards are also the product of competition in the year from which data were taken; what we would like to use as standards are the forces inherent in each sex \textit{prior} to the market. This is indeed how the standard rate matrices are used in future iterations of the population model, and ideally we would be able to backward-derive them from the data. This particular point has yet to be resolved.

Furthermore, the standard rate matrices are still static \textit{within-sex}. This is best explained by example: Say there is a spike in 25 year-old males. This will lower all rates in the adjusted male row, $\bm{W_{25,j}^{m\star}}$, and increase all rates in the adusted female row $\bm{W_{25,j}^{f\star}}$, which essentially increases age-specific fertility (or marriage rates) at all female ages. However, these changes in female rates do not then echo back into other male ages. Theoretically, 24 year-old males, $\bm{W_{24,j}^{m\star}}$, (and all other male ages) would also be affected negatively by this spike. 

\subsection{Varieties of means}



\citet{keyfitz1972mathematics}

The following sections are in various stages of progress.
Don't worry about placement or consistency with the above for the time being
% these will move around, but stuff needs to get written
\subsection{Birth Matrix Mating Rule}

\subsection{General Equilibrium Models}
% cite Lam and Sanderson
The balancing of marriages (births) has also been treated using economic models
in the General Equilibrium family of models \citep{}. The underlying
link between a marriage market and this otherwise out-of-place sounding model
family is that while effective numbers of available males and females may
change, and each will have its own utility function for mating, the number of
marriages (births) is always equal for both males and females, i.e. and
equilibrium is always found. and the two-sided supply and demand system that
arises

 each individual with a
personal set of \textit{scores} for potential partners


\input{sections/Approaches/GeneralizedMeans.tex}
\input{sections/Approaches/Henry.tex}
\FloatBarrier
\citet{mc1975models} introduced a well-established method called iterative
proportional fitting (IPF)\footnote{Also called matrix-raking.}, often used for
rescaling tables, to the two-sex problem for marriage models. We will apply
the method to fertility only, though the reader may consult
\citet{mc1975models} or \citet{Matthews2013} for details on how to apply
this method in the case of marriage matching. This method works by starting with
a known cross-tabulation of births, in our case from the base year $t$. First compute 
the marginal fertility rates for males and females (ASFR). Apply the male and
female ASFR to exposures predicted for some future year $t+n$ to produce
initially-predicted marginal birth count distributions, the sums of which will 
never agree (as was illustrated in Section~\ref{sec:divbirth}). These new
marginal distributions may be uniformly rescaled according to some mean of their
respective sums. The mean function chosen will only affect results when the
sex-gap in predicted births is very large\footnote{we have compared overall
results using arithmetic and harmonic means, and found no noteworthy difference. 
All results will be in terms of the harmonic mean for this first rescaling.}.

Now that the male and female sums for year $t+n$ agree, we iteratively
rescale the original birth matrix according to the male and female
predicted margins, alternating between male (row) and female (column) margins
until the new birth matrix margins sum exactly to the predicted margins. Of
course, the resulting matrix will differ depending on whether one begins with
rows or with column margins, and so we adopt the recommendation of
\citet{Matthews2013}, taking the elementwise average of the two
possible outcome matrices in each iteration before advancing to the next
iteration. We continue this iterative process until it no longer makes any
difference whether we first scale rows or first scale columns, and accept the
resulting \textit{raked} amtrix as our year $t+n$ prediction.

Using IPF, 1) male and female rates are guaranteed to agree, 2) structural 0s
are not problematic, and 3) the inter-age competition axiom is fulfilled,
which has not been the case with methods previously described. To illustrate
this property, we execute the following experiment: Taking initial birth count
and exposure data from US, 1975, we calculate male and female ASFR. We then
apply the male and female 1975 ASFR to exposures from 1980, and proceed with the
above-described method, finally settling on a predicted birth matrix for 1980,
from which we calculate new ASFR vectors for males and females (1980
predictions that agree). This is the base prediction that we will compare with.
Now we increase 1980 age-25 males (only) by 50\% and repeat the IMP procedure,
producing new ASFR predictions for males and for females.
Figure~\ref{fig:IPFcomp} shows the ratio of the hypothetical (age-25 male
exposure increased by 50\%) 1980 ASFR to the base 1980 ASFR prediction.

Figure~\ref{fig:IPFcomp} illustrates the competition axiom because age-25 male
rates decrease more than neighboring ages, and rate for male ages closest to 25
decrease by more than ages farther from 25 (in fact the ratio is monotonic in
either direction from 25-- ideal). Female rates increase as well, also as a
rough function of proportional \textit{intermatedness} with age-25 males in the
1975 birth matrix. Here we see only the difference in rates-- in terms of birth
counts, age-25 males would have a large increase, while all other ages would
experience decreases (lower rates applied to the same exposures). 

\begin{figure}[ht!]
        \centering  
          \caption{ASFR after increasing 1980 male exposure by 50\% compared
          with base 1980 ASFR prediction. Based on US 1975 ASFR and birth
          matrix.}
           % figure produced in/R/IPFage.R
           \includegraphics{Figures/IPFagecompetitiontest}
          \label{fig:IPFcomp}
\end{figure}

\paragraph{Iterative proportional fitting in models of population growth: }

The formulas to formalize the use of IPF fertility balancing will take on a
different appearance than those seen thus far. Note that the basic inputs to the
IPF function to constrain male and female fertility rates will be:
$IPF(B_{a,a'}(t), F_a^M(t), F_{a'}^F(t),P_a(t+n),P_{a'}(t+n))$, where $B_{a,a'}$
is the cross-classified birth matrix, $F_a^M$ and $F_{a'}^F$ are male and female ASFR,
and $P_a$ and $P_{a'}$ are future population estimates (exposures when
discrete). The function produces adjusted ASFR for \textit{both} males and females,
$F_a^{M^\ast}$, $F_{a'}^{F\ast}$. Let us define a shorthand where the year
$\tau$ is the year used as the standard for the IPF method, bearing in mind that
$IPF(\tau, p^M, p^F)$ requires the first three arguments from year $\tau$ (births and 
male and female ASFR), whereas the later two arguments, that which we would like
to adapt to, may change according to our ends. $F_a^{M^\ast(\tau,p^M, p^F)}$
will be the IPF adjusted male ASFR based on year $\tau$ data, and
$F_{a'}^{F^\ast(\tau,p^M, p^F)}$ the female ASFR output from the same procedure.
Then assuming constant mortality and continuous functions, we may define year $t$ total births as a
function of past births as:

\begin{align}
B(t) &= \int_{a=0}^\infty \varsigma B(t-a)p_aF_a^{M^\ast(\tau,p^M, p^F)}\dd a
\\ &= \int_{a'=0}^\infty (1-\varsigma) B(t-a')p_{a'}F_{a'}^{F^\ast(\tau,p^M,
p^F)}\dd a'
\end{align}
which works either with males or with females, since the IPF function already
balances fertility such that total births will be the same whether predicted by
males or females. $\varsigma$ is the proportion male at birth. If mortality is
held constant and $IPF(\tau,p^M, p^F)$ is always based on the same year $\tau$ 
constant information, the popultation will eventually begin
to grow at a constant rate $r$ which can be estimated from the following
equation:
\begin{align}
\label{eq:IPFtricky}
1 &= \int_{a=0}^\infty \varsigma e^{-ra}p_aF_a^{M^\ast(\tau,p^{M\infty},
p^{F\infty})}\dd a \\ 
&= \int_{a'=0}^\infty (1-\varsigma) e^{-ra'}p_{a'}F_{a'}^{F^\ast(\tau,p^{M\infty},
p^{F\infty})}\dd a'
\end{align}
where $p^{M\infty}$ for males is just the full age pattern of
$\varsigma e^{-ra}p_a$, and $p^{F\infty}$ is the full age pattern
from $(1-\varsigma) e^{-ra'}p_{a'}$ analogously for females. The estimation of
$r$ using this equation is unfortunately cumbersome, and is explained in
following.

\paragraph{Estimating the intrinsic growth rate: } First, note that either
version of Equation~\eqref{eq:IPFtricky} requires full information from both
males and females, so we may as well add the two right-side components and make
the equation sum to two:
\begin{equation}
\label{eq:IPFugly}
2 = \int_{a=0}^\infty \int_{a'=0}^\infty \varsigma
e^{-ra}p_aF_a^{M^\ast(\tau,p^{M\infty}, p^{F\infty})} + (1-\varsigma)
e^{-ra'}p_{a'}F_{a'}^{F^\ast(\tau,p^{M\infty}, p^{F\infty})}\dd a' \dd a
\end{equation}
As in some earlier iterative $r$-estimation instructions given in this
dissertation, one does well to allow $\varsigma$ to be determined by sex
ratios that vary over age of mother and father. This information we retain
in the sex-sex-specific fertility functions, featured elsewhere in this dissertation: $F_a^{M-M}$,
$F_a^{M-F}$,$F_{a'}^{F-F}$, $F_{a'}^{F-M}$, which will entail two IPF functions,
one for boy births and one for girl births. 
\todo{complete description, see if there's a better way to summarize}




\paragraph{Summary of the method applied to models of population growth:
}Fulfillment of the competition axiom is not a trivial achievment, and it also more or less tops off the list of important axioms: 1) The solution meets the availability
axiom, 2) is first degree homogenous, 3) is monotanous, 4) is symmetrical with respect the
sexes, 5) and is sensitive to substitution and competition. There is no
guarantee for bracketing, although the solution will always track and typically
be bracketed by the single-sex intrinsic growth rates.


\begin{figure}[ht!]
        \centering  
          \caption{IPF intrinsic growth rates, $r$, compared with
          single-sex $r^m$ and $r^f$. US, 1969-2009 and Spain, 1975-2009.}
           % figure produced in/R/IPFage.R
           \includegraphics{Figures/IPFager.pdf}
          \label{fig:IPFager}
\end{figure}



\FloatBarrier



\citet{mitra1978derivation} also relied upon the
building blocks of single-sex fertility. In this case, single-sex fertility is
conceived of in the same way as in the single-sex models, namely, males-male
and female-female. Mitra was concerned with deriving a method
to identify the two-sex intrinsic growth rate, $r$, while 1) constraining the
model to a constant sex ratio at birth in and along the trajectory to
stability, 2) fixing the \textit{shape} of fertility rates in and along
the path to stability and 3) forcing the stable $r$ to be
bracketed by $r^m$ and $r^f$. The model achieves these ends by regulating the
inherent tendency to diverge using a flexible scalar (uniform over age) weight,
$v_t$, which changes from year to year.

The scalar weight, $v_t$, requires some explanation. $v_t$ is multiplicative,
and obtains values between 0 and 1. Male fertility rates are always scaled by
$v_t$ and female rates by $1-v_t$, making the weights complimentary. The weight
must be determined in each year $t$ such that the sex ratio at birth for the
following year remains constant, thus $v_t$ changes continually until the
population arrives at a state of stability. There is one caveat, namely, that
the starting value of $v_t$, $v_0$, determines the target sex ratio at birth,
and so also determines the future values that $v_t$ will obtain along the path
to stability. Further, the stable $v^\ast$ will tend to differ from the starting
$t^\ast$. Since $r$ depends on $v_t$, which is determined by $v_0$, the
demographer's choice of $v_0$ determines $r$. In this way, the result is not
purged of subjectivity. This will be explained more in following.

Given this information, we can estimate the two-sex growth rate using the
following Lotka type unity function:
\begin{equation}
\label{mitralotka}
1 = \int _{a=0} ^\infty e^{-r^\ast a} \left(\frac{F_a ^m}{v_0} p_a^m  +
\frac{F_a ^f}{1 - v_0} p_a^f\right) \dd a
\end{equation}
where $F_a ^m$ and $F_a ^f$ are the male and female fertility probabilities
(rates when discrete), $p_a^m$ and $p_{a'}^f$ the male and female probabilities
of surviving from age $0$ to age $a$\footnote{$L_a$ from a radix-1 lifetable when
discretized}.

One can quickly converge upon a solution for Mitra's $r$ by modifying the
method proposed in \citet{coale1957new}\footnote{\citet{mitra1978derivation} alludes
to, but does not show this.}, by first calculating a trial estimate of $r$,
$\hat{r}$ and a trial two-sex mean generation length
$\widehat{T}$\footnote{For trial values, one may use simple
assumptions, such as the arithmetic means of the single sex Lotka parameters}.
Using the trial $\hat{r}^{(1)}$, one first calculates the residual, 
$\delta ^{(1)}$, from equation~\eqref{mitralotka}, then updates $\widehat{r^i}$
as:
\begin{equation}
\hat{r}^{(i+1)} = \hat{r}^{(i)} + \frac{\delta
^{(i)}}{\widehat{T} -
\frac{\delta ^{(i)}}{\hat{r}^{(i)} }}
\end{equation}
This procedure typically converges after half a dozen iterations, both faster
and more precise than a typical optimizer solution.

For a given set of starting weights, one can in this way arrive at a given two-sex growth rate, $r^\ast$. 
However, weights are also constrained to produce a constant sex ratio at birth (SRB). Given $r^\ast$ applied 
to each sex separately in the
state of stability, one notes that this sex ratio is \textit{not} maintained,
and must dervive stable weights, $v^\ast$ in order to force the final
SRB:

\begin{equation}
v^\ast =  \int _{a = 0} ^\infty e ^{ -r^\ast a} \frac{F_a ^m}{v_0} p_a^m \dd a
\end{equation}
An characteristic of this model design is that a given $v_0$ will always result
in a single, stable $v^\ast$. Mitra's two-sex growth rate, $r^\ast$, is unique for but depends upon 
the starting weights, $v_0$, and thus is not in general unique. Mitra 
suggests that a good choice for $v_0$ would be the value that minimizes the 
departure from constancy for unweighted single-sex fertility
rates. This is an attractive choice because constant rates are of course the
basis of stability. Once a population attains stability, weights, and therefore 
rates, are constant. In practice, one
chooses $v_0$ that minimizes the sum of the age-specific squared residuals
(for males and females) between $F_a$ and $F_a \times \tfrac{v_0}{v^\ast}$. 

If minimizing the difference between starting and stable rates is the criterion
for choosing $v_0$, Mitra's starting and stable weights form the time
series seen in Figure~\ref{fig:Mitra1978v0vstar}

\begin{figure}[ht!]
        \centering  
          \caption{Initial ($v_0$) and stable ($v^\ast$) weights for Spain and
          US, 1969-2009, according to the OLS criterion in
          \citet{mitra1978derivation}}
           \quad
           % /R/Mitra1978.R
           \makebox[\textwidth]{\includegraphics{Figures/Mitra1978v0vstar}}
          \label{fig:Mitra1978v0vstar}
\end{figure}

For Spain and the US throughout the period studied in this dissertation, both
$v_0$ and $v^\ast$ fell in the range $(.48,.6)$. $v_0$ was always
close to $.5$, entailing nearly equal weight for male and female rates.
The stable $v^\ast$ was consistently higher than $v_0$ and always higher than
$.5$, implying greater weights for males than females in stability. When $v > .5$, male rates weight
more than female rates, which was typically the case here, especially in the
limit, although this declined over the decades shown here. It is
tempting to interpret this result as contrary to the notion of female dominance,
which would intuit greater influence of females on overall fertility than
males. The interpretation of $v$ is unclear, and cannot necessarily in this case
be understood as direct evidence of male-leaning dominance.
\citet{mitra1978derivation} provides no guidance to interpret $v_0$, $v^\ast$,
less so a demographic meaning. It is however clear to this author that
the initial and stable weights for each sex are of greater interest
than the location of $r^\ast$, which is in any case guaranteed to be between
$r^m$ and $r^f$ ,and can therefore be summarized by some generalized mean, in
this case some mean of total exposed male and female populations.

\subsubsection{Critique of Mitra (1978)}
Initial and stable weights are attractive for purposes of the OLS
criterion and their potential for demographic interpretation (which has in any
case not been elaborated), by this author sees the use of such weights as
a superfluous byproduct of the model specification. Namely, $v_0$ and $v^\ast$
are only needed to maintain the SRB, but the SRB is only problematic due to
rote adherence to the single-sex Lotka framework, namely male-male and
female-female reproduction. Of course, males are not exclusively responsible for
the birth of boys and females are not responsible for the birth of girls. If the
model were simply changed to allow for the both-sex fertility of males and
females, one could forego balancing fertility and the SRB. As given, results are
sensitive to changes in the value of the SRB, and so this may allow for
unwelcome instability in the model.

Allowing for the full (both-sex of offspring) fertility schedules of each sex
brings to light another consideration: In this case weighting would not need to
vary between the initial and stable states, thereby making any use of weights a pure
indicator of dominance, as in \citet{goodman1967age}, but leaving the
demographer with no endogenous criterion for choosing weights, save perhaps for
the relative size of male and female exposures \citep{mitra1976effect}.
Furthermore, in either specification, males and females are treated on the same age scale, wherein the reproductive value of e.g. 20-year old males and females are
directly combined to a single sum. This sum is however not the result of age-sex
interactions between males and females.

TODO: revise text below:
Lacking from Mitra's model is allowance for variation in the SRB, age patterns
in SRB (it is a single number), weights that vary by age (the shape of
fertility is held constant), inter-age competition (all ages in the same sex are
inflated or deflated uniformly). Further the time-trajectory of weights along
the way to stability is not extracted from the model, although these would
possibly be the most interesting outputs from the model. We therefore cannot
judge the total variation in weights required in order to acheiie 2) inherent in
the state of stability. The author does not discuss this possibility or
calculate a time series in order to illustrate performance over a longer 
period, as does \citet{gupta1973, gupta1978general}. We will do both of these
things here in order to gain a better understanding of the method and possible
improvements upon it.

\citet{mitra1978derivation} also makes use of the unrealistic
notion of single-sex fertility, as have many similar solutions inspired by the
Lotka equation, though this author does not recognize the utility in doing so.
It is of course attractive and of interest to compare two-sex growth rates with the invariant $r^m$ and $r^f$, but
we need not limit ourselves to working with the same elements. The most notable
characteristic of \citep{mitra1978derivation} is the fact that in the OLS
solution for starting weights, the final $r^\ast$ is derived prior to the initial weights: since
minimizing variance in fertility rates between initial and stable states is the
criterion, weights are reduced to a convenient byproduct.

Also 1) single-sex fertility is an odd assumption and 2) results are very
sensitive to the SRB assumption and 3) the SRB in Spain was by no means constant
and 4) male and female ages are treated equally in formulas.






\FloatBarrier
\citet{gupta1978alternative} states\footnote{and this fits nicely into the flow
of our own presentation.} ``The lesson we learn from the above is that our
starting point must not be the formulation of two equations, one for $B_M(t)$ and another for
$B_F(t)$, but of a single equation for $B(t)$ with the help of a bisexual
fertility function that can explain the occurrence of births of type $(a,a')$ in
terms of the availability of both males and females''.

Das Gupta introduced a series of proposals for two-sex reproduction models
throughout the decade of the 1970s \citep{gupta1972two, gupta1973us,
gupta1976interactive, gupta1978alternative}, of which we will present the last
one. To summarize how the model works, imagine we would like to determine a
unified two-sex fertility rate, $F_{a,a'}$. Here it is clear
what to put in the numerator, as births can be tabulated by the ages of both parents.
 We thus work to define the idea of two-sex exposure for each age-combination. Das Gupta's
suggestion is derive a series of probability density functions that apply to
each age of potential mother and each age of potential father from information
contained in the matrix of observed births. Define these age-specific pdfs for
males, $U_{a,a'}$, and for females, $V_{a,a'}$ as:

\begin{align}
U_{a,a'} &= \frac{B_{a,a'}}{\int B_{a,a'} \dd a'}\\
V_{a,a'} &= \frac{B_{a,a'}}{\int B_{a,a'} \dd a}
\end{align}
In discrete terms, one establishes two matrices, arranged according to our
standard in this dissertation with male age in rows and female age over columns.
The row marginal sums for $U_{a,a'}$ all equal 1 and the column marginal sums of
$V_{a,a'}$ all equal 1\footnote{both with the exception of ages with no
fertility, which are left as 0 if undefined.}. One then calculates Das Gupta's
approximation of bisexual exposure, $E_{a,a'}$, by redistributing male and
female age-specific exposure and summing for each combination of age:
\begin{equation}
E_{a,a'} = U_{a,a'}E_a + V_{a,a'}E_{a'}
\end{equation}
which is then used as the denominator to calculate $F_{a,a'}$:
\begin{equation}
F_{a,a'} = \frac{B_{a,a'}}{E_{a,a'}}
\end{equation}
which is assumed constant in the stable model. As elsewhere, define the
sex-specific radix-1 survival functions, $p_a$, and $p_{a'}$, and a sex ratio
at birth, $S$, from which we determine the proportion male at
birth, $\varsigma=\frac{S}{1+S}$. Then Das Gupta's two-sex renewal
function becomes:
\begin{equation}
B(t) = \int_{a=0}^\infty \int_{a'=0}^\infty \Big( \varsigma U_{a,a'} B(t-a) p_a
+ (1-\varsigma)V_{a,a'}B(t-a) p_{a'}\Big)F_{a,a'} \dd a \dd a'
\end{equation}
If $U_{a,a'}$, $V_{a,a'}$, $\varsigma$ and $F_{a,a'}$ are assumed constant, then
as $t$ approaches infinity, the intrinsic rate of growth, $r$, will stabilize.
$r$ is estimated from the Lotka-type unit equation:
\begin{equation}
\label{eq:Guptaeq}
1 = \int_{a=0}^\infty \int_{a'=0}^\infty \Big( \varsigma U_{a,a'} e^{-ra} p_a
+ (1-\varsigma)V_{a,a'}e^{-ra'} p_{a'}\Big)F_{a,a'} \dd a \dd a'
\end{equation}
\paragraph{Estimating Das Gupta's $r$: } The value of $r$ that makes
Equation~\eqref{eq:Guptaeq} hold can be either optimized or found using an iterative 
process similar to that proposed by \citet{coale1957new}. We explain the latter method, as it
converges very fast:

\begin{enumerate}
  \item establish a starting value for $r$,
$r^{(0)}$ and a trial two-sex mean generation length
$\widehat{T}$. For both values, one may use simple
assumptions, such as the arithmetic means of the single sex Lotka parameters.
  \item Plug the trial $\widehat{r}^{(0)}$ into Equation~\eqref{eq:Guptaeq}
  to calculate a residual, $\delta ^{(1)}$.
  \item Improve the estimate of $r^{i+1}$ using:
  \begin{equation}
  \widehat{r}^{(i+1)} = \widehat{r}^{(i)} + \frac{\delta^{(i)}}{\widehat{T} -
\frac{\delta ^{(i)}}{\widehat{r}^{(i)} }}
  \end{equation}
  \item Use the new improved estimate, $r^{(i+1)}$ to calculate a new residual,
  and repeat steps 2 and 3 until $\delta^{(i)}$ vanishes to zero.
\end{enumerate}

\paragraph{Summary of the method: } \citet{gupta1978alternative} assumes that exposure to risk of
 age $a$ males is not evenly distributed over each age of potential female mate-
 i.e. that it is not random\footnote{As opposed to an earlier rendition of
 this method \citep{gupta1972two}}. Rather, the exposure to risk is partitioned
 over ages of potential mates according to the distribution present in a given 
 cross-classified birth matrix. In partitioning exposure in this way for each
 age of male and female, the cross-classified male and female risks are additive, and
 form the total exposure to risk. 
 
 It is attractive that this total exposure to
 risk sums to the total male and female exposures, but it is unclear whether the
 distribution should be based on cross-classified birth tabulations, which will
 likely be laden with structural artifacts. In other words, as relatively large
 cohorts pass through reproductive ages, they will tend to produce more births
 than neighboring cohorts-- even if the large cohorts also suffer lower rates.
 This will cause a spike along a particular age margin in the birth matrix,
 usually for both males and females of the larger cohort. This birth spike will
 be present in the exposure redistribution matrices, $U_{a,a'}$ and
 $V_{a,a'}$, and it will also remain evident in fertility rates, $F_{a,a'}$.
 This is problematic even in the first iteration of a projection, as the
 hypothetical large cohort will have moved up one age. This artifact will
become a characteristic of the stable population even as abrupt cohort size
differences vanish with time. The initial structural artifacts in the supposed
constant parameters thus enter into both exposures and rates. 

To a certain extent the present model also removes much of the anomolies that
result from single-sex fertility assumptions-- $m_{a,a'}$ is the ferility of
both sexes, and $\varsigma$ enters into Equation~\eqref{eq:Guptaeq} as a radix
weight for the male and female population structures. There is no dominance
parameter in this model, per se. Since the sex ratio is not use for the
splitting of births in the estimation of $r$, but rather for the weighting of exposure,
the model is not restricted to producing an estimate of the intrinsic growth 
rate that is bracketed by the male and female single-sex rates\footnote{Despite Das Gupta's effort in
explicitly producing a bracketed model in \citet{gupta1976interactive}.}. This
aspect is not mentioned explicitly in \citet{gupta1978alternative}, although
faith in the bracketing axiom was already waning by the time of its writing
\citep{yellin1977comparison}.
 
 To the extent that exposure within the model is a funciton of both males and
 females, this model may be said to be interactive. One may notice that since
 exposure is additive that the model will behave poorly in the absence of one
 potentially reproductive age-sex combination in the future (births for this
 age would not drop to 0 as they should). This possibility would not likely
 arise in practice, but it is still the most basic and necessary of
 commonly stated axioms. Further, the method is not fully age-interactive. AN
 increase in males (females) of one age will affect the fertility of all ages of
 females (males), but males have no effect on males and females have no effect
 on females.

\paragraph{The method applied to the US and Spanish data: } We estimate Das
Gupta's intrinsic growth rate for each year of the US and Spanish data. On the
whole, $r$ tracks the development of $r^m$ and $r^f$ over time, and it is
typically bracketed by them, but there are several years for both populations
where Das Gupta's $r$ was greater than either of the single-sex $r$ estimates.
These were all years in which the sex gap in $r$ was particularly narrow, and in
all cases Das Gupta's $r$ was the greater.

\begin{figure}[ht!]
        \centering  
          \caption{$r$ from Das Gupta (1978) and single sex intrinsic growth rates. US, 1969-2009, and Spain, 1975-2009}
           % /R/DasGupta.R
           \includegraphics{Figures/Gupta1978r}
          \label{fig:Gupta1978r}
\end{figure}

For purposes of prediction and ease of implementation, Das Gupta's model is
close to acceptable, though in following we will explore some models that are
somewhat more palatable and more widely studied, starting with models whose
two-sex fertility rates are derived from the harmonic (or other) mean of male
and female rates \citet{schoen1981harmonic}.

\FloatBarrier

\section{Schoen}

implemented. present Lotka-ization and results
\input{sections/Approaches/WeightedMean.tex}
