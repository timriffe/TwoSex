% 
These are some opening words to the first non-introductory chapter. This chapter introduces an exhaustive suite of
 approaches to the two-sex problem, and works through the various analytic adjustments. Authors have tended to agree on a list of conditions for a valid sex-consistency adjustment: 1) homogeneity, meaning that if the numbers of males and females doubles, then so does the number of births (marriages) (I personally do not subscribe to this one), 2) only positive fertility (marriage) rates are allowed and, 3) if one sex is missing there is no fertilty (or marriages). Then there have been a few guidlines that have been more desirable than mathematically necessary, e.g. that the two-sex rate must be bracketed by the separate male and female one-sex rates- this has been shown more than once to be not necessarilyu true \citep{yellin1977comparison}

\begin{description}
\item[Alho]: competing risks
\item[Castillo-Chavez]: logistic, minimum and harmonic
\item[Caswell]: bifurcation, exstinction
\item[Choo-Siow]: matching with spillover, utility
\item[Chung]: cycles. (still acquiring article)
\item[Das Gupta]: stable pops, general approach
\item[Decker]: extension of Choo-Siow model
\item[Feeney]: (Diss) unknown contrib, in acquisition.
\item[Fredrickson]: random mating vs strict monogamy, implications
\item[Henry]: matrix decomp (panmictic circles)
\item[Hoppensteadt] general formula for 2-sex age structures differential equation.
\item[Inaba]: Cauchy problem loose differential
\item[Karmel]: deterministic model with fixed heterogamy (4 year)
\item[Kendall]: weighted mean
\item[Keyfitz]: comparison of means methods
\item[Kirschner]: general mixing models (context HIV)
\item[Kuczynski]: arithmetic average, idea that two-sex must fall in one sex interval
\item[Lotka]: analogy to Lotka-Volterra predator prey
\item[Martcheva]: Fredrickson-Hoppensteadt model, exponential
\item[Maxin]: including divorce
\item[McFarland]: iterative contingency table (1971 Diss, have article)
\item[Milner]: partial differential equation
\item[Mitra]: instrinsic rates, building from Das Gupta
\item[Pollak]: stability framework, birth matrix mating rule (BMMR) with persistant unions
\item[Pollard]: generalized harmonic
\item[Pruess]: question of stability and exponentiality
\item[Schoen]: harmonic
\item[Tennenbaum]: analogy to foraging model (2006 Diss)
\item[Thieme]: general solution to age-structured pop with subgroups (i.e. sexes with marstat)
\item[Waldstaetter]: trying to acquire still (1990 Diss)
\item[Yang and Minlner]: logistic
\item[Zacher]: also on topic of exponentiality (group with Pruess, Martcheva)
\end{description}

\section{Means and bracketing}

\begin{singlespace}
\begin{quote}
Now of everything that is continuous and divisible, it is possible to take the larger part, or the smaller part, or an equal part, and these parts may be larger, smaller, and equal either with respect to the thing itself or relatively to us; the equal part being a mean between excess and deficiency. By the mean of the thing I denote a point equally distant from either extreme, which is one and the same for everybody; by the mean relative to us, that amount which is neither too much nor too little, and this is not one and the same for everybody.
\citetalias{rackham1947trans}
\end{quote}
\end{singlespace}


It being the case that summary measures, such as $r$, $R_0$, $B$ or $M$ based on male or female rates will nearly always differ, it may be reasonable to suppose that the true rates, those descriptive of the whole population, lie somewhere between the one-sex linear rates. This, in keeping with \citet{yellin1977comparison}, we term bracketing, and to be clear I consider it a weak assumption. If the true rates are bracketed by the single-sex rates, then one way to estimate them might be to calulate one of a variety of potential means. If necessary, the mean rates can be rescaled back to each sex such in order that they produce the same summary measure, i.e. forcing consistency. This modelling decision is not as sharply defined as it might appear at first glance. Three major refinements must be made in order to decide how to apply the strategy of means by combining some of the following considerations:

\begin{itemize}
\item One must decide what to take the mean of, and how this relates age-sex-specific rates to the final summary measure, i.e. between top-down or bottom-up averaging.
\item There are several candidate varieties of means. Demographers have most often compared the Pythagorean means: arithmetic, geometric and harmonic. For the sake of thoroughness, we will also consider logorithmic, identric, hedonic, contraharmonic, arithmetic-geometric and root mean squares.
\item Males and females can either be given equal or unequal weight. For the later, weights must be derived from data.
\end{itemize}

Results will vary based on different combinations of these considerations, have different implications for model flexibility, and entail more or less reasonable assumptions, which will be discussed in following. 

\subsection{A mean of what?}
Take for instance births, $B$, which we calculate by multiplying age-specific fertility rates to population exposed to fertility and then summing:

\begin{align}
B = \sum _{x=\alpha} ^{\omega} f_x N_x
\end{align}

Alternatively, and more intuitive for program or spreadsheet implementation, one can express this in terms of vectors, where $\bm{f}$ is a vector of ASFR, $\bm{n}$ a vector of population exposures and $\bm{b}$ a vector of births by age of progenitor (male or female as the case may be). The above formula becomes:

\begin{align}
\bm{b} &= \bm{f}\cdot\bm{n} \notag \\ \notag \\
B &= \sum \bm{b}
\label{birthvec}
\end{align}

Clearly, in the data year from which we estimate rates, calculating $B$ from either male or female rates will necessarily produce the same number, but in later years (iterations) the births calculated by males, $B^m$, and by females, $B^f$, will differ. This is the discrepancy that we wish to remedy, such that the male and female rates produce the same amount of births, either in total or by age of mother and father.

\subsubsection{Top-down rescaling}
The simplest, but most rigid, manner of forcing consistency is to take a mean of the births estimated by males and females, $\bar{B}$, and use it to monotonically rescale the single-sex rates. The resclaed rates are then taken used to estimate births in year $t$ of the model, and this procedure is repeated at each model iteration, forcing consistency throughout. This is the method described by \citet{keilman1985nuptiality} for a (then) experimental projection model in the Netherlands, and which used the harmonic mean of total marriages, $M$, to rescale male and female marriage schedules. An intuitive moniker for this method is top-down rescaling. Where $\bm{f^{\star}}$ is the vector of rescaled ASFR:

\begin{align}
\bm{f^{m\star }} &= \bm{f^m} \left(\frac{\bar{B}}{B^m}\right) \notag \\ \notag \\
\bm{f^{f\star }} &= \bm{f^f} \left(\frac{\bar{B}}{B^f}\right)
\label{simplerescale}
\end{align}

In R code, equation \ref{simplerescale} looks something like that displayed below, when \texttt{fm}, \texttt{ff}, \texttt{nm} and \texttt{nf} are defined vectors containing male and female fertility rates and population exposures, respectively. Here, the arithmetic mean, \texttt{mean()}, is implemented as an example, though this can be switched out for other another mean function.

%<<echo=F,results=hide>>=
%# here we generate some fake variables, just for a brief code demonstration:
%set.seed(1)
%nm <- rev(sort((1000+1000*runif(70))*seq(from=1,to=.65,length.out=70)))
%nf <- rev(sort((1000+1000*runif(70))*seq(from=1,to=.75,length.out=70)))
%fm <- c(rep(0,15),sort(runif(10)),rev(sort(runif(25))),rep(0,20))
%ff <- c(rep(0,12),sort(runif(7)),rev(sort(runif(20))),rep(0,31))
%@

%\singlespacing
%<<verbatim=TRUE,results=hide>>=
%# Births predicted from males and females:
%bm 		<- fm*nm
%bf		<- ff*nf
%# arithmetic average of sums:
%bbar 	<- mean(c(sum(bm),sum(bf)))
%# rescale male and female fertility:
%fmstar 	<- fm*(bbar/sum(bm))
%ffstar	<- ff*(bbar/sum(bf))
%@
%\doublespacing

This method preserves all aspects of the fertility PDF for each sex. Consider the case where one sex, say females, experiences a disproportionate increase in the number of 20-24 year-olds and all other ages for males and females remain the same. This will cause the total of births predicted by females to increase, and so increase somewhat the \textit{mean} of births predicted by male and female rates. Uniform rescaling assumes that the excess females from this one age class will be mated evenly across the distribution of males, and the other age classes of females will be equally disadvantaged by the boom in 20-24 year-olds. One could reasonably expect ripple-effects in competition across the ages from such a sudden spike, but one would also expect neighboring age groups to be more affected than distant age groups. For this reason, top-down rescaling is considered rigid; the sex-specific fertility PDFs never change in accordance with shifting age-distributions of the sexes. In a sense, all ages are affected equally by adjustments. A positive aspect of this adjustment is that it will never produce a negative number, and it will always respect zeros for ages with no fertility.

\subsubsection{Age-specific rescaling}
Still preferable would be to allow adjusted age schedules, $f_x^{\star}$, to change flexibly by preserving some amount of the age-heterogamy pattern present in the population. That is to say the above mentioned excess in 20-24 year-old females should translate more directly to increased rates for similarly aged males, but have a much dampened affect on older males. It should also predjudice the marriage prospects of 15-19 and 25-29 year-old females more than that of older females. This desirable quality in model feedback consitutes an improvement, but is itself rather difficult to implement satisfactorily. 

The simplest approach for age-specific rate rescaling is to assume fixed heterogamy, i.e. all parents and/or spouses having an exact difference in age. This value is generally taken to be the mean age difference between spouses, e.g. from 2 to 5 years in whole numbers, depending on the population and year. This was an intermediate step in \citet{karmel1947relations}, assuming 4-year fixed age heterogamy before progressing to include all age combinations, and by \citet{cabre1997tortulos} to predict a marriage squeeze in Spain, assuming 3-year fixed heterogamy. For example, assuming 3-year age differences, under fixed age heterogamy, a sudden spike in 25-year-old males will increases marriages of 22 year-old females, but have no effect on distant ages. The problem is that spillover effects are ignored entirely, with neigboring male ages unpredjudiced and neighboring female ages receiving no extra pressure to marry. This method therefore only gives a good approximation of squeezes when changes are broad and gradual, or when the variance in age heterogamy is very low (which has yet to be observed). Furthermore, older ages would tend to be disqualified from consideration, since male fertility continues well beyond female menopause.

To retain fixed heterogamy but permit spillover effects, one could assign a moving age-window of potential spouses, assigning another window for ages giving the greatest competition and taking both into consideration for each single age. However, these windows would have to change by age and would also be unnecessarily rigid. Similarly, a weighted window could be used, with weights spanning the ages of all potential spouses and a different set of weights to take into account all potential competitor ages. In either case, it is unclear how one would apply these windows, weighted or not, simultaneously so as to resolve the issue of rate adjustments. If one knew how to apply moving windows, then in principle, one could maintain this as a given set of constraints, to be applied to changing stocks each year, each age of male and female having an inherent propensity to marry, but constrained by the market and relatively loose heterogamy parameters. However, the fixing of windows and/or weighting schemes would also be in a way accidents of prior heterogamy outcomes. Apparently no studies have undertaken any variant of the present ``moving window'' proposal, but instead leap to the next level of complexity.

The most thorough method, that which comes the closest to continuous rate distributions of potential mates, is to consider all age combinations of mates or spouses. Generally this is done by calculating a rate for each \textit{potential} mate combination in a particular year, producing two rate matrices, one for males and another for females. Predicted births (or marriages) for each age combination are calculated separately from the male and female rates, producing two more matrices what will be unique from one another in nearly all non-zero entries. A mean prediction is then calculated, using a selected mean function, and this is then used to adjust the male and female rates separately.

Symbolically, where $\bm{M}$ is a matrix of counts of births (or marriages) by age of male partner and female partner, $\bm{m_0}$ and $\bm{f_0}$ are vectors of male and female exposures the same year (the jump-off year) and whose lengths correspond with the row and column dimensions of $\bm{M}$, respectively, we derive male and female rate matrices, $\bm{W^m}$ and $\bm{W^f}$:

\begin{align}
\bm{W^m} &= diag(\bm{m_0}^{-1}) \times \bm{M} \notag \\ \notag \\
\bm{W^f} &= \bm{M} \times diag(\bm{f_0}^{-1})
\end{align}

This kind of matrix operation may appear exotic to most demographers and some explanation is in order. Recalling that male ages are in the rows of $\bm{M}$ and females ages in the columns, to derive male rates, one must divide \textit{row-wise} by the vector of male exposures and \textit{column-wise} by the vector of female exposures. This translates into matrix operations by taking the inverse of the (strictly non-zero positive) vectors of exposures and converting them into diagonal matrices. Multiplying from the left of $\bm{M}$ divides row-wise (males) and multiplying on the right divides column-wise (females). The resulting rate matrices, $\bm{W^m}$ and $\bm{W^f}$, are of the same dimensions as $\bm{M}$, are age-indexed inthe exact same way, and have a straightforward interpretation. For instance $\bm{W_{30,27}^m}$ is the fertility (or marriage) rate for 30 year-old males and 27-year old females with the exposure of 30 year-old males in the denominator, and $\bm{W_{30,27}^f}$ is the same, except the exposure of 27 year-old females in the denominator. The row margins of $\bm{W^m}$ are the familiar male ASFR and the column margins of $\bm{W^f}$ are female ASFR.

As above, multiplying these sex-specific rate matrices by the original sex-specific exposures (using analogous diagonal matrix trick) yields the same count matrix $\bm{M}$, as should be the case for the year from which data were taken. Changing the male and female exposures, as happens when iterating to the next year in a model, and repeating this procedure will produce two divergent matrices of $\bm{M}$. The strategy to force consistency is analogous to the above simpler case. First, derive the two divergent sex-specific count matrices for time $t$, $\bm{M_t^m}$ and $\bm{M_t^f}$. Then, take the element-wise mean of these two matrices to yield $\bm{M^{\star}}$, and use this to rescale the male and female rate matrices. 

\begin{align}
\bm{M_{t}^{m}} &= diag(\bm{m_t}) \times \bm{W^m} \notag \\ \notag \\
\bm{M_{t}^{f}} &= diag(\bm{f_t}) \times \bm{W^f} \\ \notag \\
\bm{\bar{M_{t}}} &= meanfun(\bm{M_{t}^{m}},\bm{M_{t}^{f}}) \\ \notag \\
\bm{W_t^{m\star}} &= \left(\bm{\bar{M_{t}}} \circ \frac{1}{\bm{M_{t}^{m}}}\right) \circ \bm{W^m} \notag \\ \notag \\
\bm{W_t^{f\star}} &= \left(\bm{\bar{M_{t}}} \circ \frac{1}{\bm{M_{t}^{f}}}\right) \circ \bm{W^f}
\end{align}

\noindent, where $meanfun$ is a general mean function, and can be switched out for any of the various means discussed in the next section. Above, $\circ$ stands for the Hadamard product of two matrices, i.e. the element-wise product, rather than the standard matrix product; and $\frac{1}{\bm{M_{t}}}$ is understood as $\frac{1}{\bm{M_{i,j,t}}}$, that is to say, the element-wise inverse of the matrix, \textit{not} the standard matrix inverse.

This produces two adjusted rate matrices, $\bm{W^{m\star}}$ and $\bm{W^{f\star}}$, which when multiplied into the corresponding exposures from year $t$ (using the diagonal matrix trick), separately yield the exact same count matrix, $\bm{M_{t}^{\star}}$. In this way, the rate matrices $\bm{W^m}$ and $\bm{W^f}$ can be maintained into indefinite future iterations, or assumptions may be applied as to how they change. These matrices are used as external standards. In the end, adjusted rate matrices will always be returned that produce consistent event counts, but these may be considerably different from the standards used, due to density dependent model feedback.

An R implementation of age-combination-specific consistency adjustments turns out to be much more straightforward than the above formulas would suggest. Specifically, R allows division of a matrix by a vector without prior conversion into a diagonal matrix. Omitting this step increases code legibility. A code sample to demonstrate this point, where \texttt{\%$\ast$\%} is the R operator for matrix multiplication:


%\singlespacing
%<<keep.source=TRUE>>=
%set.seed(1)
%# a random matrix:
%A <- matrix(runif(4),2)
%# a random vector with which to do row-division:
%b <- runif(2)
%# equality of row-wise division by vector (TRUE):
%all.equal((A/b),diag(1/b)%*%A)
%# likewise, for column division (TRUE):
%all.equal(t(t(A)/b),A%*%diag(1/b))
%@
%\doublespacing

\noindent, thus later code chunks will prefer the \texttt{A/b} formulation for row-wise division by a vector, as it is also computationally lighter. To demonstrate, assume we have matrix $\bm{M}$, tabulated from data, an example is given below.

In general, after tabulating the intitial matrix $\bm{M}$, it is good practice to smooth this along both dimensions in order to reduce the effects of stochasticity among less common age-combinations. Otherwise, random events from year zero will echo through the model. This results in fewer cells containing zeros and less noise on the two-dimensional perimeters. 

Arguments have been made that methods using data based on all age combinations from a given year still do not adequately account for inter-age competition in mating. The problem is that the rates derived as standards are also the product of competition in the year from which data were taken; what we would like to use as standards are the forces inherent in each sex \textit{prior} to the market. This is indeed how the standard rate matrices are used in future iterations of the population model, and ideally we would be able to backward-derive them from the data. This particular point has yet to be resolved.

Furthermore, the standard rate matrices are still static \textit{within-sex}. This is best explained by example: Say there is a spike in 25 year-old males. This will lower all rates in the adjusted male row, $\bm{W_{25,j}^{m\star}}$, and increase all rates in the adusted female row $\bm{W_{25,j}^{f\star}}$, which essentially increases age-specific fertility (or marriage rates) at all female ages. However, these changes in female rates do not then echo back into other male ages. Theoretically, 24 year-old males, $\bm{W_{24,j}^{m\star}}$, (and all other male ages) would also be affected negatively by this spike. 

\subsection{Varieties of means}



\citet{keyfitz1972mathematics}

The following sections are in various stages of progress.
Don't worry about placement or consistency with the above for the time being
% these will move around, but stuff needs to get written
\subsection{Birth Matrix Mating Rule}

\subsection{General Equilibrium Models}
% cite Lam and Sanderson
The balancing of marriages (births) has also been treated using economic models
in the General Equilibrium family of models \citep{}. The underlying
link between a marriage market and this otherwise out-of-place sounding model
family is that while effective numbers of available males and females may
change, and each will have its own utility function for mating, the number of
marriages (births) is always equal for both males and females, i.e. and
equilibrium is always found. and the two-sided supply and demand system that
arises

 each individual with a
personal set of \textit{scores} for potential partners


\subsection{Generalized Means}

\subsection{Panmictic Circles}
\subsection{Iterative Proportion Fitting}

\textit{mc1975models} noted that a simple way to incorporate inter-age
competion (or at least inter-age sensitivity) in marriage count- balancing is
to iteratively rescale a known cross-tabulation of counts (say, from the
previous year) by the separately predicted male and female margins. That is to
say, if males are in matrix rows, one scales each row to sum to the predicted
male margin, then one scales each column to sum to the predicted female margin
(or vice versa, females then males). In rescaling rows to sum to the predicted male margin, 
followed by columns to sum to the predicted female margin, after just a few iterations the process
converges to a particular distribution. Cell counts thus shift between both male
and female ages from the original count matrix to the iteratively predicted
matrix, but stay close to the original distribution. The method satisfies the
thus-far most difficult axiom to incorportate, that of inter-age competition. 

However, the sums of the respective male and female predicted margins will of
course not agree-- After converging to a distribution, the total predicted count
will \textit{flip-flop} between the total male and female predictions, 
which will have differed. Further, \citet{Matthews2013} note that the
final distribution will depend upon whether one starts by scaling the male or
female margin. Both of these problems may be overcome, these authors suggest, 
by starting in parallel within each iteration with the male and female matrix
margins, followed by the other sex, producing two possible two-step scalings.
The starting matrix for the next iteration is taken as the average of these two
outcomes. Since the end result of each iteration is an average, the sum will be
intermediate to the divergent male and female marginal predictions, and the
biverate distribution will be indifferent to whether one started the iterative
process with males or females. The method could of course be further generalized
to take any mean of these two matrices, and not just an arithmetic mean. Results
will vary.

Neither of the above-mentioned studies used their respective iterative
adjustment procedures to predict birth counts, although the \citet{Matthews2013}
method is just a smaller part of a more complex model that includes fertility.
In this dissertation we will treat the \citet{Matthews2013} method only,
modifying it very slightly, so as to be based on a generalized mean (which
allows for a harmonic mean, for instance) as the basic for the male-female
intra-iteration averaging.




\subsection{Mitra}

\citet{mitra1978derivation}, was directly cocerned with finding a consistent
method to derive a two-sex intrinsic growth rate, $r^\ast$. Consistent here
means that 1) a constant SRB is maintained in and along the trajectory to stability, 2) the
essential \textit{shape} of fertility rates is held constant along the path to
stability and 3) the stable $r^\ast$ is guaranteed to be bracketted by $r^m$ and
$r^f$.

The method proposed by \citet{mitra1978derivation} works by assigning
complimentary (summing to 1) scalar (uniform over age) weights to male and
female single-sex fertility rates and placing the weighted rates , which are
then held constant, into a unified two-sex Lotka equation. For a given set of
weights, one can in this way arrive at a given two-sex $r$ estimate, $r^\ast$. 
However, weights are also constrained to produce a constant 
sex ratio at birth (SRB). Given $r^\ast$ applied to each sex separately in the
state of stability, one notes that this sex ratio is \textit{not} maintained, and must dervive new weights
in order to force the final SRB. These new weights are typically very close to
the original weights, which are also not very different for males and females.

The final $r^\ast$, though unique for a given set of weights, will
depend on the intitial weights chosen, and thus is not in general unique. Mitra
suggests that a good criterion for selecting starting weights would be those
that minimze the departure from constancy for unweighted single-sex fertility
rates. Constant rates are of course the essential aspect of stability- once in
the state of stability, weights no longer change, and rates turn out to be
constant, thus the criterion really deals with minimizing the departure from
initial conditions \textit{along the way} to stability.

Lacking from Mitra's model is allowance for variation in the SRB, age patterns
in SRB (it is a single number), weights that vary by age (the shape of
fertility is held constant), interage competition (all ages in the same sex are
inflated or deflated uniformly). Further the time-trajectory of weights along
the way to stability is not extracted from the model, although these would
possibly be the most interesting outputs from the model. We therefore cannot
judge the total variation in weights required in order to acheive stability.
Also of analytic interest would have been a time series of the initial and final
weights, as these can be interpreted as a kind of \textit{strength of female
dominance} 1) required to acheive lowest-effort stability and 2) inherent in
the state of stability. The author does not discuss this possibility or
calculate a time series in order to illustrate performance over a longer 
period, as does \citet{gupta1973, gupta1978general}. We will do both of these
things here in order to gain a better understanding.

\citet{mitra1978derivation} also makes use of the unrealistic
notion of single-sex fertility, as have many similar solutions, though this
author does not see the utility in doing so. It is of course attractive and of
interest to compare two-sex growth rates with the invariant $r^m$ and $r^f$, but
we need not limit ourselves to working with the same building blocks. However,
far and away the most novel and notable characterisic of
\citep{mitra1978derivation} is the fact that in the OLS solution for starting
weights, the final $r^\ast$ is derived prior to the initial weights








\FloatBarrier
\citet{gupta1978alternative} states\footnote{and this fits nicely into the flow
of our own presentation.} ``The lesson we learn from the above is that our
starting point must not be the formulation of two equations, one for $B_M(t)$ and another for
$B_F(t)$, but of a single equation for $B(t)$ with the help of a bisexual
fertility function that can explain the occurrence of births of type $(a,a')$ in
terms of the availability of both males and females''.

Das Gupta introduced a series of proposals for two-sex reproduction models
throughout the decade of the 1970s \citep{gupta1972two, gupta1973us,
gupta1976interactive, gupta1978alternative}, of which we will present the last
one. To summarize how the model works, imagine we would like to determine a
unified two-sex fertility rate, $F_{a,a'}$. Here it is clear
what to put in the numerator, as births can be tabulated by the ages of both parents.
 We thus work to define the idea of two-sex exposure for each age-combination. Das Gupta's
suggestion is derive a series of probability density functions that apply to
each age of potential mother and each age of potential father from information
contained in the matrix of observed births. Define these age-specific pdfs for
males, $U_{a,a'}$, and for females, $V_{a,a'}$ as:

\begin{align}
U_{a,a'} &= \frac{B_{a,a'}}{\int B_{a,a'} \dd a'}\\
V_{a,a'} &= \frac{B_{a,a'}}{\int B_{a,a'} \dd a}
\end{align}
In discrete terms, one establishes two matrices, arranged according to our
standard in this dissertation with male age in rows and female age over columns.
The row marginal sums for $U_{a,a'}$ all equal 1 and the column marginal sums of
$V_{a,a'}$ all equal 1\footnote{both with the exception of ages with no
fertility, which are left as 0 if undefined.}. One then calculates Das Gupta's
approximation of bisexual exposure, $E_{a,a'}$, by redistributing male and
female age-specific exposure and summing for each combination of age:
\begin{equation}
E_{a,a'} = U_{a,a'}E_a + V_{a,a'}E_{a'}
\end{equation}
which is then used as the denominator to calculate $F_{a,a'}$:
\begin{equation}
F_{a,a'} = \frac{B_{a,a'}}{E_{a,a'}}
\end{equation}
which is assumed constant in the stable model. As elsewhere, define the
sex-specific radix-1 survival functions, $p_a$, and $p_{a'}$, and a sex ratio
at birth, $S$, from which we determine the proportion male at
birth, $\varsigma=\frac{S}{1+S}$. Then Das Gupta's two-sex renewal
function becomes:
\begin{equation}
B(t) = \int_{a=0}^\infty \int_{a'=0}^\infty \Big( \varsigma U_{a,a'} B(t-a) p_a
+ (1-\varsigma)V_{a,a'}B(t-a) p_{a'}\Big)F_{a,a'} \dd a \dd a'
\end{equation}
If $U_{a,a'}$, $V_{a,a'}$, $\varsigma$ and $F_{a,a'}$ are assumed constant, then
as $t$ approaches infinity, the intrinsic rate of growth, $r$, will stabilize.
$r$ is estimated from the Lotka-type unit equation:
\begin{equation}
\label{eq:Guptaeq}
1 = \int_{a=0}^\infty \int_{a'=0}^\infty \Big( \varsigma U_{a,a'} e^{-ra} p_a
+ (1-\varsigma)V_{a,a'}e^{-ra'} p_{a'}\Big)F_{a,a'} \dd a \dd a'
\end{equation}
\paragraph{Estimating Das Gupta's $r$: } The value of $r$ that makes
Equation~\eqref{eq:Guptaeq} hold can be either optimized or found using an iterative 
process similar to that proposed by \citet{coale1957new}. We explain the latter method, as it
converges very fast:

\begin{enumerate}
  \item establish a starting value for $r$,
$r^{(0)}$ and a trial two-sex mean generation length
$\widehat{T}$. For both values, one may use simple
assumptions, such as the arithmetic means of the single sex Lotka parameters.
  \item Plug the trial $\widehat{r}^{(0)}$ into Equation~\eqref{eq:Guptaeq}
  to calculate a residual, $\delta ^{(1)}$.
  \item Improve the estimate of $r^{i+1}$ using:
  \begin{equation}
  \widehat{r}^{(i+1)} = \widehat{r}^{(i)} + \frac{\delta^{(i)}}{\widehat{T} -
\frac{\delta ^{(i)}}{\widehat{r}^{(i)} }}
  \end{equation}
  \item Use the new improved estimate, $r^{(i+1)}$ to calculate a new residual,
  and repeat steps 2 and 3 until $\delta^{(i)}$ vanishes to zero.
\end{enumerate}

\paragraph{Summary of the method: } \citet{gupta1978alternative} assumes that exposure to risk of
 age $a$ males is not evenly distributed over each age of potential female mate-
 i.e. that it is not random\footnote{As opposed to an earlier rendition of
 this method \citep{gupta1972two}}. Rather, the exposure to risk is partitioned
 over ages of potential mates according to the distribution present in a given 
 cross-classified birth matrix. In partitioning exposure in this way for each
 age of male and female, the cross-classified male and female risks are additive, and
 form the total exposure to risk. 
 
 It is attractive that this total exposure to
 risk sums to the total male and female exposures, but it is unclear whether the
 distribution should be based on cross-classified birth tabulations, which will
 likely be laden with structural artifacts. In other words, as relatively large
 cohorts pass through reproductive ages, they will tend to produce more births
 than neighboring cohorts-- even if the large cohorts also suffer lower rates.
 This will cause a spike along a particular age margin in the birth matrix,
 usually for both males and females of the larger cohort. This birth spike will
 be present in the exposure redistribution matrices, $U_{a,a'}$ and
 $V_{a,a'}$, and it will also remain evident in fertility rates, $F_{a,a'}$.
 This is problematic even in the first iteration of a projection, as the
 hypothetical large cohort will have moved up one age. This artifact will
become a characteristic of the stable population even as abrupt cohort size
differences vanish with time. The initial structural artifacts in the supposed
constant parameters thus enter into both exposures and rates. 

To a certain extent the present model also removes much of the anomolies that
result from single-sex fertility assumptions-- $m_{a,a'}$ is the ferility of
both sexes, and $\varsigma$ enters into Equation~\eqref{eq:Guptaeq} as a radix
weight for the male and female population structures. There is no dominance
parameter in this model, per se. Since the sex ratio is not use for the
splitting of births in the estimation of $r$, but rather for the weighting of exposure,
the model is not restricted to producing an estimate of the intrinsic growth 
rate that is bracketed by the male and female single-sex rates\footnote{Despite Das Gupta's effort in
explicitly producing a bracketed model in \citet{gupta1976interactive}.}. This
aspect is not mentioned explicitly in \citet{gupta1978alternative}, although
faith in the bracketing axiom was already waning by the time of its writing
\citep{yellin1977comparison}.
 
 To the extent that exposure within the model is a funciton of both males and
 females, this model may be said to be interactive. One may notice that since
 exposure is additive that the model will behave poorly in the absence of one
 potentially reproductive age-sex combination in the future (births for this
 age would not drop to 0 as they should). This possibility would not likely
 arise in practice, but it is still the most basic and necessary of
 commonly stated axioms. Further, the method is not fully age-interactive. AN
 increase in males (females) of one age will affect the fertility of all ages of
 females (males), but males have no effect on males and females have no effect
 on females.

\paragraph{The method applied to the US and Spanish data: } We estimate Das
Gupta's intrinsic growth rate for each year of the US and Spanish data. On the
whole, $r$ tracks the development of $r^m$ and $r^f$ over time, and it is
typically bracketed by them, but there are several years for both populations
where Das Gupta's $r$ was greater than either of the single-sex $r$ estimates.
These were all years in which the sex gap in $r$ was particularly narrow, and in
all cases Das Gupta's $r$ was the greater.

\begin{figure}[ht!]
        \centering  
          \caption{$r$ from Das Gupta (1978) and single sex intrinsic growth rates. US, 1969-2009, and Spain, 1975-2009}
           % /R/DasGupta.R
           \includegraphics{Figures/Gupta1978r}
          \label{fig:Gupta1978r}
\end{figure}

For purposes of prediction and ease of implementation, Das Gupta's model is
close to acceptable, though in following we will explore some models that are
somewhat more palatable and more widely studied, starting with models whose
two-sex fertility rates are derived from the harmonic (or other) mean of male
and female rates \citet{schoen1981harmonic}.

\FloatBarrier

\FloatBarrier
\label{sec:ageharmonic}
\begin{singlespace}
\begin{quote}
Now of everything that is continuous and divisible, it is possible to take the larger 
part, or the smaller part, or an equal part, and these parts may be larger, smaller, 
and equal either with respect to the thing itself or relatively to us; the equal part
 being a mean between excess and deficiency. By the mean of the thing I denote a point 
 equally distant from either extreme, which is one and the same for everybody; by the 
 mean relative to us, that amount which is neither too much nor too little, and this 
 is not one and the same for everybody.
\citetalias{rackham1947trans}
\end{quote}
\end{singlespace}

The most instinctual two-sex fertility (marriage) solution is to symmetrically
(with respect to the sexes) utilize information from the vital rates of both
sexes. Mean functions have been compared in the past\citep[see
e.g.][]{keyfitz1972mathematics}, but rated in terms of utility with difficulty.
In terms of the axioms mentioned in Section~\ref{sec:axioms}--rather than
performance-- the harmonic mean function has fared the best amongst a variety of
means. Schoen \citep{schoen1978standardized, schoen1977two, schoen1981harmonic}
provided a rationale and derivation for using the harmonic mean in order to 
balance marriage rates. \citet{martcheva2001mathematics} found evidence of
poor performance for the harmonic mean in projective scenarios. The same
strategy can be used to balance fertility rates, which is what we will do here. The method requires as inputs a matrix of birth counts cross-tabulated by age of father, $a$, and age of mother $a'$ 
and male and female exposures classified by age. The harmonic mean
\begin{equation}
\label{eq:harmonic}
H(P_a^m, P_{a'}^f) = \frac{2 P_a^m P_{a'}^f}{P_a^m + P_{a'}^f}
\end{equation}
is applied to male and female exposures in order to find an intermediate
denominator from which to calculate rates, $F_{a,a'}^H$:
 \begin{equation}
 \label{eq:harmonicrate}
 F_{a,a'}^H = \frac{B_{a,a'}}{H(P_a^m, P_{a'}^f)}
 \end{equation}
which in the stable population is assumed constant in time rather than
assuming constant male and female rates separately. In order to estimate 
a birth count in some future year $t+n$, calculate the harmonic mean
of male and female exposures and multiply into the constant harmonic rate:
 \begin{equation}
 B(t+n) = \int \int F_{a,a'}^H H\Big(P_{a}^m(t+n), P_{a'}^f(t+n)\Big) \dd a \dd
 a'
 \end{equation}
which we can rewrite to make year $t$ births a function of past births in the
renewal equation:
 \begin{equation}
 B(t) = \int \int F_{a,a'}^H H\Big(\varsigma B(t-a)p_a^m, (1-\varsigma) B(t-a)
 p_{a'}^f\Big) \dd a
 \dd a'
 \end{equation}
where $p_a^m$ and $p_{a'}^f$ are the male and female probabilities of surviving
from birth until age $a$, $a'$, and $\varsigma$ is the proportion male of
births, here assumed constant over age and time, though this may be relaxed.
Rewriting in this way brings us to a stable population framework. \citet{schoen1977two} 
proposed his own rectangular stable population framework, which 
will not be treated here. As $t$ becomes large, the annual growth factor
approaches a constant value equal to $e^r$, which can be estimated from the
following Lotka-type unity function: 

\begin{equation}
\label{eq:lotkaH}
1 = \int _{a=0}^\infty \int _{a'=0}^\infty F_{a,a'} H\Big(\varsigma
e^{-ra}p_a^m, (1-\varsigma)e^{-ra'}p_{a'}^f\Big)\dd a' \dd a
\end{equation}
where $F_{a,a'}^H$ is the constant fertility rate to be applied to the harmonic
mean of male and female exposures, $p_a^m$ and $p_{a'}^f$ are the male
and female radix-1 survival functions. $\varsigma$ serves to make the
male and female radices sum to 1, and also accounts for the fact that males and
females have slightly different $l_0$ values. 

\paragraph{Estimating $r$: } The two-sex harmonic intrinsic growth rate, $r$ can
be estimated in two ways, either assuming $\varsigma$ constant from the start
(likely based on the initial data) and using a generic optimizer, or by modifying the iterative procedure
suggested by \citet{coale1957new}, which works best if one simultaneously
estimates $r$ and $\varsigma$ (i.e. allowing $\varsigma$ to adjust to the
population structure, as it is known to vary with age). Here we will describe
the practical steps involved in the latter.

\begin{enumerate}
  \item Calculate the constant harmonic fertility rates for male and female
  births separately, $F_{a,a'}^{mH}$ and $F_{a,a'}^{fH}$
  \item Make a first estimate of the stable sex ratio at birth, $\hat{S}$; the
  initial observed sex ratio at birth is a good choice. From $S^0$ we derive a
  first estimate of the proportion male of births, $\varsigma^0$ (where
  superscripts indicate the iteration):
  \begin{equation}
  \varsigma^0 = \frac{S^0}{S^0+1}
  \end{equation}
  \item Find a first rough estimate of the net reproduction rate,
  $\widehat{R_0}$, assuming a growth rate of 0 and using the both-sex
  harmonic fertility rate $F_{a,a'}^{H} = F_{a,a'}^{mH} + F_{a,a'}^{fH}$:
  \begin{equation}
  \label{R0guessschoen}
  \widehat{R_0} = \int_{a=o}^\infty \int_{a'=0}^\infty H(\varsigma^0 p_a^m,
  (1-\varsigma^0)p_{a'}^f) F_{a,a'}^{H} \dd a' \dd a
  \end{equation}
  \item Assume a reasonable both-sex mean generation time, $\widehat{T}$.
  Weighting $a$ and $a'$ into Equation~\eqref{R0guessschoen} and then dividing
  by $\widehat{R_0}$ yields a good estimate of this. Otherwise one may simply
  choose a reasonable age, such as 30, or some mean of the male and female
  single-sex mean ages at reproduction.
  \item Calculate an initial value of $r$, $r^0$ as:
  \begin{equation}
  r^0 = \frac{log(\widehat{R_0})}{\widehat{T}}
  \end{equation}
  \item Now that we have a starting value, $r^0$, calculate a residual,
  $\delta^0$, from equation~\eqref{eq:lotkaH}:
  \begin{equation}
  \delta^i = 1 - \int _{a=0}^\infty \int _{a'=0}^\infty H(\varsigma^i p_a^m
  e^{-r^ia}, (1-\varsigma^i)p_{a'}^fe^{-r^ia'}) F_{a,a'}^H \dd a' \dd a
  \end{equation}
  \item Use $\delta^i$ to improve the estimate of $r$, $r^{i+1}$:
  \begin{equation}
  r^{i+1} = r^i - \frac{\delta^i}{\widehat{T} - \frac{\delta^i}{r^i}}
  \end{equation}
  \item Use the improved estimate of $r$ to update $\varsigma$:
  \begin{align}
  S^{i+1} &= \frac{\int_{a=o}^\infty \int_{a'=0}^\infty H(\varsigma^i
  e^{-r^{i+1}a} p_a^m, (1-\varsigma^i)^i e^{-r^{i+1}a'}p_{a'}^f) F_{a,a'}^{mH} \dd a' \dd a
  }{\int_{a=o}^\infty \int_{a'=0}^\infty H(\varsigma^i e^{-r^{i+1}a}
  p_a^m, (1-\varsigma^i)^i e^{-r^{i+1}a'}p_{a'}^f) F_{a,a'}^{fH} \dd a' \dd a }
  \\
  \varsigma^{i+1} &= \frac{S^{i+1}}{S^{i+1}+1}
  \end{align}
  \item Plug the new $\varsigma$ and $r$ estimates into step 5, to estimate a
  new residual, $\delta$, repeating steps 6-8 until $\delta$ vanishes to 0.
  Typicaly around 20 iterations are needed in order to reduce $\delta$ to
  be less than double floating point machine tolerance.
\end{enumerate}

This iterative procedure simultaneously produces an estimate of the stable
sex ratio at birth $S$ and the both-sex intrinsic growth rate, $r$. Really,
there is little room for $S$ to move between the initial and stable states,
since boy and girl births are in essence produced by (the harmonic mean of) both
males and females in this procedure. $S$
will only vary from the initial sex ratio at birth to the extent that there is
both an age pattern to the sex ratio at birth and the male and female stable age
structures differ from the initial age structures. Estimating both parameters at
the same time does not present a practical problem in the present case, and the
procedure converges faster than if $S$ is left assumed at some constant value.

One could abandon the iterative $r$ estimation procedure outlined above
and perform a standard cohort component projection, for instance using a
two-sex Leslie matrix. In this case, the fertility component of the Leslie
matrix would need to be updated between each iteration using equation~\ref{eq:asfrH} for either
males or females. One cannot easily perform standard matrix analysis of this
Leslie matrix, however, as it is not static in the standard way.

\paragraph{Other stable quantities: } Once one has identified the stable $r$ and
$S$, one may move on to estimate other stable parameters of interest, such as the 
both-sex stable birth rate, $b$:

\begin{equation}
b = \frac{1}{\int_{a = 0}^\infty e^{-ra} \varsigma p_a^m \dd a + \int_{a' =
0}^\infty e^{-ra'} \varsigma p_{a'}^f \dd a'}
\end{equation}
which may be used to calculate the male and female stable age structures, $c_a$
and $c_{a'}$:

\begin{equation}
c_a =  \varsigma  e^{-ra} p_a^m
\end{equation}
and analagously for females, where
\begin{equation}
1 = \int c_a + \int c_{a'}
\end{equation}
and the total population sex ratio, $S^{tot}$ is the ratio of these:
\begin{equation}
S^{tot} = \frac{\int c_a}{\int c_{a'}}
\end{equation}

\paragraph{Summary of the harmonic mean method: } The stable system outline here
is not taken word-for-word from Schoen's advice, but it is consistent with the 
notion of a constant \textit{force of attraction},
$F_{a,a'}^H$, and non-linear balancing of fertility rates based on the harmonic
mean of male and female exposures. The method presented here is only partially
sensitive across all ages to changes in the exposure of a single age in one sex.
That is to say, an increase in males of age $a$ will increase observed fertility rates for all ages
of females that share rates with males of age $a$. Further, females with
higher rates, $F_{a,a'}^H$, will typically observe greater increases, though this
depends on the distribution within $F^H$ and on relative exposure levels.
Lacking from this implementation are decreases in rates for males whose ages are close
to $a$, so-called spillover effects\citep{choo2006estimating}. That is to say,
an increase in age $a$ males, will not affect rates of males age $a-n$ or $a+n$, 
despite the fact that the pool of potential mates, females over
all ages $a'$, is shared. One would expect, ceteris paribus, that males of
similar ages would experience a decrease in rates, since some proportion of the
female pool will have been redirected to the increased stock of age $a$ males.
Hence, the model lacks this sense of competition. All other axioms appear to be
satisfied, except for that of bracketing, which we also deem superfluous.
Further, the harmonic mean is biased toward the minority sex, which is also intuitive.
 As stated before, one cannot empirically establish (for
humans) the ideal functional form of the fertility (marriage) function.

One satisfying property of the present method is that the harmonic mean
rates do not respond rigidly to mismatched population sizes between males and
females, but rather the mean rate is sensitive to relative size of male and
female stocks. In this way, the function is more dynamic than a weighted mean,
or Das Gupta's method presented in the previous section. Indeed, if the
demographer is not satisfied with the elasticity of the harmonic mean, one may
change $H()$ for any mean function, such as a generalized mean. An infinite number of other means
will also have the same desirable properties as the harmonic mean, such as 
dropping to 0 in the absence of one sex. Most means with this property that have
names (harmonic, geometric, logorithmic, \ldots) will produce almost
indistinguishably similar results. All such mean solutions will be symmetric
(blind) with respect to the sexes, although one could easily include weights.

\paragraph{The method applied to the US and Spanish data: }
In addittion to the harmonic mean, we have produced estimates of $r$ using the
geometric and logorithmic means, as well as the minimum function.
Figure~\ref{fig:schoenr} shows only the results of the harmonic mean and minimum
functions, as the geometric and logorithmic $r$ estimates would not be visually
distinguishable from those of the harmonic mean. From this lesson, we confirm
that if one is to use a mean function as a 2-sex fertility (marriage) function,
it really makes little difference which mean function one chooses, as long as it
satisfies the availability condition. The minimum function yields the least
consistent results, sometimes greater than the harmonic mean, sometimes less
than the harmonic mean, sometimes bracketed by the single-sex $r$ values, and
sometimes not. We note that the minimum function deviates the greatest from the
single-sex $r$ values when the sex-gap is trivial, and in these instances it is
always higher. The harmonic mean series is here always bracketed by the
single-sex $r$ values, although this is not a necessary result.

\begin{figure}[ht!]
        \centering  
          \caption{$r$ according to harmonic mean and minimum fertility
          functions compared with single sex intrinsic growth rates. US,
          1969-2009, and Spain, 1975-2009}
           % /R/Schoen1981.R
           \includegraphics{Figures/HMager}
          \label{fig:schoenr}
\end{figure}

In terms of complexity of implementation, solutions based on mean functions are
marginally less demanding than the Das Gupta solution, but this is primarily
because mean functions are more readily understood. The mean solution is seen as
onceptually simpler, yet yielding similar results and with more desirable
properties than either of the preceeding solutions. In following, we will
present two iterative fertility functions that allow for competition between age-groups of the same sex.

\FloatBarrier

\subsection{Weighted Means}
