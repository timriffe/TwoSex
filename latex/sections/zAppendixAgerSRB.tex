
\FloatBarrier
\chapter{Results from age-structured renewal models: $r$ and SRB.}
\label{appendix:ageallrestimates}

This appendix provides numerical results from all age-structured two-sex
methods treated in this dissertation applied to the US and Spanish populations
for the years 1969-2009 and 1975-2009, respectively. The two results to be listed are the intrinsic growth
rate, $r$, and the intrinsic sex ratio at birth, $S$, which for some
methods strays slightly from the initial value of the sex ratio at birth due to
changes in population structure between the initial and stable states and our inclusion of an
age-pattern to the sex ratio at birth for males and females via sex-of-birth
specific fertility rates. These results are placed into four tables, first $r$
for the US (Table~\ref{tab:USageALL}), followed by $S$
(Table~\ref{tab:USageSRBALL}) for the US, then $r$ and $S$ for the Spanish
population (Tables~\ref{tab:ESageALL} and~\ref{tab:ESageSRBALL}). Throughout, we
use superscripts in the column headers to identify the model according to the following key.

\begin{description}
  \item[$r^m$] Equation~\eqref{eq:lotkaeq} using single-sex male fertility and
  survival.
  \item[$r^f$] Equation~\eqref{eq:lotkaeq} using single-sex female fertility and
  survival. This is the standard Lotka result.
  \item[$r^{Pollard}$] Section~\ref{sec:pollardage} two-sex (mixed single-sex
  fertility) $r$. This method does not optimize the sex ratio at birth alongside $r$.
  \item[$r^{Mitra}$] Section~\ref{sec:mitraage}. two-sex $r$. This method
  assumes a constant sex ratio at birth.
  \item[$r^{\sigma=1}$] Section~\ref{sec:googmanage} weighted-dominance
  method with 100\% male information. This is identical to the single-sex male rate.
  \item[$r^{\sigma=0}$] Section~\ref{sec:googmanage} weighted-dominance
  method with 100\% female information. This is identical to the single-sex female rate.
  \item[$r^{\sigma=0.5}$] Section~\ref{sec:googmanage} weighted-dominance
  method with information split 50-50 between males and females.
  \item[$r^{Gupta}$] Section~\ref{sec:dasgupta} two-sex $r$, based on
  \citet{gupta1978alternative}.
  \item[$r^{HM}$] Section~\ref{sec:ageharmonic} mean method on the basis of
  the harmonic mean.
  \item[$r^{GM}$] Section~\ref{sec:ageharmonic} mean method on the basis of
  the geometric mean.
  \item[$r^{LM}$] Section~\ref{sec:ageharmonic} mean method on the basis of
  the logarithmic mean.
  \item[$r^{min}$] Section~\ref{sec:ageharmonic} mean method on the basis
  of the minimum function.
  \item[$r^{IPF-HM}$] Section~\ref{sec:IPF} with male and female marginal
  birth predictions balanced by the harmonic mean prior to re-estimating rates
  using iterative proportional fitting.
\end{description}

The same superscripts are used for stable sex ratios as birth, where $S(t)$
simply refers to the observed sex ratio at birth for the given year. Not
all methods optimize $S$ aloside $r$. Results with full digit precision are
available by executing the accompanying \texttt{R} code. Such precision 
should not give a false sense of exactitude, however, but
serves only for verification when reproducing results. These estimates were
arrived at by following the step-by-step instructions outlined in the text.
Notably, as mentioned in the text, the sex ratio at birth, $S$, does not vary
greatly between the initial and stable states, typically varying between
methods only in the \nth{5} digit. This should put the reader at ease if
questioning the stability of optimizing two parameters simultaneously. One can
verify that the single-sex models are degenerate cases of the Goodman model when
$\sigma$ is set to 0 or 1. Mean-based models produce very similar results
(except for the minimum function). We suggest executing the \texttt{R} code for
more detailed comparisons of these.

\begin{landscape}
\begin{table}
  \begin{adjustwidth}{-1in}{-.5in}
  \centering
    \caption{Intrinsic growth rates, $r$, from age-structured renewal models.
    US, 1969-2009.}
    \label{tab:USageALL}
        \scriptsize{% latex table generated in R 2.15.3 by xtable 1.7-1 package
% Fri May 10 09:57:43 2013
\begin{tabular}{cccccccccccccc}
  \hline
 & $r^m$ & $r^f$ & $r^{Pollard}$ & $r^{Mitra}$ & $r^{(\sigma = 1)}$ & $r^{(\sigma = 0)}$ & $r^{(\sigma = 0.5)}$ & $r^{Gupta}$ & $r^{HM}$ & $r^{GM}$ & $r^{LM}$ & $r^{min}$ & $r^{IPF-HM}$ \\ 
  \hline
1969 & 0.0100 & 0.0056 & 0.0079 & 0.0077 & 0.0100 & 0.0056 & 0.0079 & 0.0079 & 0.0080 & 0.0078 & 0.0078 & 0.0097 & 0.0078 \\ 
  1970 & 0.0100 & 0.0057 & 0.0080 & 0.0078 & 0.0100 & 0.0057 & 0.0081 & 0.0080 & 0.0081 & 0.0080 & 0.0079 & 0.0097 & 0.0079 \\ 
  1971 & 0.0070 & 0.0027 & 0.0050 & 0.0047 & 0.0070 & 0.0027 & 0.0050 & 0.0051 & 0.0050 & 0.0049 & 0.0049 & 0.0067 & 0.0049 \\ 
  1972 & 0.0026 & -0.0019 & 0.0005 & 0.0002 & 0.0026 & -0.0019 & 0.0005 & 0.0007 & 0.0005 & 0.0004 & 0.0004 & 0.0022 & 0.0004 \\ 
  1973 & 0.0001 & -0.0046 & -0.0021 & -0.0024 & 0.0001 & -0.0046 & -0.0021 & -0.0018 & -0.0021 & -0.0022 & -0.0023 & -0.0005 & -0.0022 \\ 
  1974 & -0.0007 & -0.0055 & -0.0030 & -0.0032 & -0.0007 & -0.0055 & -0.0029 & -0.0026 & -0.0029 & -0.0031 & -0.0031 & -0.0016 & -0.0031 \\ 
  1975 & -0.0019 & -0.0067 & -0.0042 & -0.0044 & -0.0019 & -0.0067 & -0.0041 & -0.0038 & -0.0041 & -0.0042 & -0.0043 & -0.0029 & -0.0042 \\ 
  1976 & -0.0025 & -0.0073 & -0.0048 & -0.0050 & -0.0025 & -0.0073 & -0.0047 & -0.0043 & -0.0048 & -0.0049 & -0.0049 & -0.0036 & -0.0049 \\ 
  1977 & -0.0019 & -0.0063 & -0.0040 & -0.0042 & -0.0019 & -0.0063 & -0.0039 & -0.0035 & -0.0039 & -0.0040 & -0.0041 & -0.0027 & -0.0040 \\ 
  1978 & -0.0026 & -0.0071 & -0.0047 & -0.0050 & -0.0026 & -0.0071 & -0.0047 & -0.0043 & -0.0047 & -0.0048 & -0.0048 & -0.0036 & -0.0048 \\ 
  1979 & -0.0019 & -0.0060 & -0.0038 & -0.0041 & -0.0019 & -0.0060 & -0.0037 & -0.0033 & -0.0038 & -0.0039 & -0.0039 & -0.0027 & -0.0039 \\ 
  1980 & -0.0016 & -0.0054 & -0.0034 & -0.0036 & -0.0016 & -0.0054 & -0.0033 & -0.0029 & -0.0034 & -0.0035 & -0.0035 & -0.0024 & -0.0034 \\ 
  1981 & -0.0020 & -0.0057 & -0.0037 & -0.0040 & -0.0020 & -0.0057 & -0.0037 & -0.0033 & -0.0037 & -0.0038 & -0.0039 & -0.0028 & -0.0038 \\ 
  1982 & -0.0021 & -0.0054 & -0.0037 & -0.0039 & -0.0021 & -0.0054 & -0.0036 & -0.0032 & -0.0037 & -0.0038 & -0.0038 & -0.0028 & -0.0037 \\ 
  1983 & -0.0029 & -0.0061 & -0.0044 & -0.0046 & -0.0029 & -0.0061 & -0.0044 & -0.0039 & -0.0044 & -0.0045 & -0.0045 & -0.0036 & -0.0044 \\ 
  1984 & -0.0030 & -0.0058 & -0.0043 & -0.0046 & -0.0030 & -0.0058 & -0.0043 & -0.0039 & -0.0043 & -0.0044 & -0.0044 & -0.0036 & -0.0044 \\ 
  1985 & -0.0025 & -0.0049 & -0.0037 & -0.0039 & -0.0025 & -0.0049 & -0.0036 & -0.0032 & -0.0037 & -0.0037 & -0.0038 & -0.0029 & -0.0037 \\ 
  1986 & -0.0029 & -0.0049 & -0.0038 & -0.0041 & -0.0029 & -0.0049 & -0.0038 & -0.0034 & -0.0038 & -0.0039 & -0.0039 & -0.0031 & -0.0038 \\ 
  1987 & -0.0026 & -0.0043 & -0.0034 & -0.0036 & -0.0026 & -0.0043 & -0.0034 & -0.0030 & -0.0034 & -0.0034 & -0.0035 & -0.0026 & -0.0034 \\ 
  1988 & -0.0018 & -0.0031 & -0.0024 & -0.0027 & -0.0018 & -0.0031 & -0.0024 & -0.0021 & -0.0024 & -0.0025 & -0.0025 & -0.0016 & -0.0024 \\ 
  1989 & -0.0007 & -0.0016 & -0.0012 & -0.0014 & -0.0007 & -0.0016 & -0.0012 & -0.0009 & -0.0011 & -0.0012 & -0.0012 & -0.0003 & -0.0012 \\ 
  1990 & 0.0002 & -0.0003 & -0.0000 & -0.0003 & 0.0002 & -0.0003 & -0.0000 & 0.0002 & -0.0000 & -0.0001 & -0.0001 & 0.0009 & -0.0000 \\ 
  1991 & -0.0002 & -0.0004 & -0.0003 & -0.0006 & -0.0002 & -0.0004 & -0.0003 & -0.0001 & -0.0003 & -0.0004 & -0.0004 & 0.0006 & -0.0003 \\ 
  1992 & -0.0006 & -0.0007 & -0.0007 & -0.0009 & -0.0006 & -0.0007 & -0.0007 & -0.0005 & -0.0007 & -0.0007 & -0.0007 & 0.0002 & -0.0007 \\ 
  1993 & -0.0012 & -0.0012 & -0.0012 & -0.0014 & -0.0012 & -0.0012 & -0.0012 & -0.0011 & -0.0012 & -0.0012 & -0.0012 & -0.0004 & -0.0012 \\ 
  1994 & -0.0015 & -0.0014 & -0.0015 & -0.0017 & -0.0015 & -0.0014 & -0.0015 & -0.0014 & -0.0015 & -0.0015 & -0.0015 & -0.0007 & -0.0015 \\ 
  1995 & -0.0019 & -0.0018 & -0.0018 & -0.0021 & -0.0019 & -0.0018 & -0.0018 & -0.0018 & -0.0018 & -0.0019 & -0.0019 & -0.0011 & -0.0018 \\ 
  1996 & -0.0018 & -0.0018 & -0.0018 & -0.0020 & -0.0018 & -0.0018 & -0.0018 & -0.0018 & -0.0018 & -0.0019 & -0.0019 & -0.0011 & -0.0018 \\ 
  1997 & -0.0018 & -0.0019 & -0.0019 & -0.0021 & -0.0018 & -0.0019 & -0.0019 & -0.0019 & -0.0019 & -0.0019 & -0.0019 & -0.0013 & -0.0019 \\ 
  1998 & -0.0013 & -0.0014 & -0.0013 & -0.0015 & -0.0013 & -0.0014 & -0.0013 & -0.0014 & -0.0013 & -0.0014 & -0.0014 & -0.0008 & -0.0013 \\ 
  1999 & -0.0011 & -0.0013 & -0.0012 & -0.0014 & -0.0011 & -0.0013 & -0.0012 & -0.0013 & -0.0012 & -0.0012 & -0.0012 & -0.0007 & -0.0012 \\ 
  2000 & -0.0004 & -0.0004 & -0.0004 & -0.0006 & -0.0004 & -0.0004 & -0.0004 & -0.0005 & -0.0004 & -0.0004 & -0.0004 & 0.0001 & -0.0004 \\ 
  2001 & -0.0008 & -0.0008 & -0.0008 & -0.0010 & -0.0008 & -0.0008 & -0.0008 & -0.0008 & -0.0008 & -0.0008 & -0.0008 & -0.0003 & -0.0008 \\ 
  2002 & -0.0007 & -0.0009 & -0.0008 & -0.0010 & -0.0007 & -0.0009 & -0.0008 & -0.0009 & -0.0008 & -0.0008 & -0.0009 & -0.0005 & -0.0008 \\ 
  2003 & -0.0002 & -0.0004 & -0.0003 & -0.0005 & -0.0002 & -0.0004 & -0.0003 & -0.0004 & -0.0003 & -0.0003 & -0.0003 & 0.0001 & -0.0003 \\ 
  2004 & 0.0000 & -0.0003 & -0.0002 & -0.0003 & 0.0000 & -0.0003 & -0.0002 & -0.0003 & -0.0002 & -0.0002 & -0.0002 & 0.0002 & -0.0002 \\ 
  2005 & 0.0001 & -0.0003 & -0.0001 & -0.0003 & 0.0001 & -0.0003 & -0.0001 & -0.0002 & -0.0001 & -0.0001 & -0.0001 & 0.0002 & -0.0001 \\ 
  2006 & 0.0010 & 0.0006 & 0.0008 & 0.0006 & 0.0010 & 0.0006 & 0.0008 & 0.0007 & 0.0008 & 0.0008 & 0.0008 & 0.0011 & 0.0008 \\ 
  2007 & 0.0012 & 0.0008 & 0.0010 & 0.0008 & 0.0012 & 0.0008 & 0.0010 & 0.0009 & 0.0010 & 0.0010 & 0.0010 & 0.0013 & 0.0010 \\ 
  2008 & 0.0006 & -0.0000 & 0.0003 & 0.0001 & 0.0006 & -0.0000 & 0.0003 & 0.0002 & 0.0003 & 0.0003 & 0.0003 & 0.0005 & 0.0003 \\ 
  2009 & -0.0004 & -0.0012 & -0.0008 & -0.0010 & -0.0004 & -0.0012 & -0.0008 & -0.0009 & -0.0008 & -0.0008 & -0.0008 & -0.0008 & -0.0008 \\ 
   \hline
\end{tabular}
}
  \end{adjustwidth}
\end{table}

\end{landscape}
\begin{landscape}
\begin{table}
  \begin{adjustwidth}{-1in}{-.5in}
  \centering
    \caption{Stable sex ratio at birth, $S$, from age-structured renewal
    models. US, 1969-2009.}
    \label{tab:USageSRBALL}
        \scriptsize{% latex table generated in R 2.15.3 by xtable 1.7-1 package
% Fri May 10 10:23:40 2013
\begin{tabular}{cccccccccc}
  \hline
 & $S(t)$ & $S^{(\sigma = 1)}$ & $S^{(\sigma = 0)}$ & $S^{(\sigma = 0.5)}$ & $S^{HM}$ & $S^{GM}$ & $S^{LM}$ & $S^{min}$ & $S^{IPF-HM}$ \\ 
  \hline
1969 & 1.05300 & 1.05273 & 1.05252 & 1.05262 & 1.05262 & 1.05262 & 1.05262 & 1.05268 & 1.05261 \\ 
  1970 & 1.05468 & 1.05442 & 1.05426 & 1.05434 & 1.05433 & 1.05434 & 1.05434 & 1.05436 & 1.05433 \\ 
  1971 & 1.05182 & 1.05157 & 1.05160 & 1.05158 & 1.05158 & 1.05159 & 1.05159 & 1.05155 & 1.05159 \\ 
  1972 & 1.05121 & 1.05071 & 1.05071 & 1.05070 & 1.05070 & 1.05070 & 1.05070 & 1.05066 & 1.05070 \\ 
  1973 & 1.05213 & 1.05165 & 1.05153 & 1.05159 & 1.05158 & 1.05159 & 1.05159 & 1.05161 & 1.05158 \\ 
  1974 & 1.05484 & 1.05433 & 1.05460 & 1.05445 & 1.05446 & 1.05446 & 1.05446 & 1.05438 & 1.05447 \\ 
  1975 & 1.05370 & 1.05285 & 1.05279 & 1.05280 & 1.05281 & 1.05281 & 1.05281 & 1.05277 & 1.05281 \\ 
  1976 & 1.05250 & 1.05242 & 1.05245 & 1.05243 & 1.05244 & 1.05243 & 1.05243 & 1.05244 & 1.05243 \\ 
  1977 & 1.05263 & 1.05203 & 1.05209 & 1.05206 & 1.05205 & 1.05206 & 1.05207 & 1.05205 & 1.05206 \\ 
  1978 & 1.05267 & 1.05243 & 1.05260 & 1.05250 & 1.05250 & 1.05251 & 1.05252 & 1.05243 & 1.05251 \\ 
  1979 & 1.05166 & 1.05134 & 1.05170 & 1.05150 & 1.05149 & 1.05151 & 1.05152 & 1.05139 & 1.05152 \\ 
  1980 & 1.05281 & 1.05250 & 1.05260 & 1.05255 & 1.05254 & 1.05255 & 1.05255 & 1.05252 & 1.05255 \\ 
  1981 & 1.05160 & 1.05094 & 1.05118 & 1.05105 & 1.05105 & 1.05107 & 1.05108 & 1.05088 & 1.05106 \\ 
  1982 & 1.05062 & 1.05047 & 1.05058 & 1.05051 & 1.05051 & 1.05052 & 1.05052 & 1.05048 & 1.05052 \\ 
  1983 & 1.05195 & 1.05164 & 1.05199 & 1.05180 & 1.05181 & 1.05181 & 1.05181 & 1.05179 & 1.05182 \\ 
  1984 & 1.05021 & 1.05003 & 1.05008 & 1.05006 & 1.05005 & 1.05005 & 1.05005 & 1.05003 & 1.05006 \\ 
  1985 & 1.05206 & 1.05191 & 1.05202 & 1.05196 & 1.05196 & 1.05197 & 1.05198 & 1.05183 & 1.05196 \\ 
  1986 & 1.05087 & 1.05065 & 1.05079 & 1.05071 & 1.05072 & 1.05072 & 1.05072 & 1.05065 & 1.05072 \\ 
  1987 & 1.04999 & 1.04974 & 1.05003 & 1.04988 & 1.04989 & 1.04988 & 1.04988 & 1.04983 & 1.04988 \\ 
  1988 & 1.04995 & 1.04999 & 1.05000 & 1.04999 & 1.04999 & 1.05000 & 1.05001 & 1.04990 & 1.04999 \\ 
  1989 & 1.04979 & 1.04970 & 1.04969 & 1.04970 & 1.04969 & 1.04971 & 1.04972 & 1.04958 & 1.04970 \\ 
  1990 & 1.04972 & 1.04985 & 1.04980 & 1.04983 & 1.04982 & 1.04983 & 1.04983 & 1.04985 & 1.04983 \\ 
  1991 & 1.04580 & 1.04581 & 1.04582 & 1.04581 & 1.04581 & 1.04582 & 1.04582 & 1.04573 & 1.04581 \\ 
  1992 & 1.04997 & 1.04998 & 1.04987 & 1.04992 & 1.04992 & 1.04993 & 1.04993 & 1.04984 & 1.04992 \\ 
  1993 & 1.04997 & 1.04988 & 1.05005 & 1.04997 & 1.04996 & 1.04998 & 1.04999 & 1.04987 & 1.04997 \\ 
  1994 & 1.04785 & 1.04796 & 1.04782 & 1.04789 & 1.04788 & 1.04789 & 1.04789 & 1.04785 & 1.04789 \\ 
  1995 & 1.04897 & 1.04926 & 1.04910 & 1.04918 & 1.04918 & 1.04918 & 1.04918 & 1.04920 & 1.04918 \\ 
  1996 & 1.04707 & 1.04712 & 1.04706 & 1.04709 & 1.04710 & 1.04709 & 1.04709 & 1.04707 & 1.04709 \\ 
  1997 & 1.04769 & 1.04770 & 1.04773 & 1.04771 & 1.04771 & 1.04772 & 1.04772 & 1.04766 & 1.04771 \\ 
  1998 & 1.04720 & 1.04734 & 1.04719 & 1.04726 & 1.04726 & 1.04727 & 1.04727 & 1.04720 & 1.04726 \\ 
  1999 & 1.04882 & 1.04881 & 1.04888 & 1.04884 & 1.04885 & 1.04885 & 1.04885 & 1.04884 & 1.04885 \\ 
  2000 & 1.04803 & 1.04792 & 1.04805 & 1.04798 & 1.04798 & 1.04799 & 1.04799 & 1.04798 & 1.04798 \\ 
  2001 & 1.04567 & 1.04564 & 1.04572 & 1.04568 & 1.04568 & 1.04568 & 1.04569 & 1.04565 & 1.04568 \\ 
  2002 & 1.04797 & 1.04814 & 1.04805 & 1.04810 & 1.04810 & 1.04809 & 1.04809 & 1.04812 & 1.04810 \\ 
  2003 & 1.04867 & 1.04857 & 1.04867 & 1.04862 & 1.04862 & 1.04862 & 1.04862 & 1.04861 & 1.04862 \\ 
  2004 & 1.04848 & 1.04847 & 1.04854 & 1.04851 & 1.04851 & 1.04851 & 1.04851 & 1.04847 & 1.04851 \\ 
  2005 & 1.04935 & 1.04935 & 1.04942 & 1.04939 & 1.04939 & 1.04939 & 1.04939 & 1.04939 & 1.04939 \\ 
  2006 & 1.04955 & 1.04960 & 1.04964 & 1.04962 & 1.04962 & 1.04962 & 1.04962 & 1.04960 & 1.04962 \\ 
  2007 & 1.04746 & 1.04739 & 1.04744 & 1.04741 & 1.04741 & 1.04741 & 1.04741 & 1.04742 & 1.04741 \\ 
  2008 & 1.04779 & 1.04772 & 1.04783 & 1.04777 & 1.04777 & 1.04777 & 1.04777 & 1.04777 & 1.04777 \\ 
  2009 & 1.04816 & 1.04804 & 1.04811 & 1.04807 & 1.04807 & 1.04808 & 1.04808 & 1.04803 & 1.04808 \\ 
   \hline
\end{tabular}
}
  \end{adjustwidth}
\end{table}
\end{landscape}

\begin{landscape}
\begin{table}
  \begin{adjustwidth}{-1in}{-.5in}
  \centering
    \caption{Intrinsic growth rates, $r$, from age-structured renewal models.
    Spain, 1975-2009.}
    \label{tab:ESageALL}
        \scriptsize{% latex table generated in R 2.15.3 by xtable 1.7-1 package
% Fri May 10 09:57:43 2013
\begin{tabular}{cccccccccccccc}
  \hline
 & $r^m$ & $r^f$ & $r^{Pollard}$ & $r^{Mitra}$ & $r^{(\sigma = 1)}$ & $r^{(\sigma = 0)}$ & $r^{(\sigma = 0.5)}$ & $r^{Gupta}$ & $r^{HM}$ & $r^{GM}$ & $r^{LM}$ & $r^{min}$ & $r^{IPF-HM}$ \\ 
  \hline
1975 & 0.0107 & 0.0092 & 0.0100 & 0.0098 & 0.0107 & 0.0092 & 0.0100 & 0.0099 & 0.0100 & 0.0099 & 0.0099 & 0.0105 & 0.0100 \\ 
  1976 & 0.0107 & 0.0095 & 0.0101 & 0.0100 & 0.0107 & 0.0095 & 0.0101 & 0.0101 & 0.0101 & 0.0101 & 0.0101 & 0.0106 & 0.0101 \\ 
  1977 & 0.0095 & 0.0079 & 0.0088 & 0.0086 & 0.0095 & 0.0079 & 0.0088 & 0.0087 & 0.0088 & 0.0087 & 0.0087 & 0.0092 & 0.0088 \\ 
  1978 & 0.0083 & 0.0063 & 0.0074 & 0.0073 & 0.0083 & 0.0063 & 0.0074 & 0.0074 & 0.0074 & 0.0074 & 0.0074 & 0.0078 & 0.0074 \\ 
  1979 & 0.0062 & 0.0040 & 0.0051 & 0.0050 & 0.0062 & 0.0040 & 0.0052 & 0.0051 & 0.0051 & 0.0051 & 0.0051 & 0.0056 & 0.0051 \\ 
  1980 & 0.0044 & 0.0016 & 0.0030 & 0.0029 & 0.0044 & 0.0016 & 0.0031 & 0.0030 & 0.0030 & 0.0030 & 0.0030 & 0.0032 & 0.0030 \\ 
  1981 & 0.0020 & -0.0016 & 0.0003 & 0.0002 & 0.0020 & -0.0016 & 0.0003 & 0.0003 & 0.0003 & 0.0002 & 0.0002 & 0.0001 & 0.0002 \\ 
  1982 & 0.0005 & -0.0032 & -0.0013 & -0.0013 & 0.0005 & -0.0032 & -0.0012 & -0.0012 & -0.0013 & -0.0013 & -0.0013 & -0.0014 & -0.0013 \\ 
  1983 & -0.0021 & -0.0057 & -0.0038 & -0.0039 & -0.0021 & -0.0057 & -0.0038 & -0.0037 & -0.0038 & -0.0038 & -0.0039 & -0.0039 & -0.0039 \\ 
  1984 & -0.0032 & -0.0072 & -0.0051 & -0.0052 & -0.0032 & -0.0072 & -0.0051 & -0.0050 & -0.0051 & -0.0052 & -0.0052 & -0.0054 & -0.0052 \\ 
  1985 & -0.0050 & -0.0088 & -0.0068 & -0.0069 & -0.0050 & -0.0088 & -0.0067 & -0.0066 & -0.0068 & -0.0068 & -0.0068 & -0.0070 & -0.0069 \\ 
  1986 & -0.0067 & -0.0106 & -0.0086 & -0.0087 & -0.0067 & -0.0106 & -0.0085 & -0.0084 & -0.0086 & -0.0086 & -0.0086 & -0.0089 & -0.0086 \\ 
  1987 & -0.0081 & -0.0120 & -0.0100 & -0.0101 & -0.0081 & -0.0120 & -0.0099 & -0.0097 & -0.0100 & -0.0100 & -0.0100 & -0.0104 & -0.0100 \\ 
  1988 & -0.0092 & -0.0130 & -0.0110 & -0.0112 & -0.0092 & -0.0130 & -0.0110 & -0.0108 & -0.0111 & -0.0111 & -0.0111 & -0.0114 & -0.0111 \\ 
  1989 & -0.0104 & -0.0142 & -0.0122 & -0.0124 & -0.0104 & -0.0142 & -0.0122 & -0.0119 & -0.0123 & -0.0123 & -0.0123 & -0.0126 & -0.0123 \\ 
  1990 & -0.0113 & -0.0150 & -0.0131 & -0.0133 & -0.0113 & -0.0150 & -0.0130 & -0.0128 & -0.0131 & -0.0131 & -0.0131 & -0.0136 & -0.0131 \\ 
  1991 & -0.0121 & -0.0157 & -0.0138 & -0.0141 & -0.0121 & -0.0157 & -0.0138 & -0.0135 & -0.0139 & -0.0139 & -0.0139 & -0.0144 & -0.0139 \\ 
  1992 & -0.0124 & -0.0158 & -0.0141 & -0.0143 & -0.0124 & -0.0158 & -0.0140 & -0.0137 & -0.0141 & -0.0141 & -0.0141 & -0.0146 & -0.0141 \\ 
  1993 & -0.0135 & -0.0171 & -0.0152 & -0.0154 & -0.0135 & -0.0171 & -0.0152 & -0.0149 & -0.0153 & -0.0153 & -0.0153 & -0.0159 & -0.0153 \\ 
  1994 & -0.0150 & -0.0185 & -0.0167 & -0.0169 & -0.0150 & -0.0185 & -0.0166 & -0.0163 & -0.0168 & -0.0168 & -0.0167 & -0.0174 & -0.0167 \\ 
  1995 & -0.0159 & -0.0192 & -0.0174 & -0.0177 & -0.0159 & -0.0192 & -0.0174 & -0.0171 & -0.0175 & -0.0175 & -0.0175 & -0.0182 & -0.0175 \\ 
  1996 & -0.0162 & -0.0193 & -0.0176 & -0.0179 & -0.0162 & -0.0193 & -0.0176 & -0.0173 & -0.0177 & -0.0177 & -0.0177 & -0.0184 & -0.0177 \\ 
  1997 & -0.0157 & -0.0188 & -0.0172 & -0.0174 & -0.0157 & -0.0188 & -0.0171 & -0.0168 & -0.0173 & -0.0172 & -0.0172 & -0.0180 & -0.0172 \\ 
  1998 & -0.0161 & -0.0194 & -0.0176 & -0.0178 & -0.0161 & -0.0194 & -0.0176 & -0.0173 & -0.0177 & -0.0177 & -0.0177 & -0.0186 & -0.0177 \\ 
  1999 & -0.0153 & -0.0181 & -0.0166 & -0.0168 & -0.0153 & -0.0181 & -0.0166 & -0.0163 & -0.0167 & -0.0167 & -0.0167 & -0.0174 & -0.0166 \\ 
  2000 & -0.0143 & -0.0171 & -0.0156 & -0.0158 & -0.0143 & -0.0171 & -0.0156 & -0.0153 & -0.0157 & -0.0157 & -0.0157 & -0.0165 & -0.0157 \\ 
  2001 & -0.0144 & -0.0166 & -0.0154 & -0.0156 & -0.0144 & -0.0166 & -0.0154 & -0.0151 & -0.0155 & -0.0155 & -0.0155 & -0.0161 & -0.0155 \\ 
  2002 & -0.0141 & -0.0162 & -0.0151 & -0.0153 & -0.0141 & -0.0162 & -0.0151 & -0.0148 & -0.0152 & -0.0152 & -0.0152 & -0.0157 & -0.0151 \\ 
  2003 & -0.0133 & -0.0149 & -0.0141 & -0.0143 & -0.0133 & -0.0149 & -0.0141 & -0.0137 & -0.0141 & -0.0141 & -0.0141 & -0.0145 & -0.0141 \\ 
  2004 & -0.0130 & -0.0146 & -0.0138 & -0.0139 & -0.0130 & -0.0146 & -0.0137 & -0.0134 & -0.0138 & -0.0138 & -0.0138 & -0.0142 & -0.0138 \\ 
  2005 & -0.0130 & -0.0140 & -0.0135 & -0.0136 & -0.0130 & -0.0140 & -0.0135 & -0.0131 & -0.0135 & -0.0136 & -0.0136 & -0.0136 & -0.0135 \\ 
  2006 & -0.0125 & -0.0133 & -0.0129 & -0.0130 & -0.0125 & -0.0133 & -0.0129 & -0.0125 & -0.0129 & -0.0129 & -0.0129 & -0.0129 & -0.0129 \\ 
  2007 & -0.0125 & -0.0129 & -0.0127 & -0.0128 & -0.0125 & -0.0129 & -0.0127 & -0.0123 & -0.0127 & -0.0128 & -0.0128 & -0.0125 & -0.0127 \\ 
  2008 & -0.0112 & -0.0114 & -0.0113 & -0.0114 & -0.0112 & -0.0114 & -0.0113 & -0.0109 & -0.0113 & -0.0114 & -0.0114 & -0.0110 & -0.0113 \\ 
  2009 & -0.0124 & -0.0128 & -0.0126 & -0.0127 & -0.0124 & -0.0128 & -0.0126 & -0.0122 & -0.0126 & -0.0127 & -0.0127 & -0.0125 & -0.0126 \\ 
   \hline
\end{tabular}
}
  \end{adjustwidth}
\end{table}
\end{landscape}

\begin{landscape}
\begin{table}
  \begin{adjustwidth}{-1in}{-.5in}
  \centering
    \caption{Stable sex ratio at birth, $S$, from age-structured renewal
    models. Spain, 1975-2009.}
    \label{tab:ESageSRBALL}
        \footnotesize{% latex table generated in R 2.15.3 by xtable 1.7-1 package
% Fri May 10 10:23:40 2013
\begin{tabular}{cccccccccc}
  \hline
 & $S(t)$ & $S^{(\sigma = 1)}$ & $S^{(\sigma = 0)}$ & $S^{(\sigma = 0.5)}$ & $S^{HM}$ & $S^{GM}$ & $S^{LM}$ & $S^{min}$ & $S^{IPF-HM}$ \\ 
  \hline
1975 & 1.07243 & 1.07235 & 1.07249 & 1.07243 & 1.07243 & 1.07244 & 1.07244 & 1.07232 & 1.07243 \\ 
  1976 & 1.06401 & 1.06405 & 1.06388 & 1.06397 & 1.06397 & 1.06398 & 1.06398 & 1.06397 & 1.06397 \\ 
  1977 & 1.06886 & 1.06892 & 1.06861 & 1.06877 & 1.06877 & 1.06877 & 1.06877 & 1.06875 & 1.06876 \\ 
  1978 & 1.07380 & 1.07372 & 1.07341 & 1.07356 & 1.07356 & 1.07357 & 1.07357 & 1.07340 & 1.07355 \\ 
  1979 & 1.06812 & 1.06786 & 1.06783 & 1.06784 & 1.06784 & 1.06784 & 1.06785 & 1.06778 & 1.06784 \\ 
  1980 & 1.07799 & 1.07802 & 1.07784 & 1.07793 & 1.07793 & 1.07792 & 1.07792 & 1.07811 & 1.07793 \\ 
  1981 & 1.09160 & 1.09161 & 1.09192 & 1.09174 & 1.09175 & 1.09175 & 1.09176 & 1.09156 & 1.09176 \\ 
  1982 & 1.08731 & 1.08687 & 1.08716 & 1.08700 & 1.08700 & 1.08701 & 1.08701 & 1.08681 & 1.08701 \\ 
  1983 & 1.07622 & 1.07607 & 1.07591 & 1.07601 & 1.07600 & 1.07599 & 1.07599 & 1.07612 & 1.07600 \\ 
  1984 & 1.08283 & 1.08329 & 1.08306 & 1.08319 & 1.08319 & 1.08318 & 1.08318 & 1.08334 & 1.08318 \\ 
  1985 & 1.07343 & 1.07326 & 1.07313 & 1.07321 & 1.07320 & 1.07320 & 1.07320 & 1.07326 & 1.07320 \\ 
  1986 & 1.07374 & 1.07343 & 1.07294 & 1.07321 & 1.07320 & 1.07320 & 1.07320 & 1.07322 & 1.07318 \\ 
  1987 & 1.07695 & 1.07723 & 1.07689 & 1.07708 & 1.07707 & 1.07705 & 1.07705 & 1.07718 & 1.07705 \\ 
  1988 & 1.07168 & 1.07202 & 1.07185 & 1.07195 & 1.07194 & 1.07193 & 1.07193 & 1.07189 & 1.07194 \\ 
  1989 & 1.07082 & 1.06994 & 1.07056 & 1.07022 & 1.07025 & 1.07026 & 1.07026 & 1.07015 & 1.07026 \\ 
  1990 & 1.06995 & 1.06892 & 1.06950 & 1.06918 & 1.06919 & 1.06918 & 1.06918 & 1.06915 & 1.06921 \\ 
  1991 & 1.07204 & 1.07239 & 1.07239 & 1.07239 & 1.07239 & 1.07239 & 1.07239 & 1.07242 & 1.07239 \\ 
  1992 & 1.06618 & 1.06605 & 1.06628 & 1.06615 & 1.06617 & 1.06617 & 1.06617 & 1.06618 & 1.06616 \\ 
  1993 & 1.06989 & 1.07032 & 1.06956 & 1.06998 & 1.06994 & 1.06993 & 1.06993 & 1.06999 & 1.06994 \\ 
  1994 & 1.06679 & 1.06634 & 1.06625 & 1.06629 & 1.06629 & 1.06630 & 1.06630 & 1.06628 & 1.06629 \\ 
  1995 & 1.06434 & 1.06360 & 1.06446 & 1.06398 & 1.06402 & 1.06401 & 1.06401 & 1.06411 & 1.06402 \\ 
  1996 & 1.06122 & 1.06005 & 1.06090 & 1.06043 & 1.06050 & 1.06051 & 1.06051 & 1.06038 & 1.06048 \\ 
  1997 & 1.06254 & 1.06277 & 1.06290 & 1.06283 & 1.06284 & 1.06283 & 1.06283 & 1.06296 & 1.06283 \\ 
  1998 & 1.07265 & 1.07105 & 1.07185 & 1.07141 & 1.07145 & 1.07145 & 1.07145 & 1.07126 & 1.07146 \\ 
  1999 & 1.06158 & 1.06063 & 1.06130 & 1.06094 & 1.06099 & 1.06100 & 1.06101 & 1.06085 & 1.06097 \\ 
  2000 & 1.07061 & 1.07029 & 1.07033 & 1.07031 & 1.07033 & 1.07031 & 1.07030 & 1.07044 & 1.07031 \\ 
  2001 & 1.05665 & 1.05589 & 1.05629 & 1.05608 & 1.05611 & 1.05612 & 1.05613 & 1.05607 & 1.05609 \\ 
  2002 & 1.06480 & 1.06412 & 1.06491 & 1.06449 & 1.06454 & 1.06452 & 1.06452 & 1.06471 & 1.06451 \\ 
  2003 & 1.06200 & 1.06112 & 1.06179 & 1.06144 & 1.06147 & 1.06148 & 1.06148 & 1.06146 & 1.06146 \\ 
  2004 & 1.06899 & 1.06853 & 1.06917 & 1.06883 & 1.06888 & 1.06887 & 1.06887 & 1.06904 & 1.06885 \\ 
  2005 & 1.06204 & 1.06208 & 1.06175 & 1.06192 & 1.06190 & 1.06191 & 1.06191 & 1.06180 & 1.06191 \\ 
  2006 & 1.06592 & 1.06423 & 1.06567 & 1.06494 & 1.06500 & 1.06498 & 1.06498 & 1.06506 & 1.06496 \\ 
  2007 & 1.06396 & 1.06350 & 1.06398 & 1.06374 & 1.06376 & 1.06374 & 1.06374 & 1.06380 & 1.06374 \\ 
  2008 & 1.06752 & 1.06737 & 1.06778 & 1.06758 & 1.06761 & 1.06759 & 1.06759 & 1.06764 & 1.06758 \\ 
  2009 & 1.07074 & 1.07018 & 1.07030 & 1.07024 & 1.07025 & 1.07022 & 1.07021 & 1.07042 & 1.07024 \\ 
   \hline
\end{tabular}
}
  \end{adjustwidth}
\end{table}
\end{landscape}


\FloatBarrier