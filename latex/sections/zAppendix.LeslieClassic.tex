\chapter{Construction of the standard one-sex Leslie matrix}
\label{Appendix:Caswell}
The Leslie matrix\citep{leslie1945use} is a tool used for age-structured cohort 
component population projections. Here we offer a very abbreviated presentation of the elements that
correspond to a simple one-sex age-structured population. More details on 
each element presented here can be found in \cite{caswell2001matrix}.

Say we have a population with $n$ discrete age-classes. Call the vector of
age-specific population counts $\textbf{p}$. The Leslie matrix, $\textbf{L}$, is a $n \times
n$ square matrix that contains probabilities of survival from age $n$ to age
$n+1$ (assuming single age population counts) in the subdiagonal with age
advancing down and to the right. The top row contains age specific fertility
rates, discounted somewhat for those births in the year that do not survive
until the end of the year. The rest of the matrix consists in zeros, and the
lower right corner contains a zero too, in order to close out the lifetable and
not accumulate population in the open age group.

Once constructed, $\textbf{p}_t$ can be projected forward one year by matrix
multiplying $\textbf{L}$ from the left:

\begin{equation}
\textbf{p}_{t+1} = \textbf{L}\textbf{p}_{t} 
\end{equation}

Say we have a population with 6 age classes, where
$S_x$ denotes survival probabilities from age $x$ to age $x+1$, $f_x$ denotes
the exact fertility probability for age class $(x,x+1]$, which must also take
into account the fact that not all potential mothers (fathers) alive on January
\nth{1} of year $t$ will survive until December \nth{31}. $\lambda$ is a discount for the part of
infant mortality to births in year $t$ that do not survive to be counted as 
age 0 at the begining of year $t+1$. Assume we have fertility in ages 1-4 only.

\begin{matrix}
\centering
\caption{An example one-sex Leslie matrix with 6 age categories}
$\bordermatrix{{Age }      & 0_t & 1_t     & 2_t &  3_t & 4_t & 5_t\cr
                0_{t+1} & 0 & \lambda f_1 & \lambda f_2 & \lambda f_3 & \lambda f_4 & 0\cr 
                1_{t+1} & S_0  &  0       & 0    & 0     & 0   & 0   \cr
                2_{t+1} & 0    &  S_1     & 0    & 0     & 0    & 0   \cr 
                3_{t+1} & 0    &  0       & S_2  & 0     & 0    & 0   \cr 
                4_{t+1} & 0    &  0       & 0    & S_3   & 0    & 0   \cr
                5_{t+1} & 0    &  0       & 0    & 0     & S_4  & 0   }$
\end{matrix}

Roughly, and for the sake of intuition, imagine that the population aged $2$
in year $t$, $p_{2,t}$, enters the matrix from the top of the \nth{3} column from
the left $a2_t$. This population then travels down the column. When the
 population runs into a probability entry, these two values
are multiplied together, and the product exits the matrix at that exact row to
the left. As $p_{2,t}$ enters the matrix from the top, it first multiplies into $\lambda
f_2$, the product of which is the contribution to births surviving to year
$t+1$ of $p_{2,t}$, the new age 0 in year $t+1$. $p_{2,t}$ then continues down
the column, multiplying into $S_2$ in the \nth{4} row and exiting out the left
to age 3, discounted by the survival probability $S_2$.

In this way, $\textbf{L}$ contains all of the information necessary to project a
fixed set of rates forward any number of years, eventually arriving at the stable
age-structure of the population. Indeed, there are ways to extract the intrinsic
growth rate and stable age structure directly from the matrix without
carrying projection computations. For details on how to calculate the individual
matrix elements, please refer to \cite{caswell2001matrix}, as mentioned before.



