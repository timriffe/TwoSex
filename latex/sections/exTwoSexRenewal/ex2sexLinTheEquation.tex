 \FloatBarrier
\label{sec:2sexlinearmain}
As mentioned, choose some weight, $\sigma$, between 0 and 1 to apply to male
rates, where the female weight is defined as $1 - \sigma$. When $\sigma = 1$
there is perfect male dominance, and when $\sigma = 0$ there is perfect female
dominance. Of course, births to girls are subject to female mortality and births
to boys are subject to male mortality. As with
Equation~\eqref{eq:exLotkafemales}, this mortality enters in the equation by way
of the $d_x$ distribution used to distribute births over life expectancies. If
one knows the sex ratio at birth, expressed as the proportion male of births,
$\varsigma$, then year $t$ births $B(t)$ can be expressed as follows:

\begin{equation}
\label{eq:lin2sexBt}
B(t) = \int_{y=0}^\infty \sigma \varsigma P_y(t) F_y \dd y + \int_{y'=0}^\infty
(1-\sigma) (1-\varsigma) P_{y'}(t) F_{y'} \dd y'
\end{equation}
where $P_y(t)$ and $P_{y'}(t)$ are the male and female year $t$ population
counts classified by exact remaining years $y$ (exposures when discrete), and
$F_y$ and $F_{y'}$ are remaining-years classified fertility rates, $e$SFR
(including both sexes of birth). Of course, $P_y$ can be expressed in terms
of past births in a roundabout way:
\begin{align}
\label{eq:rexpressPy}
P_y &= \int_{a=0}^\infty P_a \frac{d_{a+y}}{p_a} \dd a \notag \\
    &= \int_{a=0}^\infty \varsigma B(t-a)p_a \frac{d_{a+y}}{p_a} \dd a \notag \\
    &= \int_{a=0}^\infty \varsigma B(t-a)d_{a+y} \dd a
\end{align}

where $p_a$ is the probability of surviving to age $a$, which is just the same
as $\int_{x=a}^\infty d_x \dd x$. Plugging the last line into
Equation~\eqref{eq:lin2sexBt}, we have:
\begin{equation}
\begin{split}
\label{eq:lin2sexBt2}
B(t) = \int_{y=0}^\infty \int_{a=0}^\infty \sigma \varsigma B(t-a)d_{a+y} 
F_y \dd a \dd y \\+ \int_{y'=0}^\infty \int_{a=0}^\infty (1-\sigma)
(1-\varsigma) B(t-a')d_{a'+y'} F_{y'} \dd y' \dd a'
\end{split}
\end{equation}
if left to evolve for long enough the size of consecutive birth cohorts will be
related by a constant factor, $e^r$, and this property allows us to rewrite
Equation~\eqref{eq:lin2sexBt2} in terms of years $t$ births:
\begin{equation}
\begin{split}
\label{eq:lin2sexBt3}
B(t) = \int_{y=0}^\infty \int_{a=0}^\infty \sigma \varsigma B(t)e^{-ra}d_{a+y} 
F_y \dd a \dd y \\+ \int_{y'=0}^\infty \int_{a=0}^\infty (1-\sigma)
(1-\varsigma) B(t)e^{-ra'}d_{a'+y'} F_{y'} \dd y' \dd a'
\end{split}
\end{equation} 
Dividing both sides by $B(t)$ brings us to the familiar-looking
Lotka-type unity equation, which can be used to estimate the two-sex intrinsic
growth rate, $r$:
\begin{equation}
\begin{split}
\label{eq:lineartwosexrenewal}
1 = \int_{y=0}^\infty \int_{a=0}^\infty \sigma \varsigma e^{-ra}d_{a+y} 
F_y \dd a \dd y \\+ \int_{y'=0}^\infty \int_{a'=0}^\infty (1-\sigma)
(1-\varsigma) e^{-ra'}d_{a'+y'} F_{y'} \dd y' \dd a'
\end{split}
\end{equation} 
where $\varsigma$ is the proportion male at birth for the \textit{stable}
population, which may either be assumed or estimated simultaneously with $r$-- the 
iterative estimation strategy outlined below describes how to estimate these two
parameters simultaneously. Equation~\eqref{eq:lineartwosexrenewal} does 
not assume that fertility rates are
available by sex of birth (4 combinations), but these will be needed in
following in order to simultaneously calibrate the sex ratio at birth.

The dominance-weighted two-sex $r$ extracted from~\eqref{eq:lineartwosexrenewal} 
is bounded by the $e_y$-structured $r^f$ and $r^m$,
and indeed $r^f$ and $r^m$ are recovered by setting $\sigma$ to 0 and 1,
respectively. That is to say, setting $\sigma$ to 1 or 0 makes the single-sex
model a degenerate case of the present model. This works because the
dominance-weighted model uses both sexes of birth for each sex of progenitor,
but appropriately weights the radix of progenitor by the sex ratio at birth.
In the single-sex model, one may conceive of the progenitor radix as
unweighted, whereas fertility is indeed weighted. In the end, $\sigma$ has the
same effect, and the border cases are identical. The dominance-weighted model
would not have this property if only a single sex of offspring were included in
 fertility. This author does not recognize any theoretical or practical merits
 of the single-sex modelling choice, as it is not the case that males are 
 responsible for the birth of boys and females for the birth of 
 girls\footnote{Or vice versa, as we saw in
 Section~\ref{sec:pollardage}. \citet{pollard1948measurement} took this idea
 even further by swapping sexes: The fertility functions in this paper 
 are based on the births of boys to mothers and girls to 
fathers, i.e. $M-F$ and $F-M$ fertility. This we saw was parsimonious in terms 
of getting quick results that are guaranteed to fall within reasonable 
bounds, but is even less intuitively appealing.}.

It must be
noted that the two-sex value of $r$ is dependant upon the choice of $\sigma$, 
and that no guidelines are provided for choosing a good value of $\sigma$. 
This ambiguity also exists in the age-structured variant of the present model. 
For $e_y$-structured models, it has been claimed that sex-divergence is 
lesser than is the case for
age-structured models. Recall that this was the case for predictions of birth
counts, and not for the growth parameter, $r$. The
difference between the $e_y$-structured $r^f$ and $r^m$ is not necessarily 
lesser than is the case for the age-structured $r^f$ and $r^m$. This will be
discussed further along with empirical results for the two populations 
considered in this dissertation.
