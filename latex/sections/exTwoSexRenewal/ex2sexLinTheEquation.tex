 \FloatBarrier
\label{sec:2sexlinearmain}
As mentioned, choose some weight, $\sigma$, between 0 and 1 to apply to male
rates, where the female weight is defined as $1 - \sigma$. When $\sigma = 1$
there is perfect male dominance, and when $\sigma = 0$ there is perfect female
dominance. Of course, births to girls are subject to female mortality and births
to boys are subject to male mortality. As with
Equation~\eqref{eq:exLotkafemales}, this mortality enters in the equation by way
of the $d_x$ distribution used to distribute births over life expectancies. Use
the following formula to estimate the two-sex intrinsic growth rate, $r$:

\begin{equation}
\label{eq:lineartwosexrenewal}
\begin{split}
1 = (1 - \sigma) 
            \int _{y'=0}^\infty \int _{a'=y'}^\infty e^{-ra'}
                      (1 - \varsigma) d_{a'}^F \left(f_{y'}^{F-F} +
                      f_{y'}^{F-M}\right) \dd a' \dd y' \\ + \;\sigma
            \int _{y=0}^\infty \int _{a=y}^\infty e^{-ra}
                     \varsigma d_{a}^M  \left(f_{y}^{M-M} +
                     f_{y}^{M-F}\right)\dd a \dd y
\end{split}
\end{equation}
, where $a'$, $y'$, $a$ and $y$ index female age, female remaining years, male
age and male remaining years, respectively. $\varsigma$ is the
proportion male at birth for the stable population, which may either be assumed
or estimated simultaneously wth $r$-- the iterative estimation strategy
outlined here outlines how to estimate these two parameters simultaneously.
Fertility superscripts identify sex of progenitor followed by sex of offspring, 
and $d_x$ must accord with the sex of offspring. Such specific rates are chosen 
because data that would permit empirical studies of the two-sex problem are 
typically sufficiently rich to allow for cross-tabulations by age of both parents as well as sex of birth. 
Therefore, Equation~\eqref{eq:lineartwosexrenewal} assumes that rates are
available by sex of progenitor, birth (4 combinations) and age (to be transformed to remaining years), 
and no additional variable is required for the sex ratio at birth. Indeed,
Equation~\ref{eq:lineartwosexrenewal} does not require such specific rates,
since rates of reach progenitor sex are simply summed, but sex-sex-specific
rates will be needed downstream for the calculation of other stable quantities,
so it is advisable to treat them as inputs from the start.

Weights, $\sigma$ and $1-\sigma$ are divided by 2 because
total births are counted twice in total (males and females from males \&
males and females from females). One could just as easily optimize to a sum of
2 on the left-hand side rather than discount weights.

The linear two-sex $r$, $r^\upsilon$, extracted from~\eqref{eq:lineartwosexrenewal} 
is \textit{not} guaranteed to be bounded by the $e_x$-structured $r^f$ and $r^m$,
and indeed $r^f$ and $r^m$ may not be recovered by setting $\sigma$ to 0 or 1,
respectively. This is so because the model includes births of both sexes to
progenitors of each sex, which changes the age-specific fertility curves
somewhat. That is to say, manipulation of $\sigma$ is insufficient to make the
one-sex model a degenerate case of the present model. $\sigma$ can only be
understood as indicative of the balance of dominance in fertility rates between
the sexes. The later choice would both reduce the complexity of
Equation~\ref{eq:lineartwosexrenewal} and guarantee exact bounds of $r^f$ and $r^m$ 
when $\sigma$ is set to 0 and 1, respectively. This author does not recognize 
the theoretical or practical merits of the single-sex modelling choice, as it 
is not the case that males are responsible for the birth of boys and females 
for the birth of girls\footnote{\citet{pollard1948measurement} took this idea
even further by swapping sexes: The fertility functions in this paper are based 
on the births of boys to mothers and girls to fathers, i.e. $M-F$ and $F-M$ fertility. This is parsimonious 
in terms of getting quick results that are guaranteed to fall within reasonable bounds, but is less
intuitively appealing}. This stance couples with the author's choice to not
include an explicit, let alone constant, variable for the sex ratio at birth.

It must be
noted that the value of $r^\upsilon$ is dependant upon the choice of $\sigma$, 
and that no guidelines are provided for choosing a good value of $\sigma$. 
This ambiguity also exists in the age-structured variant of the present model. 
For $e_x$-structured models, it has been claimed that sex-divergence is lesser than is the case for
age-structured models. Recall that this was the case for predictions of birth
counts, and not for the growth parameter, $r^\upsilon$. The
difference between the $e_x$-structured $r^f$ and $r^m$ is not necessarily lesser than is the case for
the age-structured $r^f$ and $r^m$. This will be discussed further along with
empirical results for the two populations considered in this dissertation.
