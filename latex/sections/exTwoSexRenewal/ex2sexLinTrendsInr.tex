This procedure has been applied to the data from the US and Spain with $\sigma
= [0,.5,1]$, which correspond to the cases of female-dominace, an
intermediate value, and male-dominance, and can be seen in
Figure~\ref{fig:rupsilonlinear2sex}\footnote{The data in
    Figure~\ref{fig:rupsilonlinear2sex} are available in
    Tables~\ref{tab:ex2linRepES}~and~\ref{tab:ex2linRepUS} of
    Appendix~\ref{appendix:ex2sexlinear}, along with the stable parameters
    $R_0^\upsilon$ and $T^\upsilon$.}

\begin{figure}[!ht]
  \centering
    \caption{Two-sex linear intrinsic growth rate, $r^\upsilon$, according to
    renewal Equation~\eqref{fig:rupsilonlinear2sex}, with $\sigma = [0, .5, 1]$ US 
    and Spain, 1969-2009}
     % figure produced in /R/ExLotka2Sex.R
     \includegraphics{Figures/exLotka2sexlinear}
     \label{fig:rupsilonlinear2sex}
\end{figure}

Patterns accord with trends generally known from the age-classified $r^f$ and
$r^m$, but values of $r^\upsilon$ are higher than the
age-classified intrinsic growth rates in all of the years studied. In all years
tested here, $r^\upsilon$ was indeed bounded by the $e_x$-structured $r^f$ and $r^m$. There have been some notable crossovers in
which was the greater $r^\upsilon$ derived from the border cases of $\sigma = 0$
and  $\sigma = 1$. Notably, in the US male dominance produced a higher
$r^\upsilon$ than female dominance from the start of observation until
1985, after which time there was a crossover that has persisted. Spain has
undergone two such crossovers, with female dominace producing a higher
$r^\upsilon$ than male dominace until 1980, and again recently starting in 2007.
Note that the spread between the border cases of male and female dominance is
tighter than between the single-sex cases. This observation should put the
demographer at ease in deciding the degree of dominance to assign via $\sigma$,
and makes it particularly easy to accept an intermediate value such as $.5$.
There have been series of years in both the US and Spain where swapping dominace
would have made virtually no difference in results or conclusions!
