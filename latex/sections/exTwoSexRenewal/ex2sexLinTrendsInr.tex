 \FloatBarrier
This procedure has been applied to the data from the US and Spain with $\sigma$
given the values of 0, 0.5, and 1, which correspond to the cases of
female-dominance, an intermediate value, and male-dominance, 
and can be seen in Figure~\ref{fig:rupsilonlinear2sex}. \footnote{The data in
Figure~\ref{fig:rupsilonlinear2sex} are available in Tables~\ref{tab:ex2linRepES}~and~\ref{tab:ex2linRepUS} of
    Appendix~\ref{appendix:ex2sexlinear}, along with the stable parameters
    $R_0$ and $T$.}

\begin{figure}[!ht]
  \centering
    \caption{Two-sex linear intrinsic growth rate, $r^\upsilon$, according to
    renewal Equation~\eqref{fig:rupsilonlinear2sex}, with $\sigma$ given the
    values 0, 0.5 and 1; US and Spain, 1969-2009}
     % figure produced in /R/ExLotka2Sex.R
     \includegraphics{Figures/exLotka2sexlinear}
     \label{fig:rupsilonlinear2sex}
\end{figure}

Patterns accord with trends generally known from the age-classified $r^f$ and
$r^m$, but values of $r$ are higher than the
age-classified intrinsic growth rates in all of the years studied. In all years
tested here, $r$ was indeed bounded by the $e_y$-structured $r^f$ and $r^m$.
We can confirm that our implementation is good in that the border cases
where $\sigma$ equals 0 or 1 produce the same results as the single-sex models.

\FloatBarrier
