
Once $r$ and $\varsigma$ have been located, one may derive other stable
quantities, as described elsewhere in this dissertation. The stable both-sex
birth rate, $b$ becomes:
\begin{equation}
b = \Bigg[\int_0^\infty  \int _0^\infty \varsigma
 e^{-ra}d_{a+y}\dd a \dd y + \int_0^\infty \int _0^\infty (1-\varsigma)
 e^{-ra'}d_{a'+y'} \dd a' \dd y' \Bigg] ^{-1}
\end{equation}
and with this we may derive the stable male and female structures, $c_y$ and
$c_{y'}$ by remaining years:
\begin{align}
c_y &= b\varsigma \int _0^\infty e^{-ra}d_{a+y}\dd a \\
c_{y'} &= b(1-\varsigma) \int _0^\infty e^{-ra'}d_{a'+y'}\dd a'
\end{align}
and naturally:
\begin{equation}
1 = \int c_y + \int c_{y'}
\end{equation}
These stable structures may be used to then calculate the stable proportions of
the populations above or below some $y$ threshold, to calculate the stable
whole-population sex ratio, or any of the other typical measures. The stable age
structure that belongs to this stable population, underlies it, may be retrieved
using $r$ in the standard way.
