\FloatBarrier
Once $r$ and $\varsigma$ have been located, one may derive other stable
quantities, as described elsewhere in this dissertation. $R_0$ and $T$ are
calculated as described in the iterative procedure, except that now they use
the stable $\varsigma$. The stable both-sex birth rate, $b$ becomes:
\begin{equation}
b = \Bigg[\int_0^\infty  \int _0^\infty \varsigma
 e^{-ra}d_{a+y}\dd a \dd y + \int_0^\infty \int _0^\infty (1-\varsigma)
 e^{-ra'}d_{a'+y'} \dd a' \dd y' \Bigg] ^{-1}
\end{equation}
and with this we may derive the stable male and female structures, $c_y$ and
$c_{y'}$ by remaining years:
\begin{align}
\label{eq:schoenexcy}
c_y &= b\varsigma \int _0^\infty e^{-ra}d_{a+y}\dd a \\
c_{y'} &= b(1-\varsigma) \int _0^\infty e^{-ra'}d_{a'+y'}\dd a'
\end{align}
and naturally:
\begin{equation}
1 = \int c_y + \int c_{y'}
\end{equation}

Using the stable structure, stable male and female marginal fertility rates,
$f_y$ and $f_{y'}$, may be retrieved:
\begin{align}
  f_y &=  \frac{\int_{y'=0}^\infty F_{y,y'}^M H\Big(c_y\;,\; c_{y'}\Big)
  \dd y'}{c_y} \\
  f_{y'} &=  \frac{\int_{y=0}^\infty F_{y,y'}^M H\Big(c_y\;,\; c_{y'}\Big)
  \dd y}{c_{y'}} 
\end{align}
In following we will compare these stable marginal fertility rates with initial
rates. The stable structures, $c_y$ and $c_{y'}$, may also be used to then
calculate the stable proportions of the populations above or below some $y$ threshold, to calculate the stable
whole-population sex ratio, or any of the other typical measures. The stable
\textit{age} structure that belongs to this stable population, underlies it, may
be retrieved using $r$ in the standard way.

\FloatBarrier