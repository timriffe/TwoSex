
We mentioned that the two-sex ratio-adjustment model presented here is
characterized by a set of desirable properties. First, the ratio of observed 
to expected counts
taken from the initial year has a substantive meaning in that it summarizes
some manner of association that resembles preference or attraction. This
adjustment need not carry this particular substantive explanation, since
individual lifespans are typically unknown. An exception are of course other
 markers that are known to
individuals, but that correlate with lifespan -- and are even known as such by
individuals -- such as health, smoking, diabetes, education and other things. We
understand the ratio adjustment method as the metaphorical shadow of preference 
after translation to the remaining years perspective. Second, the use of the
association-free expected distribution prior to adjusting for shadow preference
is a way of assuming perfect mixing in the population, ergo contact
opportunities conditioned by supply.

This ratio adjustment method performs comparably with
iterative proportional fitting when predicting the distribution of 
year $t+1$ $e_y$-classified births. Both methods come very close -- as close
as 99.5\% close -- to This will be illustrated empirically in following. Third,
if one imagines the underlying age distribution within the $e_y$-classified data, the 
present method -- and indeed all $e_y$-classified methods that use fertility
rates or births crossed by the remaining years of life of both parents -- 
accounts for inter-age-competition on both sides of the exchange. This property
is not trivial, as no optimization is required in order to achieve this. 

\todo{rethink this. include estimates of r and esfr. compare stable esfr to IPF,
H}





