
It has been mentioned that the two-sex model presented here is characterized by
a set of desirable properties. First, the ratio of observed to expected counts
taken from the initial year may plausibly be taken as indicative of
attraction or preference-- this adjustment has a substantive meaning.
Second, this ratio adjustment method compares favorably against the likey
next-best contender, iterative proportional fitting, when predicting the
distribution of the year $t+1$ $e_y$-classified births. In other words, the
method captures som real empirical regularity in the data that is left
unaccounted for, or blended out by, the likely next-best method [cite]. This
will be illustrated empirically in following. Third, if one imagines the age
distribution within the $e_y$-classified data, the present method-- and indeed
all $e_y$-classified methods that use fertility rates or births crossed by the
remaining years of life of both parents-- we have managed to account for
interage-competition on both sides of the exchange. This acheivment is not
trivial, as no optimization is required in order to acheive this. This advantage
will also be demonstrated in following.

\todo{input figure for disribution prediction vs IPF}

\todo{input figure for competition experiment}







