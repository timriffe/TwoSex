
It has been mentioned that the two-sex model presented here is characterized by
a set of desirable properties. First, the ratio of observed to expected counts
taken from the initial year has a substantive meaning in that it summarizes
some manner of association that resembles preference or attraction, but which
cannot possibly be preference or attraction since individual lifespans are
unknown. For this reason, we take this as the metaphorical shadow of preference
after translation to the remaining years perspective. Second, this ratio adjustment 
method performs comparably with iterative proportional fitting when
predicting the distribution of year $t+1$ $e_y$-classified births. This will be illustrated
 empirically in following. Third, if one imagines the underlying age
 distribution within the $e_y$-classified data, the present method-- and indeed
all $e_y$-classified methods that use fertility rates or births crossed by the
remaining years of life of both parents-- accounts for
inter-age-competition on both sides of the exchange. This property is not
trivial, as no optimization is required in order to achieve this. 

\todo{rethink this. include estimates of r and esfr. compare stable esfr to IPF,
H}





