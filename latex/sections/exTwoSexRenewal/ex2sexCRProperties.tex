\FloatBarrier

The two-sex ratio-adjustment model presented here is
characterized by a set of desirable properties. The first has to do with
interpretation: the ratio of observed to expected counts taken from the initial
year has a substantive meaning in that it summarizes 
some manner of association that resembles preference or attraction. This adjustment need not carry this
particular substantive explanation, since individual lifespans are typically unknown. An exception are of course other markers that are known to individuals, but that correlate with lifespan -- and
 are even known as such by individuals -- such as health, smoking, diabetes, education 
 and other things. We understand the ratio adjustment method as the metaphorical
 shadow of preference after translation to the remaining years perspective. Second, the use of the
association-free expected distribution prior to adjusting for shadow preference
is a way of assuming perfect mixing in the population, ergo contact
opportunities conditioned by supply. The first step is therefore to treat the
population like one large panmictic circle \citep[in the
sense of][]{henry1972nuptiality}, and the second step imposes a relatively
smooth departure from perfect mixture to account for observed non-randomness in
mating according to a fixed ratio.

The ratio adjustment method performs comparably with iterative proportional
fitting when predicting the distribution of year $t+1$ $e_y$-classified births. 
Both methods come very close to the observed year $t+1$ birth distribution,
overlapping on the order of 99\% of the observed distribution. For the US joint
births-distribution, both methods achieve on average 99.20\% overlap, faring
even better for the male and female marginal distributions with around 99.45\%
overlap. For the Spanish population, both methods overlap around 98.88\% of
the observed year $t+1$ joint birth distribution, and about 99.1\% of the marginal male and
female distributions. For Spain, the ratio-adjustment method performed slightly
better in terms of the distribution prediction, and for the US performance was
close to even. This test is noteworthy because IPF could be touted for its
distributional sensitivity, given its substitution property. On this
metric, IPF shows no clear advantage over the ratio-adjustment method. In this
case, one might prefer the ratio-adjustment method because it is a simple
adjustment rather than a complete iteration. 

As with all remaining-years methods, one need not worry too much about
competition and substitution, given that the dividing lines between
remaining-years classes are not as well known to individuals in the population
-- or at least we assume that these lines are less clear and less known than is the case
for age. Furthermore, if we assume that competition and substitution should take
place in terms of age, then remaining-years models indirectly account for
these axioms as follows. If a relatively large or small cohort passes through 
the population under a remaining-years model, this cohort distributes over
all remaining-years classes. In this case, the modal age at death for neighboring
cohorts will tend to most closely match that for the oddly-sized cohort, and so
we would expect penalization (benefit) to fall more upon neighboring cohorts
than upon distant cohorts. In other words, we should expect age-heterogeneity
within remaining-years classes to take care of the competition/substitution
problem without further ado. Whether effects distribute reasonably over ages is
an open question.

The ratio-adjustment method has not been fully described, and we categorize it
as experimental at this time. Its properties appear promising, but
a more thorough comparison is needed before passing judgment or making a
recommendation to apply it. We do not assume that the model will work as well in
projective settings for age-structured populations, precisely because the
distributional distance between the observed and expected joint birth
distribution is much greater in that case.

\FloatBarrier


