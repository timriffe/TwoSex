\label{sec:rexIPFfindr}
For a given proportion male at birth, $\varsigma$, it would be straightforward
to estimate $r$ using a generic optimizer and
Equation~\eqref{eq:ex2sexIPFunity3}, with the fertility component properly
expressed in place by Equations~\eqref{eq:exipfbyyp} and \eqref{eq:exipffy} (or
vice versa for females). As described elsewhere, however, the stable $\varsigma$
is not known in advance unless one does away entirely with the remaining-years
pattern to the sex ratio at birth, and the reason for this is that the stable
age-structure itself is not known until the equations are solved for $r$.
Since there is indeed a remaining-years pattern to the sex ratio at birth and
one does not know the ultimte structure in advance, one best proceeds by
estimating $r$ and $\varsigma$ together in an iterative process based on some 
good-enough guesses of starting values. The process unfolds in like manner to 
those outlined elsewhere in this dissertation, and is based on a modified version 
of that presented by \citet{coale1957new}.
Fertility rates are specific to sex of progenitor and sex of birth, and follow
the two-superscript notation used elsewhere in this dissertation, where the
first superscript indicates sex of progenitor and the second indicates sex of
offspring.

\begin{enumerate}
  \item Establish a starting value for the sex ratio at birth. For instance, one
  may take the year $t$ observed sex ratio at birth. From this, derive
  $\varsigma^0$ as $\varsigma^0 = \frac{SRB}{1+SRB}$
  \item Establish a guess at the net reproductive rate, $\widehat{R_0}$,
  assuming $r = 0$,
  \begin{equation}
  \label{eq:R0exipf0}
  \widehat{R_0} = \int _{y=0}^\infty F_{y}^{M^0} \int_{a=0}^\infty \varsigma^0
  d_{a+y}
  \end{equation}
  where $F_{y}^{M^0}$ is the male remaining-years specific fertility rate after
  having converged on a solution to Equation~\eqref{eq:exipfbyyp} with $r=0$ and
  then using this in Equation~\eqref{eq:exipffy} where $r=0$ in the denominator
  equation.
  \item Repeat the prior step, weighting $y$ into
  Equation~\eqref{eq:R0exmean} and divide this sum by $\widehat{R_0}$ to arrive at 
  an estimate of the mean generation length, $\widehat{T}$, in terms of
  remaining years. This is just an approximation, of course.
  \item Calculate an initial value of $r$, $r^0$ as:
  \begin{equation}
  r^0 = \frac{log(\widehat{R_0})}{\widehat{T}}
  \end{equation}
  \item Now begins the iterative part. For the given $r$, $r^0$ in the first
  instance, calculate the male and female sex-specific IPF-adjusted rates,
  $F_{y}^{M\ast}$ and $F_{y'}^{F\ast}$, that belong to this $r$ using
  Equations~\eqref{eq:exipfbyyp}, \eqref{eq:exipffy}, and
  \eqref{eq:exipffyp}. This will produce
  $F_{y'}^{F-F\ast}$, $F_{y'}^{F-M\ast}$, $F_{y}^{M-M\ast}$, and
  $F_{y}^{M-F\ast}$.
  \item Use the rates from the prior step in the following equation to produce a
  residual, $\delta^i$, where $i$ indicates the present iteration:
  \begin{align}
  \label{eq:exipfroptimugly}
  \begin{split}
  \delta^i = 1 - \frac{1}{2}\Bigg( \int _{y=0}^\infty F_{y}^{M\ast}
  \int_{a=0}^\infty \varsigma^i e^{-r^ia}d_{a+y} \dd a \dd y \\
  + \int _{y'=0}^\infty F_{y'}^{F\ast} \int_{a'=0}^\infty
(1-\varsigma^i) e^{-r^ia'}d_{a'+y'} \dd a' \dd y'\Bigg)
\end{split}
  \end{align}
  This can be replaced with Equation~\eqref{eq:ex2sexIPFunity3} if one
  prefers. The fertility rates here are simply summed by sex of progenitor, e.g.
  $F_{y}^{M\ast} = F_{y}^{M-M\ast}+F_{y}^{M-F\ast}$ from the prior step.
  \item Use $\delta^i$ to improve the estimate of $r$, $r^{i+1}$:
  \begin{equation}
  r^{i+1} = r^i - \frac{\delta^i}{\widehat{T} - \frac{\delta^i}{r^i}}
  \end{equation}
  \item Use the improved $r^{i+1}$ to update the proportion male of
  births, $\varsigma^{i+1}$. One could re-optimize the IPF-adjusted births at
  this point using the new $r$ as well. This could reduce iterations, but will
  not speed computation on the whole. Instead, take the right-hand side of
  Equation~\eqref{eq:exipfroptimugly} twice, once for boy-birth fertility and
  once for girl birth fertility. The ratio of these two sums is the iteration's
  sex ratio at birth, $S^{i+1}$, and this is converted to $\varsigma^{i+1}$
  using
  \begin{equation}
  \varsigma^{i+1} = \frac{S^{i+1}}{1 + S^{i+1}}
  \end{equation}
  \item Repeat steps 5-8 until $\delta^i$ vanishes to zero. At this time both
  $r$ and $\varsigma$ will have obtained their stable values. For the data used
  in this dissertation, around 30 iterations were required to arrive at maximum
  double floating point precision.
\end{enumerate}

