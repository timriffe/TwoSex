\FloatBarrier

Clearly the stable population structure in the present generalized means model,
for which we have run on the example of the harmonic mean, will differ from
the results of the dominance-weighted model only to the extent that $r$ (and
the stable SRB) differs between the two models. We have concluded that $r$ does
not appreciably differ, and so the methods would both seem acceptable for
purposes of judging the ultimate population structure that one would expect to
see given constant application of the year $t$ vital rates. One differentiates
between the models, therefore, based on the model design itself, at times with
respect to the axioms presented for the age-classified model, which aid us here
too. 

We may differentiate these models based with the availability axiom:  The use of
a mean function that falls to zero in the absence of one or the other sex, as is the 
case with the harmonic mean and others, will already produce a more
realistic model than the dominance-weighted model, which does not have this
property. This is a hefty observation, and may suffice as far as axiom-based
judgements are concerned. In looking at the model design itself, one appreciates
the more interactive nature of this chapter's model, wherein the year $t$ rates
are conditioned based on the relative abundance of males and females in each
remaining-years class. 

Remaining-years classes are here interactive, and
the dominance-weighted model does not have this design characteristic. Thus, the
fertility rate of males with 40 remaining years, for example, is conditioned
based on the relative size of this death cohort compared to that of all female
death cohorts. Does this make sense? Staying within the realm of minimum-biased
means, such as the harmonic mean, gives relatively small cohorts
bottleneck status in the model. How then do we imagine that this interaction is
supposed to unfold when all individuals involved are unaware of their own and
others' remaining lifetimes? Clearly such bottlenecking cannot unfold
via conscious preference, unless of course, physical traits and lifestyles are
so predictive of individual mortality. We suppose that mate selection will
include such markers, and that this may lead indirectly to such interactions.

Even so, it is harder to imagine a death cohort as having an inherent 
force of fertility than it is a birth cohort, and this makes it harder to 
imagine what is going on in the population that would cause inter-cohort rates 
to tug upon each other via the harmonic or some other mean. One could just as
easily imagine that the daily churn of the mating market happens in the
conscious realm of age, but even so, preference and partnering will unfold less
with the conscious evaluation of ages than it will on the basis of other
measures of suitability such as health, beauty, income, status, lifestyle, and
myriad other categories, all of which correlate to a certain extent with age so
as to exaggerate the appearance of age-preferences, per se. In age-classified
models, and especially those with explicit preference functions, these other
tangible preferences are all subsumed by age. This is perhaps the best way to
imagine the inner-workings of any remaining-years classified two-sex model, but
especially the present relatively interactive version. Preferences at play in
mating markets correlate with remaining years, just as they correlate with years
since birth.

The present remaining-years model does not preclude an
underlying age-interactive population, as long as the underlying age
interactions are constrained and conform to the outcomes predicted by the
remaining-years model. One could of course attempt to model both perspectives
simultaneously, via an increase in the dimensionality of the problem, but most,
the present author included, would see more obstacles than advantages in this
line of development. In following, we hash out a new two-sex balancing method
designed to exploit a particular observation of the remaining-years
perspective, before moving on to a translation of the iterative proportional fitting method.

\FloatBarrier