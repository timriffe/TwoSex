
Optimizing $r$ from Equation~\eqref{eq:exMeanUnity} is straightforward if the
proportion male of births, $\varsigma$, is assumed to be some fixed quantity.
In that case, one may use a generic optimizer to find $r$. As with the
age-structured optimization akin to the present one, though, it is preferable to
calibrate the sex ratio at birth simultaneous with $r$. This is even more
important for remaining-years structured populations, since the SRB pattern over
remaining years of parents is more pronounced than is the age-pattern.
Offsetting the potential obstacle presented by the strong $e_y$-pattern to SRB
is the lucky observation that present and stable $e_y$ population structures are
typically not markedly different. In practice with the two populations tested
here, we noted no problems from over-determination, as the range of values
that SRB can take is very narrow, and indeed allowing the SRB to move in accordance
with $r$ and vice versa stabilizes the optimization rather than causing it to
shoot off in some direction. This is safely the case because each sex of parent
is responsible for each sex of birth.

The data requirement for carrying out this optimization is therefore similar to
that of the age-structured procedure from Section~\ref{sec:ageharmonic}. One
requires $d_a$ and $d_{a'}$ from the period lifetable, the joint distribution
of boy births $B_{y,y'}^M$, and the joint distribution of girls
births $B_{y,y'}^F$, along with population vectors $P_y$ and $P_{y'}$ (exposures
in discrete time), from which one calculates the mean sex-of-birth
specific joint fertility rates, $F_{y,y'}^F$ and $F_{y,y'}^M$ using some mean of
male and female exposures in the denominator. $F_{y,y'}^F$ and $F_{y,y'}^M$ are
taken as constant and used throughout. The steps to find the stable $r$ and $S$
are then as follows:
\begin{enumerate}
  \item Establish a starting value for the sex ratio at birth. For instance, one
  may take the year $t$ observed sex ratio at birth. From this, derive
  $\varsigma^0$ as $\varsigma^0 = \frac{SRB}{1+SRB}$
  \item Establish a guess at the net reproductive rate, $\widehat{R_0}$,
  assuming $r = 0$:
  \begin{equation}
  \label{eq:R0exmean}
  \widehat{R_0} = \int_0^\infty \int_0^\infty F_{y,y'} H\Bigg(\varsigma \int _0^\infty
 d_{a+y}\dd a\;\;,\;\; (1-\varsigma) \int _0^\infty
 d_{a'+y'} \dd a'\Bigg) \dd y \dd y'
  \end{equation}
  where $F_{y,y'} = F_{y,y'}^M + F_{y,y'}^F$.
  \item Weight $y$, $y'$ into Equation~\eqref{eq:R0exmean} and divide this sum
  by $\widehat{R_0}$ to arrive at an estimate of the mean generation length,
  $\widehat{T}$ in terms of remaining years. 
  \item Calculate an initial value of $r$, $r^0$ as:
  \begin{equation}
  r^0 = \frac{log(\widehat{R_0})}{\widehat{T}}
  \end{equation}
  \item Plug $r^i$ and $\varsigma^i$ into Equation~\eqref{eq:exMeanUnity} to
  produce a residual, $\delta^i$:
  \begin{equation}
 \label{eq:exMeanUnity}
 \delta^i = 1 - \int_0^\infty \int_0^\infty F_{y,y'} H\Bigg(\varsigma^i \int
 _0^\infty e^{-r^ia}d_{a+y}\dd a\;\;,\;\; (1-\varsigma^i) \int _0^\infty
 e^{-r^ia'}d_{a'+y'} \dd a'\Bigg) \dd y \dd y'
 \end{equation}
where $F_{y,y'} = F_{y,y'}^M + F_{y,y'}^F$.
  \item  Use $\delta^i$ to improve the estimate of $r$, $r^{i+1}$:
  \begin{equation}
  r^{i+1} = r^i - \frac{\delta^i}{\widehat{T} - \frac{\delta^i}{r^i}}
  \end{equation}
  \item Use the improved $r^{i+1}$ to update the proportion male of
  births, $\varsigma^{i+1}$:
  \begin{align}
  B^{M,i+1}& = \int_0^\infty \int_0^\infty F_{y,y'}^M H\Bigg(\varsigma^i \int
 _0^\infty e^{-r^{i+1}a}d_{a+y}\dd a\;\;,\;\; (1-\varsigma^i) \int _0^\infty
 e^{-r^{i+1}a'}d_{a'+y'} \dd a'\Bigg) \dd y \dd y' \\
 B^{F,i+1} &= \int_0^\infty \int_0^\infty F_{y,y'}^F H\Bigg(\varsigma^i \int
 _0^\infty e^{-r^{i+1}a}d_{a+y}\dd a\;\;,\;\; (1-\varsigma^i) \int _0^\infty
 e^{-r^{i+1}a'}d_{a'+y'} \dd a'\Bigg) \dd y \dd y' \\
 S^{i+1} &= \frac{B^{M,i+1}}{B^{F,i+1}}
 \intertext{and finally}
 \varsigma^{i+1} &= \frac{S^{i+1}}{1 + S^{i+1}}
  \end{align}
  \item Repeat steps 5-7 until the error $\delta$ vanishes to zero, which may
  take 25-30 iterations for double floating point accuracy, far fewer for most
  practical purposes.
\end{enumerate}

