 \FloatBarrier
Formulas are here couched in the harmonic
mean, but this may be generalized, given that we specify the mean itself as
a function. The harmonic mean function itself differs from
Equation~\eqref{eq:harmonic} only in its subscripts:
\begin{equation}
H(P_y^m, P_{y'}^f) = \frac{2 P_y^m P_{y'}^f}{P_y^m + P_{y'}^f}
\end{equation}
As elsewhere in this dissertation, $y$ and $y'$ index remaining years of
males and females, respectively. We begin the process by calculating a single
joint fertility rate distribution, later assumed constant
 \begin{equation}
 F_{y,y'}^H = \frac{B_{y,y'}}{H(P_y^m, P_{y'}^f)}
 \end{equation}
again, only differing from Equation~\eqref{eq:harmonicrate} in the 
remaining-years subscripts. $F_{y,y'}^H$ is the primary model component. With this, we may
calculate the births for a given year:
 \begin{equation}
 \label{eq:Bharmonicex1}
 B(t) = \int \int F_{y,y'}^H H\Big(P_{y}^m(t), P_{y'}^f(t)\Big) \dd y \dd
 y'
 \end{equation}
The population count $P_y$ is, however, easily related to past births via the
previous Equation~\eqref{eq:rexpressPy}, the result of which was
\begin{equation}
P_y = \int_{a=0}^\infty \varsigma B(t-a)d_{a+y} \dd a
\end{equation}
 where $\varsigma$ is of course the proportion
male of births and the survival function is just the sum of future deaths: $p_a
= \int _{x=a}^\infty d_x \dd x$. This identity allows us to rewrite
Equation~\eqref{eq:Bharmonicex1} in terms of past births
 \begin{equation}
 \label{eq:Bharmonicex1}
 B(t) = \int \int F_{y,y'}^H H\Bigg(\varsigma \int _0^\infty B(t-a)d_{a+y}\dd
 a\;\;,\;\; (1-\varsigma) \int _0^\infty B(t-a')d_{a'+y'} \dd a'\Bigg) \dd y \dd
 y'
 \end{equation}
which when left to renew itself for many years on-end, will eventually attain a
constant rate of growth, $r$, in which case we may rewrite
Equation~\eqref{eq:Bharmonicex1} entirely in terms of year $t$ births:
 \begin{equation}
 \label{eq:Bharmonicex2}
 B(t) = \int \int F_{y,y'}^H H\Bigg(\varsigma \int _0^\infty
 B(t)e^{-ra}d_{a+y}\dd a\;\;,\;\; (1-\varsigma) \int _0^\infty
 B(t)e^{-ra'}d_{a'+y'} \dd a'\Bigg) \dd y \dd y'
 \end{equation}
This lets us divide by $B(t)$ to arrive at our standard approachable unity
equation, which permits the estimation of the stable growth parameter, $r$:
 \begin{equation}
 \label{eq:exMeanUnity}
 1 = \int_0^\infty \int_0^\infty F_{y,y'}^H H\Bigg(\varsigma \int _0^\infty
 e^{-ra}d_{a+y}\dd a\;\;,\;\; (1-\varsigma) \int _0^\infty
 e^{-ra'}d_{a'+y'} \dd a'\Bigg) \dd y \dd y'
 \end{equation}
As in other two-sex models, $\varsigma$ is also best estimated along with $r$
rather than assumed constant from the outset.
 \FloatBarrier
