
 \FloatBarrier
\label{sec:ex2sexCRit}
Steps to practically solve Equation~\eqref{eq:ex2sexCRunity} for $r$ are
similar to those presented for the two-sex linear case in
Section~\ref{sec:exrenewalit2}. Namely, $r$ and the sex ratio at birth, $S$, are
estimated together in an iterative process, using parameter guesses as starting
values and then updating in each iteration. First derive as inputs the matrix
$R_{y,y'}$ using \eqref{eq:getR}, $e_y$-specific fertility vectors by
sex of progenitor and offspring, and the relevant $d_a$ vectors:
\begin{enumerate}
  \item Decide a starting value for $\hat{S}^0$, such as the initial observed SRB,
although 1.05 is a good enough guess. For Spain 1.07 might be more
reasonable. Use $\hat{S}^0$ to calculate $\varsigma^0$ using:
\begin{equation}
\label{eq:getvarsigma}
\varsigma^0 = \frac{\hat{S}^0}{1+\hat{S}^0}
\end{equation}
  \item A rough estimate of the net reproduction rate, $\widehat{R_0}$ (assuming
  $r=0$) is given by:
 \begin{equation}
 \label{eq:R0roughCR}
 \widehat{R_0} = \int_{y=0}^\infty \int_{y'=0}^\infty
A\left(R_{y,y'},\mathbb{M}\left(\;\int_{a=y}^\infty F_y \varsigma^0 d_a \dd a,
\int _{a'=y'}^\infty F_{y'} (1-\varsigma^0) d_{a'} \dd a'\right)\right)
 \end{equation}
  \item Weight $y$ and $y'$ into Equation~\eqref{eq:R0roughCR} and divide the
  new sum by $\widehat{R_0}$ to arrive at a first estimate of the mean
  generation time (in remaining years of life), $\widehat{T}$

  \item A good starting value $r$, $r^0$, is given by:
   \begin{equation}
   r^0 = \frac{ln(\widehat{R_0})}{\widehat{T}}
   \end{equation}
  \item Plug $r^i$ into Equation~\ref{eq:ex2sexCRunity} to calculate a
  residual, $\delta^i$
  \item Use $\delta^i$ and $\widehat{T}$ to calibrate the estimate of $r$
  using:
  \begin{equation}
  r^{i+1} = r^i + \frac{\delta^i}{\widehat{T} - \frac{\delta^i}{r^i}}
  \end{equation}
  \item Use the improved $r$ to re-estimate the sex ratio at birth, using
  sex-specific fertility rates, $F_y^{M-M}$ (father-son), $F_y^{M-F}$
  (father-daughter), $F_{y'}^{F-F}$ (mother-daughter) and $F_{y'}^{F-M}$
  (mother-son) fertility rates:
  \begin{adjustwidth}{-1in}{0in}
  \begin{equation}
  S^{i+1} = \frac{\int_{y=0}^\infty \int_{y'=0}^\infty
A\left(R_{y,y'},\mathbb{M}\left(\;\int_{a=y}^\infty F_y^{M-M} \varsigma^i
d_ae^{-r^{i+1}a} \dd a, \int _{a'=y'}^\infty F_{y'}^{F-M} (1-\varsigma^i)
d_{a'}e^{-r^{i+1}a'} \dd a'\right)\right)}{\int_{y=0}^\infty \int_{y'=0}^\infty
A\left(R_{y,y'}, \mathbb{M}\left(\;\int_{a=y}^\infty F_y^{M-F} \varsigma^i d_a
e^{-r^{i+1}a} \dd a, \int _{a'=y'}^\infty F_{y'}^{F-F} (1-\varsigma^i)
d_{a'}e^{-r^{i+1}a'} \dd a'\right)\right)}
  \end{equation}
  \end{adjustwidth}
  and then update $\varsigma$ using: $\varsigma^{i+1} =
  \frac{S^{i+1}}{1+S^{i+1}}$ .
  \item With the updated $r$ and $\varsigma$, repeat steps 5-7 until $\delta$
  reduces to 0. Typically one achieves maximum double floating point precision
  after around 20 iterations, though fewer iterations are required for
  most practical applications.
\end{enumerate}











