
 \FloatBarrier
\label{sec:ex2sexCRit}
Steps to practically solve Equation~\eqref{eq:ex2sexCRunity} for $r$ are
similar to those presented for the two-sex linear case in
Section~\ref{sec:exrenewalit2}. Namely, $r$ and the sex ratio at birth, $S$, are
estimated together in an iterative process, using parameter guesses as starting
values and then updating in each iteration. First derive as inputs the matrix
$R_{y,y'}$ using \eqref{eq:getvarsigma}, $e_y$-specific fertility vectors by
sex of progenitor and offspring, and the relevant $d_x$ vectors:
\begin{enumerate}
  \item Derive a first rough estimate of the both-sex mean remaining years of
  life at reproduction, $\widehat{T}$, akin to Lotka's mean generation time,
  $T$. One could either weight remaining years, $y$ or $y'$ into
  Equation~\ref{eq:ex2sexCRunity}, similar to Equation~\ref{eq:Tguess}, or just
  make a rough first guess, such as 60. This parameter is not updated in the
  iterative process.

\item Decide a starting value for $\hat{S}^0$. 1.05 is a good enough guess,
although for Spain 1.07 might be more reasonable. Use $\hat{S}^0$ to calculate
$\varsigma^0$ using:
\begin{equation}
\label{eq:getvarsigma}
\varsigma^0 = \frac{\hat{S}^0}{1+\hat{S}^0}
\end{equation}

  \item A rough estimate of the net reproduction rate, $\widehat{R_0}$ (assuming
  $r=0$) is given by:

 \begin{equation}
 \begin{split}
 \widehat{R_0} = \int_{y=0}^\infty \int_{y'=0}^\infty
A\left(R_{y,y'},\mathbb{M}\left(\;\int_{a=y}^\infty F_y \varsigma^0 d_a \dd a,
\int _{a'=y'}^\infty F_{y'} (1-\varsigma^0) d_{a'} \dd a'\right)\right)
 \end{split}
 \end{equation}
  \item A first rough estimate of $r$, $r^0$, is given by:
   \begin{equation}
   r^0 = \frac{ln(\widehat{R_0})}{\widehat{T}}
   \end{equation}
  \item Plug $r^0$ into Equation~\ref{eq:ex2sexCRunity} to calculate a
  residual, $\delta^0$
  \item Use $\delta^0$ and $\widehat{T}$ to calibrate the estimate of $r$
  using:
  \begin{equation}
  r^{1} = r^0 + \frac{\delta^0}{\widehat{T} - \frac{\delta^0}{r^0}}
  \end{equation}
  \item Use the improved $r$ to re-estimate the sex ratio at birth, using
  sex-specific fertility rates, $F_y^M$ (father-son), $F_y^F$
  (father-daughter), $F_{y'}^F$ (mother-daughter) and $F_{y'}^M$ (mother-son)
  fertility rates:
  \begin{equation}
  S^1 = \frac{\int_{y=0}^\infty \int_{y'=0}^\infty
A\left(R_{y,y'},\mathbb{M}\left(\;\int_{a=y}^\infty F_y^M \varsigma d_ae^{-ra}
\dd a, \int _{a'=y'}^\infty F_{y'}^M (1-\varsigma) d_{a'}e^{-ra'} \dd
a'\right)\right)}{\int_{y=0}^\infty \int_{y'=0}^\infty A\left(R_{y,y'},
\mathbb{M}\left(\;\int_{a=y}^\infty F_y^F \varsigma d_a e^{-ra} \dd a, \int
_{a'=y'}^\infty F_{y'}^F (1-\varsigma) d_{a'}e^{-ra'} \dd a'\right)\right)}
  \end{equation}
  and then update $\varsigma^1$ using Equation~\eqref{eq:getvarsigma}.
  \item With the updated $r$ and $\varsigma$, repeat steps 5-7 until $\delta$
  reduces to 0. Typically one achieves maximum double floating point precision
  after around 20 iterations, though fewer iterations are required for
  most practical applications.
\end{enumerate}











