\FloatBarrier
In this Part, we jumped from the single-sex model structured by
thanatological age to two-sex models prior to providing a complete
exploration of the properties and consequences of the former. For instance, one
might question whether the single-sex model has a unique solution. This author
was personally content to plot the residuals of a finely grained series of potential
values for $r$ to see that the solution is indeed unique along a curve of
monotonically non-decreasing values, but this will surely not be satisfactory to
the rigorous mathematician. As mentioned in the corresponding results sections,
all values estimated for stable $r$ and $S$ are available in the
Appendix~\ref{appendix:exallrestimates} for each year of US and Spanish data
used in this dissertation.

Also pertinent, as we have dealt primarily with two-sex renewal functions,
is our lack of proof that the stable rates of growth attained in the
various two-sex remaining-years structured models are unique, necessary or
ergodic.\footnote{Independent 
of initial conditions.} There is a possibility that under some real conditions
the stabilizing trajectory arrives in a limit cycle,\footnote{This author considers 
limit cycles to be a particular variety of stability.} bifurcates, or is
otherwise complex. \citet{wijewickrema1980weak} and \citet{chung1990phd, chung1994cycles}
explore the possibility of such cycles and bifurcations in age-structured
two-sex models, but this and many other dynamic properties remain to be explored for 
remaining-years structured two-sex populations. We have also omitted any
sensitivity analysis, although this would enhance our ability to compare age-classified 
and remaining-years classified models. We have in some cases measured the total
amount of oscillation in population structure between the initial and stable states, but
we have not examined the path to stability. These and most
other transient properties of the models presented in this dissertation have
been ignored, and are particularly ripe for exploration for the new family of
remaining-years structured models that we propose. These are priorities for
future research.

Assuming that the patterns to fertility by remaining years of life
are indeed meaningful, and in any case observing that they are regular (for
some this is the only requisite), we are now free to model population on the
basis of them, just as demographers have always done on the basis of age. This we began
for the single-sex case in Chapter~\ref{sec:exstructuredrenewal}, and there we
learned in the first place that the intrinsic growth rate that belongs to this
\textit{family} of model is less erratic than the age-analogue. In other
contexts, demographers have lent value to less-erratic renditions of otherwise
familiar demographic time-series. Such has been the case with tempo-adjusted
fertility rates \citep{bongaarts1998quantum}.\footnote{Although authors
\citep[e.g.,][]{kim2000quantum} have found this species of adjusted TFR to be
erratic as well, the basic desire for a relatively stable indicator remains.}
Here we have produced such a series for intrinsic growth rates (and $R_0$ and $e$SFR), 
more stable than the chronological-age analogue, and the reader must be the 
judge of whether this adds value or not. We expect that many demographers would
prefer to tinker with these methods before passing judgement. For this reason,
we make available the basic transformation of Equation~\eqref{eq:dxredist} in both a spreadsheet and
an \texttt{R} function.

As for why results of chronological age should differ from those of
thanatological age, we noted above that the pertinent rates are calculated on
the basis of different underlying exposures. There is room to experiment with
 finer segmentation of exposures, but we are uncertain (although not doubtful)
 that anything is to be gained by a more complex model. Clearly one can
coherently conceive of population renewal under the remaining-years
perspective, and we have built models that accord with this vision. A parallel
may be drawn with male and female single-sex models under either variety of
age; both models are of equal soundness, yet return results that are at odds. 
It is tautologically the case that in the year of initial conditions as well as
in hypothetical stability, the male and female models produce no discrepancy. We
may say the same of models under thanatological and chronological age: in the
initial year (as well as in hypothetical stability) these two models are congruent, but thereafter they
diverge.

With the two-sex problem, it is easy to imagine that the two sexes modeled
separately are bound to diverge, and to note that this may not be so in
observed populations, as governed by the sex ratio at birth. Any two-sex
model will provide that the two sexes project through time in unison. Model
divergence under chronological versus thanatological age, on the other hand,
will result even when both sex sexes are modeled together under the same principles under each
of the two age structures. Two-sex models for either age definition are a
summary of the growth of the entire population, yet results depend on whether
one counts up from birth or down toward death. We are happy to have demonstrated
this discrepancy, but this finding is rather unglamorous in comparison to a
hypothetical model that would contain information from both age definitions and
both sexes, or to a philosophical argument for why one definition of age
produces a superior model of population growth. Clearly a population may
have only one total growth rate or net reproduction ratio. Let us call this
conundrum for now the \textit{two-age problem}.

While the two-sex problem has not been solved in an necessary and true way, some
satisfactory solutions have arisen. These solutions have in common that they
deal somehow with mixing, with the interaction between sexes, and axioms have
been developed to help guide the way in determining ideal model properties. No
such axioms exist at this time for the two-age problem -- this author does not
even know how to properly frame it. No model has been proposed
that would unify the results of these two definitions of age. Recall that some 
of the initial responses to the pointing-out of the two-sex problem
were to produce ad hoc justifications for female dominance, and some of these
have had staying-power. These issues are worthy of more contemplation than that.
After \citet{karmel1947relations}, formal demographers came to realize the
importance of modeling the two sexes together, and a great body 
of work has been produced to this end. 

\FloatBarrier