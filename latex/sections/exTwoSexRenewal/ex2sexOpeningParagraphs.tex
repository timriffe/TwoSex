 \FloatBarrier
\label{Chapter:ex2sex}
It has been noted that divergence between the sexes, in terms of predicted birth
counts, is often dampened when projected using rates and populations that are
structured according to remaining years as compared to age-structured rates.
This does not, however, mean that the problem of the sexes is in this context
 negligible. Instead, the problem has
only become slightly more tractable. The author considers the problem more
tractable because in decreasing the magnitude of discrepancy between male and female
rates, the trade-offs inherent in the various two-sex solutions offered in the
literature also become smaller. The present Part 3 introduces some two-sex
extensions of the $e_y$-structured population model introduced in the earlier
Chapter~\ref{sec:exstructuredrenewal}:
\begin{enumerate}
  \item In Chapter~\ref{sec:ex2sexdomweights} we
  translate the dominance-weighted extension earlier presented in
  Section~\ref{sec:googmanage}. This method assuming fixed weights for male and
  female marginal fertility distributions. We provide a continuous model, an
  interactive method to estimate $r$, a two-sex projection matrix, and discuss
  stable population structure at some length.
  \item In Chapter~\ref{sec:ex2sexschoen} we propose an extension based on the
  generalized mean of the joint male-female exposures, as presented earlier in
  Section~\ref{sec:ageharmonic} for the case of age-structured populations. We
  provide the continuous model, an iterative method to estimate $r$ and discuss
  the stable fertility distribution.
  \item In Chapter~\ref{sec:ipfex} we describe the translation of iterative
  proportional fitting (IPF) to $e_y$-structured populations, as previously
  presented in Section~\ref{sec:IPF} for the case of age-structured populations.
  We provide the continuous model, an iterative method to estimate $r$, and some
  results of the stable fertility distribution.
  \item In Chapter~\ref{sec:CRchap} we consider a two-sex extension
  especially for $e_y$-structured populations, based on a constant departure
  from the association-free joint birth distribution. An iterative method to
  produce $r$ is provided, as are some basic results.
\end{enumerate}