
Much of the present implementation will follow directly from the age-oriented
implementation from Section~\ref{sec:IPF}. The primary different will be our
convention of swapping out the survival function for $e_y$-specific fertility
rates applied step-wise to elements of the death distribution, as seen elsewhere
in renewal formulas for the $e_y$-perspective. IPF itself is carried out in like
fashion to that described earlier, with total marginal birth predictions
first balanced by some mean. 

To rehash, define the IPF function,
$IPF(B_{y,y'}(t), F_y^M(t), F_{y'}^F(t),P_y(t+n),P_{y'}(t+n))$, where the first
three parameters are the joint distribution of births, and male and female
$e_y$-classified fertility rates. $P_y$ and $P_{y'}$ are the population
structures that rates iteratively adjust to until birth count predictions are
equal for males and females in each $(y,y')$ pair. The marginal distributions of
the adjusted birth counts are used to calculate the adjusted fertility rates,
$F_y^{M\ast}$ and $F_{y'}^{F\ast}$\footnote{The $M$ and $F$ superscripts are
redundant with $a$ and $a'$ in identifying sex of progenitor, but will be handy
later when doubled with a second superscript to indicate sex of offspring.}. As
before, for the first three $(t)$ arguments, those that define the initial
state, we can summarize with the single parameter $\tau$, indicative of initial
conditions: $IPF(\tau,P_y, P_{y'})$, where $P_y$, $P_{y'}$ could come from any
year or population. In short, year $t$ births are simply:
\begin{equation}
B(t) = \int _{y=0}^\infty \int _{y'=0}^\infty IPF\Big(\tau,P_y(t),
P_{y'}(t)\Big) \dd y \dd y'
\end{equation}
Assuming constant mortality and sex ratio at birth, year $t$ population counts
can be expressed as a product of past births, $P_y(t) =
\varsigma \int_{a=0}^\infty B(t-a)d_{a+y} \dd a$, which after many years of
constant conditions can be rewritten in terms of $B(t)$, $P_y(t) =
\varsigma \int_{a=0}^\infty B(t)e^{-ra}d_{a+y} \dd a$

\begin{equation}
\label{eq:ex2sexIPFunity3}
\begin{split}
B(t) = \int _{y=0}^\infty \int _{y'=0}^\infty IPF\Bigg(\tau\;,\;
 \int_{a=0}^\infty \varsigma B(t) e^{-ra}d_{a+y} \dd a\;, \\ 
 \int_{a'=0}^\infty (1-\varsigma) B(t)e^{-ra'}d_{a'+y'} \dd a'\Bigg) \dd y \dd
 y'
\end{split}
\end{equation}
which reduces to our Lotka-type unity equation:
\begin{align}
\label{eq:ex2sexIPFunity3}
1 &= \int _{y=0}^\infty F_{y}^{M\ast} \int_{a=0}^\infty \varsigma e^{-ra}d_{a+y}
\dd a \dd y \\ 
&= \int _{y'=0}^\infty F_{y'}^{F\ast} \int_{a'=0}^\infty
(1-\varsigma) e^{-ra'}d_{a'+y'} \dd a' \dd y'
\end{align}
where
\begin{align}
\label{eq:exipfbyyp}
B_{y,y'}^\ast &= IPF\Bigg(\tau\;,\;
 \int_{a=0}^\infty \varsigma e^{-ra}d_{a+y} \dd a\;,\;
 \int_{a'=0}^\infty (1-\varsigma) e^{-ra'}d_{a'+y'} \dd a'\Bigg) \\
 \label{eq:exipffy}
F_{y}^{M\ast} &= \frac{\int_{0=y'}^\infty B_{y,y'}^\ast \dd
y}{\int_{a=0}^\infty\varsigma e^{-ra}d_{a+y} \dd a} \\
 \label{eq:exipffyp}
F_{y'}^{F\ast} &= \frac{\int_{0=y}^\infty B_{y,y'}^\ast \dd
y}{\int_{a'=0}^\infty (1-\varsigma) e^{-ra'}d_{a'+y'} \dd a'} 
\end{align}




