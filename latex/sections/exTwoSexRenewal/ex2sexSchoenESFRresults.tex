\FloatBarrier
As described in Section~\ref{sec:schoenexiterative}, the iterative technique to
optimize the two-sex $r$ for the present population model is here run for each year of data. $r$ itself
is not visually distinguishable from the previous weighted dominance model where $\sigma = 0.5$,
and so we do not bother to display yet another time series of $r$. The case is
similar for the ultimate 2-sex stable population structure. Equation~\eqref{eq:schoenexcy} can 
be followed to produce structures that are also very similar to those
previously displayed for the weighted-dominance case where $\sigma =
0.5$. Instead, in order to better grasp the meaning of this particular
model, we display the initial versus stable $e_y$-specific fertility rates,
$e$SFR, for a pair of years from the US and Spanish data. This author has never
seen such comparisons done for age-classified models, even when formulas are given to
calculate the stable TFR. 

\begin{figure}[ht!]
        \centering  
          \caption{Male and female initial and stable $e$SFR (harmonic mean
          method). US and Spain, 1975 and 2009.}
           % /R/Schoen1981ex.R
           \includegraphics{Figures/eSFRharmonic}
          \label{fig:eSFRharmonic}
\end{figure}

One aspect of the present model that may not be obvious is that male and female marginal
fertility rates indeed change from the initial state in each time point along
the path to stability, and ultimately differ in the final state from the initial
$e_y$-specific rate vectors. Two aspects of fertility are held constant in the
present model: 1) The initial joint rates calculated on the basic of a given
mean of male and female exposures specific to each combination of remaining
years of life. This matrix is indeed held fixed. 2) The particular mean function
used in the first place to calculate the mean rate matrix is reapplied in each
successive year to the evolving population vectors. As population vectors
oscillate, the ultimate predicted birth count for a particular remaining-years
combination will rise or fall, as will male and female marginal birth count
predictions. The end effect is that the marginal rates themselves are also
different in the initial versus stable states. Figure~\ref{fig:eSFRharmonic}
compares the initial $e$SFR vectors for each sex with their ultimate stable
values for 1975 and 2009 in the US and Spain.

In general, initial rates will differ from stable rates as a function of the
degree of difference in the initial versus stable population structures. Where
initial and stable structures are similar, marginal fertility rates are not
expected to change much, such as 2009 US in Figure~\ref{fig:eSFRharmonic}. What
is constant in the model is the (element-wise) \textit{interaction} between
remaining years classes of males and females, as captured by a particular mean function.
Here we have used the harmonic mean, but
this can certainly be switched for any other criterion (albeit with little
consequence in our experience). To draw an example from the current figure, note
that marginal rates for 1975 US females are higher in the stable than in the
initial states. This means that females in the stable population are relatively
less abundant than in the initial population. Rates for males in this case must
on average move downward to compensate. This model property applies in like 
manner to the analogous age-classified model
presented earlier in Section~\ref{sec:ageharmonic}.

One may compare the full series of initial versus stable fertility rates, by
summing over remaining years within each year to arrive at $e$TFR, and then
taking the difference between stable and initial $e$TFR. The results of this
exercise are displayed in Figure~\ref{fig:eTFRharmonic}. As one might expect,
the male and female $e$TFR differences mirror each other approximately. These
differences are due primarily to changes in the sex structure between the initial and stable
states, and since fertility rates are calculated on the mean of male and female
exposures, male and female $e$TFR will be pulled in opposite directions. The
magnitude of the difference between initial and stable TFR under this model
definition has on the whole been decreasing over time, and it has typically been
smaller for the US population than for the Spanish population.

\begin{figure}[ht!]
        \centering  
          \caption{Difference between stable and initial $e$TFR, males and
          females (harmonic mean method). US, 1969-2009 and Spain,
          1975-2009.}
           % figure produced in /R/Schoen1981ex.R
           \includegraphics{Figures/eTFRharmonic}
          \label{fig:eTFRharmonic}
\end{figure}

\todo{shape?}
\FloatBarrier