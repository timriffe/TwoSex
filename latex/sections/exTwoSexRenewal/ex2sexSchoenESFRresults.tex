
The iterative technique to optimize the two-sex $r$ for the present
population model is here run for each year of data. $r$ itself is not visually
distinguishable from the previous weighted dominance model where $\sigma = 0.5$,
and so we do not bother to display yet another time series. The case is similar
for the ultimate 2-sex stable population structure. Formulas can be followed
from the previous recommendations to produce structures that are comparable with
those previously displayed for the weighted-dominance case where $\sigma =
0.5$. Instead, in order to better grasp the meaning of this particular
model, we display the initial versus stable $e_y$-specific fertility rates for a
pair of years from the US and Spanish data.

One aspect of the present model that may not be obvious \footnote{This
result applies in like manner to the analogous age-classified model presented
earlier in Section~\ref{sec:ageharmonic}} is that male and female marginal
fertility rates indeed change from the initial state in each time point along
the path to stability, and ultimately differ in the final state from the initial
$e_y$-specific rate vectors. Two aspects of fertility are held constant in the
present model: 1) The initial joint rates calculated on the basic of a given
mean of male and female exposures specific to each combination of remaining
years of life. This matrix is indeed held fixed. 2) The particular mean function
used in the first place to calculate the mean rate matrix is reapplied in each
successive year to the evolving population vectors. As population vectors
oscillate, the ultimate predicted birth count for a particular remaining-years
combination will rise or fall, as will male and female marginal birth count
predictions. The end effect is that the marginal rates themselves are also
different in the initial versus stable states. Figure~\ref{fig:eSFRharmonic}
compare the initial $e$SFR vectors for each sex with their ultimate stable
values for 1975 and 2009 in the US and Spain.

\begin{figure}[ht!]
        \centering  
          \caption{Male and female initial and stable $e$SFR. US and
          Spain, 1975 and 2009.}
           % /R/Schoen1981ex.R
           \includegraphics{Figures/eSFRharmonic}
          \label{fig:eSFRharmonic}
\end{figure}

In general, initial rates will differ from stable rates as a function of the
degree of difference in the initial versus stable population structures. Where
initial and stable structures are similar, marginal fertility rates are not
expected to change much. What is constant in the model is the
\textit{interaction} between remaining years classes of males and females, as
captured by a particular mean function. Here we have used the harmonic mean, but
this can certainly be switched for any other criterion (albeit with little
consequence in our experience).













