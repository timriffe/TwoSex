 \FloatBarrier
\citet{goodman1967age} offers a suite of
formulas to determine the stable age-sex composition of a population taking into
account the vital rates of both sexes, assuming one can assign a relative weight
(summing to 1) to male and female fertility. In this case there are two
trade-offs: 1) one must (arbitrarily) choose weights, and 2) these weights are
constant. The fact of having constant weights keeps the solution linear
(interaction-free), but less realistic. The final result is bracketed by the
cases of male and female dominance, but the gap between these two extremes 
also measures the demgographer's subjective leeway, which we would like to
minimize. Both of these drawbacks may be reduced in the case of $e_y$-structured
populations, since: 
\begin{enumerate}
  \item $e_y$-structured populations have a more stable (in terms of
  distributional variation from year-to-year) structure than age-structured
  populations.
\item Mate-selection with respect to remaining years of
life is nearly random in $e_y$-structured populations (see
Section~\ref{sec:exobsexpected}).
\item The difference between male and female dominance (in terms of projected
birth counts) is often reduced, thereby limiting of the impact of the
demographer's ``dominance caprice'' on results (See
Section~\ref{sec:exdivergence}).
\end{enumerate}
Points (1) and (2) reduce (but do
not eliminate) the necessity of sex-interactions in a model. By this it is meant
that the proportional difference in results from one choice of model weights
over another is simply diminished. This being so, the comparative advantage of a
more sophisticated or realistic model is to some degree diminished. 
