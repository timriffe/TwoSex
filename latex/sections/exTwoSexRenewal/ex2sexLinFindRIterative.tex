 \FloatBarrier
\label{sec:exrenewalit2}
Steps to practically solve Equation~\eqref{eq:lineartwosexrenewal} for $r$ are
similar to those presented for the one-sex case in
Section~\ref{sec:exrenewalit}
\begin{enumerate}
  \item Derive a first rough estimate of the both-sex mean remaining years of
  life at reproduction, $\widehat{T^\upsilon}$, akin to Lotka's mean generation time,
  $T$. If one assumes a growth rate of $0$, then a good-enough guess will be:

\begin{equation}
\widehat{T^\upsilon} = \frac{\splitfrac{
   \big((1 - \sigma)  \int _{y'=0}^\infty \int_{a'=y'}^\infty 
       a' d_{a'}^F \left( f_{y'}^{F-F} + f_{y'}^{F-M} \right) \dd a'\dd y'}{ + 
   \sigma \int_{y=0}^\infty \int _{a=y}^\infty a d_{a}^M  \left( f_{y}^{M-M}+
   f_{y}^{M-F} \right) \dd a \dd y \big)}}{\splitfrac{\big( (1 - \sigma) 
   \int_{y'=0}^\infty \int_{a'=y'}^\infty d_{a'}^F \left( f_{y'}^{F-F} +
   f_{y'}^{F-M} \right) \dd a'\dd y'}{ +\sigma \int _{y=0}^\infty
   \int_{a=y}^\infty d_{a}^M \left( f_{y}^{M-M} + f_{y}^{M-F} \right) \dd a \dd
   y \big)}}
\end{equation}

  \item A first rough estimate of the net reproduction rate, $\widehat{R_0}$ (assuming
  $r=0$) is given by:

 \begin{equation}
 \begin{split}
 \widehat{R_0} = \frac{(1 - \sigma)}{2}  \int _{y'=0}^\infty \int_{a'=y'}^\infty 
                d_{a'}^F \left(f_{y'}^{F-F} + f_{y'}^{F-M}\right) \dd a'\dd y'
                \\ + \frac{\sigma}{2}  \int _{y=0}^\infty \int _{a=y}^\infty 
               d_{a}^M  \left(f_{y}^{M-M}+ f_{y}^{M-F}\right) \dd a \dd y
 \end{split}
 \end{equation}
  \item A first rough estimate of $r$, $r^0$, is given by:
   \begin{equation}
   r^0 = \frac{ln(\widehat{R_0})}{\widehat{T^\upsilon}}
   \end{equation}
  \item plug $r^0$ into Equation~\ref{eq:lineartwosexrenewal} to calculate a
  residual, $\delta^0$
  \item use $\delta^0$ and $\widehat{T^\upsilon}$ to calibrate the estimate of $r$
  using:
  \begin{equation}
  r^{1} = r^0 + \frac{\delta^0}{\widehat{T^\upsilon} - \frac{\delta^0}{r^0}}
  \end{equation}
  \item repeat step (3) to to derive a new $\delta^i$, then step (4) to refine
  $r^i$, until converging on a stable $r$ after some 30 iterations,
  depending on the degree of precision desired. ($\widehat{T^\upsilon}$ is not updated
  in this process).
\end{enumerate}