 \FloatBarrier
\label{sec:exrenewalit2}
Steps to practically solve Equation~\eqref{eq:lineartwosexrenewal} for $r$ are
similar to those presented for the one-sex case in
Section~\ref{sec:exrenewalit}, except we must add a step to simultaneously
calibrate the sex ratio at birth, $S$.
\begin{enumerate}
 \item Determine a starting value for $\hat{S}^0$. 1.05 is a good enough guess,
 although for Spain 1.07 might be more reasonable. Use $\hat{S}^0$ to 
 calculate $\varsigma^0$ using
\begin{equation}
\label{eq:getvarsigmal}
\varsigma^0 = \frac{\hat{S}^0}{1+\hat{S}^0}
\end{equation}
$\varsigma$ is updated in subsequent iterations.
   \item A first rough estimate of the net reproduction rate, $\widehat{R_0}$ (assuming
  $r=0$) is given by 
 \begin{equation}
 \label{eq:ex2sexlinR0hat}
 \begin{split}
 \widehat{R_0} = (1 - \sigma)  \int _{y'=0}^\infty \int_{a'=y'}^\infty 
                (1-\varsigma^0)d_{a'}^F \left(f_{y'}^{F-F} + f_{y'}^{F-M}\right)
                \dd a'\dd y' \\ + \sigma  \int _{y=0}^\infty \int _{a=y}^\infty 
               \varsigma^0 d_{a}^M  \left(f_{y}^{M-M}+ f_{y}^{M-F}\right) \dd a
               \dd y
 \end{split}
 \end{equation}
 \item Calculate the sum of Equation~\eqref{eq:ex2sexlinR0hat} again
 after weighting in $y$ and $y'$, respectively, and divide this sum by
 $\widehat{R_0}$ to arrive at an estimate of the mean length of generation (in
 terms of remaining years), $\widehat{T}$.
  \item A first rough estimate of $r$, $r^0$, is given by
   \begin{equation}
   r^0 = \frac{ln(\widehat{R_0})}{\widehat{T}}
   \end{equation}
  \item Plug $r^0$ into Equation~\eqref{eq:lineartwosexrenewal} to calculate a
  residual, $\delta^0$.
  \item Use $\delta^0$ and $\widehat{T}$ to calibrate the estimate of $r$
  using
  \begin{equation}
  r^{1} = r^0 + \frac{\delta^0}{\widehat{T} - \frac{\delta^0}{r^0}}
  \end{equation}
  \item  Use the improved $r$ to update the sex ratio at birth, using
  sex-specific fertility rates, $F_y^{M-M}$ (father-son), $F_y^{M-F}$
  (father-daughter), $F_{y'}^{F-F}$ (mother-daughter) and $F_{y'}^{F-M}$
  (mother-son) fertility rates:
  \begin{equation}
  S^1 = \frac{ \int _{y'=0}^\infty \int _{a'=y'}^\infty e^{-r^1a'}
                      (1-\varsigma^0)d_{a'}^F f_{y'}^{F-M} \dd a' \dd y' + \int
                      _{y=0}^\infty \int _{a=y}^\infty e^{-r^1a}
                      \varsigma^0 d_{a}^F f_{y}^{M-M} \dd a \dd y}{\int
                      _{y'=0}^\infty \int _{a'=y'}^\infty e^{-r^1a'}
                      (1-\varsigma^0)d_{a'}^F f_{y'}^{F-F} \dd a' \dd y' + \int
                      _{y=0}^\infty \int _{a=y}^\infty e^{-r^1a} \varsigma^0
                      d_{a}^F f_{y}^{M-F} \dd a \dd y}
  \end{equation}
  Then update to $\varsigma^1$ using Equation~\eqref{eq:getvarsigmal}.
  \item Repeat step (5) to to derive a new $\delta^i$, then step (6) to refine
  $r^i$, adjusting $S^i$ with (7), and again steps 5-7 until converging on a
  stable $r$ (and $S$) after some 30 iterations, depending on the degree of
  precision desired ($\widehat{T}$ is not updated in this process).
\end{enumerate}
  One may rightly object that given only
  Equation~~\eqref{eq:lineartwosexrenewal} we should be able to
  solve for only one variable, $r$ or $S$, and not both. In practice, results
  are not sensitive to the choice of starting $S^0$, and the
  calibration method leads in any (reasonable) case to the same stable $r$.
  There is simply little room for $S$ to deviate from its stable value given
  that 1) the starting and stable structures are typically in this
  case not far from one another, and 2) males and females produce each sex of
  offspring, thus narrowly constraining $S$ even in the case of
  perfect dominance. As a
  sensitivity test, some extreme starting values for $S^0$ were chosen for
  select years from the data used in this dissertation (ranging between .8 and
  1.3): all lead to identical calibrated values of $r$ and $S$. At least with
  this estimation method and the data used in this dissertation, the equations
  presented here are identifiable.
