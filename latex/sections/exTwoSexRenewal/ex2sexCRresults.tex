\FloatBarrier

The method to estimate $r$ (and the stable SRB) described in the previous
section has been applied to each year of the US and Spanish data to produce the
series displayed in Figure~\ref{fig:exCRr}. Detailed results for $r$ and the
stable sex ratio at birth can be found in the tables of Appendix~\ref{appendix:exallrestimates} 
alongside those of other remaining-years renewal models. One notes immediately
that the ratio-adjustment method by far yields the most different results from
any of the other methods tested. The sex ratio at birth falls in line with estimates
produced by other methods, and so we can say that the method is still in a sense
well-balanced. Broadly, we may state that this method produces an even-less
erratic series of intrinsic growth rates than any seen thus far, often
(but not always) tending closer to zero than either of the single-sex rates. The
direction of change is always the same as the male and female series, but the
magnitude of change is typically smaller. Here we finally have a method that
yields results meaningfully different from the pack, and with a hint of
intuitive appeal. In the following section we disscuss other aspects of this
method to help judge its worth.

\begin{figure}[ht!]
        \centering  
          \caption{Two-sex $r$ calculated using the ratio-adjustment method for
          remaining-years classified data, compared with $r^m$ and $r^f$. US, 1969-2009 and
          Spain, 1975-2009.}
           % figure produced in /R/crossprodRatio.R
           \includegraphics{Figures/exCRr}
          \label{fig:exCRr}
\end{figure}

\FloatBarrier
