 \FloatBarrier
\label{sec:2sexlinearother}
Once two-sex linear $r^\upsilon$ has been found for the given $\sigma$, one may
proceed to find the corrsponding both-sex $T^\upsilon$ using:
\begin{equation}
T^\upsilon = \frac{\splitfrac{
   \big((1 - \sigma)  \int _{y'=0}^\infty \int_{a'=y'}^\infty 
       a'e^{-ra'} d_{a'}^F \left( f_{y'}^{F-F} + f_{y'}^{F-M} \right) \dd a'\dd
       y'}{ + \sigma \int_{y=0}^\infty \int _{a=y}^\infty a e^{-ra}d_{a}^M 
       \left( f_{y}^{M-M}+ f_{y}^{M-F} \right) \dd a \dd y \big)}}{\splitfrac{\big( (1 - \sigma) 
   \int_{y'=0}^\infty \int_{a'=y'}^\infty e^{-ra'} d_{a'}^F \left( f_{y'}^{F-F}
   + f_{y'}^{F-M} \right) \dd a'\dd y'}{ +\sigma \int _{y=0}^\infty
   \int_{a=y}^\infty e^{-ra} d_{a}^M \left( f_{y}^{M-M} + f_{y}^{M-F} \right)
   \dd a \dd y \big)}}
\end{equation}

Follow Equation~\eqref{eq:R0fromTy} to derive $R_0^\upsilon$ from $r^\upsilon$
and $T^\upsilon$. 

Arriving at sex-specific stable parameters from the two-sex parameters is
somewhat tricky. One might first estimate the stable total births to boys,
$\widehat{B^M}$ as:
\begin{equation}
\begin{split}
\label{eq:bboys0}
\widehat{B^M} = \frac{(1-\sigma)}{2} \int _{y'=0}^\infty \int _{a'=y'}^\infty
e^{-r^\upsilon a'} d_{a'}^F f_{y'}^{F-M} \dd a' \dd y' \\
+ \frac{\sigma}{2} \int _{y=0}^\infty \int _{a=y}^\infty e^{-r^\upsilon a}
d_{a}^M f_{y}^{M-M} \dd a \dd y
\end{split}
\end{equation}
and a first estimate of births to girls,$\widehat{B^F}$, by:
\begin{equation}
\begin{split}
\label{eq:bgirls0}
\widehat{B^F} = \frac{(1-\sigma)}{2} \int _{y'=0}^\infty \int _{a'=y'}^\infty
e^{-r^\upsilon a'} d_{a'}^F f_{y'}^{F-F} \dd a' \dd y' \\
+ \frac{\sigma}{2} \int _{y=0}^\infty \int _{a=y}^\infty e^{-r^\upsilon a}
d_{a}^M f_{y}^{M-F} \dd a \dd y
\end{split}
\end{equation}
but one notices that the stable population exposures, captured in
$e^{-r^\upsilon a'} d_{a'}^F$ and $e^{-r^\upsilon a}d_{a}^M$ do not
adequately reflect the fact that males and females are born 
into the populations with a particular ratio, $\hat{S}$,
which is approximately (but not exactly) $\frac{\widehat{B^M}}{\widehat{B^F}}$
(see Equations~\eqref{eq:bboys0}~and~\eqref{eq:bgirls0}). This initial ratio is
thus an approximation of $S$. $\hat{S}$ can then be entered into the equations,
replacing the number 2 in the denominator of $\sigma$, as follows:
\begin{equation}
\begin{split}
\label{eq:bboys0}
B^M = (1-\sigma) \frac{1}{1+\hat{S}} \int _{y'=0}^\infty \int
_{a'=y'}^\infty e^{-r^\upsilon a'} d_{a'}^F f_{y'}^{F-M} \dd a' \dd y' \\
+ \sigma \frac{\hat{S}}{1+\hat{S}} \int _{y=0}^\infty \int _{a=y}^\infty
e^{-r^\upsilon a} d_{a}^M f_{y}^{M-M} \dd a \dd y
\end{split}
\end{equation}
and analagously for girls:
\begin{equation}
\begin{split}
\label{eq:bgirls0}
B^F = (1-\sigma)\frac{1}{1+\hat{S}}\int _{y'=0}^\infty \int _{a'=y'}^\infty
e^{-r^\upsilon a'} d_{a'}^F f_{y'}^{F-F} \dd a' \dd y' \\
+ \sigma \frac{\hat{S}}{1+\hat{S}}\int _{y=0}^\infty \int _{a=y}^\infty
e^{-r^\upsilon a} d_{a}^M f_{y}^{M-F} \dd a \dd y
\end{split}
\end{equation}
giving the stable sex ratio at birth, $S^\upsilon$, as:
\begin{equation}
S^\upsilon = \frac{B^M}{B^F}
\end{equation}
$S$ only differs from $\hat{S}$ in the \nth{6} digit, and after the single
iteration just described obtains a constant value\footnote{The author offers
no proof of this single-iteration stability, but it was consistently observed
in numerical tests. There is very little ``wiggle-room'' for the sex ratio at
birth, despite the fact that the model allows $S$ to vary by $e_x$ of mother and
$e_x$ of father. Indeed, there are consistent sex-specific patterns in the sex ratio
at birth, both by mothers' and of fathers' $e_x$, but the
variariation in these is insufficient to move the stable value in continued
recursions of the above-described iteration.}. $S$ allows us to calculate the
remaining stable parameters. The both-sex stable birth rate, $b^\upsilon$ is given by:
\begin{equation}
b^\upsilon = \frac{1}{
            \splitfrac{\big(\frac{S^\upsilon}{1+S^\upsilon} \int _{y'=0}^\infty
            \int _{a'=y'}^\infty e^{-r^\upsilon a'} d_{a'}^F \dd a' \dd y'}{ + 
             \frac{1}{1+S^\upsilon} \int _{y=0}^\infty \int _{a=y}^\infty
             e^{-r^\upsilon a} d_{a}^M \dd a \dd y\big)}  }                   
\end{equation}
which can be used to derive the stable $e_x$-structure of males and females,
$c_y^\upsilon$ and $c_{y'}^\upsilon$, respectively:

\begin{equation}
c_{y'}^\upsilon = b^\upsilon \frac{1}{1+S^\upsilon} \int _{a'=y'}^\infty
e^{-ra'} d_{a'}^F \dd a'
\end{equation}
and
\begin{equation}
c_{y}^\upsilon = b^\upsilon \frac{S^\upsilon}{1+S^\upsilon} \int _{a=y}^\infty
e^{-ra} d_{a}^M \dd a'
\end{equation}
where:
\begin{equation}
1 = \int c_{y'}^\upsilon \dd y' + \int c_{y}^\upsilon \dd y
\end{equation}
the sex ratio in any given age, $S_y$, is:
\begin{equation}
S_y^\upsilon = \frac{c_{y}^\upsilon}{c_{y'}^\upsilon}
\end{equation}
and the the overall sex ratio,$S^T$ will be:
\begin{equation}
S^T= \frac{\int c_{y}^\upsilon \dd y}{\int c_{y'}^\upsilon \dd y'}
\end{equation}