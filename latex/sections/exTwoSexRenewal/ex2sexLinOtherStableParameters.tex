 \FloatBarrier
\label{sec:2sexlinearother}
Once two-sex linear $r$ and the stable proportion male of births, $\varsigma$,
have been found for the given $\sigma$, one may proceed to find the 
two-sex mean length of generation $T$ and stable $R_0$,
replacing the first guesses used in the iterative procedure described above.

We can derive the stable population sex ratio, $\bar{S}$:
\begin{equation}
\bar{S} = \frac{ \int_{y=0}^\infty \int_{a=y}^\infty \varsigma e^{-ra} d_{a}^M
\dd a \dd y}{\int_{y'=0}^\infty \int_{a'=y'}^\infty (1-\varsigma) e^{-ra'}
d_{a'}^F \dd a' \dd y'}
\end{equation}

The both-sex stable birth rate, $b$ is given by:
\begin{equation}
b = \Bigg[
            \splitfrac{\big( \int _{y'=0}^\infty
            \int _{a'=y'}^\infty e^{-r a'} (1-\varsigma) d_{a'}^F \dd a' \dd
            y'}{ + \int _{y=0}^\infty \int _{a=y}^\infty
             e^{-r a} \varsigma d_{a}^M \dd a \dd y\big)} \Bigg] ^{-1}          
\end{equation}
which can be used to derive the stable $e_y$-structure of males and females,
$c_y$ and $c_{y'}$, respectively:

\begin{align}
c_{y'} = b (1-\varsigma) \int _{a'=y'}^\infty
e^{-ra'} d_{a'}^F \dd a' \notag \\
c_{y} = b \varsigma \int _{a=y}^\infty
e^{-ra} d_{a}^M \dd a'
\label{eq:stablecy}
\end{align}
where of course,
\begin{equation}
1 = \int c_{y'} \dd y' + \int c_{y} \dd y
\end{equation}
