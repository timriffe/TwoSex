
The present method works as follows: Take constant base male and female
$e_y$-specific fertility rates, $F_y$ and $F_{y'}$. Given these rates and a male
and female $e_y$-structured population, we can generate the male and female predictions of
birth counts. We know from Section~\ref{sec:exdivergence} that the male and
female total counts tend to differ by less than if we were to repeat the
same for age-structured populations. However, the two total counts will still
differ, and so cannot be taken directly as the marginal birth count distributions from
which to calculate the association-free bivariate distribution-- the denominator
in Equation~\eqref{eq:expected}. To generate the expected count matrix, we
therefore calculate the cross-product of the male and female predictions and
divide by a mean of the male and female total predictions as follows:

\begin{equation}
\label{eq:meanexp}
\widehat{\mathbb{E}}(B_{y,y'}) = \frac{\widehat{B_y} \widehat{B_{y'}}}{M(\int
\widehat{B_y} \dd y, \int \widehat{B_{y'}} \dd y')}
\end{equation}
where $\widehat{B_y}$ is calculated using the constant base rate for males,
$F_y$, applied to male exposures, $E_y$, and likewise for females. $M()$ is a
mean function. For flexibility, one could use a generalized mean, such as the
stolarsky mean or lehmer mean, for $M()$, or any specific mean function, such as
the harmonic mean, if desired. The choice of mean function in the denominator
will be seen to have a trivial effect on the ultimate estimate of the
intrinsic growth rate.

Next, we estimate a consant ratio, $R_{y,y'}$, between the
observed and expected counts, which we take simply as:

\begin{equation}
R_{y,y'} = \frac{B_{y,y'}}{\mathbb{E}(B_{y,y'})}
\end{equation}
from the year of departure.

Using $R_{y,y'}$, we adjust the estimated expected distribution,
$\widehat{\mathbb{E}}(B_{y,y'})$ elementwise, and then rescale to sum properly
to $\mathbb{E}(B)$, the chosen mean of the male and female marginal
predictions:

\begin{equation}
\label{eq:ratioadj}
\widehat{B_{y,y'}} = R_{y,y'}\widehat{\mathbb{E}}(B_{y,y'})\frac{\int \int
\widehat{\mathbb{E}}(B_{y,y'})}{\int \int
R_{y,y'}\widehat{\mathbb{E}}(B_{y,y'})}
\end{equation}

Let us call Equation~\eqref{eq:meanexp} the mean
expected function, $\mathbb{M}(\widehat{B_y}, \widehat{B_{y'}})$, and
Equation~\eqref{eq:ratioadj} the ratio adjustment function, 
$A(R_{y,y'},\mathbb{M}(\widehat{B_y}, \widehat{B_{y'}}))$.

The marginal predictions of birth counts, $\widehat{B_y}$ and
$\widehat{B_{y'}}$, in the stable population will be determined by
fixed fertility rates and population exposures, which are a function of the
deaths distribution and the growth rate, $r$, as in the other $e_y$-structured 
models presented in this dissertation. As elsewhere, we will use sex-specific
fertility rates, i.e. specific to sex of birth and sex of progenitor, and divide
the production of each sex between the two sexes, weighted according to the sex
ratio at birth. The sex ratio at birth will not necessarily be the same in the
initial and stable populations, but does not vary by much. Assuming we know the
stable sex ratio at birth, $S$, we can estimate the growth rate, $r$, using:
 (sum( RatioAdj(RatioM, 
                                       Expect(rowSums(FexMM) * rowSums(dxM * exp(-r1 * .a)) * pmi,
                                               colSums(FexFM) * rowSums(dxF * exp(-r1 * .a)) * pfi, p))) +
                             sum(RatioAdj(RatioF, 
                                       Expect(rowSums(FexMF) * rowSums(dxM * exp(-r1 * .a)) * pmi,
                                               colSums(FexFF) * rowSums(dxF * exp(-r1 * .a)) * pfi, p))))
\begin{equation}
1 = A\left(R_{y,y'},\mathbb{M}\left(\;\int_{a=y}^\infty F_y^M
\varsigma d_a e^{-ra} \dd a, \int _{a'=y'}^\infty F_{y'}^M
(1-\varsigma) d_{a'} e^{-ra'} \dd a'\right)\right)
\end{equation} 

\todo{other 1/2 of eqn}

















