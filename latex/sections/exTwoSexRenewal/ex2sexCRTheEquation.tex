
The present method works as follows: Take constant base male and female
$e_y$-specific fertility rates, $F_y$ and $F_{y'}$. Given these rates and a male
and female $e_y$-structured population, we can generate the male and female predictions of
birth counts. We know from Section~\ref{sec:exdivergence} that the male and
female total counts tend to differ by less than if we were to repeat the
same for age-structured populations. However, the two total counts will still
differ, and so cannot be taken directly as the marginal birth count distributions from
which to calculate the association-free joint distribution-- the denominator
in Equation~\eqref{eq:expected}. To generate the expected count matrix, we
therefore calculate the cross-product of the male and female predictions and
divide by a mean of the male and female total predictions as follows:

\begin{equation}
\label{eq:meanexp}
\widehat{\mathbb{E}}(B_{y,y'}) = \frac{\widehat{B_y} \widehat{B_{y'}}}{M(\int
\widehat{B_y} \dd y, \int \widehat{B_{y'}} \dd y')}
\end{equation}
where $\widehat{B_y}$ is calculated using the constant base rate for males,
$F_y$, applied to male exposures, $E_y$, and likewise for females. $M()$ is any
mean function. For flexibility, one could use a generalized mean, such as the
Stolarsky mean or Lehmer mean, for $M()$, or any specific mean function, such as
the harmonic mean, if desired. The choice of mean function in the denominator
will be seen to have a trivial effect on the ultimate estimate of the
intrinsic growth rate.

Next, we estimate a constant ratio, $R_{y,y'}$, between the
observed and expected counts, which we take simply as:

\begin{equation}
\label{eq:getR}
R_{y,y'} = \frac{B_{y,y'}}{\mathbb{E}(B_{y,y'})}
\end{equation}
from the year of departure.

Using $R_{y,y'}$, we adjust the estimated expected distribution,
$\widehat{\mathbb{E}}(B_{y,y'})$ element-wise, and then rescale to sum properly
to $\mathbb{E}(B)$, the chosen mean of the male and female marginal
predictions:

\begin{equation}
\label{eq:ratioadj}
\widehat{B_{y,y'}} = R_{y,y'}\widehat{\mathbb{E}}(B_{y,y'})\frac{\int \int
\widehat{\mathbb{E}}(B_{y,y'})}{\int \int
R_{y,y'}\widehat{\mathbb{E}}(B_{y,y'})}
\end{equation}

Let us call Equation~\eqref{eq:meanexp} the mean
expected function, $\mathbb{M}(\widehat{B_y}, \widehat{B_{y'}})$, and
Equation~\eqref{eq:ratioadj} the ratio adjustment function, 
$A(R_{y,y'},\mathbb{M}(\widehat{B_y}, \widehat{B_{y'}}))$.

The marginal predictions of birth counts, $\widehat{B_y}$ and
$\widehat{B_{y'}}$, in the stable population will be determined by
fixed fertility rates and population exposures, which are a function of the
deaths distribution and the growth rate, $r$, as in the other $e_y$-structured 
models presented in this dissertation. 

For instance, since $\widehat{B_y} = P_yF_y$, we can determine the year $t$
births as follows:
\begin{equation}
\label{eq:ex2sexCRunity1}
B(t) = \int_{y=0}^\infty \int_{y'=0}^\infty
A\left(R_{y,y'},\mathbb{M}\left(\;Py(t)Fy, P_{y'}(t)F_{y'} \right)\right) \dd y
\dd y'
\end{equation} 
Of course population by remaining years, $P_y$, is a function of $P_a$ and the
deaths distribution, $d_a$, and we know that $P_a$ is a function of past births
and survival probabilities, $P_a = \varsigma B(t-a)p_a$, (assuming constant
mortality and proportion male of births, $\varsigma$). So, we may rewrite
Equation~\eqref{eq:ex2sexCRunity1} in terms of past births:
\begin{equation}
\label{eq:ex2sexCRunity2}
\begin{split}
B(t) = \int_{y=0}^\infty \int_{y'=0}^\infty
A\left(R_{y,y'},\mathbb{M}\left(\;Fy \int_{a=0}^\infty\varsigma
B(t-a)d_{a+y}\dd a\;\;,\right. \right. \\\left.
\left.F_{y'}\int_{a'=0}^\infty(1-\varsigma) B(t-a')d_{a'+y'} \dd a'\right)\right) \dd y \dd y'
\end{split}
\end{equation}
since the $p_a$ cancels out $\int _a^\infty d_a \dd a$ in the denominator of
Equation~\eqref{eq:dxredist}. As one may suspect, if the hypothetical
population is left to evolve endogenously under constant vital rates, $d_a$
and $F_y$, eventually the size of each new cohort will be related to the size
of the previous cohort by a fixed and constant factor equal to $e^r$, where
$r$ is the two-sex intrinsic growth rate. In this case, we may rewrite
Equation~\eqref{eq:ex2sexCRunity2} in terms of year $t$ births:
\begin{equation}
\label{eq:ex2sexCRunity3}
\begin{split}
B(t) = \int_{y=0}^\infty \int_{y'=0}^\infty
A\left(R_{y,y'},\mathbb{M}\left(\;Fy \int_{a=0}^\infty\varsigma
B(t)e^{-ra}d_{a+y}\dd a\right. \right.\;\;, \\ \left.
\left.F_{y'}\int_{a'=0}^\infty(1-\varsigma) B(t)e^{-ra'}d_{a'+y'} \dd a'\right)\right) \dd y \dd y'
\end{split}
\end{equation}
Dividing out by $B(t)$ we arrive at the familiar Lotka unity-equation form,
which allows us to isolate and estimate $r$ as a function of vital rates in the
initial year:
\begin{equation}
\label{eq:ex2sexCRunity}
1 = \int_{y=0}^\infty \int_{y'=0}^\infty
A\left(R_{y,y'},\mathbb{M}\left(\;\int_{a=y}^\infty F_y \varsigma d_a
e^{-ra} \dd a, \int _{a'=y'}^\infty F_{y'} (1-\varsigma) d_{a'} e^{-ra'} \dd
a'\right)\right) \dd y
\dd y'
\end{equation} 
Fertility rates, $F_y$ and $F_{y'}$ are standard $e$SFR, including both sexes
of offspring, and $\varsigma$ is used to weight sex of
progenitor, not sex of offspring. As will be seen below, in order to fully estimate $r$, it is best
to estimate $r$ and $\varsigma$ together, since there is a pattern to
$\varsigma$ over $y$, and the population structure is expected to change
somewhat between the initial and stable states. 


\FloatBarrier
