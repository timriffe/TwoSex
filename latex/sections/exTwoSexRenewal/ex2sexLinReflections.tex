 \FloatBarrier
I posit that there exists a formal identity to relate the various results
(e.g. $r^f{y'}$ to $r^{\upsilon (0)}$), just as \citet[pp. 56]{coale1972growth} relates 
the age-structured $r^m$ and $r^f$, but this fruit will be left on the tree
for the time being.

Most important, as is visible in Figure~\ref{fig:rupsilonlinear2sex}, there is
simply very little spread in growth rates between the positions of extreme
dominance. One intuitively wishes to see a non-linear two-sex model that
accounts for interactions between both sexes and remaining years of life, just
as one wishes, in an age-structured model to allow for fluid interactions
between sex and age. In such a model, the laws of supply and demand would move
$\sigma$ according to the relative weight of male and female exposure. However,
the distance between male and female dominance represents around twice the
maximum difference in $r$ that one would observe upon applying the more
sophisticated model. This statement assumes 1) that the interactive model is
bounded by the dominant cases presented here, and 2) that one is comparing with
the case of $\sigma = 0.5$, a prudent choice. 

As a secondary point, notice also that the present linear model holds rates
constant with respect to remaining life expectancy, but \textit{not} with
respect to age. From year to year the population structure with
respect to remaining life expectancy changes, as does the underlying age
structure. One could re-derive age-specific fertility rates from the
$e_x$-specific fertility rates used here, and would note that since the
weighting variable has changed with time, so too would the weighted sum of
the $e_x$-specific rates inherent in any age-specific rate. This observation
heeds \citet{stolnitz1949recent}, who point out several ways in which
fertility rates are indeed simply weighted sums of even more specific weights.
Prior to the formulation of the present model we have pointed out another
dimension in which age (parity-race-class) -specific rates are weighted sums, and we have exploited
that, short of holding very cross-classified rates constant, one observes
greater stability over time with $e_x$-classified rates. Holding
$e_x$-classified rates constant will force underlying age-specific rates to fold
and adapt with each passing year (albeit not much). Forcing age to adjust in
accord with constant $e_x$-specific rates appears to this author to be just as
palatable as forcing $e_x$-specific rates to change under the constraint of
constant age-specific rates-- perhaps moreso. This judegment is passed on having
compared the observed volatility in the two kinds of specific rates and deciding
$e_x$-specific rates are more reconciliable with the stable population
assumption of fixed rates. This difference is not necessarily large, and may in
any case be an accident of history, as we have not pondered upon why it is that
$e_x$-specific rates would hold more constant over time than age-specific rates.
Part of this may owe to inadequancies in the method used to redistribute
age-classified data to $e_x$-classified data, as the method is new, and has not
undergone scrutiny beyond this very dissertation.