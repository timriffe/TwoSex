 \FloatBarrier
\label{sec:ex2sexdomweightsstabstruct}
Using Equation~\eqref{eq:stablecy} we may calculate the
stable structure for any year from our test populations.
Figures~\ref{fig:stablevsinitUS}~and~\ref{fig:stablevsinitES} provide a glimpse
of how the 1975 and 2009 US and Spanish populations structured by remaining
years compare to their corresponding stable populations. For all cases,
$\sigma$ was assigned a value of 0.5. For the US, one notes that the stable
populations have differed little between these two time points. Indeed the
respective $r$ estimates for these two years, $-0.00076$ and $-0.00033$, were 
not very far from 0, which causes the \textit{walls} to be rather close to vertical in both
stable populations. Slight improvements in male and female mortality, however,
pushed the deaths distribution to higher ages, which caused the 2009 stable
population to \textit{elongate} relative to 1975. In comparing the stable with the initial
population structure, one may ask how it came to be that the initial pyramid
acquired such a high modal age-- this will be due primarily to 1) changes in
cohort sizes over time, the baby boomers having still be rather young in 1975,
but also due to slight shifting of the deaths distribution to higher ages over
time.
 
 % Figs made in exLotka2SexLinear.R 
\begin{figure}[!ht]
   \caption{US, stable ($\sigma = 0.05$) versus initial $e_y$-structures, 1975
   and 2009}
   \label{fig:stablevsinitUS}
        \centering
        \begin{subfigure}
                \centering
                \caption*{1975}
                \includegraphics[scale = .8]{Figures/exLotka2sexlinear1975USpyr}
        \end{subfigure}
        \begin{subfigure}
                \centering
                \caption*{2009}
                \includegraphics[scale = .8]{Figures/exLotka2sexlinear2009USpyr}
        \end{subfigure}
\end{figure}
 
The picture with the Spanish population is rather different, since the
respective $r$ estimates have changed so drastically over the period examined
here, dropping from $0.00870$ in 1975 to $-0.00714$ in 2009. The departure from
zero was in both years higher than that in the US, causing sharp tapering at the
base of the stable structure in 1975 and a rather \textit{pyramidal} shape in
2009, due to an endogenously shrinking population. One notes that in 1975 the
initial $e_y$ structure was very similar to the final stable form, but by 2009
these two structures were quite different. Initial (\textit{observed})
conditions were much more concentrated around the mode, due also to underlying
cohort sizes and continual and fast improvements in mortality.
 
 % Figs made in exLotka2SexLinear.R
\begin{figure}[!ht]
        \centering
        \caption{Spain, stable ($\sigma = 0.05$) versus initial
        $e_y$-structures, 1975 and 2009}
        \label{fig:stablevsinitES}
        \begin{subfigure}
                \centering
                \caption*{1975}
                \includegraphics[scale = .8]{Figures/exLotka2sexlinear1975ESpyr}
        \end{subfigure}
        \begin{subfigure}
                \centering
                \caption*{2009}
                \includegraphics[scale = .8]{Figures/exLotka2sexlinear2009ESpyr}
        \end{subfigure}
         
\end{figure}
\FloatBarrier
It has been mentioned before that the time series of observed $e_y$-structures
has held rather steady in last several decades (major wars and epidemics excluded),
due to the forces of mortality, fertility and (in)migration change having
compensated each other somewhat. Namely, for the Spanish population, decreases
in $r$ have been compensated somewhat by decreases in mortality that pushed the modal age at
death (modal remaining years at birth) to higher ages (remaining years) as well
as a large influx of relatively young migrants starting in the 1990s. The stable
model of course assumes constant mortality and fertility, as well as zero
migration, and so will not produce the same rather consistently observed tapered
base to the population structure year after year-- in the stable model the
(optimistic) tapered base will come about when low infant mortality is coupled
with high (roughly $>0$) intrinsic growth rates. Indeed, the drop in $r$ (i.e.
fertility) is the primary culprit for the wide base in the stable
$e_y$-structure for the Spanish population in 2009, which actually started some
two decades earlier. 

Figure~\ref{fig:cystabler} gives an example of the
influence of $r$ on the stable population structure for a given deaths
distribution (1975 US). In summary, growing populations tend to have higher
modal remaining years of life, shrinking populations are biased to lower
remaining life expectancies. The basic results would be the same for all years
of data for both populations dealt with in this dissertation.

Figure~\ref{fig:cystabledx} displays the 1975 US population assuming $r=0$ but
with different levels of mortality. The deaths distribution, $d_x$ was
regenerated according to 2 scalings of the underlying mortality rates, $\mu_x$.
All ages of $\mu_x$ were either scaled to 1.2 and 0.8 times lower mortality
levels-- equivalent to a bit more than 2 years lower and higher life expectancy 
at birth, respectively. $d_x$ was then rederived from the
newly scaled $\mu_x$. Notice that higher levels of mortality are on balance
lower structures with slightly wider bases, and vice versa for lower mortality
levels. 

\begin{figure}[!ht]
        \centering
        \begin{subfigure}
            \centering
            \caption{US 1975 stable population structure under
          different values of $r$.}
        	\label{fig:cystabler}
                \includegraphics[scale = .8]{Figures/exLotka2sexlinearPyrDiffr}
        \end{subfigure}
        \begin{subfigure}
            \centering
            \caption{US 1975 stable population structure under
          different levels of mortality, observed (center line), $\mu_x \times
          1.2$ and $\mu_x \times 0.8$; $r=0$.}
            \label{fig:cystabledx}
                \includegraphics[scale = .8]{Figures/exLotka2sexlinearPyrDiffdx}
        \end{subfigure}
\end{figure}
\FloatBarrier
Despite the great differences in mortality in Figure~\ref{fig:cystabledx}, the
shape of the stable population structure does not change substantially. Let us
not confuse this observation with the earlier claim that $e_y$-structured
populations are shaped in the first place by mortality and in the second place
by fertility. $e_y$-structured human populations obtain their
(rather invariant) characteristic curve due to mortality because new cohorts of
births are \textit{layered} (see Figure~\ref{fig:exrenewal}) according to the
deaths distribution, which itself has a characteristic shape. This could be seen
by comparing the present figures with those from populations in mortality crisis
years, and/or high infant mortality-- such years do no enter into study in this
thesis.

As with the single-sex case, one may measure the distributional distance between
the initial and stable conditions for two-sex $e_y$-structured populations. The
degree of separation, $\theta$, will be intermediate to those calculated for the 
single-sex cases, leaning closer to the male or
female indices depending on the value of $\sigma$ used to calculate the two-sex
stable population. Also as with the single-sex case, the damping
ratio may be calculated from the two-sex $e_y$-structured projection matrix 
presented in Section~\ref{sec:ex2sxprojmat}. Here the value is not
necessarily intermediate to the male and female single-sex cases, as seen in
Figure~\ref{fig:damping2}:

\begin{figure}[ht!]
        \centering  
        
        
          \caption{Damping ratios from two-sex $e_y$-structured projection
          matrices compared with single-sex values. US, 1969-2009 and Spain,
          1975-2009}
           % figure produced in /R/exLeslie.R
           \includegraphics{Figures/Damping2}
          \label{fig:damping2}
\end{figure}
Note that in both cases the $\sigma$ used to calculate the two-sex matrices was
.5, in principle \textonehalf~informed by male vital rates and \textonehalf~
informed by female vital rates. For the US, as one might expect, the damping
ratio was intermediate to the single-sex male and female ratios. For
the Spanish population, however, the two-sex model is expected to stabilize
faster than either of the corresponding one-sex models. We speculate that this
will be in large part due to the explicit balancing of the male and female
populations by the sex ratio at birth, which is higher in Spain than in the US. 
In the two-sex model, the Spanish population moves forward as a whole rather 
than quickly diverging due to its high sex ratio. This may be a desirable
property.

Our other summary measure of transient dynamics, the total absolute
oscillation of population structure from the initial to stable states
\citep{cohen1979cumulative}, in this case tends to be intermediate to the male
and female values (see Figure~\ref{fig:cohend22sex}). One exception are the
years 1975-6 for the Spanish population, where total oscillation in his model
would have been higher than for either single-sex model. Recall that the damping
ratio for each year of data was higher (faster stability) for the two-sex case 
than either single-sex case. Only the $\sigma$ value of 0.5 was tested, but here
we see that other values of $\sigma$ also would not guarantee damping ratios or
total oscillations bracketed by the single-sex cases. That we see this in the
simple linear combination of male and female models might be a precursor to
observing that such measures for non-linear models will also not necessarily be
bracketed by the male and female single-sex cases.

\begin{figure}[ht!]
        \centering  
          \caption{Total oscillation along the path to
       stability. Two-sex ($\sigma = 0.5$) versus single-sex $e_y$-structured
       projection trajectories; US 1969-2009 and Spain 1975-2009}
           % figure produced in /R/exLeslie.R
           \includegraphics{Figures/CohenD22sex}
          \label{fig:cohend22sex}
\end{figure}
 \FloatBarrier





