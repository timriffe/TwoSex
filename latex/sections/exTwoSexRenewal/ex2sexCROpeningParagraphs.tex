\FloatBarrier
The present section is motivated by the desire for a non-linear two-sex model of
$e_y$-structured population growth that takes advantage of the observation that
the observed joint distribution of births by remaining years of fathers 
and mothers, $B_{y,y'}$, is in our experience very close to the expected
distribution, taking the male and female marginals as given. We have noted 
that the overall distributional distance
between observed and expected counts is typically very small (see e.g.
Figure~\ref{fig:TotalVarobsexpex}), but we have not described any patterns that
may exist in the difference between these two distributions. 

There is indeed a common pattern to the departure between the observed and
expected distributions of $e_y$ structured births, as seen in Figure~\ref{fig:exCPratioexample} for
the example of 1975, US. We note the greatest departures in the four extreme
corners of this surface, which are also the locations in the joint distribution 
with the lowest birth counts. The average absolute departure percentage is here
only 9.4\% when weighted by the number of observed births.

\begin{figure}[!ht]
  \centering
    \caption{Example ratio of observed to expected joint distribution of
    $e_y$-classified births. US, 1975.}
     % figure produced in /R/crossprodRatioex.R
     \includegraphics{Figures/exCPRatioExample.pdf}
     \label{fig:exCPratioexample}
\end{figure}

The method presented here stays true to the stable population concept
 of fixed male and female $e_y$-specific fertility rates, but will add a second
 fixed component, a constant \textit{ratio} between $B_{y,y'}$ and
 $\mathbb{E}(B_{y,y'})$, which will be used as an adjustment instrument, in
 effect providing flexibility in the male and female marginal rates, while forcing
 consistency (via a mean expected count matrix), both in the total birth count
 and in the $e_y$-distribution of births. This method, described in
 following, will be seen to have several desirable properties for two-sex
 models.
 
\FloatBarrier


