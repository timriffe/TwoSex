\FloatBarrier

We have applied the iterative $r$ (and SRB) estimation procedure as outlined in
Section~\ref{sec:rexIPFfindr} to produce estimates of the intrinsic growth rate,
$r$ for each year of the US and Spanish data. Detailed results for $r$ and
the stable sex ratio at birth can be found in the tables of
Appendix~\ref{appendix:exallrestimates} alongside those of other remaining-years 
renewal models. The IPF method produces only a single
estimate of $r$, and there is less room for the demographer to influence 
results in one direction or another than there is for the dominance-weighted 
two-sex solution from Chapter~\ref{sec:ex2sexdomweights}. One can
arbitrarily choose which global mean to use at the outset for the IPF
procedure,\footnote{i.e., one may choose the mean to use in order to make the
male and female margins sum properly prior to initiating the iterative adjustment.}
but this will not move the $r$ estimate by much \textit{unless} $r$ is in
general far from 0 and the sex-gap in $r$ is large. This is not to be confused
with rate balancing by using a single joint fertility rate and the harmonic 
mean of male and female exposures.

Figure~\ref{fig:exIPFr} displays the trend in the arithmetic and harmonic
IPF-estimated intrinsic growth rates for the US and Spanish populations as
compared with the single-sex\footnote{These are identical to the
100\% sex-dominant growth rates from the weighted dominance method.} growth
rates. Results are in general consistent and believable. For the US population 
for these data it makes essentially no difference whether one were to choose the
 arithmetic or harmonic means at the initial marginal adjustment in the IPF 
 prodedure,\footnote{Both the arithmetic
and harmonic series are plotted, but there is no point in differentiating the
label as they are essentially superimposed.} but for the Spanish population the
choice entails a considerable difference in results. The arithmetic
adjustment yields results very close to the dominance-weighted two-sex $r$ with
$\sigma = 0.5$ from Chapter~\ref{sec:ex2sexdomweights} (comparison not shown),
but the harmonic rate behaves rather differently. During the mid-late 1990s the
harmonic IPF $r$ was not bracketed by the single-sex $r$ values (which we
need not demand of it). If one needed to then decide between the arithmetic and
harmonic means (or others) for the initial IPF marginal adjustment, it is
advised to base the decision on properties of the mean, in which case the
harmonic mean is preferable because it respects availability, homogeneity and
monotonicity. To be clear, the choice between means for the IPF marginal
adjustment is distinct from the choice of means for the method presented in
Chapter~\ref{sec:ex2sexschoen}. In Chapter~\ref{sec:ex2sexschoen}, mean
functions were used for exposures of each $(y,y')$ combination, whereas in the
IPF model the mean is applied globally and then birth matrix counts are \textit{shaken up}
until both margins match, which usually requires minimal \textit{shifting about}
of birth counts.

 \begin{figure}[ht!]
        \centering  
          \caption{Two-sex $r$ calculated using IPF for remaining-years
          classified data, compared with $r^m$ and $r^f$. US, 1969-2009 and
          Spain, 1975-2009.}
           % figure produced in /R/IPFex.R
           \includegraphics{Figures/exIPFr}
          \label{fig:exIPFr}
\end{figure}

Still, $r$ is not the only result of interest, and model differences in $r$
estimates are not so large that we are able to judge the practical consequences
of model choice. More information that would aid in comparing is provided on the
basis of other results, such as the \textit{intrinsic} $e$SFR, $f_y$, which is
distinct from the initial state $e$SFR, $F_y$, both for the present model and
for the case of generalized means. Failing such empirical judgment, one resorts
to other properties, such as competition and substitutability.
Figure~\ref{fig:eSFRIPF} shows initial versus stable fertility rates from the 
IPF method\footnote{Rates calculated with $r$
and $SRB$ from IPF method using initial harmonic mean marginal adjustment.}
specific to remaining years of life for males and females in two different years. 
For several years of the US data, there was virtually no difference between initial 
and stable rates (more so even than 2009 from Figure~\ref{fig:eSFRIPF}). For
the Spanish population, differences tended to be much larger, except for 1980
where the size of the initial-stable gap is similar to US 2009 (not shown). As
one may expect, differences between initial and stable rates are driven mostly 
by changes in the proportions male and female in population structure. 

\begin{figure}[ht!]
        \centering  
          \caption{Male and female initial and stable $e$SFR (IPF method). US
          and Spain, 1975 and 2009.}
           % figure produced in /R/IPFex.R
           \includegraphics{Figures/eSFRIPF}
          \label{fig:eSFRIPF}
\end{figure}

That there are differences between initial and stable $e$SFR, and that these
differences tend to all be in the same direction, on average implies a
difference between initial and stable $e_y$-total fertility rates, $e$TFR. Figure~\ref{fig:exIPFTFRdiff} displays this TFR difference (stable minus
initial $e$TFR) for each year of data, and it is informative to see that 1)
male and female $e$TFR differences roughly (not exactly) mirror each other, and
2) the female trend in this difference (or minus the male trend) follows the
overall pattern of development in $r$ for both countries. This is quite
different from the same exercise displayed in Figure~\ref{fig:eSFRharmonic} for
the harmonic mean stable population. Further, the direction of change between
the inititial and stable $e$SFR in this case is not even consistent with those
from the harmonic (or other) mean method.

\begin{figure}[ht!]
        \centering  
          \caption{Difference between stable and initial $e$TFR, males and
          females (IPF method). US, 1969-2009 and Spain,
          1975-2009.}
           % figure produced in /R/IPFex.R
           \includegraphics{Figures/exIPFTFRdiff}
          \label{fig:exIPFTFRdiff}
\end{figure}

The first observation
is to be expected for the present method, and will owe in the first place to the
harmonic-mean initial adjustment of the marginal male and female birth
predictions. Beyond the initial rescaling, further (but smaller) differences may
accrue from the iterative procedure itself, but these are reflected more in
differences in the distribution than in levels. Of course, male and female rates
in the IPF method adjust in opposite directions. 

The second observation owes
in part to the changes in stable population structure due to changes in the sex
ratio at birth and stable growth parameter. When $r$ moves down, the pyramid
becomes relatively bottom-heavy, but more so for males than females, and so the
sex ratio between 40 and 60 remaining years shifts toward females, which means
that stable TFR for females must drop in order balance with males (for whom the
movement is in just the opposite direction.). Further, decreases in SRB -- due
partly to real changes in propensity, but primarily to movement in rates
along the $e_y$-pattern to SRB -- imply increases in male rates. Note that the
dominance-weighted method also entails differences in fertility rates
between initial and stable states, but these are less worth exploring, as there
is no age-interaction or even proper male-female interaction, and these
differences may be primarily attributed to the domince parameter, $\sigma$,
which entails constant rescaling.

\begin{figure}[ht!]
        \centering  
          \caption{Difference coefficient, $\theta$, between stable and initial
          $e$SFR distributions, males and females (IPF method). US, 1969-2009
          and Spain, 1975-2009.}
           % figure produced in /R/IPFex.R
           \includegraphics{Figures/eSFRIPFdiffcoef}
          \label{fig:exIPFdiffcoef}
\end{figure}

Further worth mentioning for the IPF model are differences between the initial
and stable fertility distributions. This is notable because 1) the
dominance-weighted model has no such property, and 2) these differences behave
differently from the case for the mean-based rates presented in
Chapter~\ref{sec:ex2sexschoen}. The pattern to the
distributional difference coefficient, $\theta$, which measures the
difference between the initial and stable fertility rate distributions, follows
a trajectory that correlates closely with the absolute value of the series
presented in Figure~\ref{fig:eSFRIPF}. This we display in
Figure~\ref{fig:exIPFdiffcoef}, below. We do not take the extra step to
decompose the overall $e$SFR difference based on shape and level components, but
clearly iterative proportional fitting of birth distributions to new given
margins has the ability to mold rates -- indeed this has been touted as its
major advantage -- and this will exert an effect on $r$ in the stable model.
Further, we do not undertake any transient analysis of the present model, as was
done for the dominace-weighted model.

\FloatBarrier