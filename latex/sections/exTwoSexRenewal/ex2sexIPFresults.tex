\FloatBarrier

We have applied the iterative $r$ (and SRB) estimation procedure as outlined in
Section~\ref{sec:rexIPFfindr} to produce estimates of the intrinsic growth rate,
$r$ for each year of the US and Spanish data. The IPF method only produces a
single estimate of $r$, and there is little room for the demographer to force
results in one direction or another, as was the case for the dominance-weighted
two-sex solution from Section~\ref{sec:ex2sexdomweights}. One may
arbitrarily choose which global mean to use at the outset for the IPF procedure,
but this will not move the $r$ estimate by much in practical situations. We
compared the artithmetic and harmonic means, and only found differences when $r$
was very close to zero, in which case precision may have been more of an issue
than differences per se. We judge that the choice ought not be given too much
weight. Results presented in following used the harmonic mean for IPF. This is
not to be confused with rate balancing by using a single joint fertility rate
and the harmonic mean of male and female exposures.

Figure~\ref{fig:exIPFr} displays the trend in the intrinsic growth rate for
the US and Spanish popualtions as compared with the single-sex (sex-dominant)
growth rates. One sees that results are consistent and believable, but very
difficult to differentiate from those of the dominance-weighted $r$ from
Section~\ref{sec:ex2sexdomweights} or the generalized means (harmonic, say) of
Section~\ref{sec:ex2sexschoen}. $r$ is not the only result of interest, and so
it is more informative to compare on the basis of other results, such as the
intrinsic $e$SFR, which changes from the initial to stable states, both for the present model and
for the case of generalized means, or other properties, such as competition and
substitutability.

 \begin{figure}[ht!]
        \centering  
          \caption{Two-sex $r$ calculated using IPF for remaining-years
          classified data, compared with $r^m$ and $r^f$. US, 1969-2009 and
          Spain, 1975-2009.}
           % figure produced in /R/IPFex.R
           \includegraphics{Figures/exIPFr}
          \label{fig:exIPFr}
\end{figure}


\begin{figure}[ht!]
        \centering  
          \caption{Male and female initial and stable $e$SFR. US and
          Spain, 1975 and 2009.}
           % figure produced in /R/IPFex.R
           \includegraphics{Figures/eSFRIPF}
          \label{fig:eSFRIPF}
\end{figure}

\FloatBarrier