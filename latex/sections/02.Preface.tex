Demographers study the structure and phenomena of populations. In order to study
the magnitude and intensity of demographic phenomena, one must first remove
so-called distortions introduced by population structure. Population structure
is in turn treated as the product of demographic phenomena. Here,
demographic phenomena refer to natality and mortality and population stucture
refers to notions of time and sex. There are other forms of structure whose
effects we would also like to purge were the measurement of demographic phenomena 
our primary objective, and there are other kinds of phenomena that must also be
measured if the analysis of structure were the primary objective. This is not
so. Our objective is to study an abstraction of population processes, namely the
renewal model for closed populations structured by sex and time.

That we are concerned with the role of both sexes in the modelling of population
renewal should be no surprise, as humans reproduce sexually. That it is a
challenge for models to incorporate information from both sexes in the modelling
of fertility (marriage, reproduction) has been firmly established since
\citet{karmel1947relations}. This task is challenging because one must acheive
one result, a fertility outcome, from two information sources (males and
females), which when handled apart produce incongruous results. There is no
obviously correct way to acheive this balancing act, although a large number of
suggestions have been made. We typically call these suggestions ``solutions'',
but they are not solutions in the sense of a solution to a math problem. A
solution in the context of the present problem simply means that a reasonable
result is produced in accordance with a predefined set of modeling objectives
decided upon either by the demographer or consensus. The problem has not been
(and may never be) solved in the sense of a necessary and best model. Instead,
solutions are weighed in terms of fulfilling desirable properties versus theoretical 
or practical parsimony.

The balancing of the sexes in models without considering age is
much simpler both conceptually and in practice, as it is just a matter of
choosing some middle ground between males and females. Most of the literature on
the two-sex problem, and the properties that demographers deem desirable in
solutions, deals with the time-structuring variable, age. That modelling
decisions must be made both with respect to the interaction between sexes and
the interaction between ages makes the problem an altogether complex one. 

What is age but time passed since birth? Thus, age is \textit{time} with respect
to one of the demographic phenomena that we incorporate in models of population
renewal. Again, the reason why demographers care about age is because all
demographic phenomena vary by age in known ways, and so in order to measure the
pure force of a demographic phenomenon one does well to take age into account.
Age counts up from birth, starting at the beginning. We measure milestones and
the lifecourse in terms of age; statistics are collected by age or year of
birth, and age is in short \textit{known}. We do not know when we will die, but
this is also something that demographers think on. Namely, in properly
accounting for age (time since birth), we may faithfully approximate death
probabilities for each age, and therein know something about our probable time
of death. This later question is a subject of considerable interest!

Demographers regularly think about, estimate, a probable time of
death for persons of a particular ages-- that were born in particular years.
Might we not also venture, to take things a step further-- What if age were 
counted down to death instead of up
from birth? Literally, what might we learn about demographic phenomena and
population structure if beyond age (and due to the information we glean from
age) we structured populations by sex and remaining years of life? That is a 
big job, and we will fail to complete it, instead laying out only the groundwork
for population renewal models wherein age is exchanged for remaing years of life.

It my stance that population renewal models ought to account for both sexes, and
for this reason roughly equal attention is given in this dissertation to the
two-sex problem-- a problem that never goes away-- and population structured
by remaining years of life-- a novel concept that must be hashed out before
again complicating things with the two-sex problem. I apologize for any lack of
rigour on both fronts. Namely, I neither reproduce formal proofs for the
properties of the solutions that I treat, nor do I provide proofs for the
(many) claims that I make. The filling of this gap is left for a later day--
either someone will do the favor of proving my claims right or wrong or I
will find the time to learn to do so. Instead we are lead in this dissertation
primarily by intuition, and I have placed a premium on the
data-grounded demonstration of the methods I propose. After all, might we not
wish to free these formal demographic musings from the vacuum and see what might
be learned? There is therefore the risk that some conceptual error or
miscalculation of mine-- and all errors and miscalculations herein are mine
alone-- will bring the work down. This is my risk alone, but the possibility is
not that distressing. Rather it is inherent to the business of charting new
territory, and this I have every pretension of doing. If the maps I draw are no
good, the territory explored may still be good, and so I will rest well.

So it is that I say that sex and time
are the structuring variables of interest in two-sex models of population
renewal.

The original objective for this dissertation was to hash out a survey of two-sex
solutions and implement them in a standard and reproducible format while
applying each to contemporary datasets. I chose the topic in order to force
myself to learn formal demographic skills, which were considerably less
developed at the outset. And so I began at the begining, collecting
all the materials I could locate on the two-sex problem, and reproducing methods
in no particular order. After a couple successful attempts on ad hoc acquired
data, I came to realize that all the methods in my scope will essentially
require or the same input data-- basic exposures, and births cross-tabuated by
sex and age of father and mother-- so I diverted attention to standardizing some
datasets to use throughout this dissertation. When the mathematics or
presentation style were over my head, I typically took a few steps back to some 
earlier or less complex method, or alltogether went back to the basics 
in \citet{sharpe1911problem}, \citet{kuczynski1932fertility}, \citet{coale1972growth} 
or \citet{caswell2001matrix}. Some methods that were beyond my grasp in the 
begining \citep[e.g.][]{mitra1978derivation} were finally understood and implemented later down the road. 
Others I still don't understand\cite[e.g.][]{choo2006estimating}, despite having
reproducible code!

All along I had no vision or pretense of designing a new method, but I rather
na\"{\i}vely assumed that familiarity with the tools at hand would lead me to
some minor tweak or meaningful critique of the existing palette of methods at
hand. For three years I did not manage to produce anything novel, only a few
branches of the above-mentioned survey, and the intensity of my task waned.


