% shoot for 150 words
%The most widely used measures and models of population formally assume
%fertility as a function of females only, 'female dominance'. Early formal
%demographers proposed presently used indicators, such as $R_{0}$ and TFR as
%approximations, and accepted the assumption of female dominance as a matter of
%convenience rather than conviction. The magnitude and direction of error in
%measures of fertility and reproductivity due to this practice are not always
%clear. Various adjustment procedures have been proposed in the past several
%decades, but there remains no consensus about the most desirable analytic
%method. This dissertation recapitulates the discussion,  assesses the proposals
%that have been made about the two-sex problem, makes methods available in
%reproducible programming code and comparatively applies this suite of methods
%to two contemporary populations, Spain and the USA. Extra sections make
%recommendations for two-sex adjustments in cohort component population
%projections.

One of the foremost problems in formal demography has been that of including
information from the vital rates for both sexes in models of population renewal
and growth, the so-called two-sex problem, which may be thought of as a subset
of the analytical problems entailed by multigroup population modelling. This
dissertation characterizes the two-sex problem by means of decomposing the vital 
rate components to the sex-gap between the male and female single-sex stable growth
rates. A suite of two-sex models for age-structured models from the
literature are presented. A new variety of age-structure, age based on remaining
years of life, is presented. Analogous models of population growth for the
single-sex and two-sex cases are developed for populations structured by
remaining years of life. It is found that populations structured by remaining
years of life produce less two-sex divergence than age-structured models,
thereby reducing some of the trade-offs inherent in two-sex modelling decisions.
In general, remaining-years-structured models are found to be more stable over 
time and closer to their ultimate stable forms than age-structured models. Models of
population growth based on remaining-years structure are found to diverge from
like-designed age-structured models, and this divergence is characterized in
terms of the two-sex problem.
