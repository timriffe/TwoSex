% shoot for 150 words
The most widely used measures and models of population formally assume fertility as a function of females only, 
'female dominance'. Early formal demographers proposed presently used indicators, such as $R_{0}$ and TFR as 
approximations, and accepted the assumption of female dominance as a matter of convenience rather than conviction.
 The magnitude and direction of error in measures of fertility and reproductivity due to this practice are not 
 always clear. Various adjustment procedures have been proposed in the past several decades, but there remains no 
 consensus about the most desirable analytic method. This dissertation recapitulates the discussion,  assesses 
 the proposals that have been made about the two-sex problem, makes methods available in reproducible programming 
 code and comparatively applies this suite of methods to two contemporary populations, Spain and the USA. Extra 
 sections make recommendations for two-sex adjustments in cohort component population projections.
