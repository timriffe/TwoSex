 \FloatBarrier
It has been pointed out that the ideal functional form of a two-sex solution
cannot be empirically determined (Das Gupta, Schoen, Keyfitz, others). This is
because fertility is always undergoing secular changes, to the effect that
one cannot simply calibrate an ideal mean function (if a mean function were the
correct choice) net of outright both-sex fertility change. This we observe above
all with the Spanish data used in this dissertation: from 1975 until the mid
1990s fertility levels dropped so rapidly that in most cases the year $t+1$
brith count fell below that which would have been predicted by the year $t$
male or female rates-- even as the gap between male and
female total fertility rates was rather wide in those years. Even in less
extreme situations, where the year $t+1$ birth count is intermediate to that which would
have been predicted by the male and female year $t$ rates one is ubable to
separate the effects of relative changes in male versus female exposure from
simple changes in rates. That is to say, if there is some push and pull between
male and female rates, this cannot be measured if rates on the whole are either
rising or falling-- just as it is difficult to measure the net rising and
falling of rates when there is feedback and separation between male and female
rates. Even if one had a very large amount of data conformable to this problem
and an appropriate statistical technique so as to mete out these differences 
and estimate a function that could separate and capture the effects of our
imagined push and pull between male and female rates, it would be easy to 
suppose that this ideal function might itself change according to certain
conditions or certain periods.

This empirical obstacle has led demographers to devise a set of axioms,
necessary characteristics, that the ideal function should abide by in order that it coform with our
expectations. Here we will enumerate all such axioms located in the literature
before briefly discussing them in turn. Here, $M()$ is any function that
determines the both-sex rate using male, $P^m$, and female, $P^f$, exposures as
inputs. These exposures may be classified by some other variable, such as age, but
subscripts are ignored here unless pertinent.

\begin{description}
  \item[Availability:] $M(P^m, P^f) = 0$ if $P^m = 0$ or $P^f = 0$. Members of
  both sexes must be present in order for there to be a non-zero positive rate.
  \item[Homogeneity:] $kM(P^m, P^f) = M(kP^m, kP^f)$. Equal
  changes in the supply of males and females must lead to an equal change in the
  number of births (marriages).
  \item[Monotonicity:] for $k > 1$, $M(kP^m, P^f) \ge M(P^m, P^f)$ (and vice
  versa). If the supply of one sex increases while the other sex is held constant, the number of
  borths (marriages) cannot decrease.
  \item[Competition:] if exposure in age $x$ for males is increased by some
  factor, but all other male and female ages are held constant, monotonicity
  applies to age $x$ of males, but rates for ages $<x$ or $>x$ may only decrease
  or say the same. In this case, ages closer to $x$ should be more affected than
  ages farther from $x$. Females rates for all ages combined with male age $x$
  may only increase or stay the same.
  \item[Bracketing:] $M(P^m, P^f) > min(F^m, F^f)$ and $M(P^m, P^f) < max(F^m,
  F^f)$. The both-sex rate must be intermediate to the single-sex rates.
  \item[Proportionality in the extreme:] in situations of very extreme sex-ratio
  imbalance, changes in the amount of the minority sex should be reflected proportionately
  in the two-sex rate.
\end{description}

These axioms will now be briefly reflected upon in turn.

\paragraph{Availability:} This is the most elemental axiom, as it essentially
states a truism: If one sex is absent, there can be no reproduction in a species
that reproduces sexually. For the sake of philosophical completeness, we state
the following: 1) Assisted reproduction requires both sexes, so this is no
retort; 2) At present, technology that would negate this axiom, human
parthenogenesis, is not fully developed, although ther have been recent
advances\citep{revazova2007patient}. If and when technology would permit asexual
human reproduction, there will be legal hurdles, costs, and apoption lag. 
That is to say, potential anecdotes that would negate this axiom
will in any case not affect fertility rates in a significant way within the time
horizons that demographers currently project. This is not a tongue-in-cheek
observation, as technology in general is known to affect fertility in myriad
ways. For instance, in vitro fertilization and other forms of assisted
reproduction have had noticeable effects in the fertility and sex ratio at birth from particular
age groups.

\paragraph{Homogeneity:} This author finds the axiom of homogeneity to be on the
whole harmless, but not necessarily true. Homogeneity essentially states that
there are no scaling effects. It is easy to imagine that population size will
constrain or determine much of what happens within populations. This is
especially so when we think in terms of social organization, contact
opportunities and the countless other structural factors that may affect the
practice of mating and by extension fertility. Population size has been
given more attention in non-human ecology \citep{donalson1999population} than in
human demography, where considerations of population size have been primarily framed in terms of
carrying-capacities \citep[see e.g.][]{cohen1995human,hopfenberg2003human}. This
author is only aware of scaling in demographic process when studied as complex systems via
agent based modelling (ABM) \citep[e.g.][]{bruch2010scaling}. While ABMs have
been used to studying fertility and marriage \citep{billari2002wedding},
indirect scaling effects in such models have not been explicitly studied, not
have scaling effects been in introduced explicitly.

\paragraph{Monotonicity}: this may seem intuitive, but if viewed from a
sociological standpoint (i.e. a more complex model about how things work), it seems plausible that increased
competition could actually lead to a decrease in total births (marriages)



Bracketing: the two-sex instrinsic growth rate, $r$, must fall between the male
and female single-sex intrinsic growth rates, $r^m$, $r^f$, respectively

in instances of very large imbalances in the sex ratio, increases in the supply
of the minority sex lead to proportional increases in the number of marriages
\todo{Feeney, Pollak, McFarland, Schoen define more}
