 \FloatBarrier
It has been pointed out that the ideal functional form of a two-sex solution
cannot be empirically determined (Das Gupta, Schoen, Keyfitz, others). This is
because fertility is always undergoing secular changes, to the effect that
one cannot simply calibrate an ideal mean function (if a mean function were the
correct choice) net of outright both-sex fertility change. This we observe above
all with the Spanish data in this dissertation: from 1975 until the mid 1990s
fertility levels dropped so rapidly for males and females both that in most
years the year $t+1$ fertility fell outside the bounds of either sex of the year
$t$ fertility. This was observed even though the gap between male and female
total fertility was rather wide in those years. Even if secular fertility change
were slight enough, and the sex gap in fertility wide enough, as to virtually
always be bracketed by male and female rates, one could not faithfully separate
the effects of relative changes in exposure from changes in rates.


Monotonicity: if the supply of one sex increases while the other sex is held
constant, the number of marriages cannot decrease, and vice versa [this may
seem intuitive, but if viewed from a sociological standpoint (i.e. a more
complex model about how things work), it seems plausible that increased
competition could actually lead to a decrease in total births (marriages)]

Homogeneity: equal changes in the supply of males and females must lead to an
equal change in the number of marriages. [my stance: this is the same as saying
there are no scaling effects, and I find it an excessive and
capricious condition.]

Availability: this is a no-brainer

Competition: there are different aspects to this axiom. 1) if exposure in one
age increases, holding other ages constant, events may increase

Bracketing: the two-sex instrinsic growth rate, $r$, must fall between the male
and female single-sex intrinsic growth rates, $r^m$, $r^f$, respectively

in instances of very large imbalances in the sex ratio, increases in the supply
of the minority sex lead to proportional increases in the number of marriages
\todo{Feeney, Pollak, McFarland, Schoen define more}
