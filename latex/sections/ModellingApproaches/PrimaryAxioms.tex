 \FloatBarrier
 \label{sec:axioms}
It has been pointed out that the ideal functional form of a two-sex solution
cannot be empirically determined (Das Gupta, Schoen, Keyfitz, others). This is
because fertility is always undergoing secular changes, to the effect that
one cannot simply calibrate an ideal mean function (if a mean function were the
correct choice) net of outright both-sex fertility change. This we observe above
all with the Spanish data used in this dissertation: from 1975 until the mid
1990s fertility levels dropped so rapidly that in most cases the year $t+1$
birth count fell below that which would have been predicted by either of the
year $t$ male or female rates-- despite there having been a wide the gap between
male and female total fertility rates in those years. 

Even in less
extreme situations, where the year $t+1$ birth count is intermediate to that which would
have been predicted by the male and female year $t$ rates, one is unable to
separate the effects of relative changes in male versus female exposure from
simple changes in rates. That is to say, if there is some push and pull between
male and female rates, this cannot be measured if rates on the whole are either
rising or falling-- just as it is difficult to measure the net rising and
falling of rates when there is feedback and separation between male and female
rates. Even if one had a very large amount of data conformable to this problem
and an appropriate statistical technique so as to mete out these differences 
and estimate a function that could separate and capture the effects of our
imagined push and pull between male and female
rates\footnote{\citet{alho2000competing} come close to this ideal.}, it would be easy 
to suppose that this ideal function
might itself change according to certain conditions or certain periods.

This empirical obstacle has led demographers to devise a set of axioms,
necessary or desired characteristics, that the ideal two-sex fertility
(or marriage) function should abide by in order that it conform with our
expectations. Here we will enumerate all such axioms located in the literature 
before briefly discussing them in turn. Here, $M()$ is any function that
determines the both-sex rate using male, $P^m$, and female, $P^f$, exposures as
inputs. These exposures may be classified by some other variable, such as age, but
subscripts are ignored here unless pertinent.

\begin{description}
  \item[Availability:] $M(P^m, P^f) = 0$ if $P^m = 0$ or $P^f = 0$. Members of
  both sexes must be present in order for there to be a non-zero positive rate.
  \item[Homogeneity:] $kM(P^m, P^f) = M(kP^m, kP^f)$. Equal
  changes in the supply of males and females must lead to an equal change in the
  number of births (marriages).
  \item[Monotonicity:] for $k > 1$, $M(kP^m, P^f) \ge M(P^m, P^f)$ (and vice
  versa). If the supply of one sex increases while the other sex is held constant, the number of
  births (marriages) cannot decrease.
  \item[Symmetry:] for $P^m = P^f$, $M(kP^m, P^f) = M(P^m, kP^f)$. 
  \item[Competition:] if exposure in age $x$ for males is increased by some
  factor, but all other male and female ages are held constant, monotonicity
  applies to age $x$ of males, but rates for male ages $<x$ or $>x$ may only
  decrease or say the same. 
  \item[Subsitution:] The size of competition effects varies directly with
  age-proximity to $x$ among males. For instance, males of age 24 are closer
  substitutes for males of age 25 than are males of age 20.
  \item[Bracketing:] $M(P^m, P^f) > min(F^m, F^f)$ and $M(P^m, P^f) < max(F^m,
  F^f)$. The both-sex rate must be intermediate to the single-sex rates.
  \item[Proportionality in the extreme:] in situations of very extreme sex-ratio
  imbalance, changes in the amount of the minority sex should be reflected proportionately
  in the two-sex rate.
\end{description}
 
These axioms will now be briefly reflected upon in turn.

\paragraph{Availability:} This is the most elemental axiom, as it essentially
states a truism: If one sex is absent, there can be no reproduction in a species
that reproduces sexually. For the sake of philosophical completeness, we state
the following: 1) Assisted reproduction requires both sexes, so this is no
retort; 2) At present, technology that would negate this axiom, human
parthenogenesis, is not fully developed, although ther have been recent
advances\citep{revazova2007patient}. If and when technology would permit asexual
human reproduction, there will be legal hurdles, costs, and apoption lag. 
That is to say, potential anecdotes that would negate this axiom
will in any case not affect fertility rates in a significant way within the time
horizons that demographers currently project. This is not a tongue-in-cheek
observation, as technology in general is known to affect fertility in myriad
ways. For instance, in vitro fertilization and other forms of assisted
reproduction have had noticeable effects in the fertility and sex ratio at birth from particular
age groups.

\paragraph{Homogeneity:} This author finds the axiom of homogeneity to be on the
whole harmless, but not necessarily true. Homogeneity essentially states that
there are no scaling effects. It is easy to imagine that population size will
constrain or determine much of what happens within populations. This is
especially so when we think in terms of social organization, contact
opportunities and the countless other structural factors that may affect the
practice of mating and by extension fertility. Population size has been
given more attention in non-human ecology \citep{donalson1999population} than in
human demography, where considerations of population size have been primarily framed in terms of
carrying-capacities \citep[see e.g.][]{cohen1995human,hopfenberg2003human}. This
author is only aware of scaling in demographic process when studied as complex systems via
agent based modelling (ABM) \citep[e.g.][]{bruch2010scaling}. While ABMs have
been used to studying fertility and marriage \citep{billari2002wedding},
indirect scaling effects in such models have not been explicitly studied, not
have scaling effects been in introduced explicitly.

\paragraph{Monotonicity:} This may seem intuitive, but if viewed from a
sociological standpoint, it seems plausible that increased competition could actually lead to a
decrease in total births (marriages) via different mechansisms that we will
briefly hash out. Imagine a more complex model wherein individuals must
apportion time / effort / resources between mate search costs and competition. 
In the case of an increase in males while holding females
constant, increased competition between males in mate selection could
 scale non-linearly to-- and offset-- the standard predicted increase in matings
 that would result from increased male pressure on the market. In a different
 scenario, females faced with abundant potential mates may actually decrease their 
 search efforts and postpone the mate search until a later time,
 thereby acting to supress rates. Were this later effect present in the model,
 the effect of increases in one sex would be ambiguous, as it would depend on
 the relative forces of male pressure and female deprioritization. In yet
 another model scenario, a proportion of males faced with increased competition may 
 indeed cease to compete, and remove themselves from the market, thereby 
 decreasing pressure from the side of abundant males. Other similar effects may
 be dreamed up wherein the results of an increase in one sex only could be
 coplex and counterintuitive. 
 
 None of these complex model scenarios are particularly amenable to inclusion in
 a practical analytic model of mating / marriage / fertility markets. However,
 in indicating such potential countervailing forces-- all reasonable in the mind
 of this author-- one may at least question the necessity of holding
 monotonicity as \textit{axiomatic} in the sense of a functional necessity- a
 criterion by which the adequacy of a model may be judged. 

\paragraph{Symmetry:} It appears that symmetry, treated as an axiom, is also
inappropriate. Males and females differ not only with respect to vital rates,
but with respect to mate preferences and behavior \citep{buss1989sex}. There is
also evidence for variation in these differences by group size
\citep{fisman2006gender}, which plays into the previous axiom of monotonicity.
Clearly, if males and females have different preferences and also react
differently to differences in group size, we should expect different outcomes
from hypothetical sex-complementary compositional changes in the mating market.
For this reason we may also conclude that symmetry, though likely to be a
characteristic of the functional form assigned to the male-female
dependant fertility (marriage) function, ought not be given the status of an
axiomatic requirement for a good and proper model. That the functional forms
often used for marriage and fertlity often were symmetric with respect to the
sexes need not be drawback, but we ought not grant this characteristic post hoc
status as an axiom.

\paragraph{Competition:} It seems reasonable that, holding mate supply constant,
increases in matings in age $x$ either decrease or have no effect on ages
close to $x$ of the same sex. Some two-sex models have accounted for this axiom
\citep{henry1972nuptiality, mc1975models, choo2006estimating}, sometimes via
explicit preference functions \citep{parlett1972can, pollard1993interaction} but
many have not. These models are considerably more complex to implement than the 
alternatives. It is unclear to this author whether this axiom should be treated
as a requirement or a desirable property. 

\paragraph{Substitution:} In the case of inter-age (or inter-group) competition
for mates, it is intuitive that, since age can be thought of as
continuous, competition effects should vary inversely in magnitude as a function
of distance to the age that hypothetically experiences a sudden change in
effective population. In the case that explicit preference functions are used, this axiom
is already dealt with, and \citet{choo2006estimating} also has this
characteristic. \citet{keilman1999female} only
detected small effects for inter-age competition using data from Norway.

\paragraph{Bracketing:} The interpretation of this axiom depends on context.
In the first instance, it states that the two-sex instrinsic growth rate, $r$,
must fall between the male and female single-sex intrinsic growth rates, $r^m$, $r^f$, 
respectively. Many authors have treated this axiom as a requirement
\citep{pollard1948measurement,
coale1972growth, gupta1976interactive, mitra1978derivation}, others have
argued otherwise \citep{gupta1973,schoen1981harmonic}, and indeed it has even
been proven an unreasonable condition \citep{yellin1977comparison}. This author
agrees that the single-sex growth parameters will not serve as two-sex bounds
because they are calculated in unreasonable isolation, namely, each constrained
by its own sex-specific fertility rates and without interaction between ages of
each sex. That is to say, in isolation the single-sex models may behave
strangely and not bound the true trajectory of the total population.

A second domain of bracketing could be in terms of the total births predicted by
males and females for year $t+1$ using the ASFR and sex-specific exposures. In
this case, the main difference is that the offspring of each sex is of both
sexes. In this case, bracketing appears a less troublesome condition, as we
essentially remove fertility sex-ratio constraints from the boundary
predictions. Absent secular change in birth rates or the age-pattern of fertility, 
we would expect one sex to ovestimate and the other to underetimate the birth count to be observed in
future years. 

\paragraph{Proportionality in the extreme:} In other words, at some point along
the continuum of potential sex ratios, the minority sex should experience
\textit{saturation}, in the sense that further increases in the majority will
not result in increased matings. In this same scenario, one may imagine that,
while still within the same extreme order of sex ratio magnitude, a unit
increase in effective population of the minority sex will lead to a unit
increase in predicted births (marriages). It is doubtful that this
situation would ever arise in a real projective scenario 
