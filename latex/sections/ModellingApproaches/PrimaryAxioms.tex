 \FloatBarrier
It has been pointed out


Monotonicity: if the supply of one sex increases while the other sex is held
constant, the number of marriages cannot decrease, and vice versa [this may
seem intuitive, but if viewed from a sociological standpoint (i.e. a more
complex model about how things work), it seems plausible that increased
competition could actually lead to a decrease in total births (marriages)]

Homogeneity: equal changes in the supply of males and females must lead to an
equal change in the number of marriages. [my stance: this is the same as saying
there are no scaling effects, and I find it an excessive and
capricious condition.]

Availability: this is a no-brainer

Competition: there are different aspects to this axiom. 1) if exposure in one
age increases, holding other ages constant, events may increase

Bracketing: the two-sex instrinsic growth rate, $r$, must fall between the male
and female single-sex intrinsic growth rates, $r^m$, $r^f$, respectively

in instances of very large imbalances in the sex ratio, increases in the supply
of the minority sex lead to proportional increases in the number of marriages
\todo{Feeney, Pollak, McFarland, Schoen define more}
