
A complete survey of two-sex models would have been a gargantuan task, given
that the modelling challenge has in recent years branched into various other
disciplines and approaches. Covering all of these would have precluded the
developments to be presented in the remaining chapters of this dissertation,
which we deem of higher value. Here we attempt to summarize other avenues that
work on the two-sex problem has taken, some of which could have been included in this work. Most were excluded because they either exceeded the
complexity desired in this dissertation, exceeded data constraints, exceeded our
comprehension, or exceeded our implementation ability. Nonetheless, we will give
superficial attention to the varieties of two-sex models otherwise excluded.

\subsection{Henry's panmictic circles} 
Another widely-known iterative method, 
most comparable with IPF, is that described in \citet{henry1972nuptiality},
Louis Henry's method of matrix decomposition via \textit{panmictic circles}, which entails somewhat more
implementation effort than does IPF, and is likely more substantively appealing. 
\citet{mc1975models} only compared the
IPF method with that of Henry, and \citet{wijewickrema1980weak} used this method 
in his dissertation study of weak ergodicity in the two-sex problem in the
context of marriage\footnote{And most interestingly, was with little
effort able to produce apparent limit cycles in the stable population.}. We do
not implement this method\footnote{There are some ambiguities in terms of 
implementing the method when working with unabridged data that also contains 
many zeros. We did not succeed in translating the method from an
abridged table to single ages without also producing negative numbers.}, though
it most resembles IPF, and it has all of the same desirable properties, yet
gives different results. This extension is left for future work.

\subsection{Combined marriage-fertility models} 
\citet{karmel1947relations}
first proposed solving the male-female fertility discrepancy by shifting
fertility to \textit{couples} as the primary unit of reproduction. This choice
makes perfect sense theoretically, but does not solve the essential modelleing
challenge so much as displace it, as this family of models is
must somehow balance the numbers of couples. Balancing couple formation is the
same operationally as balancing birth predictions. It is easy enough to see that
this family of models, when built well and based on appropriate data, is
superior to our own modelling of unconditional fertility. However, the
obstacles are greater to operationalizing this family of model, as one must
decide what to consider a couple (marriage, cohabitation, sexual
partner), how to incorporate ruptures (divorce, separation, widowhood) and
couple re-formation. The more couple varieties in the model, the more fertility
rates must be specified, and these must of course be estimable. Further
complexity may be added by considering durations, such as duration since 
couple formation, or duration since last birth, states such as
education, or of course any of the proximate determinants of
fertility\citep{bongaarts1982fertility, bongaarts1978framework,
bongaarts1983fertility}. 

In a projective setting with some simplifying assumptions, some mix of the above
considerations is indeed possible given that the demographer is in possession of
the appropriate population stocks, couple transition probabilities and
vital rates\footnote{e.g. \citep{schoen1987modeling} gives implementation
guidance.}. In practice 1) there are (still) not many populations for which this
endeavor is possible, 2) it is not clear whether the two-sex stable population
structure or growth rate would differ from that produced by the simpler model,
3) one increases the possible sources of error in that more data sources are
required and 4) assumptions (or data constraints) about what kind of couples
matter for fertility will likely affect results. In short, for this
dissertation, these other considerations would have been more of a distraction.
We have kept effort to modelling the part of the model that demographers have
often called the \textit{marriage-function}, albeit applied to fertility.

In this vein, we have in this dissertation notably neglected the work of Pollak
\citep{pollak1986reformulation, pollak1987two, pollak1990two}, who
\textit{solves} the two-sex problem by separating couple-formation and birth
functions into two model components in the so-called BMMR (birth matrix mating
rule) model. As mentioned above, the couple-formation component of the model is
subject to the same modelling considerations as our own fertility component in
this dissertation. This model would have been commensurate with our own line of
presentation had the requisite data been on hand.

\subsection{Diferential equations} 
Many recent advances in the two-sex problem
have come from mathematicians and epidemiologists, and much (but not all) of
this effort has been motivated by the need to model sexual mixing in
populations for the study of disease-- most prominently HIV. Differential
equations (ODEs, or ordinary differential equations) are the model of choice in
this case because transitions may ocurr in intervals of less than a year 
(the standard in discrete demography), just as in
life \citep{hoppensteadt1975mathematical}. This is especially true of sexually
transmitted disease, which was a motivator mentioned by \citet{hadeler1988models}, and which sparked a wide blossoming
of two-sex, multi-state model development
\citep{dietz1988epidemiological,hadeler1989pair,busenberg1991general,blythe1991toward}.

ODE formulations of the two-sex population models typically maintain the
couple-formation (marriage-function) component to the model
\citep{Fredrickson1971,inaba1992age}, 
\citep[see e.g.][for a good overview]{ianelli2005gender}, and they have often entered into
territory seldom formally considered by practicing demographers. For instance,
ODEs at times incorporate logistic growth
functions\citep{castillo1995logistic,yang2009logisticwb,yang2009logistic} rather
than assuming exponential growth\citep{martcheva1999exponential}. Interdependencies in ODEs branch in more directions than in any demographic
projection model-- \citet{maxin2010two}, to take an example, incorporates a
divorce rate that depends on external pressure from the proportion still single,
rather than some constant rate for couples. Couple-formation may be
specified to occur with a maturation period without
loss of desirable model properties\citep{hadeler1993pair}. Some such models have
been shown to have unique solutions \citep{martcheva1999two} and stable age
structures \citep{inaba2000persistent}. While work has been done to
discretize some two-sex ODEs \citep{arbogast1989finite,
martcheva2001mathematics, ianelli2005gender}, continuous time models are
regretfully absent from this dissertation.

\subsection{A parametric solution?} 
All two-sex solutions presented thus far in
this dissertation have been framed in terms of single-age data commensurable with lifetable
methods. The age schedules for the demographic phenomena underlying these
methods have not been summarized here in terms of a reduced set of parameters.
Our end, the estimation and measurement of population reproductivity, is primarily 
a non-parametric endeavor. To summarize a two-sex version of net reproduction or
the intrinsic growth rate in terms of a reduced set of parameters is
possible given that: 
\begin{enumerate}
  \item There are several parameterizations of mortality.
  \textit{heligman1980age} provide an especially popular parameterization, which 
  summarizes an entire age-pattern in terms of eight parameters. To do this
  separately for males would entail a total of 16 mortality parameters, unless
  any of the original eight are plausibly equal.
  \item There are also several competing parameterizations of age-specific
  fertility schedules. For instance, \citet{coale1974model} propose a fertility
  model wherein marriage schedules (proportion ever married) are summarized by
  two parameters and marital fertility is summarized in terms of departure from
  a standard natural fertility pattern with two further parameters. Presumably,
  something similar could be done for males. We know of no attempt to
  parametrically model the two-sex fertility surface\footnote{e.g. as displayed
  in Figure~\ref{fig:US1970obsexp}}, although this is apparently within reach,
  as \citet{marriage1981warren,
sanderson1983two} has done this for marriage markets (described in following).
  \item The sex ratio at birth is essentially linear over ages of mothers and
  fathers, entailing two parameters each for males and females, and possibly
  fewer if parsimony is sought.
\end{enumerate}

Insofar as each model input can be parameterized, one could plausibly
parameterize the entire renewal equation. Indeed, \citet{coale1974model} mention
that their fertility parameterization was a subset of a project at
Princeton to find the complex roots of the Lotka equation, which
was later presented in \citet{james1977determinants}. These models have not 
been adapted in the direction of two-sex reproductivity. Thus, this is an avenue
for potential further exploration in the two-sex problem.

\subsection{The general equilibrium perspective} 
\citet{marriage1981warren,
sanderson1983two} made substantial headway in parameterizing a two-sex model to predict marriages. Part of this effort was
inspired by and based on the well-known 2-parameter Coale-McNeil
model\citep{coale1972distribution, coale1971age}\footnote{This marriage model is
a subset of the Coale-Trussel fertility model.} for estimating
single-age female marriage schedules from sparse, noisey or grouped
data. Coale and McNeil had namely found one of demography's most enduring 
and appealing empirical regularities-- 1) that nearly 
all observed marriage schedules at that time could be fit to a single standard parametric
curve, 2) that this curve could be fit by adjusting only three parameters, 3)
that these parameters could be estimated exogenously and separate from one
another, and 4) that each of the three parameters had a clear demographic
interpretation-- all of this with no reference to a \textit{standard}
schedule. Sanderson's challenge was to acheive the same degree of parsimony in a
two-sex setting, namely, where the numbers of available mates affects marriage
rates for each sex. 

Marginal distributions in this model are fit to the
Coale-McNeil equations. The model uses a large number of
age-age-specific scarcity and demand parameters to balance male and female
rates, which are in turn summarized in terms of a smaller set of parameters.
Sanderson is able to use supply and demand notions in an exchange economy
framework, as he describes marriages as exchanges of vows, wherein the number
of vows exchanged between brides and grooms in each age combination must
match. This conceptual framework is obviously imported from economics,
specifically from the extended linear expenditure system of
\citet{lluch1977patterns} in the general equilibrium family of models, which entails solving a large number of linear
equations simultaneously. This model is not implemented in this dissertation, 
in part because the conceptual and programming overhead is much higher
than the other models treated here, and in part because considerable
modifications would be necessary to bring the model to bear upon fertility and
work it into a full two-sex reproduction model. No posterior implementations of
Sanderson's model have been located in the literature.

\citet{bergstrom1994sweden} take a similar tack from the economic literature,
viewing the two-sex problem (from a suggestion in \citet{becker1973theory}) as a
particular case of the \textit{assignment problem} in combinatorial optimization, wherein given male
and female preferences for ages at marriage\footnote{i.e. as opposed to
preferences for ages of partnerts, these authors pose preferences in terms of
one's own age at marriage and leave the rest to the market.} combine with the
market. One problem is that a payoff, or cost, matrix-- the penalty for not
marrying at one's prefered age-- must be exogenously specified in order to
optimize the system. These authors were not satisfied with their own
empirical results and attributed this to certain assumptions in their model.
This particular modelling strategy appears not to have been followed further in
the literature.

\subsection{Choo and Siow's econometric perspective} 
The line of work from Choo
and Siow \citet{choo2006estimating, siow2008does, seitz2010collective} is construed 
from a similar market-based perspective to the above. These
authors conceive of the marriage market as several overlapping markets-- one
for each age / characteristic combination--, all of which must be optimized
(cleared). Interior to the model is a geometric mean marriage
function\footnote{The geometric mean respects homogeneity, but not
monotonicity. I do not find this problematic.}, but with additional
parameters to allow for inter-age substitution via supply and demand. The model
is identifiable, and it requires a kind of global optimization of a similar
order of complexity to the Sanderson or Bergstrom-Lam models. \citet{siow2008does} have
been able to test their model empirically by using a natural experiment that
drastically affected cohort size\footnote{The abrupt cohort size change at the
start and end of China's \textit{Great Leap Forward} from the province of
Sichuan.}, and found the model to match some expected marriage market
dynamics. Hypothetically, it would be possible to migrate this model framework
to the phenomena of fertility, though this has not been done yet, and the
substantive arguments would need to change. Likewise, it would be possible for
the Choo-Siow model to be an interior component to a larger fertility model
wherein marital states affect fertility rates. Implementing the Choo-Siow model
and / or translating it into this dissertations focus on reproductivity is
beyond our present scope. 

\subsection{Agent-based models and \textit{marriage-matching} per se} 
Other
propositions have been advanced in a similar line\citep{jacquemet2011marriage, Dagsvik1998report},
though often in an agent-based framework, of which \textit{marriage matching}
 algorithms form a widely studied subset. The two-sex problem in this
arena is not necessarily viewed as a problem of population modelling but as an
optimization or algorithmic assignment problem. \citet{dagsvik2000aggregation}
shows how such micro-level models can yield macro-level results of interest for
demography (and other disciplines), although this author knows of no efforts to
make macro-demographic predictions about population reproductivity by
aggregating from such agent-based models. Algorithms used in 
the marriage matching problem are applicable in other matching problems (e.g.
firms-buyers, firms-locations). 

The most famous such matching algorithm is
the Gale-Shapley algorithm \citep{gale1962college}\footnote{For instance, a variant of this algorithm is used 
to assign medical graduates to hospital residencies in the United States.}
wherein each male and female member of a population begins with a ranking of
each potential partner according to any criteria. The algorithm begins with one
side (males, let us say) \textit{proposing} to the
highest-ranked unmarried individual (females) in the preference list to which
he has not yet proposed. Second, each female that received proposals
provisionally accepts the proposal from the suitor that was highest in her own
ranking list (gets enganged). In the next round proposals may be made to engaged
or single females, but engaged males do not propose. An engagement may in this
case be broken if a higher ranked male proposed. The algorithm continues until
no new engagements are made. It has been shown that this algorithm is a solution
to the \textit{stable marriage problem}, meaning that once the each individual
is matched there is no male-female combination in the population wherein each
would prefer to leave their partner. When this condition is met the marriages
are said to be stable. In this algorithm, the proposing side (males in our
description) will always converge to have partners that were ranked higher on
their initial list than the accepting side. This algorithm is not an agent-based
model of the marriage market, per se, but rather a potential component of one.

Agent-based models (ABMs) are attractive for the two-sex problem
precisely because the problem in this setting changes its nature from being one of \textit{internal} consistency 
to one of dynamics or matching. Internal consistency is dealt with precisely
because individuals in such simulations mate due to interactions,
in which case a single marriage is assigned to each, and the accounting
constraint is fulfilled without further ado. \citet{billari2002wedding} have put
this framework to productive use, reproducing aggregate-level age-at-marriage
patterns based on sociologically and psychologically informed micro-level
interactions of the marriage market. Namely, potential mates marry not only as a
function of mate availability-- that primarily accounted for by demographers
in two-sex models-- but also as a function of marriages taking place in the
agents' own social networks, in essence granting part of agent marriage
propensity (willingness in the model) to peer effects (i.e. contagion).
\citet{walker2013modelling} recently built a similarly conceived ABM which
permitted the authors to test sociological theory about various kinds of 
homogamy against observed census data. In general,
ABMs are of use to demographers interested in the two-sex problem because they permit the exploration of the implications of
particular hypotheses about decision-making, social interactions and individual-environment
interations for demographic and other population processes. ABMs will not be
useful in the context of stable populations or reproductivity in the sense
studied in this dissertation precisely because such models are \textit{complex}
and may never stabilize or have unique trajectories\citep{RePEc:ssb:dispap:247}.




