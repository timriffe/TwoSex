 \FloatBarrier

The magnitude and vital rate components to the two-sex problem in age-structured
populations has now been presented for the cases of the US and Spanish
populations. That the fundamental discrepancy exists was already demonstrated by
\citet{kuczynski1932fertility} in the early 1930s, and formally introduced to
the discipline of demography in the late 1940s by \citet{karmel1947relations},
since which time interest has continued in waves. It is namely the case that the
discrepancy thus far has no real solution in the sense of a unique and true
solution. Instead what is meant by a two-sex \textit{solution} is a method to
balance male and female vital rates so as to produce the same estimate of
total births, or else the same intrinsic growth rate. This is perhaps part of
the reason why interest has continued-- there are many ways in which this goal
might be achieved. All methods proposed have incurred some degree of tradeoff
between convenience, simplicity, realistic design and results that are
consistent with expectations.
 
The point of departure for the goal in formal demography of balancing male and
female rates is the following: For the single-sex case-- the classic Lotka
model captured in Equation~\eqref{eq:lotkaeq}-- we have a coherent model that
works for each sex separately but produces undesirable results when modelled in parallel for both
sexes. What modifications must we introduce to the model, such that a single
estimate of the intrinsic growth rate, $r$, is produced while maintaining a
reasonable sex ratio (both total sex ratio and the sex ratio at birth) and
maintaining constant male and female vital rates? An alternative formulation
could be summarized in terms of producing a single prediction of births in
future projected years. Namely, what changes should we admit to the cohort
component projection model such that the model accepts both male and
female inputs but produces consistent output in the form of single estimates of
male and female births.



 
 
 
 
 
\FloatBarrier