 \FloatBarrier

\begin{singlespace}
\begin{quote}
The purpose of models is not to fit data but to sharpen questions-- Samuel
Karlin\footnote{Quote from the \nth{11} annual R. A. Fisher Memorial Lecture
given at the Royal Society of London on April \nth{20}, 1983.}
\end{quote}
\end{singlespace}

The an assessment of the magnitude and vital rate components to the two-sex
problem in age-structured populations was presented presented in the previous
chapter for the cases of the US and Spanish populations. That the fundamental
discrepancy exists was already empirically demonstrated by
\citet{kuczynski1932fertility} in the early 1930s, and formally introduced to
the discipline of demography in the late 1940s by \citet{karmel1947relations},
since which time interest has continued in waves. It is the case that the
discrepancy thus far has no real solution in the sense of a necessarily true
solution. Instead what is meant by a two-sex \textit{solution} is a method to
balance male and female vital rates so as to produce the same estimate of
total births, or else the same intrinsic growth rate. This is perhaps part of
the reason why interest has continued-- there are many ways in which this goal
might be achieved. All methods proposed have incurred some degree of tradeoff
between convenience, simplicity, realistic design and results that are
consistent with expectations.
 
The point of departure for the goal in formal demography of balancing male and
female rates is the following: For the single-sex case-- the classic Lotka
model captured in Equation~\eqref{eq:lotkaeq}-- we have a coherent model that
works for each sex separately but produces undesirable results when modelled in parallel for both
sexes. What modifications must we introduce to the model, such that a single
estimate of the intrinsic growth rate, $r$, is produced while maintaining a
reasonable sex ratio (both total sex ratio and the sex ratio at birth) and
maintaining constant male and female vital rates? 

An alternative formulation
could be summarized in terms of producing a single prediction of births in
future projected years. Namely, what changes should we admit to the cohort
component projection method such that the model accepts both male and
female inputs but produces consistent output in the form of single estimates of
male and female births.

Classifying two-sex models into families of models that share similar qualities
is non-trivial. We will present an imperfectly arranged subset of models that
have thus far appeared in the literature, focusing primarily on those models
that we will modeify later in this dissertation for the case of
population structured by remaining years of life. 

We begin by classifying into a single group those two-sex models whose
consistuent parts have essentially been the male and female single-sex models; i.e. those
models that have adhered to the concept of single-sex fertility. In this group
we could first place the exceedingly simple solution that consists in taking some 
mean of the male and female single-sex intrinsic growth rates to produce a 
both-sex $r$, without digging into the workings of the model
itself\citep[e.g.][]{kuczynski1932fertility}. We will discuss two models that
fall into this class, one parsimonious and effective, another intricate. Both
models will be seen to yield results in line with expectations, but to suffer
particular drawbacks. The first model which we will briefly discuss is that
which appeared in \citet{pollard1948measurement}. This will be followed by
a longer explanation of the less intuitive model in \citet{mitra1978derivation}.
Neither of these two models will be translated to the remaining years
perspective.

Next we will briefly explain a linear model that makes use of a constant
\textit{dominance} parameter--a weighted arithemtic mean-- to regulate the
influence of male and female vital rates on intrinsic growth rates. In this
model, both males and females produce offspring of both sexes, thereby
removing any additional complications in sex ratios implied by the use of
single-sex fertility rates. This model will later be translated to the
remaining-years perspective later in this dissertation.

Third, we will present two more models, \citet{gupta1978alternative} and
\citet{schoen1981harmonic} (and mention several others) that make use of
fertility rates cross-classified by age of mother and age of father. Finally 
we will present two more models that also rely on age
cross-classified fertility information, but that have some more desirable
properties, but that are difficult to examine in an analytic framework, namely,
\citet{henry1972nuptiality} and \citet{mc1975models}. Some
results will be compared and assessed in light of the axioms presented
previously.

 
 
 
 
 
\FloatBarrier