\paragraph{Componets to the model: }
Define two fertility functions, $F_{a'}^{F-M}$ and $F_a^{M-F}$, where the first
superscript indicates the sex of progenitor, the second superscript indicates
the sex of birth, $a$ indexes male age and $a'$ indexes female age. In
other words, the female births are determined by male rates and vice versa.
$p_a$ and $p_{a'}$ are the male and female probabilities of surviving to age 
$a$, $a'$. It will be convenient to combine these two items into \textit{net}
opposite-sex offspring functions
\begin{align}
m_a &= p_aF_a^{M-F}\\
m_{a'} &= p_{a'}F_{a'}^{F-M}
\end{align}
Note that these function names are the same as in standard single-sex models,
but that sex of progenitor and offspring have been juxtaposed.
\paragraph{The renewal function(s): }
Given $m_a$ and $m_{a'}$, the renewal function to determine births in yeat $t$
is given by
\begin{align}
B^F &= \int_0^\infty m_a \dd a\\
B^M &= \int_0^\infty m_{a'} \dd a'\\
\end{align}
Which can be converted such that sex of offspring and reference
progenitor are the same by moving back one generation and combining net
offspring functions:
\begin{align}
B^F &= \int_0^\infty \int_0^\infty B^F_{t-a-a'} m_am_{a'}\dd a \dd a'\\
B^M &= \int_0^\infty \int_0^\infty B^M_{t-a-a'} m_am_{a'}\dd a \dd a'\\
\end{align}
These later two functions combine into a single convenient renewal function
\begin{equation}
B^T = \int_0^\infty \int_0^\infty B_{t-a-a'}m_am_{a'}\dd a \dd a'
\end{equation}
Conveniently, all of these five renewal functions will converge to the same
ultimate intrinsic growth rate, $r$, which is the real root of the following equation: 
\begin{equation}
1 = \int_0^\infty \int_0^\infty e^{-(a+a')r}m_am_{a'}\dd a \dd a'
\end{equation}

In Pollard's model the sex ratio at birth and overall sex ratio
of the population are regulated by cris-crossed sex-specific fertility. Pollard proves that
the resulting estimate of $r$ will be intermediate to the male and female
single-sex intrinsic growth rates-- assuming a constant sex ratio at birth-- and
the function remains linear. Further, the function has the advantage of being 
relatively easy to understand.
This author finds the method clever, but it has been rather ignored in the
literature because authors typically find the assumption of cris-crossed
fertility unrealistic. This seems like a fair criticism if the goal is to
faithfully reflect fertiltiy dynamics. It would seem that Pollard's goal 
was to approximate the value of the two-sex growth rate while maintaining
a small set of desirable model qualities (bracketing, linearity, homogeneity
simplicity), but not to approximate true reproducive dynamics. Another drawback
is that the sex ratio at birth, if not assumed constant, depends on the initial
conditions. 

Given an optimized value of $r$, one can retrieve the stable age structure and
sex ratio at birth\footnote{Some advice is given in
\citet{pollard1948measurement} for arriving at the stable sex ratio, but it
would be easier to either just assume a sex ratio at birth or else iterate
forward to stability and derive it empirically.} and other stable parameters of
interest. Empirical results of Pollard's method will be compared later with
others.
