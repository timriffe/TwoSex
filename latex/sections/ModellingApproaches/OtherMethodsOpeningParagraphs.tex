
Another widely-known iterative method, most comparable with the IPF method, is
that described in \citet{henry1972nuptiality}, Louis Henry's method of matrix
decomposition via \textit{panmictic circles}, which entails somewhat more
implementation effort than does IPF, and is likely more substantively appealing. 
\citet{mc1975models} only compared the
IPF method with that of Henry, and \citet{wijewickrema1980weak} used this method 
in his dissertation study of weak ergodicity in the two-sex problem in the
context of marriage\footnote{And most interestingly, was with little
effort able to produce apparent limit cycles in the stable population.}. We do
not implement this method\footnote{There are some ambiguities in terms of 
implementing the method when working with unabridged data that also contains 
many zeros. We did not succeed in translating the method from an
abridged table to single ages without also producing negative numbers.}, though
it most resembles IPF, and it has all of the same desirable properties, yet
gives different results. This extension is left for future work.





