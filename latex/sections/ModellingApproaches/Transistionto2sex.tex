Classifying two-sex models into families of models that share similar qualities
is non-trivial. We will present an imperfectly arranged subset of models that
have thus far appeared in the literature, focusing primarily on those models
that we will modeify later in this dissertation for the case of
population structured by remaining years of life. 

We begin by classifying into a single group those two-sex models whose
consistuent parts have essentially been the male and female single-sex models; i.e. those
models that have adhered to the concept of single-sex fertility. In this group
we could first place the exceedingly simple solution that consists in taking some 
mean of the male and female single-sex intrinsic growth rates to produce a 
both-sex $r$, without digging into the workings of the model
itself\citep[e.g.][]{kuczynski1932fertility}. We will discuss two models that
fall into this class, one parsimonious and effective, another intricate. Both
models will be seen to yield results in line with expectations, but to suffer
particular drawbacks. The first model which we will briefly discuss is that
which appeared in \citet{pollard1948measurement}. This will be followed by
a longer explanation of the less intuitive model in \citet{mitra1978derivation}.

\subsubsection{Pollard 1949}

Define two fertility functions, $F_{x'}^{F-M}$ and $F_x^{M-F}$, where the first
superscript indicates the sex of progenitor, the second superscript indicates
the sex of birth, $x$ indexes male age and $x'$ indexes female age. $p_x$ and
$p_{x'}$ are the male and female probabilities of surviving to age $x$, $x'$. It
will be convenient to combine these items into \textit{net} offspring functions
\begin{align}
m_x &= p_xF_x^{M-F}\\
m_{x'} &= p_{x'}F_{x'}^{F-M}
\end{align}
Then the renewal function to determine births in yeat $t$ is given by
\begin{align}
B^F &= \int_0^\infty B^M_{t-x}m_x \dd x\\
B^M &= \int_0^\infty B^F_{t-x'}m_{x'} \dd x'\\
\end{align}
which can be reversed to single-sex components by moving the reference
progenitor back one generation and combining net offspring functions
\begin{align}
B^F &= \int_0^\infty \int_0^\infty B^F_{t-x-x'} m_xm_{x'}\dd x \dd x'\\
B^M &= \int_0^\infty \int_0^\infty B^M_{t-x-x'} m_xm_{x'}\dd x \dd x'\\
\end{align}
These later two function combine into a single convenient renewal function
\begin{equation}
B^T = \int_0^\infty \int_0^\infty B_{t-x-x'}m_xm_{x'}\dd x \dd x'
\end{equation}
All of these five functions will converge to the same ultimate intrinsic growth
rate, $r$, which is the real root of the following equation 
\begin{equation}
1 = \int_0^\infty \int_0^\infty e^{-(x+y)r}m_xm_{x'}\dd x \dd x'
\end{equation}

other words, the sex ratio at birth and overall sex ratio of the population are
regulated by cris-crossed sex-specific fertility.


\subsubsection{Mitra 1978}

\subsubsection{}


We shed light on two
particualr models, namely that of Pollard and that of Mitra










