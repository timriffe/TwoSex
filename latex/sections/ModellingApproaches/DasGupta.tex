
\citet{gupta1978alternative} states\footnote{and this fits nicely into the flow
of our own presentation.} ``The lesson we learn from the above is that our
starting point must not be the formulation of two equations, one for $B_M(t)$ and another for
$B_F(t)$, but of a single equation for $B(t)$ with the help of a bisequal
fertility function that can explain the occurrence of births of type $(a,a')$ in
terms of the availability of both males and females''.


\citet{gupta1972two, gupta1973growth, gupta1976interactive, gupta1978alternative}
was concerned primarily with deriving a method to force consistency between male and female
vital rates so as to yield the same population summary growth measures rate,
$r$, and $R_0$.





