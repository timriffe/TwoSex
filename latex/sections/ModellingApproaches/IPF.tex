\FloatBarrier
\citet{mc1975models} introduced a well-established method called iterative
proportional fitting (IPF)\footnote{Also called matrix-raking.}, often used for
rescaling tables, to the two-sex problem for marriage models. We will apply
the method to fertility only, though the reader may consult
\citet{mc1975models} or \citet{Matthews2013} for details on how to apply
this method in the case of marriage matching. This method works by starting with
a known cross-tabulation of births, in our case from the base year $t$. First compute 
the marginal fertility rates for males and females (ASFR). Apply the male and
female ASFR to exposures predicted for some future year $t+n$ to produce
initially-predicted marginal birth count distributions, the sums of which will 
never agree (as was illustrated in Section~\ref{sec:divbirth}). These new
marginal distributions may be uniformly rescaled according to some mean of their
respective sums. The mean function chosen will only affect results when the
sex-gap in predicted births is very large\footnote{we have compared overall
results using arithmetic and harmonic means, and found no noteworthy difference. 
All results will be in terms of the harmonic mean for this first rescaling.}.

Now that the male and female sums for year $t+n$ agree, we iteratively
rescale the original birth matrix according to the male and female
predicted margins, alternating between male (row) and female (column) margins
until the new birth matrix margins sum exactly to the predicted margins. Of
course, the resulting matrix will differ depending on whether one begins with
rows or with column margins, and so we adopt the recommendation of
\citet{Matthews2013}, taking the elementwise average of the two
possible outcome matrices in each iteration before advancing to the next
iteration. We continue this iterative process until it no longer makes any
difference whether we first scale rows or first scale columns, and accept the
resulting \textit{raked} amtrix as our year $t+n$ prediction.

Using IPF, 1) male and female rates are guaranteed to agree, 2) structural 0s
are not problematic, and 3) the inter-age competition axiom is fulfilled,
which has not been the case with methods previously described. To illustrate
this property, we execute the following experiment: Taking initial birth count
and exposure data from US, 1975, we calculate male and female ASFR. We then
apply the male and female 1975 ASFR to exposures from 1980, and proceed with the
above-described method, finally settling on a predicted birth matrix for 1980,
from which we calculate new ASFR vectors for males and females (1980
predictions that agree). This is the base prediction that we will compare with.
Now we increase 1980 age-25 males (only) by 50\% and repeat the IMP procedure,
producing new ASFR predictions for males and for females.
Figure~\ref{fig:IPFcomp} shows the ratio of the hypothetical (age-25 male
exposure increased by 50\%) 1980 ASFR to the base 1980 ASFR prediction.

Figure~\ref{fig:IPFcomp} illustrates the competition axiom because age-25 male
rates decrease more than neighboring ages, and rate for male ages closest to 25
decrease by more than ages farther from 25 (in fact the ratio is monotonic in
either direction from 25-- ideal). Female rates increase as well, also as a
rough function of proportional \textit{intermatedness} with age-25 males in the
1975 birth matrix. Here we see only the difference in rates-- in terms of birth
counts, age-25 males would have a large increase, while all other ages would
experience decreases (lower rates applied to the same exposures). 

\begin{figure}[ht!]
        \centering  
          \caption{ASFR after increasing 1980 male exposure by 50\% compared
          with base 1980 ASFR prediction. Based on US 1975 ASFR and birth
          matrix.}
           % figure produced in/R/IPFage.R
           \includegraphics{Figures/IPFagecompetitiontest}
          \label{fig:IPFcomp}
\end{figure}

\paragraph{Iterative proportional fitting in models of population growth: }

The formulas to formalize the use of IPF fertility balancing will take on a
different appearance than those seen thus far. Note that the basic inputs to the
IPF function to constrain male and female fertility rates will be:
$IPF(B_{a,a'}(t), F_a^M(t), F_{a'}^F(t),P_a(t+n),P_{a'}(t+n))$, where $B_{a,a'}$
is the cross-classified birth matrix, $F_a^M$ and $F_{a'}^F$ are male and female ASFR,
and $P_a$ and $P_{a'}$ are future population estimates (exposures when
discrete). The function produces adjusted ASFR for \textit{both} males and females,
$F_a^{M^\ast}$, $F_{a'}^{F\ast}$. Let us define a shorthand where the year
$\tau$ is the year used as the standard for the IPF method, bearing in mind that
$IPF(\tau, p^M, p^F)$ requires the first three arguments from year $\tau$ (births and 
male and female ASFR), whereas the later two arguments, that which we would like
to adapt to, may change according to our ends. $F_a^{M^\ast(\tau,p^M, p^F)}$
will be the IPF adjusted male ASFR based on year $\tau$ data, and
$F_{a'}^{F^\ast(\tau,p^M, p^F)}$ the female ASFR output from the same procedure.
Then assuming constant mortality and continuous functions, we may define year $t$ total births as a
function of past births as:

\begin{align}
B(t) &= \int_{a=0}^\infty \varsigma B(t-a)p_aF_a^{M^\ast(\tau,p^M, p^F)}\dd a
\\ &= \int_{a'=0}^\infty (1-\varsigma) B(t-a')p_{a'}F_{a'}^{F^\ast(\tau,p^M,
p^F)}\dd a'
\end{align}
which works either with males or with females, since the IPF function already
balances fertility such that total births will be the same whether predicted by
males or females. $\varsigma$ is the proportion male at birth. If mortality is
held constant and $IPF(\tau,p^M, p^F)$ is always based on the same year $\tau$ 
constant information, the popultation will eventually begin
to grow at a constant rate $r$ which can be estimated from the following
equation:
\begin{align}
\label{eq:IPFtricky}
1 &= \int_{a=0}^\infty \varsigma e^{-ra}p_aF_a^{M^\ast(\tau,p^{M\infty},
p^{F\infty})}\dd a \\ 
&= \int_{a'=0}^\infty (1-\varsigma) e^{-ra'}p_{a'}F_{a'}^{F^\ast(\tau,p^{M\infty},
p^{F\infty})}\dd a'
\end{align}
where $p^{M\infty}$ for males is just the full age pattern of
$\varsigma e^{-ra}p_a$, and $p^{F\infty}$ is the full age pattern
from $(1-\varsigma) e^{-ra'}p_{a'}$ analogously for females. The estimation of
$r$ using this equation is unfortunately cumbersome, and is explained in
following.

\paragraph{Estimating the intrinsic growth rate: } First, note that either
version of Equation~\eqref{eq:IPFtricky} requires full information from both
males and females, so we may as well add the two right-side components and make
the equation sum to two:
\begin{equation}
\label{eq:IPFugly}
2 = \int_{a=0}^\infty \int_{a'=0}^\infty \varsigma
e^{-ra}p_aF_a^{M^\ast(\tau,p^{M\infty}, p^{F\infty})} + (1-\varsigma)
e^{-ra'}p_{a'}F_{a'}^{F^\ast(\tau,p^{M\infty}, p^{F\infty})}\dd a' \dd a
\end{equation}
As in some earlier iterative $r$-estimation instructions given in this
dissertation, one does well to allow $\varsigma$ to be determined by sex
ratios that vary over age of mother and father. This information we retain
in the sex-sex-specific fertility functions, featured elsewhere in this dissertation: $F_a^{M-M}$,
$F_a^{M-F}$,$F_{a'}^{F-F}$, $F_{a'}^{F-M}$, which will entail two IPF functions,
one for boy births and one for girl births. 
\todo{complete description, see if there's a better way to summarize}




\paragraph{Summary of the method applied to models of population growth:
}Fulfillment of the competition axiom is not a trivial achievment, and it also more or less tops off the list of important axioms: 1) The solution meets the availability
axiom, 2) is first degree homogenous, 3) is monotanous, 4) is symmetrical with respect the
sexes, 5) and is sensitive to substitution and competition. There is no
guarantee for bracketing, although the solution will always track and typically
be bracketed by the single-sex intrinsic growth rates.


\begin{figure}[ht!]
        \centering  
          \caption{IPF intrinsic growth rates, $r$, compared with
          single-sex $r^m$ and $r^f$. US, 1969-2009 and Spain, 1975-2009.}
           % figure produced in/R/IPFage.R
           \includegraphics{Figures/IPFager.pdf}
          \label{fig:IPFager}
\end{figure}



\FloatBarrier

