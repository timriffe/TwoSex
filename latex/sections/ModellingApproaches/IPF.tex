\FloatBarrier
\citet{mc1975models} introduced a well-established method called iterative
proportional fitting (IPF)\footnote{Also called matrix-raking.}, often used for
rescaling tables, to the two-sex problem for marriage models. We will apply
the method to fertility only, though the reader may consult
\citet{mc1975models} or \citet{Matthews2013} for details on how to apply
this method in the case of marriage matching. This method works by starting with
a known cross-tabulation of births, in our case from the base year $t$. First compute 
the marginal fertility rates for males and females (ASFR). Apply the male and
female ASFR to exposures predicted for some future year $t+n$ to produce
initially-predicted marginal birth count distributions, the sums of which will 
never agree (as was illustrated in Section~\ref{sec:divbirth}). These new
marginal distributions may be uniformly rescaled according to some mean of their
respective sums. The mean function chosen will only affect results when the
sex-gap in predicted births is very large\footnote{we have compared overall
results using arithmetic and harmonic means, and found no noteworthy difference. 
All results will be in terms of the harmonic mean for this first rescaling.}.

Now that the male and female sums for year $t+n$ agree, we iteratively
rescale the original birth matrix according to the male and female
predicted margins, alternating between male (row) and female (column) margins
until the new birth matrix margins sum exactly to the predicted margins. Of
course, the resulting matrix will differ depending on whether one begins with
rows or with column margins, and so we adopt the recommendation of
\citet{Matthews2013}, taking the elementwise average of the two
possible outcome matrices in each iteration before advancing to the next
iteration. We continue this iterative process until it no longer makes any
difference whether we first scale rows or first scale columns, and accept the
resulting \textit{raked} amtrix as our year $t+n$ prediction.

Using IPF, 1) male and female rates are guaranteed to agree, 2) structural 0s
are not problematic, and 3) the inter-age competition axiom is fulfilled,
which has not been the case with methods previously described. To illustrate
this property, we execute the following experiment: Taking initial birth count
and exposure data from US, 1975, we calculate male and female ASFR. We then
apply the male and female 1975 ASFR to exposures from 1980, and proceed with the
above-described method, finally settling on a predicted birth matrix for 1980,
from which we calculate new ASFR vectors for males and females (1980
predictions that agree). This is the base prediction that we will compare with.
Now we increase 1980 age-25 males (only) by 50\% and repeat the IMP procedure,
producing new ASFR predictions for males and for females.
Figure~\ref{fig:IPFcomp} shows the ratio of the hypothetical (age-25 male
exposure increased by 50\%) 1980 ASFR to the base 1980 ASFR prediction.

Figure~\ref{fig:IPFcomp} illustrates the competition axiom because age-25 male
rates decrease more than neighboring ages, and rate for male ages closest to 25
decrease by more than ages farther from 25 (in fact the ratio is monotonic in
either direction from 25-- ideal). Female rates increase as well, also as a
rough function of proportional \textit{intermatedness} with age-25 males in the
1975 birth matrix. Here we see only the difference in rates-- in terms of birth
counts, age-25 males would have a large increase, while all other ages would
experience decreases (lower rates applied to the same exposures). 

\begin{figure}[ht!]
        \centering  
          \caption{ASFR after increasing 1980 male exposure by 50\% compared
          with base 1980 ASFR prediction. Based on US 1975 ASFR and birth
          matrix.}
           % figure produced in/R/IPFage.R
           \includegraphics{Figures/IPFagecompetitiontest}
          \label{fig:IPFcomp}
\end{figure}

\begin{figure}[ht!]
        \centering  
          \caption{IPF intrinsic growth rates, $r$, compared with
          single-sex $r^m$ and $r^f$. US, 1969-2009 and Spain, 1975-2009.}
           % figure produced in/R/IPFage.R
           \includegraphics{Figures/IPFager.pdf}
          \label{fig:IPFager}
\end{figure}



\FloatBarrier

