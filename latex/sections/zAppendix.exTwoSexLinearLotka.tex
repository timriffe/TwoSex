\chapter{Equation~\ref{eq:lineartwosexrenewal} applied to US and Spanish data}
This appendix provides numerical results from the application of the
estimation techniques described in Section~\ref{sec:exLotka2linear}. Estimates
of $r^\upsilon$, $T^\upsilon$ and $R_0^\upsilon$ under three values of $\sigma$.
Recall that $\sigma$ determines the relative dominance of each sex, where 0 is
female dominance, 1 is male dominant, and .5 is an intermediate level of
dominance (which is not necessarily an arithmetic mean). As stated in the
main text, the border cases where $\sigma = [0,1]$ do not revert to the
single-sex $e_x$-structured values presented in Section~\ref{sec:exstructuredrenewal} and
Appendix~\ref{appendix:exlotka1sex}.

\begin{table}
  \begin{adjustwidth}{-1in}{-.5in}
  \centering
    \caption{Two-sex linear intrinsic growth rate, $r^\upsilon$, mean remaining
years of life at reproduction, $T^\upsilon$, and net reproduction,
$R_0^\upsilon$, according to renewal Equation~\eqref{eq:lineartwosexrenewal},
with $\sigma = [0, .5, 1]$ US, 1969-2009.}
    \label{tab:ex2linRepES}
        % latex table generated in R 2.15.3 by xtable 1.7-1 package
% Wed May  1 09:28:38 2013
\begin{tabular}{cccccccccc}
  \hline
 & $r^{\upsilon (\sigma = 0)}$ & $r^{\upsilon (\sigma = .5)}$ & $r^{\upsilon (\sigma = 1)}$ & $T^{\upsilon (\sigma = 0)}$ & $T^{\upsilon (\sigma = .5)}$ & $T^{\upsilon (\sigma = 1)}$ & $R_0^{\upsilon (\sigma = 0)}$ & $R_0^{\upsilon (\sigma = .5)}$ & $R_0^{\upsilon (\sigma = 1)}$ \\ 
  \hline
1969 & 0.0050 & 0.0060 & 0.0069 & 50.62 & 46.01 & 41.63 & 1.289 & 1.316 & 1.330 \\ 
  1970 & 0.0058 & 0.0068 & 0.0077 & 51.08 & 46.45 & 42.06 & 1.346 & 1.370 & 1.380 \\ 
  1971 & 0.0038 & 0.0047 & 0.0056 & 50.91 & 46.30 & 41.92 & 1.211 & 1.243 & 1.263 \\ 
  1972 & 0.0004 & 0.0013 & 0.0023 & 50.38 & 45.73 & 41.33 & 1.018 & 1.063 & 1.098 \\ 
  1973 & -0.0013 & -0.0003 & 0.0007 & 50.27 & 45.63 & 41.26 & 0.936 & 0.987 & 1.028 \\ 
  1974 & -0.0015 & -0.0003 & 0.0007 & 50.72 & 46.08 & 41.72 & 0.929 & 0.985 & 1.029 \\ 
  1975 & -0.0019 & -0.0008 & 0.0003 & 51.10 & 46.39 & 41.97 & 0.908 & 0.965 & 1.011 \\ 
  1976 & -0.0020 & -0.0008 & 0.0002 & 51.18 & 46.50 & 42.11 & 0.904 & 0.962 & 1.009 \\ 
  1977 & -0.0007 & 0.0004 & 0.0014 & 51.73 & 47.03 & 42.60 & 0.966 & 1.020 & 1.062 \\ 
  1978 & -0.0010 & 0.0002 & 0.0012 & 51.75 & 47.05 & 42.64 & 0.952 & 1.007 & 1.051 \\ 
  1979 & 0.0003 & 0.0014 & 0.0023 & 52.34 & 47.61 & 43.15 & 1.017 & 1.068 & 1.106 \\ 
  1980 & 0.0010 & 0.0021 & 0.0030 & 52.30 & 47.64 & 43.24 & 1.055 & 1.103 & 1.138 \\ 
  1981 & 0.0010 & 0.0020 & 0.0029 & 52.42 & 47.76 & 43.36 & 1.052 & 1.099 & 1.134 \\ 
  1982 & 0.0012 & 0.0022 & 0.0031 & 52.63 & 47.99 & 43.59 & 1.067 & 1.111 & 1.144 \\ 
  1983 & 0.0006 & 0.0016 & 0.0025 & 52.37 & 47.77 & 43.42 & 1.031 & 1.078 & 1.113 \\ 
  1984 & 0.0007 & 0.0016 & 0.0025 & 52.35 & 47.77 & 43.42 & 1.037 & 1.080 & 1.113 \\ 
  1985 & 0.0013 & 0.0021 & 0.0030 & 52.39 & 47.79 & 43.43 & 1.068 & 1.108 & 1.137 \\ 
  1986 & 0.0010 & 0.0019 & 0.0026 & 52.34 & 47.75 & 43.36 & 1.056 & 1.093 & 1.120 \\ 
  1987 & 0.0013 & 0.0021 & 0.0028 & 52.39 & 47.81 & 43.44 & 1.072 & 1.106 & 1.130 \\ 
  1988 & 0.0020 & 0.0027 & 0.0033 & 52.45 & 47.89 & 43.52 & 1.109 & 1.137 & 1.156 \\ 
  1989 & 0.0029 & 0.0036 & 0.0042 & 52.90 & 48.34 & 43.95 & 1.167 & 1.189 & 1.201 \\ 
  1990 & 0.0037 & 0.0043 & 0.0048 & 53.23 & 48.69 & 44.30 & 1.216 & 1.231 & 1.238 \\ 
  1991 & 0.0031 & 0.0036 & 0.0040 & 53.22 & 48.69 & 44.30 & 1.177 & 1.190 & 1.196 \\ 
  1992 & 0.0023 & 0.0029 & 0.0034 & 53.15 & 48.65 & 44.29 & 1.131 & 1.151 & 1.163 \\ 
  1993 & 0.0014 & 0.0020 & 0.0025 & 52.70 & 48.25 & 43.95 & 1.077 & 1.100 & 1.116 \\ 
  1994 & 0.0008 & 0.0013 & 0.0018 & 52.60 & 48.19 & 43.93 & 1.041 & 1.066 & 1.084 \\ 
  1995 & 0.0000 & 0.0006 & 0.0012 & 52.40 & 48.05 & 43.86 & 1.001 & 1.030 & 1.052 \\ 
  1996 & -0.0003 & 0.0003 & 0.0008 & 52.41 & 48.19 & 44.12 & 0.985 & 1.014 & 1.037 \\ 
  1997 & -0.0007 & -0.0001 & 0.0005 & 52.45 & 48.36 & 44.42 & 0.966 & 0.997 & 1.022 \\ 
  1998 & -0.0004 & 0.0001 & 0.0007 & 52.57 & 48.56 & 44.69 & 0.978 & 1.007 & 1.030 \\ 
  1999 & -0.0006 & 0.0000 & 0.0006 & 52.46 & 48.54 & 44.75 & 0.971 & 1.001 & 1.025 \\ 
  2000 & 0.0000 & 0.0006 & 0.0011 & 52.61 & 48.74 & 44.99 & 1.002 & 1.028 & 1.049 \\ 
  2001 & -0.0004 & 0.0001 & 0.0006 & 52.54 & 48.70 & 44.97 & 0.980 & 1.006 & 1.026 \\ 
  2002 & -0.0006 & -0.0001 & 0.0004 & 52.53 & 48.70 & 45.00 & 0.967 & 0.995 & 1.017 \\ 
  2003 & -0.0003 & 0.0003 & 0.0007 & 52.66 & 48.89 & 45.24 & 0.986 & 1.013 & 1.034 \\ 
  2004 & -0.0002 & 0.0003 & 0.0008 & 53.04 & 49.29 & 45.66 & 0.988 & 1.015 & 1.036 \\ 
  2005 & -0.0002 & 0.0003 & 0.0008 & 53.11 & 49.38 & 45.77 & 0.989 & 1.015 & 1.036 \\ 
  2006 & 0.0006 & 0.0011 & 0.0015 & 53.65 & 49.94 & 46.34 & 1.033 & 1.056 & 1.074 \\ 
  2007 & 0.0009 & 0.0013 & 0.0017 & 53.98 & 50.29 & 46.71 & 1.048 & 1.068 & 1.083 \\ 
  2008 & 0.0002 & 0.0006 & 0.0011 & 53.84 & 50.21 & 46.69 & 1.009 & 1.033 & 1.052 \\ 
  2009 & -0.0009 & -0.0003 & 0.0002 & 53.84 & 50.23 & 46.74 & 0.955 & 0.984 & 1.007 \\ 
   \hline
\end{tabular}

  \end{adjustwidth}
\end{table}

\begin{table}
  \begin{adjustwidth}{-1in}{-.5in}
    \centering
    \caption{Two-sex linear intrinsic growth rate, $r^\upsilon$, mean remaining
years of life at reproduction, $T^\upsilon$, and net reproduction,
$R_0^\upsilon$, according to renewal Equation~\eqref{eq:lineartwosexrenewal}, with
$\sigma = [0, .5, 1]$ Spain, 1975-2009.}
    \label{tab:ex2linRepES}
        % latex table generated in R 2.15.3 by xtable 1.7-0 package
% Mon Apr  1 22:43:43 2013
\begin{tabular}{cccccccccc}
  \hline
 & $r^{\upsilon (\sigma = 0)}$ & $r^{\upsilon (\sigma = .5)}$ & $r^{\upsilon (\sigma = 1)}$ & $T^{\upsilon (\sigma = 0)}$ & $T^{\upsilon (\sigma = .5)}$ & $T^{\upsilon (\sigma = 1)}$ & $R_0^{\upsilon (\sigma = 0)}$ & $R_0^{\upsilon (\sigma = .5)}$ & $R_0^{\upsilon (\sigma = 1)}$ \\ 
  \hline
1975 & 0.0078 & 0.0087 & 0.0095 & 50.14 & 46.02 & 42.12 & 1.479 & 1.492 & 1.492 \\ 
  1976 & 0.0081 & 0.0089 & 0.0095 & 50.70 & 46.57 & 42.60 & 1.510 & 1.510 & 1.499 \\ 
  1977 & 0.0067 & 0.0076 & 0.0083 & 51.04 & 46.86 & 42.88 & 1.409 & 1.425 & 1.429 \\ 
  1978 & 0.0053 & 0.0063 & 0.0071 & 51.15 & 46.92 & 42.92 & 1.313 & 1.343 & 1.358 \\ 
  1979 & 0.0033 & 0.0042 & 0.0051 & 51.42 & 47.10 & 43.03 & 1.186 & 1.221 & 1.244 \\ 
  1980 & 0.0012 & 0.0024 & 0.0034 & 51.46 & 47.18 & 43.19 & 1.065 & 1.119 & 1.160 \\ 
  1981 & -0.0015 & 0.0000 & 0.0013 & 51.21 & 46.86 & 42.91 & 0.927 & 1.000 & 1.058 \\ 
  1982 & -0.0026 & -0.0011 & 0.0002 & 51.41 & 47.03 & 43.04 & 0.875 & 0.949 & 1.008 \\ 
  1983 & -0.0046 & -0.0032 & -0.0020 & 50.86 & 46.48 & 42.47 & 0.792 & 0.862 & 0.919 \\ 
  1984 & -0.0056 & -0.0041 & -0.0028 & 51.07 & 46.55 & 42.44 & 0.752 & 0.827 & 0.889 \\ 
  1985 & -0.0068 & -0.0053 & -0.0041 & 50.81 & 46.25 & 42.08 & 0.709 & 0.781 & 0.841 \\ 
  1986 & -0.0081 & -0.0066 & -0.0053 & 50.71 & 46.20 & 42.11 & 0.664 & 0.737 & 0.799 \\ 
  1987 & -0.0090 & -0.0075 & -0.0062 & 50.75 & 46.18 & 42.06 & 0.632 & 0.707 & 0.771 \\ 
  1988 & -0.0097 & -0.0082 & -0.0069 & 50.70 & 46.06 & 41.87 & 0.613 & 0.686 & 0.750 \\ 
  1989 & -0.0105 & -0.0090 & -0.0077 & 50.64 & 45.91 & 41.65 & 0.589 & 0.662 & 0.726 \\ 
  1990 & -0.0110 & -0.0095 & -0.0082 & 50.51 & 45.75 & 41.46 & 0.573 & 0.646 & 0.710 \\ 
  1991 & -0.0115 & -0.0100 & -0.0086 & 50.39 & 45.61 & 41.32 & 0.561 & 0.635 & 0.700 \\ 
  1992 & -0.0114 & -0.0099 & -0.0086 & 50.69 & 45.81 & 41.41 & 0.562 & 0.635 & 0.700 \\ 
  1993 & -0.0123 & -0.0107 & -0.0094 & 50.35 & 45.52 & 41.19 & 0.539 & 0.614 & 0.679 \\ 
  1994 & -0.0133 & -0.0118 & -0.0105 & 50.12 & 45.27 & 40.93 & 0.512 & 0.586 & 0.652 \\ 
  1995 & -0.0138 & -0.0123 & -0.0110 & 49.92 & 45.03 & 40.64 & 0.502 & 0.575 & 0.640 \\ 
  1996 & -0.0139 & -0.0124 & -0.0111 & 49.88 & 44.98 & 40.57 & 0.501 & 0.573 & 0.636 \\ 
  1997 & -0.0134 & -0.0119 & -0.0107 & 50.20 & 45.37 & 41.04 & 0.510 & 0.582 & 0.646 \\ 
  1998 & -0.0139 & -0.0123 & -0.0110 & 50.09 & 45.28 & 41.01 & 0.497 & 0.572 & 0.638 \\ 
  1999 & -0.0128 & -0.0114 & -0.0102 & 50.23 & 45.43 & 41.08 & 0.526 & 0.596 & 0.659 \\ 
  2000 & -0.0118 & -0.0103 & -0.0090 & 50.75 & 45.98 & 41.69 & 0.551 & 0.623 & 0.687 \\ 
  2001 & -0.0112 & -0.0099 & -0.0089 & 51.08 & 46.31 & 41.93 & 0.566 & 0.631 & 0.689 \\ 
  2002 & -0.0108 & -0.0095 & -0.0084 & 51.23 & 46.46 & 42.12 & 0.574 & 0.642 & 0.702 \\ 
  2003 & -0.0097 & -0.0086 & -0.0075 & 51.26 & 46.59 & 42.29 & 0.608 & 0.671 & 0.727 \\ 
  2004 & -0.0093 & -0.0081 & -0.0070 & 51.84 & 47.11 & 42.77 & 0.618 & 0.683 & 0.740 \\ 
  2005 & -0.0088 & -0.0078 & -0.0069 & 51.85 & 47.16 & 42.79 & 0.633 & 0.691 & 0.743 \\ 
  2006 & -0.0081 & -0.0071 & -0.0063 & 52.48 & 47.79 & 43.41 & 0.654 & 0.712 & 0.762 \\ 
  2007 & -0.0079 & -0.0070 & -0.0062 & 52.55 & 47.86 & 43.46 & 0.661 & 0.715 & 0.763 \\ 
  2008 & -0.0066 & -0.0058 & -0.0050 & 52.93 & 48.32 & 43.99 & 0.703 & 0.756 & 0.801 \\ 
  2009 & -0.0081 & -0.0071 & -0.0063 & 52.78 & 48.18 & 43.88 & 0.651 & 0.709 & 0.759 \\ 
   \hline
\end{tabular}

  \end{adjustwidth}
\end{table}



