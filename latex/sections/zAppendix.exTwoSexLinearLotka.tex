\chapter{Equation~\ref{eq:lineartwosexrenewal} applied to the US and Spanish
data}
\label{appendix:ex2sexlinear}
This appendix provides numerical results from the application of the
estimation techniques described in Section~\ref{sec:exLotka2linear}. Estimates
of $r^\upsilon$, $T^\upsilon$ and $R_0^\upsilon$ under three values of $\sigma$.
Recall that $\sigma$ determines the relative dominance of each sex, where 0 is
female dominance, 1 is male dominant, and .5 is an intermediate level of
dominance (which is not necessarily an arithmetic mean). As stated in the
main text, the border cases where $\sigma = [0,1]$ do not revert to the
single-sex $e_x$-structured values presented in Section~\ref{sec:exstructuredrenewal} and
Appendix~\ref{appendix:exlotka1sex}.

\begin{table}
  \begin{adjustwidth}{-1in}{-.5in}
  \centering
    \caption{Two-sex linear intrinsic growth rate, $r^\upsilon$, mean remaining
years of life at reproduction, $T^\upsilon$, and net reproduction,
$R_0^\upsilon$, according to renewal Equation~\eqref{eq:lineartwosexrenewal},
with $\sigma = [0, .5, 1]$ US, 1969-2009.}
    \label{tab:ex2linRepUS}
        % latex table generated in R 2.15.3 by xtable 1.7-1 package
% Wed May  1 09:28:38 2013
\begin{tabular}{cccccccccc}
  \hline
 & $r^{\upsilon (\sigma = 0)}$ & $r^{\upsilon (\sigma = .5)}$ & $r^{\upsilon (\sigma = 1)}$ & $T^{\upsilon (\sigma = 0)}$ & $T^{\upsilon (\sigma = .5)}$ & $T^{\upsilon (\sigma = 1)}$ & $R_0^{\upsilon (\sigma = 0)}$ & $R_0^{\upsilon (\sigma = .5)}$ & $R_0^{\upsilon (\sigma = 1)}$ \\ 
  \hline
1969 & 0.0050 & 0.0060 & 0.0069 & 50.62 & 46.01 & 41.63 & 1.289 & 1.316 & 1.330 \\ 
  1970 & 0.0058 & 0.0068 & 0.0077 & 51.08 & 46.45 & 42.06 & 1.346 & 1.370 & 1.380 \\ 
  1971 & 0.0038 & 0.0047 & 0.0056 & 50.91 & 46.30 & 41.92 & 1.211 & 1.243 & 1.263 \\ 
  1972 & 0.0004 & 0.0013 & 0.0023 & 50.38 & 45.73 & 41.33 & 1.018 & 1.063 & 1.098 \\ 
  1973 & -0.0013 & -0.0003 & 0.0007 & 50.27 & 45.63 & 41.26 & 0.936 & 0.987 & 1.028 \\ 
  1974 & -0.0015 & -0.0003 & 0.0007 & 50.72 & 46.08 & 41.72 & 0.929 & 0.985 & 1.029 \\ 
  1975 & -0.0019 & -0.0008 & 0.0003 & 51.10 & 46.39 & 41.97 & 0.908 & 0.965 & 1.011 \\ 
  1976 & -0.0020 & -0.0008 & 0.0002 & 51.18 & 46.50 & 42.11 & 0.904 & 0.962 & 1.009 \\ 
  1977 & -0.0007 & 0.0004 & 0.0014 & 51.73 & 47.03 & 42.60 & 0.966 & 1.020 & 1.062 \\ 
  1978 & -0.0010 & 0.0002 & 0.0012 & 51.75 & 47.05 & 42.64 & 0.952 & 1.007 & 1.051 \\ 
  1979 & 0.0003 & 0.0014 & 0.0023 & 52.34 & 47.61 & 43.15 & 1.017 & 1.068 & 1.106 \\ 
  1980 & 0.0010 & 0.0021 & 0.0030 & 52.30 & 47.64 & 43.24 & 1.055 & 1.103 & 1.138 \\ 
  1981 & 0.0010 & 0.0020 & 0.0029 & 52.42 & 47.76 & 43.36 & 1.052 & 1.099 & 1.134 \\ 
  1982 & 0.0012 & 0.0022 & 0.0031 & 52.63 & 47.99 & 43.59 & 1.067 & 1.111 & 1.144 \\ 
  1983 & 0.0006 & 0.0016 & 0.0025 & 52.37 & 47.77 & 43.42 & 1.031 & 1.078 & 1.113 \\ 
  1984 & 0.0007 & 0.0016 & 0.0025 & 52.35 & 47.77 & 43.42 & 1.037 & 1.080 & 1.113 \\ 
  1985 & 0.0013 & 0.0021 & 0.0030 & 52.39 & 47.79 & 43.43 & 1.068 & 1.108 & 1.137 \\ 
  1986 & 0.0010 & 0.0019 & 0.0026 & 52.34 & 47.75 & 43.36 & 1.056 & 1.093 & 1.120 \\ 
  1987 & 0.0013 & 0.0021 & 0.0028 & 52.39 & 47.81 & 43.44 & 1.072 & 1.106 & 1.130 \\ 
  1988 & 0.0020 & 0.0027 & 0.0033 & 52.45 & 47.89 & 43.52 & 1.109 & 1.137 & 1.156 \\ 
  1989 & 0.0029 & 0.0036 & 0.0042 & 52.90 & 48.34 & 43.95 & 1.167 & 1.189 & 1.201 \\ 
  1990 & 0.0037 & 0.0043 & 0.0048 & 53.23 & 48.69 & 44.30 & 1.216 & 1.231 & 1.238 \\ 
  1991 & 0.0031 & 0.0036 & 0.0040 & 53.22 & 48.69 & 44.30 & 1.177 & 1.190 & 1.196 \\ 
  1992 & 0.0023 & 0.0029 & 0.0034 & 53.15 & 48.65 & 44.29 & 1.131 & 1.151 & 1.163 \\ 
  1993 & 0.0014 & 0.0020 & 0.0025 & 52.70 & 48.25 & 43.95 & 1.077 & 1.100 & 1.116 \\ 
  1994 & 0.0008 & 0.0013 & 0.0018 & 52.60 & 48.19 & 43.93 & 1.041 & 1.066 & 1.084 \\ 
  1995 & 0.0000 & 0.0006 & 0.0012 & 52.40 & 48.05 & 43.86 & 1.001 & 1.030 & 1.052 \\ 
  1996 & -0.0003 & 0.0003 & 0.0008 & 52.41 & 48.19 & 44.12 & 0.985 & 1.014 & 1.037 \\ 
  1997 & -0.0007 & -0.0001 & 0.0005 & 52.45 & 48.36 & 44.42 & 0.966 & 0.997 & 1.022 \\ 
  1998 & -0.0004 & 0.0001 & 0.0007 & 52.57 & 48.56 & 44.69 & 0.978 & 1.007 & 1.030 \\ 
  1999 & -0.0006 & 0.0000 & 0.0006 & 52.46 & 48.54 & 44.75 & 0.971 & 1.001 & 1.025 \\ 
  2000 & 0.0000 & 0.0006 & 0.0011 & 52.61 & 48.74 & 44.99 & 1.002 & 1.028 & 1.049 \\ 
  2001 & -0.0004 & 0.0001 & 0.0006 & 52.54 & 48.70 & 44.97 & 0.980 & 1.006 & 1.026 \\ 
  2002 & -0.0006 & -0.0001 & 0.0004 & 52.53 & 48.70 & 45.00 & 0.967 & 0.995 & 1.017 \\ 
  2003 & -0.0003 & 0.0003 & 0.0007 & 52.66 & 48.89 & 45.24 & 0.986 & 1.013 & 1.034 \\ 
  2004 & -0.0002 & 0.0003 & 0.0008 & 53.04 & 49.29 & 45.66 & 0.988 & 1.015 & 1.036 \\ 
  2005 & -0.0002 & 0.0003 & 0.0008 & 53.11 & 49.38 & 45.77 & 0.989 & 1.015 & 1.036 \\ 
  2006 & 0.0006 & 0.0011 & 0.0015 & 53.65 & 49.94 & 46.34 & 1.033 & 1.056 & 1.074 \\ 
  2007 & 0.0009 & 0.0013 & 0.0017 & 53.98 & 50.29 & 46.71 & 1.048 & 1.068 & 1.083 \\ 
  2008 & 0.0002 & 0.0006 & 0.0011 & 53.84 & 50.21 & 46.69 & 1.009 & 1.033 & 1.052 \\ 
  2009 & -0.0009 & -0.0003 & 0.0002 & 53.84 & 50.23 & 46.74 & 0.955 & 0.984 & 1.007 \\ 
   \hline
\end{tabular}

  \end{adjustwidth}
\end{table}

\begin{table}
  \begin{adjustwidth}{-1in}{-.5in}
    \centering
    \caption{Two-sex linear intrinsic growth rate, $r^\upsilon$, mean remaining
years of life at reproduction, $T^\upsilon$, and net reproduction,
$R_0^\upsilon$, according to renewal Equation~\eqref{eq:lineartwosexrenewal}, with
$\sigma = [0, .5, 1]$ Spain, 1975-2009.}
    \label{tab:ex2linRepES}
        % latex table generated in R 2.15.2 by xtable 1.7-0 package
% Wed Mar  6 13:00:01 2013
\begin{tabular}{cccccccccc}
  \hline
 & $r^{\upsilon (\sigma = 0)}$ & $r^{\upsilon (\sigma = .5)}$ & $r^{\upsilon (\sigma = 1)}$ & $T^{\upsilon (\sigma = 0)}$ & $T^{\upsilon (\sigma = .5)}$ & $T^{\upsilon (\sigma = 1)}$ & $R_0^{\upsilon (\sigma = 0)}$ & $R_0^{\upsilon (\sigma = .5)}$ & $R_0^{\upsilon (\sigma = 1)}$ \\ 
  \hline
1975 & 0.0085 & 0.0085 & 0.0084 & 49.91 & 46.04 & 42.16 & 1.531 & 1.476 & 1.426 \\ 
  1976 & 0.0087 & 0.0086 & 0.0085 & 50.43 & 46.56 & 42.68 & 1.550 & 1.493 & 1.440 \\ 
  1977 & 0.0072 & 0.0073 & 0.0074 & 50.76 & 46.85 & 42.97 & 1.443 & 1.409 & 1.374 \\ 
  1978 & 0.0058 & 0.0060 & 0.0062 & 50.86 & 46.91 & 43.01 & 1.342 & 1.327 & 1.307 \\ 
  1979 & 0.0035 & 0.0040 & 0.0044 & 51.08 & 47.07 & 43.16 & 1.194 & 1.207 & 1.209 \\ 
  1980 & 0.0014 & 0.0022 & 0.0027 & 51.15 & 47.14 & 43.30 & 1.075 & 1.107 & 1.126 \\ 
  1981 & -0.0013 & -0.0002 & 0.0006 & 50.91 & 46.84 & 43.01 & 0.938 & 0.990 & 1.025 \\ 
  1982 & -0.0025 & -0.0013 & -0.0004 & 51.09 & 46.99 & 43.16 & 0.879 & 0.939 & 0.982 \\ 
  1983 & -0.0048 & -0.0034 & -0.0023 & 50.52 & 46.41 & 42.63 & 0.785 & 0.854 & 0.906 \\ 
  1984 & -0.0058 & -0.0043 & -0.0031 & 50.71 & 46.48 & 42.61 & 0.743 & 0.820 & 0.877 \\ 
  1985 & -0.0073 & -0.0055 & -0.0042 & 50.42 & 46.15 & 42.29 & 0.694 & 0.775 & 0.836 \\ 
  1986 & -0.0086 & -0.0068 & -0.0054 & 50.32 & 46.10 & 42.32 & 0.648 & 0.732 & 0.797 \\ 
  1987 & -0.0096 & -0.0077 & -0.0062 & 50.34 & 46.07 & 42.29 & 0.616 & 0.703 & 0.771 \\ 
  1988 & -0.0104 & -0.0083 & -0.0067 & 50.26 & 45.93 & 42.13 & 0.592 & 0.682 & 0.753 \\ 
  1989 & -0.0114 & -0.0091 & -0.0074 & 50.17 & 45.77 & 41.94 & 0.566 & 0.659 & 0.732 \\ 
  1990 & -0.0120 & -0.0097 & -0.0079 & 50.04 & 45.60 & 41.76 & 0.549 & 0.643 & 0.718 \\ 
  1991 & -0.0125 & -0.0101 & -0.0083 & 49.92 & 45.46 & 41.62 & 0.537 & 0.633 & 0.708 \\ 
  1992 & -0.0124 & -0.0100 & -0.0082 & 50.19 & 45.64 & 41.74 & 0.536 & 0.633 & 0.710 \\ 
  1993 & -0.0133 & -0.0108 & -0.0090 & 49.88 & 45.36 & 41.49 & 0.516 & 0.612 & 0.688 \\ 
  1994 & -0.0144 & -0.0119 & -0.0100 & 49.63 & 45.10 & 41.26 & 0.489 & 0.585 & 0.662 \\ 
  1995 & -0.0150 & -0.0124 & -0.0105 & 49.42 & 44.84 & 40.98 & 0.477 & 0.574 & 0.651 \\ 
  1996 & -0.0151 & -0.0125 & -0.0106 & 49.38 & 44.79 & 40.91 & 0.476 & 0.572 & 0.649 \\ 
  1997 & -0.0145 & -0.0120 & -0.0102 & 49.75 & 45.20 & 41.33 & 0.487 & 0.581 & 0.656 \\ 
  1998 & -0.0149 & -0.0124 & -0.0106 & 49.67 & 45.13 & 41.26 & 0.478 & 0.571 & 0.646 \\ 
  1999 & -0.0138 & -0.0115 & -0.0097 & 49.80 & 45.26 & 41.35 & 0.503 & 0.595 & 0.668 \\ 
  2000 & -0.0126 & -0.0104 & -0.0088 & 50.35 & 45.85 & 41.93 & 0.532 & 0.621 & 0.692 \\ 
  2001 & -0.0121 & -0.0100 & -0.0085 & 50.65 & 46.15 & 42.20 & 0.542 & 0.629 & 0.698 \\ 
  2002 & -0.0116 & -0.0097 & -0.0082 & 50.83 & 46.33 & 42.37 & 0.554 & 0.639 & 0.707 \\ 
  2003 & -0.0104 & -0.0087 & -0.0074 & 50.89 & 46.47 & 42.51 & 0.589 & 0.668 & 0.731 \\ 
  2004 & -0.0099 & -0.0082 & -0.0070 & 51.48 & 47.00 & 42.98 & 0.602 & 0.680 & 0.741 \\ 
  2005 & -0.0094 & -0.0080 & -0.0068 & 51.49 & 47.04 & 43.00 & 0.615 & 0.688 & 0.746 \\ 
  2006 & -0.0086 & -0.0073 & -0.0062 & 52.13 & 47.69 & 43.61 & 0.639 & 0.707 & 0.762 \\ 
  2007 & -0.0084 & -0.0071 & -0.0062 & 52.21 & 47.76 & 43.65 & 0.646 & 0.711 & 0.763 \\ 
  2008 & -0.0069 & -0.0059 & -0.0052 & 52.63 & 48.25 & 44.15 & 0.694 & 0.751 & 0.796 \\ 
  2009 & -0.0084 & -0.0073 & -0.0064 & 52.50 & 48.11 & 44.02 & 0.643 & 0.705 & 0.755 \\ 
   \hline
\end{tabular}

  \end{adjustwidth}
\end{table}



