% 

\chapter{Fertility rates by remaining years of life under different assumed
reproductive spans}
\label{Appendix:reprospans}
One may rightly wish to see $e_x$-classified fertility rates calculated where
exposures in the denominator are taken only from ages within the known
reproductive span. For many, this will more closely represent the population exposed. Bounding the
original age-clsasified exposures introduces a second problem, namely that of
determining the which age-bounds to use for males and females. Results are
sensitive to the choice, especially when comparing males and females, since 1)
the male reproductive span is much longer than the female span, and 2) the
$e_x$-distributed population shows a greater and steadier sex-imbalance than the
age-classified population. As expected, results are sensitive to the choice of
bounds. In following, Figures~\ref{fig:exSFRsurfUS},~\ref{fig:exSFRsurfES}
and~\ref{fig:exTFR} are reproduced after first limiting original
age-classified exposures to certain reproductive bounds. These include:

\begin{itemize}
  \item ages 15-55 for both males and females.
  \item ages 13-49 for females and 15-64 for males.
  \item ages higher than the 1st and lower than the 99th quantiles of ASFR for
  males and females separately and over the entire period studied.
  \item ages higher than the 1st and lower than the 99th quantiles of ASFR for
  each year for males and females separately. ($e_x$-TFR only).
\end{itemize}

\pagebreak

\section{ages 15-55 for both males and females}

\begin{figure}[ht!]
        \centering
        \begin{subfigure}
                \centering
                \caption{Male and Female $e_x$-SFR surfaces, 1969-2009, USA,
                redistributing exposures only from ages 15-55}
                \includegraphics[scale = .8]{Figures/eSFRsurfacesUSlim1555}
                \label{fig:exSFRsurfUSlim15_55}
        \end{subfigure}
        \begin{subfigure}
                \centering
                \caption{Male and Female $e_x$-SFR surfaces, 1975-2009, Spain,
                redistributing exposures only from ages 15-55}
                \includegraphics[scale = .8]{Figures/eSFRsurfacesESlim1555} 
                \label{fig:exSFRsurfESlim15_55}
        \end{subfigure}
\end{figure}

\begin{figure}[ht!]
        \centering  
          \caption{Male and Female $e_x$-total fertility rates, Spain
          and USA, 1969-2009}
           % figure produced in
           % /R/Parents_ex.R
           \includegraphics{Figures/exTFRlim1555}
          \label{fig:exTFRlim15_55}
\end{figure}
\pagebreak

\section{ages 13-49 for females and 15-64 for males}

\begin{figure}[ht!]
        \centering
        \begin{subfigure}
                \centering
                \caption{Male and Female $e_x$-SFR surfaces, 1969-2009, USA,
                redistributing exposures only from ages 13-49 for females and 15-64 for males}
                \includegraphics[scale = .8]{Figures/eSFRsurfacesUSlim1364mixed}
                \label{fig:exSFRsurfUSlim1364}
        \end{subfigure}
        \begin{subfigure}
                \centering
                \caption{Male and Female $e_x$-SFR surfaces, 1975-2009, Spain,
                redistributing exposures only from ages 13-49 for females and 15-64 for males}
                \includegraphics[scale = .8]{Figures/eSFRsurfacesESlim1364mixed} 
                \label{fig:exSFRsurfESlim1364}
        \end{subfigure}
\end{figure}

\begin{figure}[ht!]
        \centering  
          \caption{Male and Female $e_x$-total fertility rates, Spain
          and USA, 1969-2009}
           % figure produced in
           % /R/Parents_ex.R
           \includegraphics{Figures/exTFRlim1364mixed}
          \label{fig:exTFRlim13_64}
\end{figure}

\pagebreak
\section{ages higher than the 1st and lower than the 99th quantiles of ASFR,
full period}

\begin{figure}[ht!]
        \centering
        \begin{subfigure}
                \centering
                \caption{Male and Female $e_x$-SFR surfaces, 1969-2009, USA,
                redistributing exposures only from the 1st-99th quantiles of
                ASFR over the full period}
                \includegraphics[scale = .8]{Figures/eSFRsurfacesUSlimBquant}
                \label{fig:exSFRsurfUSlimBquant}
        \end{subfigure}
        \begin{subfigure}
                \centering
                \caption{Male and Female $e_x$-SFR surfaces, 1975-2009, Spain,
                redistributing exposures only from the 1st-99th quantiles of
                ASFR over the full period}
                \includegraphics[scale = .8]{Figures/eSFRsurfacesESlimBquant} 
                \label{fig:exSFRsurfESlimBquant}
        \end{subfigure}
\end{figure}

\begin{figure}[ht!]
        \centering  
          \caption{Male and Female $e_x$-total fertility rates, Spain
          and USA, 1969-2009}
           % figure produced in
           % /R/Parents_ex.R
           \includegraphics{Figures/exTFRlimBquant}
          \label{fig:exTFRlimBquant}
\end{figure}
\pagebreak

\section{ages higher than the 1st and lower than the 99th quantiles of ASFR,
each year}

In comparing Figures~\ref{fig:exTFRlimBquant}~and~\ref{fig:exTFRlimBquantyr},
one notes that flexibly changing the age bounds included in $e_x$-classified
exposures according to year-to-year changing ASFR quantiles does not make much
difference as compared to holding the same bounds over the entire period. If
the central 98\% of fertility moves over age with time, then year-to-year
flexbility may be desirable. These data do not undergo large enough changes in
these thresholds to justify this practice. Further, surfaces are best rendered
based upon constant bounds.

 \begin{figure}[ht!]
        \centering  
          \caption{Male and Female $e_x$-total fertility rates, Spain
          and USA, 1969-2009}
           % figure produced in
           % /R/Parents_ex.R
           \includegraphics{Figures/exTFRlimBquantyr}
          \label{fig:exTFRlimBquantyr}
\end{figure}
