
Schoen \citep{schoen1978standardized, schoen1977two, schoen1981harmonic}
provided a rationale and derivation for using the harmonic mean in order to
balance marriage rates. The same strategy can be used to balance fertility
rates, which is what we will do here. The method requires as inputs a matrix of
birth counts cross-tabulated by age of mother $a'$ and age of father $a$, and
male and female exposures classified by age. The harmonic mean: 

\begin{equation}
H(P_a^m, P_{a'}^f) = \frac{2 P_a^m P_{a'}^f}{P_a^m + P_{a'}^f}
\end{equation}
is applied to exposures, in order to find an intermediate denominator
from which to calculate rates, $F_{a,a'}^H$:
 
 \begin{equation}
 F_{a,a'}^H = \frac{B_{a,a'}}{H(P_a^m, P_{a'}^f)}
 \end{equation}
which in the stable population is assumed constant in time rather than
assuming constant male and female rates separately. In order to estimate 
a birth count in some future year $t+n$, calculate the harmonic mean
of male and female exposures and multiply into the constant harmonic rate:

 \begin{equation}
 B_{a,a',t+n} = F_{a,a'}^H H(P_{a, t+n}^m, P_{a',t+n}^f)
 \end{equation}

Since the effective birth count in the future is a function of both male and
female exposures, the method is inherently non-linear. 

Let us place this fertility balancing strategy into a stable
population framework. \citet{schoen1977two} proposed his own rectangular stable
population framework, which will not be treated here. Instead, we will make some minimal
changes to the basic components of the familiar Lotka system. In the first
place, the new Lotka equation, based on equation~\eqref{eq:lotkaeq} will now
contain both male and female survival, $p_a^m$ and $p_{a'}^f$, respectively.
Further, we must have some notion of the stable versus the present sex ratio at
birth, from which we derive proportions male, $\varsigma$, and female
$1-\varsigma$. Assume for the time being that we know the stable proportion male
of births. Then we can estimate the two-sex harmonic growth rate using:

\begin{equation}
\label{eq:lotkaH}
1 = \int _{a=0}^\infty \int _{a'=0}^\infty \frac{2 \varsigma e^{-ra} p_a^m
(1 - \varsigma) e^{-ra'} p_{a'}^f}{\varsigma e^{-ra} p_a^m  + (1 -
\varsigma) e^{-ra'} p_{a'}^f} F_{a,a'}^H \dd a' \dd a
\end{equation}
where $F_{a,a'}^H$ is the constant fertility rate to be applied to the harmonic
mean of male and female exposures, $p_a^m$ and $p_{a'}^f$ are the male
and female radix-1 survival functions. $\varsigma$ serves to make the
male and female radices sum to 1, and also accounts for the fact that males and
females have slightly different $l_0$ values. 

The two-sex harmonic intrinsic growth rate, $r$ can be estimated by modifying
the iterative procedure suggested by \citet{coale1957new}. In practice, one
estimates the stable sex ratio at birth at the same time. Here we will describe
the practical steps involved in the estimation

\begin{enumerate}
  \item Calculate the constant harmonic fertility rates for male and female
  births separately, $F_{a,a'}^{mH}$ and $F_{a,a'}^{fH}$
  \item Make a first estimate of the stable sex ratio at birth, $\hat{S}$,
  assuming an intrinsic growth rate of 0. Here $H()$ will be used as shorthand for the
  harmonic mean function:
  \begin{equation}
  S^0 = \frac{\int_{a=o}^\infty \int_{a'=0}^\infty H(p_a^m, p_{a'}^f)
  F_{a,a'}^{mH} \dd a' \dd a }{\int_{a=o}^\infty \int_{a'=0}^\infty H(p_a^m, p_{a'}^f)
  F_{a,a'}^{fH} \dd a' \dd a }
  \end{equation}
  From $S^0$ we derive a first estimate of the proportion male of
  births, $\varsigma^0$, as:
  \begin{equation}
  \varsigma^0 = \frac{S^0}{S^0+1}
  \end{equation}
  \item Find a first rough estimate of the net reproduction rate,
  $\widehat{R_0}$, assuming a growth rate of 0 and using the both-sex
  harmonic fertility rate $F_{a,a'}^{H} = F_{a,a'}^{mH} + F_{a,a'}^{fH}$:
  \begin{equation}
  \widehat{R_0} = \int_{a=o}^\infty \int_{a'=0}^\infty H(\varsigma^0 p_a^m,
  (1-\varsigma^0)p_{a'}^f) F_{a,a'}^{H} \dd a' \dd a
  \end{equation}
  \item Assume a reasonable both-sex mean generation time, $\widehat{T}$, such
  as 35, and calculate an intitial value of $r$, $r^0$ as:
  \begin{equation}
  r^0 = \frac{log(\widehat{R_0})}{\widehat{T}}
  \end{equation}
  \item Now that we have a starting value, $r^0$, calculate a residual,
  $\delta^0$, from equation~\eqref{eq:lotkaH}, where the fraction is
  again replaced with $H()$ as shorthand:
  \begin{equation}
  \delta^0 = 1 - \int _{a=0}^\infty \int _{a'=0}^\infty H(\varsigma^0 p_a^m,
  (1-\varsigma^0)p_{a'}^f) F_{a,a'}^H \dd a' \dd a
  \end{equation}
  \item Use $\delta$ to improve the esimate of $r$:
  \begin{equation}
  r^{i+1} = r^i - \frac{\delta^i}{\widehat{T} - \frac{\delta^i}{r^i}}
  \end{equation}
  \item Use the improved estiamte of $r$ to update $\varsigma$:
  \begin{align}
  S^{i+1} &= \frac{\int_{a=o}^\infty \int_{a'=0}^\infty H(\varsigma^i
  e^{-r^{i+1}a} p_a^m, (1-\varsigma^i)^i e^{-r^{i+1}a'}p_{a'}^f) F_{a,a'}^{mH} \dd a' \dd a
  }{\int_{a=o}^\infty \int_{a'=0}^\infty H(\varsigma^i e^{-r^{i+1}a}
  p_a^m, (1-\varsigma^i)^i e^{-r^{i+1}a'}p_{a'}^f) F_{a,a'}^{fH} \dd a' \dd a }
  \\
  \varsigma^{i+1} &= \frac{S^{i+1}}{S^{i+1}+1}
  \end{align}
  \item Plug the new $\varsigma$ and $r$ estimates into step 5, to estimate a
  new residual, $\delta$, repeating steps 5-7 until $\delta$ is reduced to 0.
  Typicaly around 20 iterations are needed in order to reduce $\delta$ to
  be less than double floating point machine tolerance.
\end{enumerate}

The present iterative procedure simulaneously produces an estimate of the stable
sex ratio at birth $S$ and the both-sex intrinsic growth rate, $r$. Really,
there is little room for $S$ to move between the initial and stable states,
since boy and girl births are in essence produced by (the harmonic mean of) both
males and females in this procedure. $S$
will only vary from the intial sex ratio at birth to the extent that there is
an age pattern to the sex ratio at birth and the male and female stable age
structures differ from the initial age structures. Estimating both parameters at
the same time does not present a practical problem in the present case.

Once one has the stable $r$ and $S$, one may move on to estimate other stable
parameters of interest, such as the both-sex stable birth rate, $b$:

\begin{equation}
b = \frac{1}{\int_{a = 0}^\infty e^{-ra} \varsigma p_a^m \dd a + \int_{a' =
0}^\infty e^{-ra'} \varsigma p_{a'}^f \dd a'}
\end{equation}
which may be used to calculate the male and female stable age structures, $c_a$
and $c_{a'}$:

\begin{equation}
c_a =  \varsigma  e^{-ra} p_a^m
\end{equation}
and analagously for females, where
\begin{equation}
1 = \int c_a + \int c_{a'}
\end{equation}
and the total population sex ratio, $S^{tot}$ is the ratio of these:
\begin{equation}
S^{tot} = \frac{\int c_a}{\int c_{a'}}
\end{equation}

Of interest, male and female stable ASFR may also be estimated, where
$B_{a,a'}$ are cross-classified birth counts:
\begin{equation}
B_{a,a'} = F_{a,a'}^H H(c_a,c_{a'})
\end{equation}
we integrate over male (female) age to get the balanced marginal rates:
\begin{equation}
\label{eq:asfrH}
F_a^m = \int_a ^\infty B_{a,a'} c_a \dd a
\end{equation}
and analagously for females. 

One could prescind of the iterative $r$ estimation procedure outlined above and
perform a standard cohort component projection, for instance using a
two-sex Leslie matrix. In this case, the fertility component of the Leslie
matrix would need to be updated between each iteration using equation~\ref{eq:asfrH} for either
males or females. One cannot easily perform standard matrix analysis of this
Leslie matrix, however, as it is not static in the standard way.

The stable system outline here is not taken word-for-word from Schoen's advice,
but is consistent with his notions of a constant \textit{force of attraction}





