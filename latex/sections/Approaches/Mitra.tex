\subsection{Mitra}

\citet{mitra1978derivation}, was directly cocerned with finding a consistent
method to derive a two-sex intrinsic growth rate, $r^\ast$. Consistent here
means that 1) a constant SRB is maintained in and along the trajectory to stability, 2) the
essential \textit{shape} of fertility rates is held constant along the path to
stability and 3) the stable $r^\ast$ is guaranteed to be bracketted by $r^m$ and
$r^f$.

The method proposed by \citet{mitra1978derivation} works by assigning
complimentary scalar (uniform over age) weights, $v_0$ and $1 - v_0$, to male
and female single-sex fertility rates and placing the initially weighted
fertility rates, which are then held constant, into a unified two-sex Lotka
equation in order to determine $r^\ast$:

\begin{equation}
\label{mitralotka}
1 = \int _{a=0} ^\infty e^{-r^\ast a} \left(\frac{F_a ^m}{v_0} p_a^m  +
\frac{F_a ^f}{1 - v_0} p_a^f\right) \dd a
\end{equation}
, where $F_a ^m$ and $F_a ^f$ are the male and female fertility probabilities
(rates when discrete), $p_a^m$ and $p_a^f$ the male and female probabilities of
surviving from age $0$ to age $a$ ($L_a$ from a radix-1 lifetable when
discretized).

One can quickly converge upon a solution for Mitra's $r^\ast$ by modifying the
method proposed in \citet{coale1957new}\footnote{\citet{mitra1978derivation} alludes
to, but does not show this.}, by first calculating a trial estimate of $r\ast$,
$\widehat{r^\ast}$ and a trial two-sex mean generation length
$\widehat{T^\ast}$, say, the arithmetic means of the single sex Lotka
parameters. Using the trial $\widehat{r^\ast}^{(1)}$, one first calculates the
residual, $\delta ^{(1)}$, from equation~\eqref{mitralotka}, then updates $\widehat{r^\ast}$ as:
\begin{equation}
\widehat{r^\ast}^{(i+1)} = \widehat{r^\ast}^{(i)} + \frac{\delta
^{(i)}}{\widehat{T^\ast} -
\frac{\delta ^{(i)}}{\widehat{r^\ast}^{(i)} }}
\end{equation}
This procedure typically converges after half a dozen iterations, both faster
and more precise than a typical optimizer solution.

For a given set of starting weights, one can in this way arrive at a given two-sex growth rate, $r^\ast$. 
However, weights are also constrained to produce a constant sex ratio at birth (SRB). Given $r^\ast$ applied 
to each sex separately in the
state of stability, one notes that this sex ratio is \textit{not} maintained,
and must dervive stable weights, $v^\ast$ in order to force the final
SRB:

\begin{equation}
v^\ast =  \int _{a = 0} ^\infty e ^{ -r^\ast a} \frac{F_a ^m}{v_0} p_a^m \dd a
\end{equation}
, that is: a given $v_0$ will always result in a single, stable $v^\ast$.
Mitra's two-sex growth rate, $r^\ast$, is unique for but depends upon the starting weights, $v_0$, and thus is not in general unique. Mitra suggests that a good choice for $v_0$ would be the value that minimzes the 
departure from constancy for unweighted single-sex fertility
rates. This is an attractive choice because constant rates are of course the
basis of stability. Once a population attains stability, weights, and therefore rates, are constant. In practice, one
choses the $v_0$ that minimizes the sum of the age-specific squared residuals
(for males and females) between $F_a$ and $F_a \times \tfrac{v_0}{v^\ast}$. 

If minimizing the difference between starting and stable rates is the criterion
for choosing $v_0$, Mitra's starting and stable weights form the time
series seen in Figure~\ref{fig:Mitra1978v0vstar}

\begin{figure}[ht!]
        \centering  
          \caption{Initial ($v_0$) and stable ($v^\ast$) weights for Spain and
          US, 1969-2009, according to the OLS criterion in
          \citet{mitra1978derivation}}
           \quad
           % /R/Mitra1978.R
           \makebox[\textwidth]{\includegraphics{Figures/Mitra1978v0vstar}}
          \label{fig:Mitra1978v0vstar}
\end{figure}

For Spain and the US throughout the period studied in this dissertation, both
$v_0$ and $v^\ast$ fell in the range $(.48,.6)$. $v_0$ was always
close to $.5$, entailing nearly equal weight for male and female rates.
The stable $v^\ast$ was consistently higher than $v_0$ and always higher than
$.5$, implying greater weights for males than females in stability. When $v > .5$, male rates weight
more than female rates, which was typically the case here, especially in the
limit, although this declined over the decades shown here. It is
tempting to interpret this result as contrary to the notion of female dominance,
which would intuit greater influence of females on overall fertility than
males. The interpretation of $v$ is unclear, and cannot necessarily in this case
be understood as direct evidence of male-leaning dominance.
\citet{mitra1978derivation} provides no guidance to interpret $v_0$, $v^\ast$,
less so a demographic meaning. It is however clear to this author that
the intial and stable weights for each sex are of greater interest
than the location of $r^\ast$, which is in any case guaranteed to be between
$r^m$ and $r^f$ ,and can therefore be summarized by some generalized mean, in
this case some mean of total exposed male and female populations.

\subsubsection{Critique of Mitra (1978)}
Initial and stable weights are attractive for purposes of the OLS
criterion and their potential for demographic interpretation (which has in any
case not been elaborated), by this author sees the use of such weights as
a superfluous byproduct of the model specification. Namely, $v_0$ and $v^\ast$
are only needed to maintain the SRB, but the SRB is only problematic due to
rote adherence to the single-sex Lotka framework, namely male-male and
female-female reproduction. Of course, males are not exlusively responsible for
the birth of boys and females are not responsible for the birth of girls. If the
model were simply changed to allow for the both-sex fertility of males and
females, one could forego balancing fertility and the SRB. As given, results are
sensitive to changes in the value of the SRB, and so this may allow for
unwelcome instability in the model.

Allowing for the full (both-sex of offspring) fertility schedules of each sex
brings to light another consideration: In this case weighting would not need to
vary between the initial and stable states, thereby making any use of weights a pure
indicator of dominance, as in \citet{goodman1967age}, but leaving the
demographer with no endogenous criterion for choosing weights, save perhaps for
the relative size of male and female exposures \citep{mitra1976effect}.
Furthermore, in either specification, males and females are treated on the same age scale, wherein the reproductive value of e.g. 20-year old males and females are
directly combined to a single sum. This sum is however not the result of age-sex
interactions between males and females.

TODO: revise text below:
Lacking from Mitra's model is allowance for variation in the SRB, age patterns
in SRB (it is a single number), weights that vary by age (the shape of
fertility is held constant), interage competition (all ages in the same sex are
inflated or deflated uniformly). Further the time-trajectory of weights along
the way to stability is not extracted from the model, although these would
possibly be the most interesting outputs from the model. We therefore cannot
judge the total variation in weights required in order to acheive stability.
Also of analytic interest would have been a time series of the initial and final
weights, as these can be interpreted as a kind of \textit{strength of female
dominance} 1) required to acheive lowest-effort stability and 2) inherent in
the state of stability. The author does not discuss this possibility or
calculate a time series in order to illustrate performance over a longer 
period, as does \citet{gupta1973, gupta1978general}. We will do both of these
things here in order to gain a better understanding of the method and possible
improvements upon it.

\citet{mitra1978derivation} also makes use of the unrealistic
notion of single-sex fertility, as have many similar solutions inspired by the
Lotka equation, though this author does not recognize the utility in doing so.
It is of course attractive and of interest to compare two-sex growth rates with the invariant $r^m$ and $r^f$, but
we need not limit ourselves to working with the same elements. The most notable characterisic of
\citep{mitra1978derivation} is the fact that in the OLS solution for starting
weights, the final $r^\ast$ is derived prior to the initial weights: since
minimizing variange in fertility rates between initial and stable states is the
criterion, weights are reduced to a convenient byproduct.

Also 1) single-sex fertility is an odd assumption and 2) results are very
sensitive to the SRB assumption and 3) the SRB in Spain was by no means constant
and 4) male and female ages are treated equally in formulas.





