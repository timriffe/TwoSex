\subsection{Mitra}

\citet{mitra1978derivation}, was directly cocerned with finding a consistent
method to derive a two-sex intrinsic growth rate, $r^\ast$. Consistent here
means that 1) a constant SRB is maintained in and along the trajectory to stability, 2) the
essential \textit{shape} of fertility rates is held constant along the path to
stability and 3) the stable $r^\ast$ is guaranteed to be bracketted by $r^m$ and
$r^f$.

The method proposed by \citet{mitra1978derivation} works by assigning
complimentary (summing to 1) scalar (uniform over age) weights to male and
female single-sex fertility rates and placing the weighted rates , which are
then held constant, into a unified two-sex Lotka equation. For a given set of
weights, one can in this way arrive at a given two-sex $r$ estimate, $r^\ast$. 
However, weights are also constrained to produce a constant 
sex ratio at birth (SRB). Given $r^\ast$ applied to each sex separately in the
state of stability, one notes that this sex ratio is \textit{not} maintained, and must dervive new weights
in order to force the final SRB. These new weights are typically very close to
the original weights, which are also not very different for males and females.

The final $r^\ast$, though unique for a given set of weights, will
depend on the intitial weights chosen, and thus is not in general unique. Mitra
suggests that a good criterion for selecting starting weights would be those
that minimze the departure from constancy for unweighted single-sex fertility
rates. Constant rates are of course the essential aspect of stability- once in
the state of stability, weights no longer change, and rates turn out to be
constant, thus the criterion really deals with minimizing the departure from
initial conditions \textit{along the way} to stability.

Lacking from Mitra's model is allowance for variation in the SRB, age patterns
in SRB (it is a single number), weights that vary by age (the shape of
fertility is held constant), interage competition (all ages in the same sex are
inflated or deflated uniformly). Further the time-trajectory of weights along
the way to stability is not extracted from the model, although these would
possibly be the most interesting outputs from the model. We therefore cannot
judge the total variation in weights required in order to acheive stability.
Also of analytic interest would have been a time series of the initial and final
weights, as these can be interpreted as a kind of \textit{strength of female
dominance} 1) required to acheive lowest-effort stability and 2) inherent in
the state of stability. The author does not discuss this possibility or
calculate a time series in order to illustrate performance over a longer 
period, as does \citet{gupta1973, gupta1978general}. We will do both of these
things here in order to gain a better understanding of the method and possible
improvements upon it.

\citet{mitra1978derivation} also makes use of the unrealistic
notion of single-sex fertility, as have many similar solutions inspired by the
Lotka equation, though this author does not recognize the utility in doing so.
It is of course attractive and of interest to compare two-sex growth rates with the invariant $r^m$ and $r^f$, but
we need not limit ourselves to working with the same elements. The most notable characterisic of
\citep{mitra1978derivation} is the fact that in the OLS solution for starting
weights, the final $r^\ast$ is derived prior to the initial weights: since
minimizing variange in fertility rates between initial and stable states is the
criterion, weights are reduced to a convenient byproduct.

Also 1) single-sex fertility is an odd assumption and 2) results are very
sensitive to the SRB assumption and 3) a constant SRB in Spain couldn't have
been farther from the truth.





