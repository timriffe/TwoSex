\subsection{Iterative Proportion Fitting}

\textit{mc1975models} noted that a simple way to incorporate inter-age
competion (or at least inter-age sensitivity) in marriage count- balancing is
to iteratively rescale a known cross-tabulation of counts (say, from the
previous year) by the separately predicted male and female margins. That is to
say, if males are in matrix rows, one scales each row to sum to the predicted
male margin, then one scales each column to sum to the predicted female margin
(or vice versa, females then males). In rescaling rows to sum to the predicted male margin, 
followed by columns to sum to the predicted female margin, after just a few iterations the process
converges to a particular distribution. Cell counts thus shift between both male
and female ages from the original count matrix to the iteratively predicted
matrix, but stay close to the original distribution. The method satisfies the
thus-far most difficult axiom to incorportate, that of inter-age competition. 

However, the sums of the respective male and female predicted margins will of
course not agree-- After converging to a distribution, the total predicted count
will \textit{flip-flop} between the total male and female predictions, 
which will have differed. Further, \citet{Matthews2013} note that the
final distribution will depend upon whether one starts by scaling the male or
female margin. Both of these problems may be overcome, these authors suggest, 
by starting in parallel within each iteration with the male and female matrix
margins, followed by the other sex, producing two possible two-step scalings.
The starting matrix for the next iteration is taken as the average of these two
outcomes. Since the end result of each iteration is an average, the sum will be
intermediate to the divergent male and female marginal predictions, and the
biverate distribution will be indifferent to whether one started the iterative
process with males or females. The method could of course be further generalized
to take any mean of these two matrices, and not just an arithmetic mean. Results
will vary.

Neither of the above-mentioned studies used their respective iterative
adjustment procedures to predict birth counts, although the \citet{Matthews2013}
method is just a smaller part of a more complex model that includes fertility.
In this dissertation we will treat the \citet{Matthews2013} method only,
modifying it very slightly, so as to be based on a generalized mean (which
allows for a harmonic mean, for instance) as the basic for the male-female
intra-iteration averaging.



