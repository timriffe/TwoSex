 \FloatBarrier
\label{sec:exrenewalit}
\citet{coale1957new} offers a fast-converging iterative approach to estimate the
intrinsic growth rate for age-structured populations. For the $e_y$-structured
renewal equation, a similar approach may be designed, with some slight
modifications. The following steps can be followed to estimate $r$ from
Equation~\ref{eq:exLotkafemales}:

\begin{enumerate}
  \item Derive a first rough estimate of the mean remaining years of life at
  reproduction, $\widehat{T^y}$, akin to Lotka's mean generation time, $T$. If
  one assumes a growth rate of $0$, then a good guess will be: \footnote{$\widehat{T^y}$ appears to range between 50 and 70, judging by the
 two populations studied in this dissertation. \textit{True} $T^y$ is around 10
 years lower, ranging from 40-50.}
\begin{equation}
\widehat{T^y} = \frac{\int _{y=0}^\infty \int _{a=y}^\infty y d_a f_y \dd a
\dd y}{\int _{y=0}^\infty \int _{a=y}^\infty d_a f_y \dd a \dd y}
\end{equation}
  \item A first rough guess at the net reproduction rate, $R_0$ is given by
 \begin{equation}
  R_0 = \int _{y=0}^\infty \int _{a=y}^\infty d_a f_y \dd a
\dd y
\end{equation}
  \item A first rough estimate of $r$, $r^0$, is given by
   \begin{equation}
   r^0 = \frac{ln(R_0)}{\widehat{T^y}}
   \end{equation}
  \item Plug $r^0$ into Equation~\eqref{eq:exLotkafemales} to calculate a
  residual, $\delta^0$.
  \item Use $\delta^0$ and $\widehat{T^y}$ to calibrate the estimate of $r$
  using
  \begin{equation}
  r^{1} = r^0 + \frac{\delta^0}{\widehat{T^y} - \frac{\delta^0}{r^0}}
  \end{equation}
  \item Repeat step (4) to to derive a new $\delta^i$, then step (5) to refine
  $r^i$, until converging on a stable $r$ after some 30 iterations,
  depending on the degree of precision desired ($\widehat{T^y}$ is not updated
  in this process).
\end{enumerate}

The above procedure is both faster and more precise than minimizing the absolute
residual of Equation~\eqref{eq:exLotkafemales} using a generic
optimizer\footnote{Use of a Newton-Raphson optimizer with analytic objective
and gradient functions may prove even more efficient, but I have not tried
this, since the present routine is more than efficient enough for practical
purposes.}.
