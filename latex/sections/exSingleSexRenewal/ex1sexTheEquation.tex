 \FloatBarrier
The present section will formalize the mathematical relations between vital
rates as they pertain to population growth in single-sex $e_y$-structured
populations. The entire system to be presented here finds its analogue in the
familiar stable population model, as typically applied to age-classified
demographic data. Given the renewal process described above, it is perhaps now
intuitive to see that the stable structure of the $e_y$-structured population is determined
primarily by the deaths distribution and the rate of growth of the
population. Indeed, upon transforming fertility rates to the earlier-presented
$e$SFR, one is just a few short steps away from a full Lotka-type renewal
model, namely (for females):

\begin{align}
\label{eq:exLotkafemales}
1 &= \int _{y'=0}^\infty \int _{a'=y'}^\infty e^{-ra'} d_{a'}^F f_{y'}^{F-F} \dd
a' \dd y'
\end{align}
where $a'$ indexes female age, $y'$ indexes female remaining years of life,
$d_{a'}^F$ is the age-distribution of female deaths from the radix-1 period
lifetable, and $f_{y'}^{F-F}$ are exact female-female (mother-daughter)
fertility probablilities by remaining years of life ($e$SFR, see Equation~\eqref{eq:eSFR}). Likewise for males:

\begin{align}
1 &= \int _{y=0}^\infty \int _{a=y}^\infty e^{-ra} d_a^M f_y^{M-M} \dd a \dd y
\end{align}

Equation~\ref{eq:exLotkafemales} is indeed similar to the
original age-structured Lotka equation, introduced in
Equation~\eqref{eq:lotkaeq}. First, note that the survival function $p_a$
inside Equation~\ref{eq:lotkaeq} can also be expressed in terms of
$d_a$ (current livings are the sum of future deaths):

\begin{equation}
p_a = \int _{x = a} ^\infty d_x \dd x
\end{equation}
in which case Equation~\ref{eq:lotkaeq} can be rewritten as:

\begin{align}
\label{eq:lotkadx}
1 &= \int _{a=0}^\infty \int _{b = a}^\infty e^{-ra} d_b m_a \dd b \dd a
\end{align}

All we have changed in order to derive Equation~\ref{eq:exLotkafemales}
is to turn $l_a$ and $m_a$ sideways, so to speak, multiplying the two vectors
together where they coincide in terms of remaining years instead of in terms of age. This
transformation is a simple change of perspective. $r$ still applies to sucessive 
time steps, but in terms of remaining years of life, it must be applied incrementally 
over the newcomers to
each grouping of remaining years of life, i.e. over the time-layers of the
$e_y$-structured pyramid.
