 \FloatBarrier
 \label{sec:ex2sexequation}
The present section will formalize the mathematical relations between vital
rates as they pertain to population growth in single-sex $e_y$-structured
populations. The entire system to be presented here finds its analogue in the
familiar stable population model, as typically applied to age-classified
demographic data. Given the renewal process described above, it is perhaps now
intuitive to see that the stable structure of the $e_y$-structured population is determined
primarily by the deaths distribution and the rate of growth of the
population. Indeed, upon transforming fertility rates to the earlier-presented
$e$SFR, one is just a few short steps away from a full Lotka-type renewal
model, namely (for females), births in year $t$, $B_{t}$ are given by:
\begin{equation}
B_t = \int_{y=0}^\infty F_{y,t} P_{y,t} \dd y
\end{equation}
Where $F_y$ and $P_y$ are $e_y$-specific fertility probabilities and population
counts (rates and exposures when discrete). The population with
exact remaining years $y$, $P_y$, is composed of multiple birth cohorts, thus:
\begin{equation}
B_t = \int_{y=0}^\infty \int_{n=0}^\infty F_{y,t}
\frac{P_{y+n,t}d_{y+n,t}}{\int_{a=y+n}^\infty d_{a,t} \dd a} \dd y
\end{equation}
where $y+n$ can be thought of as age. Present population can be related to past
births in the same way:
\begin{equation}
B_t = \int_{y=0}^\infty \int_{n=0}^\infty F_y
\frac{B_{t-n}d_{y+n}}{\int_{a=y+n}^\infty d_a \dd a} \dd n \dd y
\end{equation}
and when the population is subject to constant vital rates it will eventually
enter into an invariant structure, where the births in each year are related to
the births in the previous year by a constant proportion:
\begin{equation}
B_t = \lambda B_{t-1}
\end{equation}
and in continuous time:
\begin{equation}
B_t = e^{rn}B_{t-n}
\end{equation}
where $r$ is Lotka's intrinsic rate of growth. This identity means that $B_t$
can, in the case of stability, also be written in terms of itself:
\begin{equation}
\label{eq:exrenewal1sex}
B_t = \int_{y=0}^\infty \int_{n=0}^\infty F_y
\frac{B_{t}e^{-rn}d_{y+n}}{\int_{a=y+n}^\infty d_a \dd a} \dd n \dd y
\end{equation}
which give us the so-called renewal equation. THe fundamental equation to
estimate the growth rate, $r$, is given by:
\begin{align}
\label{eq:exLotkafemales}
1 &= \int _{y=0}^\infty \int _{a=y}^\infty e^{-ra} d_{a} F_{y} \dd a \dd y
\end{align}
where $a$ indexes age, $y$ indexes remaining years of life,
$d_{a}$ is the age-distribution of female deaths from the radix-1 period
lifetable, and $F_{y}$ are exact single sex fertility probabilities
(mother-daughter or father-son) by remaining years of
life ($e$SFR, see Equation~\eqref{eq:eSFR}). Equation~\ref{eq:exLotkafemales} is indeed similar to the
original age-structured Lotka equation, introduced in
Equation~\eqref{eq:lotkaeq}. First, note that the survival function $p_a$
inside Equation~\ref{eq:lotkaeq} can also be expressed in terms of
$d_a$ (current livings are the sum of future deaths):

\begin{equation}
p_a = \int _{x = a} ^\infty d_x \dd x
\end{equation}
in which case Equation~\ref{eq:lotkaeq} can be rewritten as:

\begin{align}
\label{eq:lotkadx}
1 &= \int _{a=0}^\infty \int _{b = a}^\infty e^{-ra} d_b m_a \dd b \dd a
\end{align}

All we have changed in order to derive Equation~\ref{eq:exLotkafemales}
is to turn $l_a$ and $m_a$ sideways, so to speak, multiplying the two vectors
together where they coincide in terms of remaining years instead of in terms of age. This
transformation is a simple change of perspective. $r$ still applies to
successive time steps, but in terms of remaining years of life, it must be applied incrementally 
over the newcomers to
each grouping of remaining years of life, i.e. over the time-layers of the
$e_y$-structured pyramid.
