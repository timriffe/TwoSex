  \FloatBarrier
  \label{sec:trendsinrex1sex}
 We have applied the iterative procedure to estimate $r^m$ and $r^f$ for each
 eyar of the US and Spanish data, and the results are displayed in
  Figure~\ref{fig:rex1sex} alongside those for the age-structured single-sex
  $r$. For the US, in nearly all years studied, $e_y$-structured $r$ has been
  greater than the age-structured $r$. The exceptions are the first couple of
  observations, as well as for females in the most recent years, where results
  have been nearly identical. In the case of US males, in no year studied has the $e_y$-structured
  $r^m$ dropped below 0. For the Spanish population,  $e_y$-structured $r$
  has tended to have the same sign as age-structured $r$, but it has also
  tended closer to 0. Broadly, one sees less erratic series for both the US and
  for Spain, although overall pattern of change has been very similar in both
  cases. The convergence seen between male and female $r$ for the age-classified 
  model in the 1990s does not appear as completely in the $e_y$-classified
  model. This is because the opposing forces of mortality, fertility, and the sex ratio at birth 
  combine differently in the $e_y$-structured model, as will be decomposed later.

\begin{figure}[!ht]
  \centering
    \caption{One-sex intrinsic growth rates, $r^m$ and $r^f$, according to
    renewal Equation~\eqref{eq:exLotkafemales}, US and Spain, 1969-2009.}
     % figure produced in /R/ExLotka1Sex.R
     \includegraphics{Figures/exLotka1sex2}
     \label{fig:rex1sex}
\end{figure}




 \FloatBarrier
