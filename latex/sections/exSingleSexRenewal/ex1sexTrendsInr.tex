  \FloatBarrier
 When applied to the data from the US and Spain, we see the trends
displayed in Figure~\ref{fig:rex1sex}. In all years studied, $e_y$-structured
$r$ has been higher than the age-structured $r$, though the overall patterns of
change have been very similar. In the case of US males, in no year studied
has the $e_y$-structured $r^m$ dropped below 0. The convergence seen between
male and female $r$ for the age-classified model in the 1990s does not appear in
the $e_y$-classified model. This is because the opposing forces of mortality,
fertility, and the sex ratio at birth combine differently in the
$e_y$-structured model, as will be decomposed later.

\begin{figure}[!ht]
  \centering
    \caption{One-sex intrinsic growth rates, $r^m$ and $r^f$, according to
    renewal Equation~\eqref{eq:exLotkafemales}, US and Spain, 1969-2009.}
     % figure produced in /R/ExLotka1Sex.R
     \includegraphics{Figures/exLotka1sex}
     \label{fig:rex1sex}
\end{figure}

 \FloatBarrier
