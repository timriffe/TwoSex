 \FloatBarrier
At this point we have demonstrated that the two-sex problem persists in
$e_y$-structured populations, we have given some measures of its
magnitude, and these have been compared with analogous results from
age-structured populations. These measures have included the gap in the
intrinsic growth rate, $r$ between males and females, as well as
divergence in projected birth counts and some
temporal notion of sex separation, as previously
presented for age-structured populations in Section~\ref{sec:Divergence}. We
will now repeat the decomposition exercise that was the topic of
 Section~\ref{sec:Decompr}, but for the male and female intrinsic growth rates 
 derived from the $e_y$-structured model. Specifically, we partition the 
 gap, $r^m-r^f$ into three components: Differences due to fertility, 
 mortality and the sex
ratio at birth. Again, we add a parameter to Equation~\eqref{eq:exLotkafemales}
to account for the sex ratio at birth, $\varsigma _{y}$, making $F_{y}^M$
the both-sex fertility of males and $F_{y'}^F$ the both-sex fertility of females
by remaining years. $\varsigma _{y}$ is then defined as $\frac{SRB_y}{1+SRB_y}$
for males and $\varsigma _{y'}$ as $\frac{1}{1+SRB_{y'}}$ for females, i.e.
allowing the sex ratio at birth to vary by remaining years of life separately 
for males and females.

Figures~\ref{fig:exDecomprUS} and~\ref{fig:exDecomprES} provide a graphical
display of the decomposition for each year of data. Some
aspects of the pattern resemble those of the same exercise for age-structured 
populations (see page~\pageref{fig:DecomprUS}), and others differ.
Specifically, the effect of the sex ratio at birth is more or less the same as
in the age-structured decomposition-- rather uniformly in favor of males.
Mortality effects are also observed to be in favor of females in each year, as
one would expect. However, in the case of $e_y$-structured populations,
mortality usually assumes a much greater role in determining the difference between growth
rates-- one exception is in the mid 1970s for Spain. Fertility is also seen to
be more regularly, but not exclusively in favor of males. The magnitude of
fertility effects were in some years greater in the $e_y$-structured model--
especially years 1980 onward in Spain. Both the age-structured model and the
$e_y$-structured model show rather stable forces contributing to the gap in
sex-specific intrinsic growth rates from around 1990 onward. In most years the
total magnitude of opposing forces was greater than that for

The primary curiosity is that the effects of fertility and mortality appear to
mirror each other rather consistently in the present model. We must determine
whether this is coincidentally so, whether it is an artifact of the method, or
whether this is an observation that might bear lessons. Much, even most, of this
owes to the fact that changes in mortality leave an imprint on $e$SFR, because
the death distribution is used to redistribute ASFR. Further, the stable
population structure is determined exclusively by the deaths distribution and
growth rate. This combination works to \textit{somewhat} align the modal ages of
fertility and population structure\footnote{Recall that in the populations
treated here, the stable population structure (and observed structures, for
that matter) is tapered at the base.}. In this way, the fertility component in
the present decomposition is not fully purged of mortality effects: $e$-SFR has
been taken for granted, namely. Mortality and fertility do not, in the present
case, fully offset each other. Evidently, more work is required to fully
understand the dynamics at play in the present decomposition.

\begin{figure}
        \centering
        \begin{subfigure}
                \centering
                \caption{Components to difference in single-sex intrinsic growth
                rates ($r^m - r^f$) when population is structured by remaining
                years, US, 1969-2009}
                \includegraphics[scale = .8]{Figures/DecomprExUS}
                \label{fig:exDecomprUS}
        \end{subfigure}
        \begin{subfigure}
                \centering
                \caption{Components to difference in single-sex intrinsic growth
                rates ($r^m - r^f$) when population is structured by remaining
                years, Spain, 1975-2009}
                \includegraphics[scale = .8]{Figures/DecomprExES}
                \label{fig:exDecomprES}
        \end{subfigure}
\end{figure}
 \FloatBarrier
As with the single-sex age-structured decomposition presented in
Section~\ref{sec:Decompr}, we may break down the present decomposition even
further, so as to separate the effects of fertility shape from those of
fertility level. This we do using a similar strategy, wherein $F_y$ ($eSFR$) is
broken into two multiplicative pieces, first the overall level, $\tau = eTFR =
\int F_y$ and second $F_y$ rescaled to sum to 1, $\rho_y = \frac{F_y}{\tau}$.
The results of this second decomposition are displayed in
Figures~\ref{fig:exDecomprUS2}~and~\ref{fig:exDecomprES2}.

\begin{figure}
        \centering
        \begin{subfigure}
                \centering
                \caption{Additional decomposition into the components
                to difference in single-sex intrinsic growth rates
                ($r^m - r^f$) for $e_y$-structured population, US, 1969-2009}
                \includegraphics[scale = .8]{Figures/DecomprExUS2}
                \label{fig:exDecomprUS2}
        \end{subfigure}
        \begin{subfigure}
                \centering
                \caption{Additional decomposition into the components to difference in single-sex intrinsic growth
                rates ($r^m - r^f$) for $e_y$-structured population, Spain,
                1975-2009}
                \includegraphics[scale = .8]{Figures/DecomprExES2}
                \label{fig:exDecomprES2}
        \end{subfigure}
\end{figure}

 Here we note that most fertility effects in the sex-gap to $e_y$-structured
 population growth rates are due to the shape of fertility and not the level of
 fertility. Recall that in the age-structured decomposition the weight was
 flipped for the US and roughly equally divided for Spain. We also conclude that
 both the fertility shape effects and the fertility level effects are of
 ambiguous sign, although fertility shape effects have been
 consistently in the favor of $r^m > r^f$ in the US over the period studied. It
 has been seen consistently throughout the results in this dissertation that
 the massive fertility decline in the Spanish population through the first two
 decades of these data echoes through nearly all indicators, no matter how
 transformed, but most importantly that it effected males and females differently. 
 Here we note that the shape-penalty to
 this fertility decline was observed much more among females than among males.
 In recent years, fertility shape effects for Spain have levelled off, and
 females have recuperated in aggregate fertility levels.
 
 One lingering question we may have is why the fertility and mortality effects
 so often nearly mirror each other. Of course, in the $e_y$-perspective, all
 data are derived in the first place from age-specific information, and all mortality
 effects are redistributed in terms of remaining years of life on the basis of
 age-specific mortality data. That is to say, fertility information in the
 $e_y$-perspective depends greatly on mortality information. The decomposition
 has been conducted such that fertility is transformed to the $e_y$-structure
 prior to decomposition, whereas the mortality information, in the
 decomposition, enters only into the Lotka Equation~\eqref{eq:exLotkafemales}.
 In other words, the shape of fertility with respect to remaining years of life is
 taken for granted, whereas the stable population structure is determined in the
 first place by the deaths distribution (derived from $\mu_x$). To this extent,
 the fertility shape effects could once again be broken down into two parts,
 namely, shape effects due to shape of age-specific fertility and shape effects
 due to $\mu_x$. This exercise is left for later work. 
 
The interplay between fertility and mortality in the present model is therefore
complex, and the apparent \textit{mirroring} only seen in
Figures~\ref{fig:exDecomprUS2}~and~\ref{fig:exDecomprES2} would seem to
oversimplify the story. Mortality effects are much more consistent than
fertility effects, but we do not see this when summed over $y$, as was done
above for the sake of parsimony. To illustrate the underlying complexity, not
necessarily apparent in the above, it will for the time being suffice to take a
glimpse at the $e_y$-pattern to the sex-gap in growth rates from some particular
year. In this case, we display 1990, Spain in Figure~\ref{fig:exDecompr1990ES}.
The $eTFR$ effect is left out of the figure, as it is not specific to
remaining years-- this effect was in the favor of males ($0.000658$). There is
of course a time-pattern to that displayed here, a complex evolution. An exploration 
of this pattern must wait for future work as well. Here we merely aim to illustrate 
that the apparent counterweighting of fertility and mortality in the present decomposition is only
apparent-- most of the counterweighting occurs within the shape of fertility
itself over thanatological age! It would also appear that around 50 years from
death, the shape of fertility, SRB and mortality offset each other close to
perfectly. As one would expect, male advantage in fertility is apparent in low
remaining years of life (late life in the age-perspective), and females have a
fertility shape advantage when many years remain until death (early
reproductive ages, on average). 

 \begin{figure}
                \centering
                \caption{Components to difference in single-sex
                $e_y$-structured intrinsic growth rates ($r^m - r^f$) by
                remaining years of life, Spain, 1990}
                \includegraphics{Figures/DecomprExES1990}
                \label{fig:exDecompr1990ES}
\end{figure}
        
 \FloatBarrier























