At this point we have demonstrated that the two-sex problem persists in
$e_x$-structured populations as well, and we have given some measures of its
magnitude. In previous sections we have seen the gap in the intrinsic growth
rate, $r$ between males and females, and calculated some of the same 
measures of magnitude as were used in Section~\ref{sec:Divergence}. We
will now repeat the decomposition exercise that was the topic of 
Section~\ref{sec:Decompr}, but for the male and female intrinsic growth rates
derived from the $e_x$-structured model, namely, partitioning the gap, $r^m-r^f$
into three components: the difference due to fertility, mortality and the sex
ratio at birth. Again, we add a parameter to Equation~\eqref{eq:exLotkafemales}
to account for the sex ratio at birth, $\varsigma _{y}$, making $F_{y}^M$
the both-sex fertility of males and $F_{y'}^F$ the both-sex fertility of females
by remaining years. $\varsigma _{y}$ is then defined as $\frac{SRB_y}{1+SRB_y}$
for males and $\varsigma _{y'}$ as $\frac{1}{1+SRB_{y'}}$ for females, i.e.
allowing the sex ratio at birth to vary by remaining years of life separately for males and females.
\todo{discuss}
\begin{figure}
        \centering
        \begin{subfigure}
                \centering
                \caption{Components to difference in single-sex intrinsic growth
                rates ($r^m - r^f$) when population is structured by remaining
                years, US, 1969-2009}
                \includegraphics[scale = .8]{Figures/DecomprExUS}
                \label{fig:exDecomprUS}
        \end{subfigure}
        \begin{subfigure}
                \centering
                \caption{Components to difference in single-sex intrinsic growth
                rates ($r^m - r^f$) when population is structured by remaining
                years, Spain, 1975-2009}
                \includegraphics[scale = .8]{Figures/DecomprExES}
               
                \label{fig:exDecomprES}
        \end{subfigure}
\end{figure}