At this point we have demonstrated that the two-sex problem persists in
$e_y$-structured populations, we have given some measures of its
magnitude, and these have been compared with analogous results from
age-structured populations. These measures have included the gap in the
intrinsic growth rate, $r$ between males and females, as well as
divergence in projected birth counts and some
temporal notion of sex separation, as previously
presented for age-structured populations in Section~\ref{sec:Divergence}. We
will now repeat the decomposition exercise that was the topic of
 Section~\ref{sec:Decompr}, but for the male and female intrinsic growth rates 
 derived from the $e_y$-structured model. Specifically, we partition the 
 gap, $r^m-r^f$ into three components: Differences due to fertility, 
 mortality and the sex
ratio at birth. Again, we add a parameter to Equation~\eqref{eq:exLotkafemales}
to account for the sex ratio at birth, $\varsigma _{y}$, making $F_{y}^M$
the both-sex fertility of males and $F_{y'}^F$ the both-sex fertility of females
by remaining years. $\varsigma _{y}$ is then defined as $\frac{SRB_y}{1+SRB_y}$
for males and $\varsigma _{y'}$ as $\frac{1}{1+SRB_{y'}}$ for females, i.e.
allowing the sex ratio at birth to vary by remaining years of life separately 
for males and females.

Figures~\ref{fig:exDecomprUS} and~\ref{fig:exDecomprES} provide a graphical
display of the decomposition for each year of data. Some
aspects of the pattern resemble those of the same exercise for age-structured 
populations (see page~\pageref{fig:DecomprUS}), and others differ.
Specifically, the effect of the sex ratio at birth is more or less the same as
in the age-structured decomposition-- rather uniformly in favor of males.
Mortality effects are also observed to be in favor of females in each year, as
one would expect. However, in the case of $e_y$-structured populations,
mortality usually assumes a much greater role in determining the difference between growth
rates-- one exception is in the mid 1970s for Spain. Fertility is also seen to
be more regularly, but not exclusively in favor of males. The maginitude of
fertilty effects were in some years greater in the $e_y$-structured model--
especially years 1980 onward in Spain. Both the age-structured model and the
$e_y$-structured model show rather stable forces contributing to the gap in
sex-specific intrinsic growth rates from around 1990 onward. In most years the
total maginitude of opposing forces was greater than that for

The primary curiosity is that the effects of fertility and mortality appear to
mirror each other rather consistently in the present model. We must determine
whether this is coincidentally so, whether it is an artefact of the method, or
whether this is an observation that might bear lessons. Much, even most, of this
owes to the fact that changes in mortality leave an imprint on $e$SFR, because
the death distribution is used to redistribute ASFR. Further, the stable
population structure is determined exlusively by the deaths distribution and
growth rate. This combination works to \textit{somewhat} align the modal ages of
fertility and population structure\footnote{Recall that in the populations
treated here, the stable population structure (and observed structures, for
that matter) is tapered at the base.}. In this way, the fertility component in
the present decomposition is not fully purged of mortality effects: $e$-SFR has
been taken for granted, namely. Mortality and fertiity do not, in the present
case, fully offset each other. Evidently, more work is required to fully
understand the dynamics at play in the present decomposition.

\begin{figure}
        \centering
        \begin{subfigure}
                \centering
                \caption{Components to difference in single-sex intrinsic growth
                rates ($r^m - r^f$) when population is structured by remaining
                years, US, 1969-2009}
                \includegraphics[scale = .8]{Figures/DecomprExUS}
                \label{fig:exDecomprUS}
        \end{subfigure}
        \begin{subfigure}
                \centering
                \caption{Components to difference in single-sex intrinsic growth
                rates ($r^m - r^f$) when population is structured by remaining
                years, Spain, 1975-2009}
                \includegraphics[scale = .8]{Figures/DecomprExES}
                \label{fig:exDecomprES}
        \end{subfigure}
\end{figure}


























