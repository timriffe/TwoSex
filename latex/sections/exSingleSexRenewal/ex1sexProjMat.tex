\label{sec:ex1sexleslie}
This section explains the construction of the projection matrix that corresponds
to the one-sex $e_x$-structured population model presented above. The objective is
to offer a practical discrete implementation of the prior
formulas, which may aid in understanding main differences with the classic
one-sex Lotka renewal model and be of practical use for projections.

If the reader is not familar with the construction
of age-structured Leslie matrices, a brief description may be found in
Appendix~\ref{Appendix:Caswell}, which is essentially a paraphrase of the
detailed description offered in \cite{caswell2001matrix}. As with
age-structured Leslie matrices, $e_x$-structured projection matrices,
$\textbf{Y}$, are square and of dimension $n \times n$, where $n$ is the number
of remaining years classifications into which the population is divided. The matrix contains
elements for survival and elements for fertility. Unlike Leslie matrices,
$\textbf{Y}$ is not sparse, but is primarily populated with non-zero entries.

Recall the description of renewal in an $e_x$-structured population offered in
Section~\ref{sec:exrenewal} and illustrated in Figure~\ref{fig:exrenewal}. Of
interest is that mortality only occurs in the population class with zero
remaining years of life. Age 1 in year $t$ moves to age o in year $t+1$. In this
way, populations shift \textit{down} rather than up with each time iteration.
Thus, instead of in the subdiagonal, we place survival in the superdiagonal, and
indeed all survival values are 1, since there is no decrement, and the
upper-left corner contains no entry for survival. As in
Appendix~\ref{Appendix:Caswell}, we illustrate using a 6$\times$6 matrix. The
survival component of $\textbf{Y}$ is organized as in
Matrix~\ref{matrix:ex1sexsurvival}.

\begin{matrix}[h!]
\centering
\caption{Survival component of one-sex remaining years
($e_y$)-structured projection matrix, $\textbf{Y}$} 
\label{matrix:ex1sexsurvival}
$\bordermatrix{{e_y } & 0_t & 1_t & 2_t & 3_t & 4_t & 5_t\cr 
                0_{t+1} & 0    &  1   & 0    & 0    & 0    & 0   \cr
                1_{t+1} & 0    &  0   & 1    & 0    & 0    & 0   \cr 
                2_{t+1} & 0    &  0   & 0    & 1    & 0    & 0   \cr 
                3_{t+1} & 0    &  0   & 0    & 0    & 1    & 0   \cr 
                4_{t+1} & 0    &  0   & 0    & 0    & 0    & 1   \cr
                5_{t+1} & 0    &  0   & 0    & 0    & 0    & 0   }$
\end{matrix}

 Fertility inputs to the matrix are derived from $e$SFR and the lifetable $d_x$
 distribution, where $x$ indexes age, but is translated to $y$, remaining years
 of life. Recall that fertility in an $e_y$-structured population occurs in all
 but the highest remaining years classes. Say for our example, that fertility is
 observed in classes 0-4, whereas the final class has no fertility. Where $f_y$
 indicates the fertility probability for class $y$ in the year $t$ entering
 population (in the matrix columns). Each $f_y$ is then distributed according to
 $d_x$, indeed with no further translation, since the $d_x$ column refers to age
 0, as such. Thus the fertility entry in row $m$ and column $n$ of $\textbf{Y}$
 will be $f_n \cdot d_m$. We assume that those dying over the course of year
 $t$ (the first column) are exposed to fertility for \textonehalf of the
 year\footnote{One might be tempted to not allow for fertility at all for
 females dying in year $t$, but recall that fertility is measured in the moment of
 birth, and not conception.},
 and so discount the fertility entry accordingly. Further, infant mortality, 
 $f_y \cdot d_0$, located in the first row, must also be discounted, since part
 ]of the mortality will occur in the same year $t$ and the rest in year $t+1$. 
 The first row of fertility must be further discounted by a factor, $\lambda$ in 
 order to account for the fact that infant mortality is higher in the lower Lexis 
 triangle than in the upper, i.e.
 of those infants that die in the first year of life, a proportion equal to
 $\lambda$ do not make it to December \nth{31} of the calendar year in which
 they were born\footnote{$\lambda$ can be derived directly from death counts
 data classified by Lexis triangles. In the US, lambda has behaved similarly
 for males and females, falling steadily from around $0.9$ in 1969 to $0.86$
 around 1990, since which time it has steadily risen to around $0.87$. That is
 to say, lambda has varied, but not drastically. Likewise for Spain, $\lambda$
 fell from around $0.885$ in 1975 to $0.86$ in the mid 1990s, since which time it
 has risen another \textonehalf\%. In Spain  $\lambda$ has been around \textonehalf\%
 higher for males than females. These numbers are just meant to give a feel
 for the ranges that $\lambda$ can be expected to receive. If the demographer
 does not have information to derive $\lambda$ directly, ad hoc semidirect
 methods may be used to assign a reasonable proportion. } . The
 fertility component of $\textbf{Y}$ is then composed as in Matrix~\ref{matrix:ex1sexfertility}.

\begin{matrix}[h!]
\centering
\caption{Fertility component of one-sex remaining years
($e_y$)-structured projection matrix, $\textbf{Y}$} 
\label{matrix:ex1sexfertility}
$\bordermatrix{
  {e_y } \vspace{.6em}&                0_t  & 1_t  & 2_t  & 3_t  & 4_t  & 5_t\cr 
   0_{t+1} \vspace{.6em}& (1-\lambda) \tfrac{f_0d_0}{2} & (1-\lambda) f_1d_0 & (1-\lambda)
   f_2d_0 & (1-\lambda) f_3d_0 & (1-\lambda) f_4d_0 & 0 \cr 
   1_{t+1} \vspace{.6em}& \tfrac{f_0d_1}{2} & f_1d_1 & f_2d_1 & f_3d_1 & f_4d_1
   & 0   \cr 2_{t+1} \vspace{.6em}& \tfrac{f_0d_2}{2} & f_1d_2 & f_2d_2 & f_3d_2 & f_4d_2
   & 0   \cr 3_{t+1} \vspace{.6em}& \tfrac{f_0d_3}{2} & f_1d_3 & f_2d_3 & f_3d_3 & f_4d_3
   & 0   \cr 4_{t+1} \vspace{.6em}& \tfrac{f_0d_4}{2} & f_1d_4 & f_2d_4 & f_3d_4 & f_4d_4
   & 0   \cr 5_{t+1} \vspace{.6em}& \tfrac{f_0d_5}{2} & f_1d_5 & f_2d_5 & f_3d_5 & f_4d_5
   & 0   }$
\end{matrix}

The survival and fertility components of $\textbf{Y}$ add together elementwise,
thus the full 6$\times$6 matrix is composed as in Matrix~\ref{matrix:ex1sex}.

\begin{matrix}[h!]
\centering
\caption{A full one-sex remaining years ($e_y$)-structured projection
matrix, $\textbf{Y}$} 
\label{matrix:ex1sex}
$\textbf{Y} = \bordermatrix{
  {e_y } \vspace{.6em} & 0_t  & 1_t  & 2_t  & 3_t  & 4_t  & 5_t\cr 
  0_{t+1} \vspace{.6em}&  (1-\lambda) \tfrac{f_0d_0}{2} & (1-\lambda) f_1d_0 + 1 &
  (1-\lambda) f_2d_0 & (1-\lambda) f_3d_0 & (1-\lambda) f_4d_0 & 0 \cr 
    1_{t+1} \vspace{.6em}& \tfrac{f_0d_1}{2} & f_1d_1 & f_2d_1 + 1 & f_3d_1 & f_4d_1 & 0 \cr 
    2_{t+1} \vspace{.6em}& \tfrac{f_0d_2}{2} & f_1d_2 & f_2d_2 & f_3d_2 + 1 & f_4d_2 & 0 \cr 
   3_{t+1} \vspace{.6em}& \tfrac{f_0d_3}{2} & f_1d_3 & f_2d_3 & f_3d_3 & f_4d_3 + 1 & 0 \cr 
   4_{t+1} \vspace{.6em}& \tfrac{f_0d_4}{2} & f_1d_4 & f_2d_4 & f_3d_4 & f_4d_4 & 1 \cr 
   5_{t+1} \vspace{.6em}& \tfrac{f_0d_5}{2} & f_1d_5 & f_2d_5 & f_3d_5 & f_4d_5 & 0 }$
\end{matrix}

Remaining years classes should ideally terminate at the highest value permitted
by data. For the data used in this dissertation, there are 111 total age
classes, which translate to 111 total remaining years classes (0-110+). In practice 
$\textbf{Y}$ becomes
as 111$\times$111 matrix, with most entries non-zero. Construction may appear
tedious for this reason . However, note that the bulk of fertility entries can
be derived as the outer (tensor) product $d_x \otimes f_y$, leaving only the 
first row and first column discounting followed by the addition of the survival
superdiagonal. In most statistical programming languages constructing $\textbf{Y}$ entails only
a couple more lines of code than constructing a Leslie matrix.

As with Leslie matrices, the above projection matrix may be manipulated using
generic matrix techniques in order to extract such information as the intrinsic
growth rate, or the stable $e_x$ structure. The former is the natural log of the
largest real eigenvalue, and the later is the real part of the eigenvector that
corresponds to the largest real eigenvalue, rescaled to sum to 1\footnote{see
\citet[pp 86-87]{caswell2001matrix}}.


