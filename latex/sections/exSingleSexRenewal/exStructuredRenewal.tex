 \FloatBarrier
\label{sec:exrenewal}
We begin by describing renewal in age-structured populations, using the
population pyramid as a mental image. The description might appear to be a
statement of the obvious, but it serves as a guide to the following description of
$e_y$-structured renewal, which is not at first glance intuitive. The
age-structured pyramid shifts upward by one year with each passing year, with 
some decrement occurring in each age of life, such that the essential shape, 
primarily the result of past fertility,\footnote{Thanks are owed to Kirk Scott 
for first imparting this heuristic to me.} takes several decades to be erased 
from memory.\footnote{Thanks are owed to Anna Cabr\'{e} for
first imparting this heuristic to me.} Births from the age structured population 
are produced by a wide range of ages in the population pyramid; these
are assigned to the bottom of the pyramid and are grouped together into a single
cohort. This cohort is heterogeneous with respect to future age (year) at death, 
but it is homogeneous with respect to the year of birth. We are familiar with
the way the age-structured population model unfolds, as it reflects both our
experience of life and the history of demography. The key characteristics are to
note where on the pyramid increment and decrement occur, and the direction of
movement in the pyramid with each passing year.

The $e_y$ structured pyramid, on the other hand, (see Figures~\ref{fig:exPyrUS}
and \ref{fig:exPyrES}) shifts down by one year each year. There are no deaths,
except for in the bottommost layer, those whose $y = 0$. Those with a life
expectancy of 20 move the next year into 19, and so forth, experiencing
increments from newly added births, but no decrement to mortality. 
Each $e_y$ class is heterogeneous with 
respect to year of birth (age) but homogeneous with respect to remaining 
years of life, forming what could be called a \textit{death cohort}. Fertility
can arise from individuals with nearly any remaining life expectancy; the 
age-boundedness of fecundity belongs to the age
perspective of demography. Thus the entire pyramid produces 
offspring.\footnote{The only exception
to this statement is the very top of the $e_y$-pyramid, consisting only of
pre-menarchical girls and pre-semenarchical boys who will have very long
lives.} Total births, $B$, are proportioned to the pyramid using the ``radix-1''
deaths distribution, $d_x$; for example, $P_{e_1}$ is incremented by $d_1 \cdot
B$, and so forth for all ages, adding a new layer whose total over $y$ equals $B$. In 
this way births
increment most heavily around the modal age at death, typically very high in the
pyramid, depending on the year and population. Some are
unfortunate and decrement out of the pyramid in the same year as they are
incremented (births where $y = 0$). See Figure~\ref{fig:exrenewal} for a
schematic visualization of $e_y$-structured population renewal.

\begin{figure}[ht!]
\begin{adjustwidth}{-1in}{1in}
        \centering  
          \caption{Schematic diagram of the renewal process in a population
          structured by remaining years of life.}
           % figure produced in /R/exRenewalDiagram.R
           \includegraphics[scale=.95]{Figures/exRenovationDiagram.pdf}
          \label{fig:exrenewal}
          \end{adjustwidth}
\end{figure}

%The elemental formula on the right side of the Figure~\ref{fig:exrenewal}
%diagram says that remaining years-structure births ($B_{y,t}$) are calculated
%on the basis of remaining years-structured fertility rates ($F_{y,t}$) and
%exposures ($E_{y,t}$), summed to total births ($B_t$) and then distributed over
%the pyramid according to the deaths distribution ($d_y = d_x$).
\FloatBarrier