 \FloatBarrier
\label{sec:exrenewal}
The $e_y$ structured pyramid (see Figures~\ref{fig:exPyrUS},~\ref{fig:exPyrES})
shifts down by one year each year. There are no deaths, except for in 
the bottommost layer, those whose $y = 0$. Those with a life
expectancy of 20 move the next year into 19, and so forth, experiencing
increments from newly added births, but no decrement to mortality. 
Each $e_y$ class is heterogenous with 
respect to year of birth (age) but homogenous with respect to remaining 
years of life. Fertility can arise from individuals with nearly any remaining life
expectancy; the age-boundedness of fecundity belongs to the age
perspective of demography. Thus the entire pyramid produces 
offspring\footnote{The only exception
to this statement is the very top of the $e_y$-pyramid, consisting only of
pre-menarchical girls and pre-semenarchical boys that will have very long
lives}. Total births, $B$, are proportioned to the pyramid using the ``radix-1''
deaths distribution, $d_x$; e.g. $P_{e_1}$ is incremented by $d_1 \cdot B$, and
so forth for all ages, adding a new layer whose total over $y$ equals $B$. In 
this way births
increment most heavily around the modal age at death, typically very high in the
pyramid, between 60 and 80, depending on the year and population. Some are
unfortunate and decrement out of the pyramid in the same year as they are
incremented (births where $y = 0$). See Figure~\ref{fig:exrenewal} for a
schematic visualization of $e_y$-structured population renewal.


\begin{figure}[ht!]
\begin{adjustwidth}{-1in}{1in}
        \centering  
          \caption{Schematic diagram of the renewal process in a population
          structured by remaining years of life.}
           % figure produced in /R/exRenewalDiagram.R
           \includegraphics[scale=.95]{Figures/exRenovationDiagram.pdf}
          \label{fig:exrenewal}
          \end{adjustwidth}
\end{figure}

