 \FloatBarrier
A final calculation of $T^y$ is given by:

\begin{equation}
\label{eq:Ty}
 T^y =  \frac{\int _{y=0}^\infty \int _{a=y}^\infty y e^{-ra} d_a f_y \dd a
\dd y}{\int _{y=0}^\infty \int _{a=y}^\infty e^{-ra} d_a f_y \dd a \dd y}
\end{equation}
using $r$ from the iterative procedure. The net reproduction rate, $R_0$ is
related by, e.g.:
\begin{equation}
\label{eq:R0fromTy}
R_0 = e^{r T^y}
\end{equation}

The birth rate, $b$, is given by:
\begin{equation}
b = \frac{1}{\int _{y=0}^\infty \int _{a=y}^\infty e^{-ra} d_a \dd a
\dd y}
\end{equation}

The stable age structure, $c$, where $c_y$ is the
proportion of the stable population with remaining years to live $y$, is given
by:

\begin{equation}
c_y = b \int _{a=y}^\infty e^{-ra} d_a \dd a
\end{equation}

Other possibly interesting stable parameters may be estimated by
similarly translating the various definitions in the glossary of
\citet{coale1972growth} to the present perspective. Before presenting 
results or extending the present one-sex renewal
formula to two-sex linear and non-linear situations, the heart of this 
thesis, we first describe the construction of the Leslie matrix that corresponds to the
present model.