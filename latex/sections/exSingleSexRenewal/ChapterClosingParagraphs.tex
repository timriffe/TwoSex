\FloatBarrier

This chapter has been rather ambitious in its material, and it has opened
several branches of remaining work, likely producing more questions
than there is material covered. We began by imagining how single-sex population
renewal would work when viewed from the perspective of remaining years of life. 
Indeed much changes -- the orientation of increment and decrement most
especially, and there is more to it than simply inverting the pyramid. It turns
out that the intrinsic growth rates belonging to one and the same population 
differ when calculated from an age-structured or from a thanatologically-structured population -- intrinsic growth rates in the present
system will often, but likely not necessarily, fall closer to zero than their
age-structured single-sex equivalent. It has been demonstrated using our 
example data that observed US and Spanish $e_y$-structured populations
are in the first place closer to their ultimate stable forms, but will also 
obtain stability faster and with less oscillation
than the very same population when structured by age. No proof has been offered
as to whether this observation is necessary for all populations or is
accidental, but we speculate that this will typically be the case. 

Further, no explanation
has been offered as to why it is that intrinsic growth rates differ from classic 
intrinsic growth rates. One could relate these two intrinsic growth
rates formally using Coale's line of thinking mentioned in
Section~\ref{sec:divlotkar}, but this would do little to explain why the
difference should occur in the first place. One may conceive of this
discrepancy as reconcilable in that it owes to the marginal distributions
of a matrix containing one and the same data, as is the case with birth counts
in the age-structured two-sex problem. 

To illustrate, note that with the two-sex
problem, births cross-classified by age of mother and age of father begin 
in a single matrix, from which the marginal
sums of the rows equal the male distribution (the convention in this
dissertation only) and the column margins give the female distribution. Either
of these once-marginal distributions \textit{after} being reapportioned into
remaining-years classes also has this property. Specifically, if instead of
integrating Equation~\eqref{eq:dxredist} over age, one leaves the redistributed
data in a matrix, the (in our case) row margins are equal to the
$e_y$-distribution, and the column margins are equal to the original age
distribution. The primary reorientation behind the present
$e_y$-structured model is in the fertility rates, $F_y$, as the $d_a$ used
herein relates in a direct way to $l_a$ in the age-structured model. 

In this
way, one can easily arrive at a births matrix\footnote{Indeed the fertility-rate
matrix must go back to its two origin matrices -- births and exposures, from
which rates are derived. The total births will sum correctly in the intial
year. The redistributed exposure matrix, as treated here, will not sum to
the exposures used in age-classified rates. Recall the discussion in
Section~\ref{sec:exrates} as to the effective population to use in rates, and
that we have decided to take exposures from the whole population for simplicity
and consistency.} wherein the row margins give $e_y$-structured birth counts and
the column margins give age-structured birth counts. This matrix would be
the link matrix, as per the age cross-classified birth matrix for males and
females. As with the male and female exposures in the two-sex problem, the sums 
of age-structured and
$e_y$-structured exposures will not match, and the problem would shift to the 
determination of a proper denominator, or effective population. That is, such a
link could be made so as to use information from both age perspectives to arrive
at a single estimate of $r$, or other growth parameter. This adventure would
indeed square the degree of complexity of the problem at hand, calling
for a function to use information from $e_y$-rates, age-structured rates, and
each sex -- four combinations to be dealt with. Imagine then the final
cross-classified array in single ages and per the dimensions used in this dissertation: it would
contain $111^4$ (over 150 million) cells for just a single year, and this with
no added variables for nuptial states! This observation is of a speculative nature, and despite
temptation, we will not explore this avenue. Instead we aim to work out some 
common solutions to the two-sex problem in this particular variety of population structure.

Other avenues at our disposal have not been explored -- for example, can our
earlier re-orientation of Fisher's reproductive value (see
Section~\ref{sec:fisher}) also be extracted from the discrete projection
matrix by way of an eigenvector? There are also surely refinements to be made to the discretization of our model
in the corresponding projection matrix outlined in
Section~\ref{sec:ex1sexleslie}, although we still have been able to make good
use of it in measuring the transient dynamics of the present model.

The single-sex model outlined here can be said to be minimal, in that many of
its properties are left unexplored. This author has been content to establish a
working and coherent model, so as to move on to a treatment of the two-sex
problem within it. This is the topic of the following chapter.
