 \FloatBarrier
 \label{sex:doublingex}
The basic projection matrix has been described for the single-sex
$e_x$-structured model. This tool permits us to repeat the illustrative
excercise from Section~\ref{sec:ageSRdoubling} wherein male and female populations are
projected separately and in parallel until such time as one sex outnumbers the
other sex by a factor of two. Long waiting times indicate less divergence, short
waiting times strong divergence. This exercise is close to being just another
viewpoint on the intrinsic growth rate, except that initial conditions are
expected to be far from stable, and may therefore influence results. Again,
human sex ratios of 2 or \textonehalf are simply absurd, and this species of
indicator merely serves to compare.

In Figure~\ref{fig:exSRdoubling}, results from the age-structured model (dashed
lines) are compared with those of the $e_x$-structured model (solid lines).
These results were arrived at using the respective Leslie-matrices. Recall that
higher values indicate less or slower divergence, by this definition. For some 
years in both Spain and the US, the single-sex $e_x$-structured models were less 
divergent, and in other years the single-sex age-structured models were less divergent.
 For the age-structured models, very long waiting times are associated
with crossovers in $r$-- $r^m$ and $r^f$ have undergone no such crossovers for
the $e_x$-structured model in either Spain or the US, as was seen in
Figure~\ref{fig:rex1sex}. The rate of divergence for the $e_x$-structured models
was for this reason, relatively consistent over the range of years studied. 

\begin{figure}[ht!]
        \centering  
          \caption{$log(\mathrm{Years})$ until one sex is twice the size as the
          other, given separate single-sex projections using annual vital rates and initial
          conditions, $e_x$-structured model vs age-structured model, Spain
          and US, 1969-2009}
           % figure produced in /R/exLeslie.R
           \includegraphics{Figures/ExrSRdoubling}
          \label{fig:exSRdoubling}
\end{figure}

The pace of divergence will be determined in the long run by the sex-gap in $r$.
As we saw for the age-structured model, the sex-gap in $r$ owes to various vital
rate components, which were revealed in a decomposition in
Section~\ref{sec:Decompr}. Likewise, the sex-gap in the $e_x$-structured model
is not the whole story, and it will be better understood if we examine the role
of each vital rate in determining its magnitude.

 \FloatBarrier