 \FloatBarrier
Upon viewing a variety of $e_y$-classified population \textit{leaves}, one finds
abundant anecdotal evidence for the existence of a characteristic shape. It has
been claimed in this dissertation that the range of shapes that might be
observed for this variety of population structure is relatively narrow--
relative with respect to age-classified pyramids. The author offers no
mathematical proof that this is so, but it is evident that the deaths
distribution is the primary force behind the $e_y$-structure, and demographers
recognize a characteristic shape to both $d_x$, and the force of mortality from
which it is derived. These characteristics are negative senescent youth, a hump
from the teenage age until ages 30-40, followed by gompertz mortality, which
might eventually taper off to constant, albeit high, mortality in the oldest of
old ages. The $e_y$ structured population will tend to conform then to the
distribution derived from the characteristic shape of the force of mortality,
while the affect of fertility change will be to weight the deaths distribution,
as new generations are added to the population. When fertility is assumed
constant, as in the stable population, the deaths distribution, weighted by 
the growth rate, becomes the only determinant of the shape. 

This being so, we may venture to complement the original claim, that observed
$e_y$-structures tend not to vary far from their characteristic shape, by
comparing observed with stable structures. To do this, we will use the same 
measure of distribution separation seen elsewhere in this dissertation, the
difference coefficient, $\theta$, which is the complement to the proportional
overlap between two distributions. This we show in
Figure~\ref{fig:exstablepyr}\footnote{Trends actually indicate 95\% confidence
regions, which in this case are quite narrow. We Have allowed for
stochasticity in birth counts and death counts, as elsewhere in this
dissertation, but taken the growth rate, population counts, and original
exposures as given.}, where we see that for the US and Spanish populations, the
observed and stable distributions for males and females obtained some 80-95\%
overlap over the period studied. Single-sex male populations tended to be closer
to their stable form.

\begin{figure}
       \centering
       \caption{Distribution dissimilarity of $e_y$-structured populations in
       year $t$ and corresponding year $t$ stable distributions; US 1969-2009
       and Spain 1975-2009}
        \includegraphics{Figures/exPyramidPresentvsStableDivergence}
        \label{fig:exstablepyr}
\end{figure}

It is speculated that the drammatic fertility drop in Spain caused the distance
from the present to the stable structure to increase via abrupt changes in the
growth rate, which will have moved the modal $e_y$-class. We will not decompose
changes in dissimilarity over time into fertility and
mortality components in this dissertation, though this would be an informative
exercise. It results that the degree of separation between
observed and stable age-structured populations follows a similar pattern over
time. For these two populations in the period studied, age-$\theta$ has always
been higher than $e_y$-$\theta$, indicating that the stable population structure
is more different. Figure~\ref{fig:exPyramidthetaratio} displays the ratio of
these two measures of separation. High values in this figure indicate that the
$e_y$-structure was much closer to its stable form than the age-structure.

\begin{figure}
       \centering
       \caption{Ratio of observed versus stable dissimilarity in $e_y$- and
       age-structured populations; US 1969-2009 and Spain 1975-2009}
        \includegraphics{Figures/exPyramidthetaratio}
        \label{fig:exPyramidthetaratio}
\end{figure}

The degree of distributional separation between the present and stable
structure is not the entire story, as the path to stability may be either
rough and slow or smooth and fast. We may measure such things as the speed at
which convergence occurs or the maginitude of the oscillations undergone in
population structure along the path to stability. For an indicator of the speed
of convergence, we may extract the so-called damping ratio from the
respective projection matrices. We may also measure the total departure from
stability along the path

\todo{plot damping ratio}

\begin{figure}
       \centering
       \caption{Cohen's measure of total oscillation along the path to
       stability. Age-clsasified versus $e_y$-classified trajecoties; US
       1969-2009 and Spain 1975-2009}
        \includegraphics{Figures/CohenD2}
        \label{fig:exCohenD2}
\end{figure}

\citet[p. 101]{caswell2001matrix} describes the
use of the damping ratio method of \citet{cohen1979cumulative} for measuring the
cumulative deviation from the stable structure along the path to stability.

that rely on analysis of the
projection matrix, or on actually carrying out projections. For a greater sense of completeness, we will compare two further

\citet[p. 101]{caswell2001matrix} describes the
method of \citet{cohen1979cumulative} for measuring the cumulative deviation
from the stable structure along the path to stability.




It would appear accidentally rather than necessarily the case that
$e_y$-structure was consistently closer to its stable form than age-structure to
its stable form over the period studied, though we may speculate that, absent
drammatic changes in vital rates, this will tend to be the case. The author
considers this observation to be another element of anecdotal evidence in favor
of the claim that $e_y$-structured populations are, in general, more stable than
age-structured populations, at least for the methods shown here.




 \FloatBarrier




