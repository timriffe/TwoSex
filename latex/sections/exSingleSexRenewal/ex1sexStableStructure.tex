 \FloatBarrier
Upon viewing a variety of $e_y$-classified population \textit{leaves}, one finds
abundant anecdotal evidence for the existence of a characteristic shape. It has
been claimed in this dissertation that the range of shapes that might be
observed for this variety of population structure is relatively narrow--
relative with respect to age-classified pyramids. The author offers no
mathematical proof that this is so, but it is evident that the deaths
distribution is the primary force behind the $e_y$-structure, and demographers
recognize a characteristic shape to both $d_x$, and the force of mortality from
which it is derived. These characteristics are negative senescent youth, a hump
from the teenage age until ages 30-40, followed by gompertz mortality, which
might eventually taper off to constant, albeit high, mortality in the oldest of
old ages\citep{horiuchi1998deceleration, vaupel1997trajectories}. The $e_y$
structured population will tend to conform then to the distribution derived from 
the characteristic shape of the force of mortality,
while the affect of fertility change will be to weight the deaths distribution,
as new generations are added to the population. When fertility is assumed
constant, as in the stable population, the deaths distribution, weighted by 
the growth rate, becomes the only determinant of the shape. 

This being so, we may venture to complement the original claim, that observed
$e_y$-structures tend not to vary far from their characteristic shape, by
comparing observed with stable structures. To do this, we will use the same 
measure of distribution separation seen elsewhere in this dissertation (see
Equation~\eqref{eq:coefdiff}), the difference coefficient, $\theta$, which is
the complement to the proportional overlap between two distributions. This we show in
Figure~\ref{fig:exstablepyr}\footnote{Trends actually indicate 95\% confidence
regions, which in this case are quite narrow. We Have allowed for
stochasticity in birth counts and death counts, as elsewhere in this
dissertation, but taken the growth rate, population counts, and original
exposures as given.}, where we see that for the US and Spanish populations, the
observed and stable distributions for males and females obtained some 80-95\%
overlap over the period studied. Single-sex male populations tended to be closer
to their stable form.

\begin{figure}[ht!]
       \centering
       \caption{Distribution dissimilarity of $e_y$-structured populations in
       year $t$ and corresponding year $t$ stable distributions; US 1969-2009
       and Spain 1975-2009}
        \includegraphics{Figures/exPyramidPresentvsStableDivergence}
        \label{fig:exstablepyr}
\end{figure}

The drammatic fertility drop in Spain is likely to have caused the distance
from the present to the stable structure to increase via abrupt changes in the
growth rate, which will have noticeably moved the modal $e_y$-class. We
will not decompose changes in dissimilarity over time into fertility and
mortality components in this dissertation, though this would be an informative
exercise and is left for future work. The degree of separation
between observed and stable age-structured populations follows a similar
year-to-year pattern. For the Spanish and US populations in the period studied,
age-$\theta$ has always been higher than $e_y$-$\theta$, indicating greater
separation between the stable and observed structures.
Figure~\ref{fig:exPyramidthetaratio} displays the ratio of these two measures 
of separation. High values in this figure indicate that the $e_y$-structure was
 much closer to its stable form than the age-structure to its stable form. This
evidence is used in support of the claim that $e_y$-structures are \textit{more
stable} than age-structure. We will now complement this evidence with other
perspectives on stability.

\begin{figure}[ht!]
       \centering
       \caption{Ratio of observed versus stable dissimilarity in $e_y$- and
       age-structured populations; US 1969-2009 and Spain 1975-2009}
        \includegraphics{Figures/exPyramidthetaratio}
        \label{fig:exPyramidthetaratio}
\end{figure}

The degree of distributional separation between the present and stable
structure is not the entire story-- it represents only the starting and
theoretical stable states, but says nothing about the changes in struture that
would unfold in the process of convergence toward stability. The path to
stability may entail abrupt oscillations that last a few generatinos, or it may proceed
quickly and smoothly. We may measure such things as the speed at which
convergence occurs or the maginitude of the oscillations undergone in 
population structure along the path to stability. 

\begin{figure}[ht!]
       \centering
       \caption{Damping ratios. Age-classified versus $e_y$-classified
       trajectories; US 1969-2009 and Spain 1975-2009}
        \includegraphics{Figures/Damping}
        \label{fig:damping}
\end{figure}

Figure~\ref{fig:damping} displays the so-called damping ratio from the
respective projection matrices gives an indicator of the speed of convergence,
supersctipts indicate sex\footnote{These results were derived by eigenvector
analysis of the respective male and female, age-structured and
$e_y$-structured projection matrices using statistical tools from the
\texttt{popbio} package \citep{popbio2007} in the
\texttt{R} programming language \citep{Rcitation}. The \texttt{popbio} package
is primarily based on \textit{caswell2001matrix}.}. The damping ratio is the 
ratio of the largest to the second largest real eigenvalue from the projection 
matrix\citep[p. 101]{caswell2001matrix}. Higher values indicate faster 
convergence, while lower values indicate likely slower convergence. One notes 
that females here tend to undergo faster convergence than
males by this indicator, though this difference has been more consistent and
more marked in the US than for Spain. The US population would also have had a
theoretically faster journey to stability than the Spanish population, save for
the year range 1975-1985. The lengthening of the likely duration to stability in
Spain will have owed to the rapid decline in fertility which quickly changed the
shape of the stable structure, while the observed population structure only
changed slowly over the same time span. This couples with the information from
Figure~\ref{fig:exstablepyr}, where we saw a drammatic increase in
dissimilarity between the observed and stable populations for Spain. Of interest
in the present discussion is that $e_y$-structured populations, with great difference, 
are seen here to converge faster than age-structured populations. With this we 
have another piece of evidence to support the claim that $e_y$-structured populations are more stable than
age-structured populations, namely: $e_y$-structured populations have a shorter
trip to the stable structure.

This information we complement further by measuring the total departure from
stability from the initial to stable states, as proposed by
\citet{cohen1979cumulative}. The method works by projecting a given starting
population (the year $t$ population) forward a large number of years. For each
year $t+n$ of the projection, we measure the distributional difference from
the stable structure ($c_a$, or $c_y$) using the difference coefficient from
Equation~\ref{eq:coefdiff} (having scaled the year $t+n$ population and the
stable structure to each sum to 1), and integrate these differences over time.
Explicitly, and in discrete form, since this exercise is best varried out with
projection matrices, define the $e_y$-structured projection matrix,
$\textbf{Y}$, the year $t$ $e_y$-classified population vector
$\textbf{p}_y$, and the stable population vector, $\textbf{c}_y$

\begin{equation}
\label{eq:totaloscillation}
\mathrm{Total~Oscillation} = \sum _{t=0} ^\infty 1 - \sum _{y=0} ^\omega
min \left( \frac{\boldsymbol{Y} \boldsymbol{p}_{y,t}}{\sum \boldsymbol{Y}
\boldsymbol{p}_{y,t}}, \boldsymbol{c}_y \right)
\end{equation}
where
\begin{equation}
\boldsymbol{p}_{y,t+1} = \boldsymbol{Y} \boldsymbol{p}_{y,t}
\end{equation}

The population vector $\textbf{p}_{y,t}$ changes in each iteration based on the
projection matrix. Eventually the age structure stabilizes, after which time the central sum will
equal 0. This is in essence a measure of the total absolute departure from the
stable structure from the intial population until the stable population,
Cohen's \textit{D2} \citep{caswell2001matrix}. The process works the same way
for age-classified data, changing the subscript to $a$. The results of applying
Equation~\eqref{eq:totaloscillation} to the Spanish and US data are displayed
in Figure~\ref{fig:exCohenD2}. Larger values of this indicator signify larger
oscillations, which take longer to diminish to 0. One could
simplistically understand this as a measure of the difficulty, or friction along
the path to stability.

Results are mostly consistent with previous indicators shown in this section--
$e_y$-structured populations oscillate less in the process of converging. This
is because the oscillations are smaller, which is because the distributional
overlap is greater, producing smaller waves in structure that dissappear faster
and smoother. Curiosuly, females have a larger total oscillation than males,
save for the start and end of the Spanish age-classified series-- this is
curious because, according to the damping ratio, females should approach
stability faster. On the whole, there has been a downward trend in this
indicator for the US population, and the trend in the Spanish population
coincides from the trend in overall departure from the stable form, as seen in
Figure~\ref{fig:exstablepyr}. The peaks for Spain in Figure~\ref{fig:exCohenD2}
also correspond with dips in the Figure~\ref{fig:damping} damping ratio, as
expected.

\begin{figure}[ht!]
       \centering
       \caption{Total oscillation along the path to
       stability. Age-classified versus $e_y$-classified trajectories; US
       1969-2009 and Spain 1975-2009}
        \includegraphics{Figures/CohenD2}
        \label{fig:exCohenD2}
\end{figure}

By now we have presented different pieces of evidence in support of the
statement that $e_y$-structured populations are more stable than
age-structured populations. No formal proof will be offered. There is some
risk that the evidence presented here has been accidental rather
than essential in nature. Namely, the range of years presented here for these
two populations may have coincidentally fallen at a point in time where
conditions were such as to make our claim only appear true. It must be the case
that if a population is 






 \FloatBarrier




