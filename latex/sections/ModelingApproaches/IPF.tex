\FloatBarrier
\label{sec:IPF}
\citet{mc1975models} introduced a well-established method called iterative
proportional fitting (IPF),\footnote{Also called matrix-raking.} often used for
rescaling tables, to the two-sex problem for marriage models. We will apply
the method to fertility only, though the reader may consult
\citet{mc1975models} or \citet{Matthews2013} for details on how to apply
this method in the case of marriage matching. This method works by starting with
a known cross-tabulation of births, in our case from the base year $t$. First compute 
the marginal fertility rates for males and females (ASFR). Apply the male and
female ASFR to exposures predicted for some future year $t+n$ to produce
initially-predicted marginal birth count distributions, the sums of which will 
never agree (as was illustrated in Section~\ref{sec:divbirth}). These new
marginal distributions may be uniformly rescaled according to some mean of their
respective sums. The mean function chosen will affect results only when the
sex-gap in predicted births is very large.\footnote{we have compared overall
results using arithmetic and harmonic means, and found no noteworthy difference. 
All results will be in terms of the harmonic mean for this first rescaling.}

Now that the male and female sums for year $t+n$ agree, we iteratively
rescale the original birth matrix according to the male and female
predicted margins, alternating between male (row) and female (column) margins
until the new birth matrix margins sum exactly to the predicted margins. Of
course, the resulting matrix will differ depending on whether one begins with
rows or with column margins, and so we adopt the recommendation of
\citet{Matthews2013}, taking the element-wise average of the two
possible outcome matrices in each iteration before advancing to the next
iteration. We continue this iterative process until it no longer makes any
difference whether we first scale rows or first scale columns, and accept the
resulting \textit{raked} matrix as our year $t+n$ prediction. Other algorithms
exist for IPF, and IPF is also often used internally in log-linear model
implementations, but we adhere to these easy-to-understand steps.

Using IPF, 1) male and female rates are guaranteed to agree, 2) structural zeros
are not problematic, and 3) the inter-age competition axiom is fulfilled,
which has not been the case with methods previously described. To illustrate
this property, we execute the following experiment. Taking initial birth count
and exposure data from US, 1975, we calculate male and female ASFR. We then
apply the male and female 1975 ASFR to exposures from 1980, and proceed with the
above-described method, finally settling on a predicted birth matrix for 1980,
from which we calculate new ASFR vectors for males and females (1980
predictions that agree). This is the base prediction that we will compare with.
Now we increase 1980 age-25 males (only) by 50\% and repeat the IMP procedure,
producing new ASFR predictions for males and for females.
Figure~\ref{fig:IPFcomp} shows the ratio of the hypothetical (age-25 male
exposure increased by 50\%) 1980 ASFR to the base 1980 ASFR prediction.

Figure~\ref{fig:IPFcomp} illustrates the competition axiom because age-25 male
rates decrease more than neighboring ages, and rate for male ages closest to 25
decreases by more than ages farther from 25 (in fact the ratio is monotonic in
either direction from 25 -- ideal). Female rates increase as well, also as a
rough function of proportional \textit{intermating} with age-25 males in the
1975 birth matrix. Here we see only the difference
in rates -- in terms of birth counts, age-25 males would have a large increase,
while all other ages would experience decreases (lower rates applied to the same exposures).

\begin{figure}[ht!]
        \centering  
          \caption{ASFR after increasing 1980 male exposure by 50\% compared
          with base 1980 ASFR prediction. Based on US 1975 ASFR and birth
          matrix.}
           % figure produced in/R/IPFage.R
           \includegraphics{Figures/IPFagecompetitiontest}
          \label{fig:IPFcomp}
\end{figure}

\paragraph{Iterative proportional fitting in models of population growth: }

The formulas to formalize the use of IPF fertility balancing will take on a
different appearance than those seen thus far. Note that the basic inputs to the
IPF function to constrain male and female fertility rates will be
$IPF(B_{a,a'}(t), F_a^M(t), F_{a'}^F(t),P_a(t+n),P_{a'}(t+n))$, where $B_{a,a'}$
is the cross-classified birth matrix, $F_a^M$ and $F_{a'}^F$ are male and female ASFR,
and $P_a$ and $P_{a'}$ are future population estimates (exposures when
discrete). The function produces adjusted ASFR for \textit{both} males and females,
$F_a^{M^\ast}$, $F_{a'}^{F\ast}$. Let us define a shorthand where the year
$\tau$ is the year used as the standard for the IPF method, bearing in mind that
``$\tau$'' in $IPF(\tau, p^M, p^F)$ indicates the first three arguments from
year $\tau$ (births and male and female ASFR), whereas the latter two arguments,
which we would like to adjust to, may change according to our ends. $F_a^{M^\ast(\tau,p^M, p^F)}$
will be the IPF-adjusted male ASFR based on year $\tau$ data, and
$F_{a'}^{F^\ast(\tau,p^M, p^F)}$ the female ASFR output from the same procedure.
Then assuming constant mortality and continuous functions, we can define year
$t$ total births as a function of past births as:

\begin{align}
B(t) &= \int_{a=0}^\infty \varsigma B(t-a)p_aF_a^{M^\ast(\tau,p^M, p^F)}\dd a
\\ &= \int_{a'=0}^\infty (1-\varsigma) B(t-a')p_{a'}F_{a'}^{F^\ast(\tau,p^M,
p^F)}\dd a'
\end{align}
which works either with males or with females, since the IPF function already
balances fertility such that total births will be the same whether predicted by
males or females. $\varsigma$ is the proportion male at birth. If mortality is
held constant and $IPF(\tau,p^M, p^F)$ is always based on the same year $\tau$ 
constant information, the population will eventually begin
to grow at a constant rate $r$ which can be estimated from the following
equation:
\begin{align}
\label{eq:IPFtricky}
1 &= \int_{a=0}^\infty \varsigma e^{-ra}p_aF_a^{M^\ast(\tau,p^{M\infty},
p^{F\infty})}\dd a \\ 
&= \int_{a'=0}^\infty (1-\varsigma) e^{-ra'}p_{a'}F_{a'}^{F^\ast(\tau,p^{M\infty},
p^{F\infty})}\dd a'
\end{align}
$p^{M\infty}$ for males is just the full age pattern of
$\varsigma e^{-ra}p_a$, and $p^{F\infty}$ is the full age pattern
from $(1-\varsigma) e^{-ra'}p_{a'}$ analogously for females. 

\paragraph{Estimating the intrinsic growth rate: } The estimation of
$r$ using this equation is based on the same principles that have been presented
earlier, but differs in some details. First, note
that either version of Equation~\eqref{eq:IPFtricky} requires full information from both
males and females, so we may as well add the two right-side components and make
the equation sum to two:
\begin{equation}
\label{eq:IPFugly}
2 = \int_{a=0}^\infty \int_{a'=0}^\infty \varsigma
e^{-ra}p_aF_a^{M^\ast(\tau,p^{M\infty}, p^{F\infty})} + (1-\varsigma)
e^{-ra'}p_{a'}F_{a'}^{F^\ast(\tau,p^{M\infty}, p^{F\infty})}\dd a' \dd a
\end{equation}
As in some earlier iterative $r$-estimation instructions given in this
dissertation, one does well to allow $\varsigma$ to be determined by the
exposure-weighted average of sex ratios that vary over age of mother and father.
This information we retain in the four sex-specific fertility functions:
$F_a^{M-M}$, $F_a^{M-F}$, $F_{a'}^{F-F}$, $F_{a'}^{F-M}$, which therefore enter two
separate IPF functions, one for boy births and one for girl births. For
notational convenience, we indicate the sex of birth with a second superscript,
where $F_a^{M-M^\ast(\tau,p^{M\infty}, p^{F\infty})}$ and
$F_{a'}^{F-M^\ast(\tau,p^{M\infty}, p^{F\infty})}$ indicate IPF-adjusted
father-son and mother-son fertility. The second superscript indicates that
the same sex-specificity applies to the three arguments from year $\tau$: the 
cross-classified birth matrix for boy births, and the two sex-sex-specific ASFR
vectors, $F_a^{M-M}$ and $F_{a'}^{F-M}$. $F_a^{M-F^\ast(\tau,p^{M\infty},
p^{F\infty})}$ and $F_{a'}^{F-F^\ast(\tau,p^{M\infty}, p^{F\infty})}$ are the
respective girl-birth adjusted fertility vectors. With these components, we can
now describe the iterative procedure to locate $r$ and simultaneously the stable
SRB.

\begin{enumerate}
  \item Make a rough guess at $R_0$, $\widehat{R_0}$, assuming that $r = 0$
  \begin{equation}
  \widehat{R_0} = \int_{a=0}^\infty \varsigma p_aF_a^{M^\ast(\tau,p^{M},
  p^{F})} \dd a
  \end{equation}
  in other words, where the IPF arguments $p^{M}$ and $p^{F}$ are simply the
  male and female survival functions ($L_a$ discrete). One could also write this
  in terms of female and obtain the same result.
  \item Make a rough guess at the both-sex mean time between generations,
  $\widehat{T}$ by weighting $a$ and $a'$ into Equation~\eqref{eq:IPFugly},
  and dividing the resulting sum by $2\widehat{R_0}$.
  \item With these two quantities, establish a first guess at $r$,
  $r^{(0)}$:
  \begin{equation}
  r^{(0)} = \frac{ln(\widehat{R_0})}{\widehat{T}}
  \end{equation}
  Further, make a first guess at the stable SRB, $S^0$, using
  the observed year $t$ sex ratio at birth, and derive $\varsigma^{(0)}$.
  \item With these starting values, $r^{(0)}$ and $\varsigma^{(0)}$, we begin
  the iterative process by first using IPF to determine the male and female
  both-sex fertility rates (i.e., standard ASFR) that correspond with
  $\varsigma^{(0)} p_a^Me^{r^{(0)}a}$ and $(1-\varsigma^{(0)})p_{a'}^Fe^{r^{(0)}a'}$, for
  instance, $F_a^{M^\ast(\tau,p^{M}(0), p^{F}(0))}$ for males, and plugging
  these two fertility vectors, along with $r^{(0)}$ and $\varsigma^{(0)}$ into
  Equation~\eqref{eq:IPFugly}, which produces a residual, $2\delta^{(i)}$.
  \footnote{i.e., simply divide the residual by two to get the effective
  $\delta^{(i)}$.}
  \item Next, improve the estimate of $r^{i+1}$ using: 
  \begin{equation}
  r^{(i+1)} = r^{(i)} + \frac{\delta^{(i)}}{\widehat{T} -
\frac{\delta ^{(i)}}{r^{(i)} }}
  \end{equation}
  \item Using the updated $r^{(i+1)}$, redo the sex-sex-specific IPF-adjusted
  fertility rates, and then update the running estimate of the stable
  sex-ratio at birth, $S$
  \begin{equation}
  S^{(i+1)} = \frac{\int_{a=0}^\infty \varsigma^{(i)}
  e^{-r^{(i+1)}a}p_aF_a^{M-M^\ast(\tau,p^{M(i+1)}, p^{F(i+1)})}}{\int_{a=0}^\infty
  \varsigma^{(i)} e^{-r^{(i+1)}a}p_aF_a^{M-F^\ast(\tau,p^{M(i+1)}, p^{F(i+1)})}}
  \end{equation}
  from which we update the proportion male for the next iteration:
  \begin{equation}
  \varsigma^{(i+1)} = \frac{S^{(i+1)}}{S^{(i+1)}+1}
  \end{equation} 
  Note that since the IPF adjustment balances the male and female fertility
  rates, we would arrive at the same value using females as the reference.
  \item Use the new $\varsigma^{(i+1)}$ and $r^{(i+1)}$ to restart the process
  in step 4, repeating steps 4-6 until $\delta^{(i)}$ drops to zero, which
  in our experience typically happens in a mere 5-7 iterations.
\end{enumerate}

We have been explicit in these instructions because the implementation of this
method is not obvious, and it requires more moving parts than other
methods. Nonetheless this particular method converges much faster.

\paragraph{Summary of IPF method applied to models of population growth:
}
Models of two-sex population renewal that incorporate IPF-adjustment of
fertility rates\footnote{Other configurations are surely possible, such as the
substantively more complete model described by \citet{Matthews2013}, wherein IPF
is used for marriage-balancing and fertility rates are marital-state specific
as well.} perform well with respect to many of our axiomatic considerations.
Particularly, fulfillment of the competition axiom is not a trivial achievement, 
and it also more-or-less tops off the list of important axioms: 1) the solution
 meets the availability axiom, 2) is first-degree homogeneous, 3) is monotonous, 4) is symmetrical with
respect the sexes, and 5) is sensitive to substitution and competition.
There is no guarantee for bracketing, although the solution will always track and typically
be bracketed by the single-sex intrinsic growth rates. 

IPF fertility (marriage) balancing, properly attributed to \citet{mc1975models}
in this context, is regularly mentioned in reviews of possible two-sex
solutions,\footnote{See e.g., the review inside \citet{ianelli2005gender}.} but
is not typically evaluated alongside more analytical
methods\footnote{The two most widely cited
reviews of methods are \citet{keyfitz1972mathematics} and
\citet{pollard1973mathematical}, both of which precede the introduction of IPF
to marriage or fertility models. \citet{ianelli2005gender} mentions but does not
evaluate the method, possibly because it is not commensurable with the
differential equation framework employed by these authors. Individual authors proposing
two-sex solutions have rarely evaluated the method, possibly because the implementation
is cumbersome.} precisely due to its iterative nature. This ought not be a
drawback to us, given that our method for optimizing $r$ is also iterative,
rather than analytic. Properties may be judged empirically (as we have done),
but have to this author's knowledge never been proven with mathematical rigor,
and this may never be accomplished. As such, the method may be categorized as ad
hoc, but apparently no more than other methods presented
here.\footnote{\citet{mc1975models} provides an unconvincing sociological justification for IPF in marriage
models.}

\paragraph{The method applied to the US and Spanish data:}

We have used the IPF method described earlier to calculate the two-sex intrinsic
growth rate for each year of the US and Spanish data. Detailed 
results for $r$ and the stable sex ratio at birth can be found in the tables of
Appendix~\ref{appendix:ageallrestimates} alongside those of other
age-structured renewal models. Within the IPF process, we
have mentioned that the demographer has a choice of mean functions for the
initial balancing of the male and female marginal sums. Both arithmetic and
harmonic means were tested, and produced no visually discernible differences.
This is good, if we would like to minimize the effects of the demographer's
subjectivity in obtaining results. In Figure~\ref{fig:IPFager}, we display
results from using the harmonic mean internally in the IPF fitting. Note that
$r$ is not bracketed in some years for the US, but that these are years where 
the sex-gap itself was trivial, and so may be due to
rounding. That the result is not bracketed need not be of any concern, as we
need not guarantee it. On the whole, the method falls squarely between $r^m$ and
$r^f$, as do most other methods presented thus far. In keeping with common
practice in two-sex modeling, the judgement of the method will rest not on this
result, but rather on the method's earlier-discussed properties.

\begin{figure}[ht!]
        \centering  
          \caption{IPF intrinsic growth rates, $r$, compared with
          single-sex $r^m$ and $r^f$. US, 1969-2009 and Spain, 1975-2009.}
           % figure produced in/R/IPFage.R
           \includegraphics{Figures/IPFager.pdf}
          \label{fig:IPFager}
\end{figure}




\FloatBarrier

