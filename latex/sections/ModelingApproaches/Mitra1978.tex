\label{sec:mitraage}
\citet{mitra1978derivation}, like Pollard, also limited two-sex models to being
based on the building blocks of single-sex fertility. In this case, single-sex
fertility is conceived of as in the single-sex models, using male-male and female-female
fertility rates. Mitra aimed to produce a consistent
method to derive a two-sex intrinsic growth rate, $r$. Consistent here means 
that 1) a constant SRB is maintained in and along the trajectory to stability, 2) 
the essential \textit{shape} of fertility rates is held constant along the path 
to stability and 3) the stable $r$ is guaranteed to be bracketted by $r^m$ and $r^f$.

\paragraph{Components to the model: }
The model requires a fixed sex ratio at birth, $S$, although this need only
enter into formulas explicitly if one implements Mitra's formulas iteratively,
which we will not present here. The method proposed by \citet{mitra1978derivation} works
by assigning complementary scalar (uniform over age) weights, $\frac{1}{v_0}$ 
and $\frac{1}{1 - v_0}$, to male and female single-sex fertility rates, $F_a
^{M-M}$ and $F_{a'} ^{F-F}$. As elsewhere, the model requires the male
and female age-specific survival functions, $p_a^m$ and
$p_{a'}^f$, respectively. The initially weighted fertility rates are held
constant and placed into a unified two-sex Lotka unity equation in order to
determine $r$:

\begin{equation}
\label{eq:mitralotka}
1 = \int _{a=0} ^\infty e^{-r a} \frac{F_a ^{M-M}}{v_0} p_a^m \dd a + \int
_{a'=0} ^\infty e^{-r a'} \frac{F_{a'} ^{F-F}}{1 - v_0} p_{a'}^f \dd a'
\end{equation}
Upon determining the combined-sex growth rate, $r$, however, one must readjust
the weights, $v_\infty$, to correct the stable sex ratio at birth:
\begin{equation}
\label{eq:mitrastablev}
v_\infty = \int _0^\infty e^{-ra} \frac{F_a ^{M-M}}{v_0} p_a^m \dd a
\end{equation}

\paragraph{Estimating Mitra's $r$: } One can quickly converge upon a solution
to Equation~\eqref{eq:mitralotka} by modifying the method proposed in
\citet{coale1957new}\footnote{\citet{mitra1978derivation} alludes to this, but
does not get into any specifics.}:
\begin{enumerate}
  \item Calculate a trial estimate of $r$,
$\widehat{r}$ and a trial two-sex mean generation length
$\widehat{T}$. For trial values, one can use simple
assumptions, such as the arithmetic means of the single-sex Lotka parameters.
  \item Plug the trial $\widehat{r}^{(1)}$ into Equation~\eqref{eq:mitralotka}
  to calculate the residual, $\delta ^{(1)}$.
  \item Improve the estimate of $r^{i+1}$ using:
  \begin{equation}
  \widehat{r}^{(i+1)} = \widehat{r}^{(i)} + \frac{\delta^{(i)}}{\widehat{T} -
\frac{\delta ^{(i)}}{\widehat{r}^{(i)} }}
  \end{equation}
  \item Use the new improved estimate, $r^{(i+1)}$, to calculate a new residual,
  and repeat steps 2 and 3 until $\delta^{(i)}$ vanishes to zero.
\end{enumerate}

This method converges quickly and with greater precision than most
generic optimizers. Once $r$ is found, one takes the extra step of calculating
the stable weights, $v_\infty$ using Equation~\eqref{eq:mitrastablev}.

\paragraph{Summary of the method: } A characteristic of Mitra's model design is
that a given starting weight, $v_0$, will always result in a single, stable
$v_\infty$. Mitra's two-sex growth rate,
$r$, is unique for but depends upon the starting weights, $v_0$, and thus is not 
in general unique to a given set of vital rates, which is a drawback. Mitra
suggests that a good choice for $v_0$ would be the value that minimizes the 
departure from constancy for weighted single-sex fertility rates. This is an 
attractive choice because constant rates are of course the
basis of stability. Once a population attains stability, weights, and therefore 
rates, are constant. In practice, one chooses the $v_0$ that minimizes the sum
of the age-specific squared residuals (for males and females) between $F_a$ and 
$F_a \times \tfrac{v_0}{v^\ast}$.

\paragraph{Mitra's weights in the initial versus stable states: } If minimizing
the difference between starting and stable rates is the criterion for choosing $v_0$, 
then there is indeed a single stable $r$ that
corresponds to a given set of vital rates. We calculate Mitra's starting and
stable weights for the US and Spanish data and display them
in Figure~\ref{fig:Mitra1978v0vstar}.

\begin{figure}[ht!]
        \centering  
          \caption{Initial ($v_0$) and stable ($v$) weights according to the OLS
          criterion. US, 1969-2009, and Spain, 1975-2009 
          \citet{mitra1978derivation}}
           \quad
           % /R/Mitra1978.R
           \includegraphics{Figures/Mitra1978v0vstar}
          \label{fig:Mitra1978v0vstar}
\end{figure}

For Spain and the US throughout the period studied in this dissertation, both
$v_0$ and $v^\ast$ fell in the range $(.48,.6)$. $v_0$ was always
close to $.5$, entailing nearly equal weight for male and female rates.
The stable $v^\ast$ was consistently higher than $v_0$ and always higher than
$.5$, implying greater weights for males than females in stability. When $v > .5$, male rates weight
more than female rates, which was typically the case here, especially in the
limit, although this declined over the decades shown here. It is
tempting to interpret this result as contrary to the notion of female dominance,
which would intuit greater influence of females on overall fertility than
males. However, the interpretation of $v$ is unclear, and cannot necessarily in
this case be understood as direct evidence of male-leaning dominance.
\citet{mitra1978derivation} provides no guidance to interpret $v_0$, $v_\infty$,
less so a demographic meaning. 
\FloatBarrier
\paragraph{Critique of Mitra, 1978: } Initial and stable weights are attractive
for purposes of the OLS criterion and their potential for demographic interpretation, which has in any
case not been elaborated. This author considers this particular variety of
weights to be a superfluous byproduct of the model specification. Namely, $v_0$
and $v_\infty$ are needed only to maintain the SRB, and the SRB is
only problematic due to use of the single-sex fertility rates. Of course, males
are not exclusively responsible for the birth of boys and females are not 
responsible for the birth of girls -- the same critique applied to Pollard's
fertility rates applies here too. If the model were simply changed to allow for
the both-sex fertility of males and females, one could forego the
intricacies of balancing fertility and the SRB. As given, results are
sensitive to changes in the value of the SRB, and so this admits
unwelcome instability into the model. Further, since SRB varies by age,
changes in the age-sex-structure of the population ought to result in changes in
SRB, but Mitra's model forces a constant SRB. This decision reverberates
throughout the model specification.

If the model were to include the full fertility schedules of each sex (i.e.
births of both sexes to parents of each sex), then weights would not need to
vary between the initial and stable states. In this case, weights would only serve as a pure indicator of
dominance, as in \citet{goodman1967age}. The drawback, in this case would be
that the demographer is left with no endogenous criterion for choosing weights, save
perhaps for the relative size of male and female exposures \citep{mitra1976effect}. 
Furthermore, in either specification, males and females are treated on the same 
age scale, wherein the reproductive values of for example, 20-year old males and
females are directly combined to a single sum -- i.e., the model lacks age-sex
interactions and fertility schedules are rigid. 

We compare the results of this method with those from Pollard in
the following Section.

\FloatBarrier