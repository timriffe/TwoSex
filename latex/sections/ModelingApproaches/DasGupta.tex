\FloatBarrier
\label{sec:dasgupta}
\citet{gupta1978alternative} states\footnote{and this fits nicely into the flow
of our own presentation.} ``The lesson we learn from the above is that our
starting point must not be the formulation of two equations, one for $B_M(t)$ and another for
$B_F(t)$, but of a single equation for $B(t)$ with the help of a bisexual
fertility function that can explain the occurrence of births of type $(a,a')$ in
terms of the availability of both males and females''.

Das Gupta introduced a series of proposals for two-sex reproduction models
throughout the decade of the 1970s \citep{gupta1972two, gupta1973us,
gupta1976interactive, gupta1978alternative}, of which we will present the last
one. To summarize how the model works, imagine we would like to determine a
unified two-sex fertility rate, $F_{a,a'}$. Here it is clear
what to put in the numerator, as births can be tabulated by the ages of both parents.
 We thus work to define the idea of two-sex exposure for each age-combination. Das Gupta's
suggestion is to derive a series of probability density functions that apply to
each age of potential mother and each age of potential father from information
contained in the matrix of observed births. Define these age-specific pdfs for
males, $U_{a,a'}$, and for females, $V_{a,a'}$ as:

\begin{align}
U_{a,a'} &= \frac{B_{a,a'}}{\int B_{a,a'} \dd a'}\\
V_{a,a'} &= \frac{B_{a,a'}}{\int B_{a,a'} \dd a}
\end{align}
In discrete terms, one establishes two matrices, arranged according to our
standard in this dissertation with male age in rows and female age over columns.
The row marginal sums for $U_{a,a'}$ all equal 1 and the column marginal sums of
$V_{a,a'}$ all equal 1\footnote{both with the exception of ages with no
fertility, which are left as 0 if undefined.}. One then calculates Das Gupta's
approximation of bisexual exposure, $E_{a,a'}$, by redistributing male and
female age-specific exposure and summing for each combination of age:
\begin{equation}
E_{a,a'} = U_{a,a'}E_a + V_{a,a'}E_{a'}
\end{equation}
which is then used as the denominator to calculate $F_{a,a'}$:
\begin{equation}
F_{a,a'} = \frac{B_{a,a'}}{E_{a,a'}}
\end{equation}
which is assumed constant in the stable model. As elsewhere, define the
male and female radix-1 survival functions, $p_a$, and $p_{a'}$, and a sex ratio
at birth, $S$, from which we determine the proportion male at
birth, $\varsigma=\frac{S}{1+S}$. Then Das Gupta's two-sex renewal
function becomes:
\begin{equation}
B(t) = \int_{a=0}^\infty \int_{a'=0}^\infty \Big( \varsigma U_{a,a'} B(t-a) p_a
+ (1-\varsigma)V_{a,a'}B(t-a) p_{a'}\Big)F_{a,a'} \dd a \dd a'
\end{equation}
If $U_{a,a'}$, $V_{a,a'}$, $\varsigma$ and $F_{a,a'}$ are assumed constant, then
as $t$ approaches infinity, the intrinsic rate of growth, $r$, will stabilize.
$r$ is estimated from the Lotka-type unity equation:
\begin{equation}
\label{eq:Guptaeq}
1 = \int_{a=0}^\infty \int_{a'=0}^\infty \Big( \varsigma U_{a,a'} e^{-ra} p_a
+ (1-\varsigma)V_{a,a'}e^{-ra'} p_{a'}\Big)F_{a,a'} \dd a \dd a'
\end{equation}
\paragraph{Estimating Das Gupta's $r$: } The value of $r$ that solves
Equation~\eqref{eq:Guptaeq} can either be found using an iterative 
process similar to that proposed by \citet{coale1957new}, which we present
because it converges very fast:

\begin{enumerate}
  \item establish a starting value for $r$,
$r^{(0)}$ and a trial two-sex mean generation length
$\widehat{T}$. For both values, one may use simple
assumptions, such as the arithmetic means of the single sex Lotka parameters.
  \item Plug the trial $r^{(0)}$ into Equation~\eqref{eq:Guptaeq}
  to calculate a residual, $\delta ^{(1)}$.
  \item Improve the estimate of $r^{i+1}$ using:
  \begin{equation}
  r^{(i+1)} = r^{(i)} + \frac{\delta^{(i)}}{\widehat{T} -
\frac{\delta ^{(i)}}{r^{(i)} }}
  \end{equation}
  \item Use the new improved estimate, $r^{(i+1)}$ to calculate a new residual,
  and repeat steps 2 and 3 until $\delta^{(i)}$ vanishes to zero.
\end{enumerate}

\paragraph{Summary of the method: } \citet{gupta1978alternative} assumes that exposure to risk of
 age $a$ males is not evenly distributed over each age of potential female mate-
 i.e. that it is not random\footnote{As opposed to an earlier rendition of
 this method \citep{gupta1972two}}. Rather, the exposure to risk is partitioned
 over ages of potential mates according to the distribution present in a given 
 cross-classified birth matrix. In partitioning exposure in this way for each
 age of male and female, the cross-classified male and female risks are additive, and
 form the total exposure to risk. 
 
 It is attractive that this total exposure to
 risk sums to the total male and female exposures, but it is unclear whether the
 distribution should be based on cross-classified birth tabulations, which will
 likely be laden with structural artifacts. For example, as relatively large
 cohorts pass through reproductive ages, they will tend to produce more births
 than neighboring cohorts-- even if the large cohorts also suffer lower rates.
 This will cause a spike along a particular age margin in the birth matrix,
 usually for both males and females of the larger cohort. This birth spike will
 be present in the exposure redistribution matrices, $U_{a,a'}$ and
 $V_{a,a'}$, and it will also remain evident in fertility rates, $F_{a,a'}$.
 This is problematic even in the first iteration of a projection, as the
 hypothetical large cohort will have moved up one age. This artifact will
become a characteristic of the stable population even as abrupt cohort size
differences vanish with time. The initial structural artifacts in the supposed
constant parameters thus enter into both exposures and rates. 

The present model also removes some of the anomalies that
result from single-sex fertility assumptions-- $F_{a,a'}$ is the fertility of
both sexes, and $\varsigma$ enters into Equation~\eqref{eq:Guptaeq} as a radix
weight for the male and female population structures. There is no dominance
parameter in this model, per se. \citet{gupta1978alternative} does not mention
whether the method will always produce an $r$ that is bracketed by the
single-sex $r$ values, although in a previous paper
\citep{gupta1976interactive} he appeared to give this property axiom
status\footnote{yellin1977comparison had since proved bracketing to be an
extraneous constraint.}.
 
 To the extent that exposure within the model is a function of both males and
 females, this model may be said to be interactive. One may notice that since
 exposure is additive that the model will behave poorly in the absence of one
 potentially reproductive age-sex combination in the future (births for this
 age would not drop to 0 as they should). This possibility would not likely
 arise in practice, but it is still the most basic and necessary of
 commonly stated axioms. Further, the method is not fully age-interactive. An
 increase in males (females) of one age will affect the fertility of all ages of
 females (males), but males have no effect on males and females have no effect
 on females. In this way, the model lacks competition.

\paragraph{The method applied to the US and Spanish data: } We estimate Das
Gupta's intrinsic growth rate for each year of the US and Spanish data. In each
year of data for both populations, $r$ is bracketed by $r^m$ and $r^f$. The
relative poisition between $r^m$ and $r^f$ does not appear to follow any
particular mean function-- it is consistently greater than any of the mean
functions that satisfy the availability axiom, such as the geometic, harmonic or
logorithmic means (to be explored in following).

\begin{figure}[ht!]
        \centering  
          \caption{$r$ from Das Gupta (1978) and single sex intrinsic growth rates. US, 1969-2009, and Spain, 1975-2009}
           % /R/DasGupta.R
           \includegraphics{Figures/Gupta1978r}
          \label{fig:Gupta1978r}
\end{figure}

For purposes of prediction and ease of implementation, Das Gupta's model is
close to acceptable, though in following we will explore some models that are
somewhat more palatable and more widely studied, starting with models whose
two-sex fertility rates are derived from mean functins of the male
and female rates \citet{schoen1981harmonic}.

\FloatBarrier
