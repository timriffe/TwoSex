\FloatBarrier
As mentioned, models that adhere to the notion of single-sex fertility are
characterized by difficulty in keeping the sex ratio at birth under control.
\citet{pollard1948measurement} partially resolves this issue, keeping
sex-divergence at bay by swapping the generation of birth of each sex to
progenitors of the other sex. The method has the drawback of reliance upon
starting population structure \citep{yntema1952mathematical}, and so cannot be
said to be 100\% ergodic. This later criticism applies only to the derivation of
sex-structure in the stable state, as no sex ratio assumptions are required to arrive at Pollard's two-sex
$r$. Similar conclusions may be made for the model of
\citet{mitra1978derivation}- the demographer has not been liberated from making
decisions, as the initial weights must be decided upon, and the OLS criterion
used earlier is just one such choice. Further, this choice is forced upon the
demographer due to the use of single-sex fertility information and use of the
sex ratio at birth as a governing parameter. To a certain extent, this is to say
that both models' major pitfalls originate in the use of single-sex fertility,
carried over more-or-less directly from the single-sex model framework.

Figure~\ref{fig:PollardMitrar} displays the results of applying these two
methods to the US and Spanish data to arrive at estimates of the two-sex 
intrinsic growth rate for each year. We see that
Pollard's method yields a somewhat higher estimate than the Mitra (OLS
criterion) method, but that differences are minor. Both methods yield two-sex
estimates of $r$ that are bracketed by $r^m$ and $r^f$, and this property was
one of the primary motives in the design of both models. Neither of these models
is seen to allow for interactions between the sexes, or between ages.

\begin{figure}[ht!]
        \centering  
          \caption{$r$ from Pollard (1948), Mitra (1978; OLS criterion),
          and single-sex intrinsic growth rates. US, 1969-2009, and Spain, 1975-2009}
           % /R/Mitra1978.R
           \includegraphics{Figures/PollardMitrar}
          \label{fig:PollardMitrar}
\end{figure}

In the following, we investigate models that allow for fertility
rates to be a function of the ages of both parents.

\FloatBarrier


