\FloatBarrier
\label{sec:ageharmonic}
\begin{singlespace}
\begin{quote}
Now of everything that is continuous and divisible, it is possible to take the larger 
part, or the smaller part, or an equal part, and these parts may be larger, smaller, 
and equal either with respect to the thing itself or relatively to us; the equal part
 being a mean between excess and deficiency. By the mean of the thing I denote a point 
 equally distant from either extreme, which is one and the same for everybody; by the 
 mean relative to us, that amount which is neither too much nor too little, and this 
 is not one and the same for everybody -- \textit{Aristotle}\footnote{\citetalias{rackham1947trans}}
\end{quote}
\end{singlespace}

The most instinctual two-sex fertility (marriage) solution is to symmetrically
(with respect to the sexes) utilize information from the vital rates of both
sexes. Mean functions have been compared in the past\citep[see
e.g.,][]{keyfitz1972mathematics}, but rated in terms of utility with difficulty.
In terms of the axioms mentioned in Section~\ref{sec:axioms} -- rather than
performance -- the harmonic mean function has fared the best amongst a variety
of means. \citet{schoen1978standardized, schoen1977two, schoen1981harmonic}
provided a rationale and derivation for using the harmonic mean in order to 
balance marriage rates. \citet{martcheva2001mathematics} found evidence of
poor performance for the harmonic mean in projective scenarios. The same
strategy can be used to balance fertility rates, which is what we will do here. 
The method requires as inputs a matrix of birth counts cross-tabulated by 
age of father, $a$, and age of mother, $a'$,
and male and female exposures classified by age. The harmonic mean
\begin{equation}
\label{eq:harmonic}
H(P_a^m, P_{a'}^f) = \frac{2 P_a^m P_{a'}^f}{P_a^m + P_{a'}^f}
\end{equation}
is applied to male and female exposures in order to find an intermediate
denominator from which to calculate rates, $F_{a,a'}^H$,
 \begin{equation}
 \label{eq:harmonicrate}
 F_{a,a'}^H = \frac{B_{a,a'}}{H(P_a^m, P_{a'}^f)}
 \end{equation}
which in the stable population is assumed constant in time rather than
assuming constant male and female rates separately. In order to estimate 
a birth count in some future year $t+n$, calculate the harmonic mean
of male and female exposures and multiply into the constant harmonic rate
 \begin{equation}
 B(t+n) = \int \int F_{a,a'}^H H\Big(P_{a}^m(t+n), P_{a'}^f(t+n)\Big) \dd a \dd
 a'
 \end{equation}
which we can rewrite to make year $t$ births a function of past births in the
renewal equation
 \begin{equation}
 B(t) = \int \int F_{a,a'}^H H\Big(\varsigma B(t-a)p_a^m, (1-\varsigma) B(t-a)
 p_{a'}^f\Big) \dd a
 \dd a'
 \end{equation}
where $p_a^m$ and $p_{a'}^f$ are the male and female probabilities of surviving
from birth until age $a$, $a'$, and $\varsigma$ is the proportion male of
births, here assumed constant over age and time, though this may be relaxed.
Rewriting in this way brings us to a stable population framework. \citet{schoen1977two} 
proposed his own rectangular stable population framework, which 
will not be treated here. As $t$ becomes large, the annual growth factor
approaches a constant value equal to $e^r$, which can be estimated from the
following Lotka-type unity function: 

\begin{equation}
\label{eq:lotkaH}
1 = \int _{a=0}^\infty \int _{a'=0}^\infty F_{a,a'} H\Big(\varsigma
e^{-ra}p_a^m, (1-\varsigma)e^{-ra'}p_{a'}^f\Big)\dd a' \dd a
\end{equation}
$F_{a,a'}^H$ is the constant fertility rate to be applied to the harmonic
mean of male and female exposures, $p_a^m$ and $p_{a'}^f$ are the male
and female radix-1 survival functions. $\varsigma$ serves to make the
male and female radices sum to 1, and also accounts for the fact that males and
females have slightly different $l_0$ values. 

\paragraph{Estimating $r$: } The two-sex harmonic intrinsic growth rate, $r$ can
be estimated in two ways, either assuming $\varsigma$ to be constant from the
start (likely based on the initial data) and using a generic optimizer, or by modifying the iterative procedure
suggested by \citet{coale1957new}, which works best if one simultaneously
estimates $r$ and $\varsigma$ (i.e., allowing $\varsigma$ to adjust to the
population structure, as it is known to vary with age). Here we describe
the practical steps involved in the latter.

\begin{enumerate}
  \item Calculate the constant harmonic fertility rates for male and female
  births separately, $F_{a,a'}^{mH}$ and $F_{a,a'}^{fH}$.
  \item Make a first estimate of the stable sex ratio at birth, $\hat{S}$; the
  initial observed sex ratio at birth is a good choice. From $S^0$ we derive a
  first estimate of the proportion male of births, $\varsigma^0$ (where
  superscripts indicate the iteration):
  \begin{equation}
  \varsigma^0 = \frac{S^0}{S^0+1}
  \end{equation}
  \item Find a first rough estimate of the net reproduction rate,
  $\widehat{R_0}$, assuming a growth rate of 0 and using the both-sex
  harmonic fertility rate $F_{a,a'}^{H} = F_{a,a'}^{mH} + F_{a,a'}^{fH}$:
  \begin{equation}
  \label{R0guessschoen}
  \widehat{R_0} = \int_{a=o}^\infty \int_{a'=0}^\infty H(\varsigma^0 p_a^m,
  (1-\varsigma^0)p_{a'}^f) F_{a,a'}^{H} \dd a' \dd a
  \end{equation}
  \item Assume a reasonable both-sex mean generation time, $\widehat{T}$.
  Weighting $a$ and $a'$ into Equation~\eqref{R0guessschoen} and then dividing
  by $\widehat{R_0}$ yields a good estimate of this. Otherwise one may simply
  choose a reasonable age, such as 30, or some mean of the male and female
  single-sex mean ages at reproduction.
  \item Calculate an initial value of $r$, $r^0$:
  \begin{equation}
  r^0 = \frac{log(\widehat{R_0})}{\widehat{T}}
  \end{equation}
  \item Now that we have a starting value, $r^0$, calculate a residual,
  $\delta^0$, from equation~\eqref{eq:lotkaH}:
  \begin{equation}
  \delta^i = 1 - \int _{a=0}^\infty \int _{a'=0}^\infty H(\varsigma^i p_a^m
  e^{-r^ia}, (1-\varsigma^i)p_{a'}^fe^{-r^ia'}) F_{a,a'}^H \dd a' \dd a
  \end{equation}
  \item Use $\delta^i$ to improve the estimate of $r$, $r^{i+1}$:
  \begin{equation}
  r^{i+1} = r^i - \frac{\delta^i}{\widehat{T} - \frac{\delta^i}{r^i}}
  \end{equation}
  \item Use the improved estimate of $r$ to update $\varsigma$:
  \begin{align}
  S^{i+1} &= \frac{\int_{a=o}^\infty \int_{a'=0}^\infty H(\varsigma^i
  e^{-r^{i+1}a} p_a^m, (1-\varsigma^i)^i e^{-r^{i+1}a'}p_{a'}^f) F_{a,a'}^{mH} \dd a' \dd a
  }{\int_{a=o}^\infty \int_{a'=0}^\infty H(\varsigma^i e^{-r^{i+1}a}
  p_a^m, (1-\varsigma^i)^i e^{-r^{i+1}a'}p_{a'}^f) F_{a,a'}^{fH} \dd a' \dd a }
  \\
  \varsigma^{i+1} &= \frac{S^{i+1}}{S^{i+1}+1}
  \end{align}
  \item Plug the new $\varsigma$ and $r$ estimates into step 5, to estimate a
  new residual, $\delta$, repeating steps 6-8 until $\delta$ vanishes to 0.
  Typicaly around 20 iterations are needed in order to reduce $\delta$ to
  zero (at least as close as double floating-point precision can get).
\end{enumerate}

This iterative procedure simultaneously produces an estimate of the stable
sex ratio at birth $S$ and the both-sex intrinsic growth rate, $r$. Really,
there is little room for $S$ to move between the initial and stable states,
since boy and girl births are in essence produced by (the harmonic mean of) both
males and females in this procedure. $S$
will vary from the initial sex ratio at birth only to the extent that there is
both an age pattern to the sex ratio at birth and the male and female stable age
structures differ from the initial age structures. Estimating both parameters at
the same time does not present a practical problem in the present case, and the
procedure converges faster than if $S$ is left assumed at some constant value.

One could abandon the iterative $r$ estimation procedure outlined above
and perform a standard cohort component projection, for instance using a
two-sex Leslie matrix. In this case, the fertility component of the Leslie
matrix would need to be updated between each iteration using
equation~\ref{eq:harmonicrate} for either males or females. One cannot easily
perform standard matrix analysis of this Leslie matrix, however, as it is not static in the standard way.

\paragraph{Other stable quantities: } Once one has identified the stable $r$ and
$S$, one can move on to estimate other stable parameters of interest, such as
the both-sex stable birth rate, $b$:

\begin{equation}
b = \frac{1}{\int_{a = 0}^\infty e^{-ra} \varsigma p_a^m \dd a + \int_{a' =
0}^\infty e^{-ra'} \varsigma p_{a'}^f \dd a'}
\end{equation}
This can be used to calculate the male and female stable age structures, $c_a$
and $c_{a'}$:

\begin{equation}
c_a =  \varsigma  e^{-ra} p_a^m
\end{equation}
and analagously for females, where
\begin{equation}
1 = \int c_a + \int c_{a'}
\end{equation}
and the total population sex ratio, $S^{tot}$ is the ratio of these:
\begin{equation}
S^{tot} = \frac{\int c_a}{\int c_{a'}}
\end{equation}

\paragraph{Summary of the harmonic mean method: } The stable system outlined
here is not taken word-for-word from Schoen's advice, but it is consistent with the 
notion of a constant \textit{force of attraction},
$F_{a,a'}^H$, and non-linear balancing of fertility rates based on the harmonic
mean of male and female exposures. The method presented here is only partially
sensitive across all ages to changes in the exposure of a single age in one sex.
That is to say, an increase in males of age $a$ will increase observed fertility rates for all ages
of females that share rates with males of age $a$. Further, females with
higher rates, $F_{a,a'}^H$, will typically observe greater increases, though this
depends on the distribution within $F^H$ and on relative exposure levels.
Lacking from this implementation are decreases in rates for males whose ages are close
to $a$, so-called spillover effects \citep{choo2006estimating}. That is to say,
an increase in age $a$ males will not affect rates of males age $a-n$ or $a+n$, 
despite the fact that the pool of potential mates, females over
all ages $a'$, is shared. One would expect, ceteris paribus, that males of
similar ages would experience a decrease in rates, since some proportion of the
female pool will have been redirected to the increased stock of age $a$ males.
Hence, the model lacks this sense of competition. All other axioms appear to be
satisfied, except for that of bracketing, which we also deem superfluous.
Further, the harmonic mean is biased toward the minority sex, which is also intuitive.
 As stated before, one cannot empirically establish (for
humans) the ideal functional form of the fertility (marriage) function.

One satisfying property of the present method is that the harmonic mean
rates do not respond rigidly to mismatched population sizes between males and
females, but rather the mean rate is sensitive to relative size of male and
female stocks. In this way, the function is more dynamic than a weighted mean
or Das Gupta's method presented in the previous section. Indeed, if the
demographer is not satisfied with the elasticity of the harmonic mean, one may
change the function, $H()$, for any mean function, such as a generalized mean.
An infinite number of other means will also have the same desirable properties as the harmonic mean, such as 
dropping to 0 in the absence of one sex. Most means with this property that have
names (harmonic, geometric, logorithmic,\ldots) will produce almost
indistinguishably similar results. All such mean solutions will be symmetric
(blind) with respect to the sexes, although one could easily include weights.

\paragraph{The method applied to the US and Spanish data: }
In addittion to the harmonic mean, we have produced estimates of $r$ using the
geometric and logorithmic means, as well as the minimum function. Detailed 
results for $r$ and the stable sex ratio at birth under these four mean
functions can be found in the tables of Appendix~\ref{appendix:ageallrestimates}
alongside those of other age-structured renewal models. Figure~\ref{fig:schoenr} shows the results of
only the harmonic mean and minimum functions, as the geometric and logorithmic $r$ estimates would not be visually
distinguishable from those of the harmonic mean. From this lesson, we confirm
that if one is to use a mean function as a two-sex fertility (marriage)
function, it really makes little difference which mean function one chooses, as long as it
satisfies the availability condition. The minimum function yields the least
consistent results, sometimes greater than the harmonic mean, sometimes less
than the harmonic mean, sometimes bracketed by the single-sex $r$ values, and
sometimes not. We note that the minimum function deviates the greatest from the
single-sex $r$ values when the sex-gap is trivial, and in these instances it is
always higher. The harmonic mean series is here always bracketed by the
single-sex $r$ values, although this is not a necessary result.

\begin{figure}[ht!]
        \centering  
          \caption{$r$ according to harmonic mean and minimum fertility
          functions compared with single-sex intrinsic growth rates. US,
          1969-2009, and Spain, 1975-2009}
           % /R/Schoen1981.R
           \includegraphics{Figures/HMager}
          \label{fig:schoenr}
\end{figure}

In terms of complexity of implementation, solutions based on mean functions are
marginally less demanding than the Das Gupta solution, but this is primarily
because mean functions are more readily understood. The mean solution is seen as
conceptually simpler, yet yielding similar results and with more desirable
properties than either of the preceding solutions. In the following, we will
present two iterative fertility functions that allow for competition between 
ages within the same sex.

\FloatBarrier
