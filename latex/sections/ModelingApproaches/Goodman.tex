\FloatBarrier
\label{sec:googmanage}
\citet{goodman1967age}, in an age-extension to \citet{goodman1953population},
provided a series of discrete formulas for calculating the
stable age-sex-structure of a given series of vital rates similar to those
treated earlier. In particular, let us define the radix-1 survival functions,
$p_a$ for males and $p_{a'}$ for females, as well as four fertility
functions specific to sex of progenitor and sex of birth:
$F_a^{M-M}$, $F_a^{M-F}$, $F_{a'}^{F-F}$, and  $F_{a'}^{F-M}$, where the first
superscript indicates sex of progenitor and the second superscript indicates sex
of birth. If $P_a(t)$ are males of age $a$ in year $t$, and $P_{a'}(t)$ are
females, then everything aligns properly in a tautological way:
\begin{align}
B(t) &= \int_{a=0}^\infty P_a(t) (F_a^{M-M}(t)+F_a^{M-F}(t)) \dd a
     &= \int_{a'=0}^\infty P_{a'}(t) (F_{a'}^{F-F}(t)+F_{a'}^{F-M}(t)) \dd a'
\end{align}
and so forth for each sex of birth separately. However, $n$ years hence this
will no longer be the case. The female dominant model declares that in general
for any given year
\begin{equation}
\label{eq:fdominanceSimple}
B(t) = \int_{a'=0}^\infty P_{a'}(t) (F_{a'}^{F-F}+F_{a'}^{F-M}) \dd a'
\end{equation}
and it is assumed that male rates will simply adjust in accordance with this
such that the model is internally consistent. If assumed  to be constant, female
rates in this case could refer to any year and there will be no room for
inconsistency, and so we drop the $t$ index.
Equation~\eqref{eq:fdominanceSimple} is just the same as this:
\begin{equation}
\label{eq:fdominancesigma}
B(t) = 1 \cdot \int_{a'=0}^\infty P_{a'}(t) (F_{a'}^{F-F}+F_{a'}^{F-M}) \dd a' +
       0 \cdot \int_{a=0}^\infty P_a(t) (F_a^{M-M}+F_a^{M-F}) \dd a
\end{equation}
where the 1 before females gives them 100\% of the weight in determining births,
and the 0 before the male integral gives 0\% of the weight to males. The way
Goodman describes it, females in Equation~\ref{eq:fdominancesigma} determine
births 100\% of the time and males 0\% of the time. One could just as easily
swap the 0 and the 1 to have a male-dominant model, or in general assign two
weights that sum to 1 for a mixed-dominance model. If we define the male weight
as $\sigma$ and the female weight as $1-\sigma$, then we have the general
weighted dominance model:
\begin{equation}
\label{eq:dominancesigma}
B(t) = (1-\sigma) \cdot \int_{a'=0}^\infty P_{a'}(t) (F_{a'}^{F-F}+F_{a'}^{F-M})
\dd a' + \sigma \cdot \int_{a=0}^\infty P_a(t) (F_a^{M-M}+F_a^{M-F}) \dd a
\end{equation}
and everything is accounted for. As per usual, we may go on to define $P_a$ as
male births from $t-a$ years ago, $B^M(t-a)$ , discounted by the probability of
surviving to age $a$, $P_a = B^M(t-a)p_a$, and likewise for females. If the
proportion male at birth is captured in $\varsigma$, then we can rewrite the
latter as $P_a = \varsigma B(t-a)p_a$, and analogously for females. Plugging
these into Equation~\eqref{eq:dominancesigma}, we obtain year $t$ births in
terms of past births
\begin{equation}
\label{eq:dominancebirths}
\begin{split}
B(t) = (1-\sigma) \cdot \int_{a'=0}^\infty (1-\varsigma) B(t-a')p_a'
(F_{a'}^{F-F}+F_{a'}^{F-M}) \dd a' \\
+ \sigma \cdot \int_{a=0}^\infty \varsigma B(t-a)p_a (F_a^{M-M}+F_a^{M-F}) \dd
a
\end{split}
\end{equation}
which when left to evolve according to fixed rate schedules for many years will
eventually stabilize to
\begin{equation}
\label{eq:dominancebirthsstab}
\begin{split}
B(t) = (1-\sigma) \int_{a'=0}^\infty (1-\varsigma) B(t)p_ae^{-ra'}
(F_{a'}^{F-F}+F_{a'}^{F-M}) \dd a' \\
+ \sigma \int_{a=0}^\infty \varsigma B(t)p_{a'}e^{-ra'}
(F_a^{M-M}+F_a^{M-F}) \dd a
\end{split}
\end{equation}
where $r$ is a constant growth rate equal for males and females, and year $t$
births can then be endogenously related. Dividing by $B(t)$ we arrive at the
Lotka-type unity equation
\begin{equation}
\label{eq:goodmanunity}
\begin{split}
1 = (1-\sigma) \int_{a'=0}^\infty (1-\varsigma) e^{-ra'}p_a
(F_{a'}^{F-F}+F_{a'}^{F-M}) \dd a' \\
+ \sigma \int_{a=0}^\infty \varsigma e^{-ra'}p_{a'} (F_a^{M-M}+F_a^{M-F})
\dd a
\end{split}
\end{equation}
from which we need only estimate $r$ in order to derive the full suite of stable
population parameters, such as two-sex mean generation length and stable
population structure. In the following, we describe the steps to estimate $r$
iteratively.

\paragraph{Estimating $r$: } 
Assuming some fixed proportion male at birth, one can simply use a generic
optimizer on Equation~\eqref{eq:goodmanunity} to estimate the stable growth
rate, $r$. However, since males and females each have an age-pattern to
the sex-ratio at birth, changes in population structure between the initial and
stable states will entail a different total SRB, as it is just a weighted
average of the sex-age-specific sex ratios at birth. For this reason, we 
calibrate the stable SRB, $S$, simultaneously with $r$. In
practice, this presents no problems, as the SRB is rather limited in its
movement between the stable and initial states, and it only subtly tweaks $r$
compared to simply assuming some $S$. The steps to estimate $r$ and $S$ are
similar to those outlined elsewhere in this dissertation, and are based on a
modification of \citet{coale1957new}, which converges very quickly and is easy
to implement. For a given $\sigma$ between 0 and 1, follow the these steps
to estimate $r$.

\begin{enumerate}
  \item Establish a rough estimate of the net reproduction rate, $R_0$, assuming
  that $r = 0$ and assuming some value of SRB (such as the year $t$ observed
  SRB) and derive $\hat{\varsigma} = \frac{SRB(t)}{1 + SRB(t)}$
  \begin{equation}
  \label{eq:R0itgoodman}
  \begin{split}
\widehat{R_0} = (1-\sigma) \int_{a'=0}^\infty (1-\hat{\varsigma})
p_{a'}(F_{a'}^{F-F}+F_{a'}^{F-M}) \dd a' \\ + \sigma  \int_{a=0}^\infty
\hat{\varsigma} p_a(F_a^{M-M}+F_a^{M-F}) \dd a
\end{split}
  \end{equation}
  This value only differs slightly from the stable $R_0$ in that the stable SRB
  has not yet been determined.
  \item Establish a guess at the mean generation length, $\widehat{T}$, by
  weighting $a$ and $a'$, respectively, into Equation~\eqref{eq:R0itgoodman} and
  then dividing by $\widehat{R_0}$. With $\widehat{T}$ and $\widehat{R_0}$,
  derive the starting value of $r$, $r^{0}$, as
  \begin{equation}
  r^{0} = \frac{log(\widehat{R_0})}{\widehat{T}}
  \end{equation}
  \item Plug $r^{i}$ and $\varsigma^i$ into
  Equation~\eqref{eq:goodmanunity}, producing a residual, $\delta^{i}$.
  \item Use $\delta^{i}$ to update the estimate of $r$ using
   \begin{equation}
  r^{i+1} = r^i - \frac{\delta^i}{\widehat{T} - \frac{\delta^i}{r^i}}
  \end{equation}
  \item Now update the estimate of $S$ using
   \begin{equation}
        S^{i+1} = \frac{(1-\sigma) \int_{a'=0}^\infty
         (1-\varsigma^i)p_{a'}e^{-r^{i+1}a'} F_{a'}^{F-M} \dd a' + 
         \sigma  \int_{a=0}^\infty \varsigma^i p_ae^{-r^{i+1}a} F_a^{M-M} \dd
         a}{ (1-\sigma) \int_{a'=0}^\infty
         (1-\varsigma^i)p_{a'}e^{-r^{i+1}a'} F_{a'}^{F-F} \dd a' + 
         \sigma  \int_{a=0}^\infty \varsigma^i p_ae^{-r^{i+1}a} F_a^{M-F} \dd a}
  \end{equation}
  from which we derive $\varsigma^{i+1} = \frac{S^{i+1}}{1+S^{i+1}}$
  \item Repeat steps 3-5 until $\delta$ is reduced to 0, which takes around 20
  iterations (fewer for most practical purposes).
\end{enumerate}

\paragraph{The method applied to the US and Spanish data: } We apply the
above-described method to the US and Spanish data for each year to produce
estimates of $r$ according to $\sigma = 0$, $\sigma = 1$, and $\sigma = 0.5$,
corresponding to female dominance, male dominance, and mixed dominance. Detailed 
results for $r$ and the stable sex ratio at birth can be found in the tables of
Appendix~\ref{appendix:ageallrestimates} alongside those of other
age-structured renewal models. The
results, displayed in Figure~\ref{fig:Goodmanr}, show the mixed-dominance case
to be intermediate to the single-sex dominant series. 

\begin{figure}[ht!]
        \centering  
          \caption{$r$ according to dominance-weights, $\sigma = 0, 0.5, 1$.
          US, 1969-2009, and Spain, 1975-2009}
           % /R/Goodman.R
           \includegraphics{Figures/Goodmanager}
          \label{fig:Goodmanr}
\end{figure}

\paragraph{Summary of the method: }
Female dominance in this case is identical to the female single-sex model, and
analogously for males, and so we see that Goodman's model is bracketed. One concludes that the present
model is indeed expedient -- more so than \citet{mitra1978derivation} and
similar in complexity to \citet{pollard1948measurement}. The model has a desirable
design feature that neither of the preceding models has in that 
births of both sexes from each parent are accounted for, in a sense liberating
the model from limited single-sex rate dependence, but with the added cost of 
including a parameter to weight the male and female radices according 
to a sex ratio at birth. The sex ratio at birth, which was the complicating factor in
\citet{mitra1978derivation} is here made endogenous and inherent with ease
simply because rates of each sex of birth are considered. Rather than an
overdetermining obstacle, the SRB is an aid in optimizing (in our experience).
One shortcoming, since the male and female components to the model are
additive, is that the availability axiom is not met. Homogeneity and
monotonicity are indeed met, but all axioms of an interactive nature are left
unattended to. For populations within the range of age-structures often
observed, we would not expect anomalous results in projective scenarios. The
expediency of this model lends itself to encapsulation in a two-sex Leslie
matrix, which remains fixed throughout a projection. This model is
implemented in the remaining-years perspective to be explored in
Chapter~\ref{sec:ex2sexdomweights} of this dissertation, and for that case we
present the corresponding projection matrix in addition to other stable quantities.

\FloatBarrier
