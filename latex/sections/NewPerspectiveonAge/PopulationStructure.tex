The key
observation is that we can sidestep the $e_x$ column of the lifetable, since
$e_x$ is essentially a weighted mean of $d_x$. The resulting population pyramid
is heterogeneous with respect to age within any given level of remaining life
expectancy, and looks like
Figures~\ref{fig:exPyrUS} and \ref{fig:exPyrES}\footnote{The idea to
redistribute the population pyramid in this way is due to a conversation with 
John MacInnes, and appears in \citep{MacInnes2013pop} (unpublished) using a
different method.} for the years 1975 and 2009 in the US and Spain. As a helpful
pointer, note that the population at the base of the pyramid is expected to decrement
within the \textit{next year}, thus the vertical axis can also can also be
thought of as year $t+y$, although $e_x$ more clearly identifies the pyramid
with year $t$ mortality conditions. The pyramid should not be
taken out of context as a forecast. Note that this pyramid represents the exact same
population as an age-classified pyramid: Underlying males sum to the correct total on 
the left and females sum to the correct total on the right. Only the definition of age has
changed; instead of counting forward from birth we count \textit{age} in reverse
starting from death. For individuals, this feat would be impossible, but given
 the information contained in a period lifetable,  one can to great utility 
 redistribute population aggregates accoridng to $e_x$\footnote{To undertake
 the same but assumping future mortality changes (improvements) one might
 better undertake a fertility-free cohort component projection and collect the
 deaths from each future year $t+y$ until extinction. This possibility is not
 treated in the present dissertation.}. Both pyramids have been rescaled
 to sum to 100, in order to more comparably represent population structure\footnote{Incidentally, 
 a time series of remaining life expectancy pyramids for any given Western country will show 
 incredible stability over time, which is remarkable in light of observed ageing in the observed
  population pyramid. The simple interpretation of this kind of pyramid adds to its utility, and
   this author 
believes that $e_x$-specific population structure, and other here-unmentioned indicators 
that can be derived from this method sould make up a valuable new component to the contemporary
 demographer's toolbox, as well as help inform current population debates.}.

\begin{figure}
        \centering
        \begin{subfigure}
                \centering
                \caption{US years-lived by $e_x$, 1975 and 2009}
                \includegraphics[scale = .8]{Figures/exPyramidUS}
                \label{fig:exPyrUS}
        \end{subfigure}
        \begin{subfigure}
                \centering
                \caption{Spain years-lived by $e_x$, 1975 and 2009}
                \includegraphics[scale = .8]{Figures/exPyramidES}
               
                \label{fig:exPyrES}
        \end{subfigure}
\end{figure}