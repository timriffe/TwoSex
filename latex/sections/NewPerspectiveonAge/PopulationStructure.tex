 \FloatBarrier

The key
observation is that we can sidestep the $e_x$ column of the lifetable, since
$e_x$ is essentially the $d_x$-weighted mean of age. The resulting
population pyramid is heterogeneous with respect to age within any given level of remaining life
expectancy, and looks like
Figures~\ref{fig:exPyrUS} and \ref{fig:exPyrES}\footnote{The idea to
redistribute the population pyramid in this way is due to a conversation with 
John MacInnes, and appears in \citep{MacInnes2013pop} (unpublished) using a
different method.} for the years 1975 and 2009 in the US and Spain. As a helpful
pointer, note that the population at the base of the pyramid is expected to decrement
within the \textit{next year}, thus the vertical axis can also can also be
thought of as year $t+y$, although $e_y$ more clearly identifies the pyramid
with year $t$ mortality conditions. The pyramid should not be
taken out of context as a forecast. Note that this pyramid represents the exact same
population as an age-classified pyramid: Underlying males sum to the correct total on 
the left and females sum to the correct total on the right. Only the definition of age has
changed; instead of counting forward from birth we count \textit{age} in reverse
starting from death. For individuals, this feat would be impossible, but given
 the information contained in a period lifetable, one can to great utility 
 redistribute population aggregates according to $e_y$\footnote{To undertake
 the same but assuming future mortality changes (improvements) one might
 better undertake a fertility-free cohort component projection and collect the
 deaths from each future year $t+y$ until extinction. This possibility is not
 treated in the present dissertation.}. Both pyramids have been rescaled
 to sum to 100, in order to more comparably represent population structure.

\begin{figure}
        \centering
        \begin{subfigure}
                \centering
                \caption{US population by remaining years, 1975 and 2009}
                \includegraphics[scale = .8]{Figures/exPyramidUS}
                \label{fig:exPyrUS}
        \end{subfigure}
        \begin{subfigure}
                \centering
                \caption{Spain population by remaining years, 1975 and 2009}
                \includegraphics[scale = .8]{Figures/exPyramidES}
               
                \label{fig:exPyrES}
        \end{subfigure}
\end{figure}

A time series of remaining life expectancy pyramids for any given Western 
country (excluding war years and epidemics, and especially after the rapid fall
in infant mortality) will show incredible stability over time, which is
remarkable in light of ageing in the observed population pyramid. The simple 
interpretation of this kind of pyramid adds to its utility, and this author 
believes that $e_y$-specific population structure, and
indicators that can be derived from this method (not treated here) should 
make up a valuable new component to the contemporary demographer's toolbox, as well 
as help inform current population debates. 

It will suffice for the time being to
illustrate that for Spain and the US in the years trated in this dissertation,
the remaining-years-structured population pyramid has been many times more
stable. This we will demonstrate by again making use of the difference
coefficient, $\theta$ from Equation~\ref{eq:coefdiff}, where $f_1$ is the
population structure for year $t$ and $f_2$ is the population structure from
year $t+1$ (males and females, together, scaled to sum to 1). We iteratively
produce $\theta$, comparing year $t$ and $t+1$ for age-structured pyramids in
the first place and for $e_y$-structured pyramids on the other. Pyramids are in
general very stable, so the difference $\theta$ in both cases will nearly always
fall below 0.02. However, $e_y$-$\theta$ is consistently and considerably lower
than the age-$\theta$. It will suffice to take the ratio of the two indicators,
 $e_y$-$\theta$ divided by age-$\theta$ over the period of study for both
 countries, as seen in Figure~\ref{fig:PyramidStability}.

\begin{figure}
      \centering
      % Figure made in PyramidStability.R
      \caption{Relative stability of $e_y$-structured populations to
                age-structured populations, year $t$ vs $t+1$, ratio of
                $\theta$, Spain and US, 1969-2009} 
         \includegraphics{Figures/PyramidStabilityThetaRatio}
      \label{fig:PyramidStability} 
\end{figure}

In Figure~\ref{fig:PyramidStability}, a value of 1 would indicate that the two
ways of structuring population are equally stable between years $t$ and $t+1$;
values less than 1 indicate that the $e_y$-structured population is more stable.
For instance, .5 means that the $e_y$-structured population was twice as stable,
.2 means 5 times more stable, and so forth. In all years in this dissertation,
$e_y$-structuring acted to stabilize the population somewhat. As a heuristic,
runs of years with continuous and modest improvements in mortality will produce
the most stable $e_y$-structured pyramids. This measure of stability compounds
as well: that is to say, and $e_y$-structured population in year $t$
compared with that from year $t+10$ will be much more stable than the same
comparison for the standard population. This lesson will reap dividends
throughout the remainder of this dissertation; we will exploit this observation
without investigating much further into its causes.

\paragraph{Uncertainty in $e_y$-structured population pyramids:} One may
reasonably question whether the structure observed in
Figures~\ref{fig:exPyrUS}~and~\ref{fig:exPyrES} is as certain as its graphical
representation would imply-- There are namely no confidence intervals in the
plot, despite the fact that we've used the deaths distribution $d_x$ to
redistribute population counts, and $d_x$ is naturally subject to random
fluctuations. Aside from typical sources of error for population estimates, and
assuming that age in the first place is correctly recorded, we may wish to asses
how much the present population structure could suffer distortion from noise in
$d_x$. 

To answer shed light on this question, the following exercise has been
carried out: 1) for each age of original data, 1000 random deviates were drawn
from the poisson distribution, using the observed death count as the parameter
$\lambda$. 2) Dividing these simulated death counts by observed exposures gives
simulated death rates $\mu_x$. 3) $d_x$ is derived from $\mu_x$ using the HMD
methods protocol. 4) The population is redistributed 1000 times according to the
1000 random $d_x$ disrtibutions. 5) From these simulated population structure,
the 0.025 and 0.975 quantiles are extracted from each class of remaining years.
This yields some rough 95\% uncertainty bands.

It happens that uncertainty due to randomness in $d_x$ is so minor in both the
US and Spanish populations that confidence bands on Figure~\ref{fig:exPyrUS}
would not be visible to the naked eye. There is a pattern to uncertainty in
$e_y$-structured populations, however, that we display in
Figure~\ref{fig:PyramidUncertainty}. Confidence bands are larger for the Spanish
population than for the US population due to difference in population size. The
pattern over remaing years is for uncertainty to decrease from the lowest life
expectancies until $e_y$ is around 40 or 50, and then to
increase geometrically until the highest life expectancies. The highest
uncertainty corresponds with the thinnest part of the pyramid, and the lowest
uncertainty falls around the mode of the pyramid. Uncetainties for males and
females is similar in low $e_y$, but tends to become higher for males and
females as $e_y$ increases geometrically.

\begin{figure}
      \centering
      % Figure made in PyramidStability.R
      \caption{Width of 95\% uncertainty bands as percentage for each
      remaining years category, Spain and US, 1975}
         \includegraphics{Figures/exPyramiduncertainty1975}
      \label{fig:PyramidUncertainty} 
\end{figure}

%\paragraph{Affects of mortality improvement on remaining-years classified
%pyramids:} The pyramid, or leaf, structured by remaining years of life can be
%conceived of as a period indicator, and is not necessarily a prediction about
%the future, especially in the highest levels of $e_y$, many years away.
%Expected deaths to the population presently living several decades from the
%moment of observation are subject to greater uncertainty than that presented in
%Figure~\ref{fig:PyramidUncertainty}, namely uncertainy about future changes in
%mortality. Common experience of the past several decades, with few exceptions, 
%is for mortality rates to have improved from year to year. Of course, different
%ages may undergos differing rates of change\footnote{It is necessary to speak
%of age herre because $d_x$ is an age-classified vector dervided from
%age-specific mortality rates, $\mu_x$}, but on the whole the direction of
%change has been one of improvement.



