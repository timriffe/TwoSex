 \FloatBarrier

The resulting
population structure from applying Formula~\eqref{eq:dxredist} to age-classified
population data is diachronous\footnote{heterogeneous with respect to age} 
within any given level of remaining life expectancy, and 
looks like Figures~\ref{fig:exPyrUS} and \ref{fig:exPyrES}\footnote{The idea to
redistribute the population pyramid in this way is due to a conversation with 
John MacInnes, and appears in \citep{MacInnes2013pop} (unpublished) using a
different method.} for the years 1975 and 2009 in the US and
Spain\footnote{The unlabeled inside cover artwork is the same 2009 pyramid (in
green) but preceeded by history (grey) and continued with a deterministic
projection (blue) under specific assumptions (Spain left, US right, vertical
axes comparable, horizontal axes not comparable).}. As a helpful pointer, note that the
population at the base of the pyramid is expected to decrement within the \textit{next year}, thus the vertical axis can also can also be thought of as year $t+y$, although $e_y$ more clearly identifies the pyramid with year $t$ mortality conditions. The pyramid should not be taken out of context as a forecast. Note that this pyramid represents the exact same
population as an age-classified pyramid: Underlying males sum to the correct total on 
the left and females sum to the correct total on the right. Only the definition of age has
changed; instead of counting forward from birth we count \textit{age} in reverse
starting from death. For individuals, this feat would be impossible, but given
 the information contained in a period lifetable, one can to great utility 
 redistribute population aggregates according to $e_y$\footnote{To undertake
 the same but assuming future mortality changes (improvements) one might
 better undertake a fertility-free cohort component projection and collect the
 deaths from each future year $t+y$ until extinction. This possibility is not
 treated in the present dissertation.}. Both pyramids have been rescaled
 to sum to 100, in order to more comparably represent population structure.

\begin{figure}
        \centering
        \begin{subfigure}
                \centering
                \caption{US population by remaining years, 1975 and 2009}
                \includegraphics[scale = .8]{Figures/exPyramidUS}
                \label{fig:exPyrUS}
        \end{subfigure}
        \begin{subfigure}
                \centering
                \caption{Spain population by remaining years, 1975 and 2009}
                \includegraphics[scale = .8]{Figures/exPyramidES}
               
                \label{fig:exPyrES}
        \end{subfigure}
\end{figure}

A time series of remaining life expectancy pyramids for any given Western 
country (excluding war years and epidemics, and especially after the rapid fall
in infant mortality) will show incredible stability over time, which is
remarkable in light of ageing in the observed population pyramid. The simple 
interpretation of this kind of pyramid adds to its utility, and this author 
believes that $e_y$-specific population structure, and
indicators that can be derived from this method (not treated here) should 
make up a valuable new component to the contemporary demographer's toolbox, as well 
as help inform current population debates. 

It will suffice for the time being to
illustrate that for Spain and the US in the years trated in this dissertation,
the remaining-years-structured population pyramid has been many times more
stable. This we will demonstrate by again making use of the difference
coefficient, $\theta$ from Equation~\ref{eq:coefdiff}, where $f_1$ is the
population structure for year $t$ and $f_2$ is the population structure from
year $t+1$ (males and females, together, scaled to sum to 1). We iteratively
produce $\theta$, comparing year $t$ and $t+1$ for age-structured pyramids in
the first place and for $e_y$-structured pyramids on the other. Pyramids are in
general very stable, so the difference $\theta$ in both cases will nearly always
fall below 0.02. However, $e_y$-$\theta$ is consistently and considerably lower
than the age-$\theta$. It will suffice to take the ratio of the two indicators,
 $e_y$-$\theta$ divided by age-$\theta$ over the period of study for both
 countries, as seen in Figure~\ref{fig:PyramidStability}.

\begin{figure}
      \centering
      % Figure made in PyramidStability.R
      \caption{Relative stability of $e_y$-structured populations to
                age-structured populations, year $t$ vs $t+1$, ratio of
                $\theta$, Spain and US, 1969-2009} 
         \includegraphics{Figures/PyramidStabilityThetaRatio}
      \label{fig:PyramidStability} 
\end{figure}

In Figure~\ref{fig:PyramidStability}, a value of 1 would indicate that the two
ways of structuring population are equally stable between years $t$ and $t+1$;
values less than 1 indicate that the $e_y$-structured population is more stable.
For instance, .5 means that the $e_y$-structured population was twice as stable,
.2 means 5 times more stable, and so forth. In all years in this dissertation,
$e_y$-structuring acted to stabilize the population somewhat. As a heuristic,
runs of years with continuous and modest improvements in mortality will produce
the most stable $e_y$-structured pyramids. This measure of stability compounds
as well: that is to say, and $e_y$-structured population in year $t$
compared with that from year $t+10$ will be much more stable than the same
comparison for the standard population. This lesson will reap dividends
throughout the remainder of this dissertation; we will exploit this observation
without investigating much further into its causes.

\FloatBarrier