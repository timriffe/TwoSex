\FloatBarrier

This chapter aimed to introduce a new variety of population structure--
thanatological age, or age counted as time until death. The basic steps needed
to carry out the transformation presented in Equation~\ref{eq:dxredist} were
indeed available to demographers via the work of
\citet{miller2001increasing}, \citet{lee2002approach}, \citet{vaupel2009life}
and possibly other unbeknownst to this author. Indeed the perspective as a
whole is widespread in demography, this concept of age as time until death
was already in existence, and the name given to it is of known coinage. Our only
addition was to integrate over age, arriving at a wholly 
redistributed population (count, rate) vector entailing loss of age
information\footnote{If this has indeed been done before, then this author has
not seen it, and apologies will be issued}. In a sense this is more of a loss than an addition,
as we throw away information in doing so. 

It has been with fresh eyes that we have investigated fertility rates in terms
of thanatological age. Recall the words of Coale:
``One of the characteristics of demographic research is a search for empirical regularities, particularly in the
age-schedules of rates of marriage, birth, death, and
migration''\citep{coale1996development}. We may claim to have found
empirical regularities in the restructured fertility rates. Indeed, these
patterns appear to be parametrically tractable-- we mark this possibility with
a cairn and continue onward. There is in any case no need to smooth data that
have been redistributed in this fashion! However, if it turns out that the
remaining-years pattern to fertility is more regular, and hence more
predicatable, than the age-pattern to fertility, one may find a model curve to
be of use. We have not given an explicit account of the regularity of
$e_y$-structured fertility over time, though a glance at the surfaces in
figures~\ref{fig:exSFRsurfUS}~and~\ref{fig:exSFRsurfES} invites speculation in
this direction.

We have also found that the cross-classified remaining-years births distribution
is regular to the extent that it greatly resembles its own association-free
distribution. Either this observation is indeed so, or it is the result of the
overly permissive assumption of homogeneity with respect to mortality in
procreating pair formation. Namely, if there is noticeable selection with
respect to mortality fitness in the mate search, which is rather imaginable
\citep{gangestad1993pathogen, roberts2008good}, the patterns seen in
Figure~\ref{fig:US1970obsexpex} will not reflect the true distribution. The
question is then how far off our homogeneity assumption is. This could be
answered by means of linked register data, as has often been done for countries
such as Sweden and Denmark. We may already surmise that procreating individuals
are on average more fit than non-procreating individuals. To the extent that
marriage is a proxy for mating, we already know that married males have a
mortality advantage over unmarried males, although
there are likely intervening factors\citep[see
e.g.][]{rogers1995marriage, waite1995does}, and we do not know for sure whether
this is due to causation or selection\citep[see e.g.][]{goldman1993marriage}. For the
present use, we do not necessarily care whether the differential might be due to
selection or causality. All this is to say that we ought to take in the
observations of cross-classified fertility with a grain of salt. Furthermore, as
is pointed out in the text, selection may disproportionately affect the tails 
in the lowest remaining years classes\footnote{This possibility would be just
the opposite in high maternal mortality settings, which does not affects
our two populations.}. The best we can do in this instance is imagine the
direction of bias, as has we have tried to make clear.

In general, we have seen that period fertility indices calculated using
remaining-years classified data track well to those calculated age-classified
data, but also that they return higher estimates of period fertility. This is
not altogther an expected result, given that we have admitted persons in
non-reproductive ages into our exposures. Part of this may owe to fertility
spreading out over a wider range of classes than in the age-structured setting.
No claims are made as to whether some \textit{true} rate lies between the age
and $e_y$-structured rates, although we will think more on this later in this
dissertation.

We completed the chapter with some rather indefinite speculation about the
potential use of our $v_y$, Fisher's reproductive value rethought to correspond
to the thanatological perspective, in cognate fields of demography, or
evolutionary demography itself. One could just as well reframe $v_y$ in terms of
inter-age transfers rather than reproduction, as per \citet{lee2003rethinking},
although the perspective change will still pertain.

The purpose of this chapter was to define and explore the vital rates to be used
in the remained of this dissertation. We will proceed by defining a model of
population renovation akin to Lotka's renewal equation. From this model we will
extract and explore the intrinsic growth rate and some other stable
parameters that belong to the thanatological perspective. 

\FloatBarrier



