\FloatBarrier
\label{sec:structuncertainty}
The reader may reasonably question whether the structure observed in
Figures~\ref{fig:exPyrUS} and \ref{fig:exPyrES} is as certain as its graphical
representation would imply -- there are namely no confidence intervals in the
plot, despite the fact that we've used the deaths \textit{distribution}, $d_x$,
to redistribute population counts, and $d_x$ is naturally subject to random
fluctuations. Aside from typical sources of error for population estimates, and
assuming that age in the first place is correctly recorded, we may wish to asses
how much the present population structure could suffer distortion from noise in
$d_x$. 

To shed light on this question, the following exercise has been
carried out. 1) For each age of original data 1000 random deviates were drawn
from the poisson distribution, using the observed death count as the parameter
$\lambda$. 2) Dividing these simulated death counts by observed exposures gives
simulated death rates $\mu_x$. 3) $d_x$ is derived from $\mu_x$ using the HMD
methods protocol. 4) The population is redistributed 1000 times according to the
1000 random $d_x$ distributions. 5) From these simulated population structures,
the 0.025 and 0.975 quantiles are extracted from each class of remaining years.
This yields some rough 95\% uncertainty bands.

It happens that uncertainty due to randomness in $d_x$ is so minor in both the
US and Spanish populations that these confidence bands superimposed on
Figure~\ref{fig:exPyrUS} would not be visible to the naked eye. There is
nonetheless a pattern to uncertainty in $e_y$-structured populations, 
which we display in Figure~\ref{fig:PyramidUncertainty}. Confidence bands are
larger for the Spanish population than for the US population due to
the difference in population size. The pattern over remaing years is for
uncertainty to decrease from the lowest life expectancies until $e_y$ is around 40 or 50, and then to
increase geometrically until the highest life expectancies. The highest
uncertainty corresponds with the thinnest part of the pyramid, however, and the
lowest uncertainty falls around the mode of the pyramid. Uncertainty for males
and females is similar in low $e_y$, but tends to become greater for males as
$e_y$ increases geometrically.

\begin{figure}
      \centering
      % Figure made in PyramidStability.R
      \caption{Width of 95\% uncertainty bands as percentage for each
      remaining-years class, Spain and US, 1975}
         \includegraphics{Figures/exPyramiduncertainty1975}
      \label{fig:PyramidUncertainty} 
\end{figure}

By far the greatest source of uncertainty in this
pseudo-projection arises if the demographer decides to account for
improvements in mortality. This later uncertainty arises not only from
random fluctuations, but also due to the projection assumptions used. This
variety of adjustment, adding improvement to the mix, is discussed in the
following section.

\FloatBarrier