 \FloatBarrier
 \label{sec:etfr}
This evolution in rates can, as with ASFR, also be
summarized with an indicator akin to TFR, which we here call $e$TFR:

\begin{equation}
\label{eq:exTFR}
e\mathrm{TFR} = \int _{y=0}^\infty F_y \dd y
\end{equation}
where $y$ indexes remaining years of life. A time series of this indicator
for the period studied is displayed in Figure~\ref{fig:exTFR}.

\begin{figure}[ht!]
        \centering  
          \caption{Male and Female $e_y$-total fertility rates versus standard
          TFR, USA, 1969-2009 and Spain, 1975-2009}
           % figure produced in /R/Parents_ex.R
           \includegraphics{Figures/exTFR}
          \label{fig:exTFR}
\end{figure}

Canonical TFR can conveniently be imagined as the total number of
offspring that that an average female (male) will have in a lifetime assuming
no mortality and constant fertility rates of the present year.
Since a lifetime measured in age counting from birth is the same length as a
lifetime measured in age counting backward from death, $e$TFR in fact has the
same interpretation. Why is this? Age-classified rates are of course
heterogeneous within age with respect to remaining life expectancy, and here we
have produced an synthetic index based on the reverse idea. The age-classified
distribution of births and populations are quite different (there being an age
pattern to fertility rates). $e_y$-reclassifying these data not only changes the
center of gravity of numerator and denominator distributions, but asymmetrically
shifts underlying schedules, uniquely reshaping the pattern of
fertility. Summing over $e_y$-rates will almost yield a different total, our
synthetic $e$TFR. 

Figures~\ref{fig:exSFRsurfES}, ~\ref{fig:exSFRsurfUS}~and~\ref{fig:exTFR} are
reproduced according to various definitions of the reproductive span in
Appendix~\ref{Appendix:reprospans}. Rates are shown to be sensitive to the
choice of reproductive span. For the remainder of this dissertation, we ignore
issues of age boundaries in the reproductive span for simplicity and
consistency, although this issue deserves further attention if the
remaining-years-perspective is deemed of merit.