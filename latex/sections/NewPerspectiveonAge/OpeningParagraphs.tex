Recall the observations of \citet{sanderson2005average}, more directly
relating to the popuation pyramid. These authors note that despite ageing in a
population, the mean \textit{remaining} years to be lived may increase. This is due 
to improvements in mortality offsetting (or more than offsetting) increases in
the mean age of a population. In general, a population looked at from the
perspective of life expectancy looks different from, behaves differently from,
and yields complimentary information to one looked at from the canonical
perspective of age. This observation is the point of departure for the
linear and non-linear two-sex models that I will introduce in this dissertation.
However, such steps are large, and require an involved demonstration of some
key differences between age clsasified and \textit{remaining years}-classified,
henceforth $e_x$-classified demographic data. Indeed a whole new one-sex
model must first be developed prior to re-delving into its two-sex extensions. 
I will first present a method
to exactly redistribute population counts (or exposures) according to period 
remaining years of life, rather than according to age per se. A reexamination 
of recent fertility patterns according to remaining years of life will follow.
