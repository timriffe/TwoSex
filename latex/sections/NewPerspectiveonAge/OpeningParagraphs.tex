 \FloatBarrier
 
\begin{singlespace}
\begin{quote}
Jedes letzte Jetzt ist \textit{als Jetzt} je immer \textit{schon} ein
Sofort-nicht-mehr, also Zeit im Sinne des Nicht-mehr-jetzt, der Vergangenheit;
jedes erste Jetzt ist je ein Soeben-noch-nicht, mithin Zeit im Sinne des
Noch-nicht-jetzt, der $\langle$Zukunft$\rangle$.
\citep[p424][]{heidegger1977sein}
\end{quote}
\textit{translation:}
\begin{quote}
Every last now, as a now, is always already a right-away that is no longer, thus
it is time in the sense of the sense of the no longer now, of the past. Every
first now is always a just-now-not-yet, thus it is time in the sense of the
not-yet-now, the ``future''. \citep{stambaugh1996being}
\end{quote}
\end{singlespace}

In general, a population viewed from the
perspective of remaining years of life displays different patterns 
from and yields complimentary information to one looked at from the
perspective of chronological age, time since birth. This observation is the
point of departure for the population models that will be presented in the remainder of this
dissertation. 

This perspective is known in other branches of social science as well.
 \citet{carstensen2006influence,carstensen1999taking}, 
for instance, argue that various aspects of emotion and cognition are best 
relativized to one's perception of time until
death rather than time since birth. Carstensen argues, inter alia, that the
precision of chronological age in measuring the life course loses precision in old age,
whereas one's intuition of remaining time gains in precision for marking various
kinds of cognitive transitions. \citet{meyer2008altern} emphasizes that the
individual's experience of time is put into play upon the transition out of
productive work, which may include the notion of moving toward death. Meyer
argues that the dominace of various experiences of time over phases of the
lifecourse coupled with changes in the demographic composition may provide
opportunities to redefine the dominant understanding of ageing. Namely, is
the experience of age backward-looking, a sequence of significant events,
forward-looking, or something else? It is evident that interest in the
remaining-years perspective in social science goes beyond the
confines of quantitative population studies. Here we will use the perspective in
a strictly formal demographic framework. Secondary uses of and insights from the
strictly demographic results of applying this perspective to population data
evidently may reach beyond the limited ends of population accounting or
economic planning.

Counting age as time until death rather than time
since birth, applied to the extent that is to
follow, has been lacking from the discipline of demography, although there is a
widespread desire for such tools, and much work of high value has been done in
the same direction. \citet{hersch1944demographie}
introduced the idea of \textit{potential years of life}, PYL, the
total years of remaining life expectancy for a particular age class of
population. This measure has proven very useful to demographers
\citep{panush1996potential}, but it is considerably more aggregate in 
nature than the methods to be proposed
here, and is not a true departure from the age perspective, as it is calculated
for age classes. \citet{ryder1975notes} as well calculated exact ages at which
particular life expectancies were attained using the Coale-Demeny model
life tables, taking the extra step of calculating (stable) proportions
of population with a particular remaining life expectancy.

Recently, 
\citet{sanderson2005average,sanderson2010remeasuring} and have made much headway
 in using the notion of remaining years of life in
 order to adjust measures of population ageing and life expectancy. For instance, 
 these authors offer an index of average remaining life expectancy (PARYL), which 
 is calculated as the weighted average of age-specific 
 remaining life expectancies, a measure which summarizes that of
 \citet{hersch1944demographie}. The interpretation of this index of
 course belongs to the remaining-years perspective.
 \citet{sanderson2005average}, for instance, note that despite ageing 
 in a population, the mean \textit{remaining} years to 
 be lived may increase. This is due to improvements in mortality offsetting 
 (or more than offsetting) increases in the mean age of a population.

The author whose work most resembles what will be
presented here is \citet{miller2001increasing}, who takes a more
exact approach than the above, looking at particular age-specific death
\textit{distributions}\footnote{age-specific death distributions are more
 specific than age-specific remaining life expectancy because the later are weighted 
 averages of the former.} in order to calculate indices
 of projected health expenditures. This method has been extended somewhat into
 the domain of health care expenditure projection
 \citep{lee2002approach,lee2007demographic,topoleski2004uncertainty}. We will
 relate our own method to \citet{miller2001increasing} in the
 following section. \citet{stearns2004time} and \citet{seshamani2004longitudinal} as well apply a regression approach in
 order to account for time until-death-effects-- it is clear that there is great
 interest among cognate disciplines for demographic data classified by remaining
 years of life.
 
 These contributions are of great importance, but they take a different 
 strategy than that proposed here, working primarily with particular ages, in a 
 regression framework, or in a stochastic cohort component projection framework.
  A full embrace of the remaining-years perspective would require us to answer
 the following question: ``How many persons in this population have a remaining
 life expectancy of $y$?''. In answering this question for each remaining life
 expectancy, $y$, one arrives at a population structure \textit{by} remaining
 years of life, \textit{thanatological age}\footnote{This term was coined by
 Ken Wachter and Tim Miller.} in which case the population may be said to be
 structured by remaining years of life. PYL and similar measures do not answer this question for us. The concept in itself is not new, and it owes to the work of 
 many demographers. 
 
 Our objective in this dissertation is to take the concept of
 remaining time, time until death, three steps further, 1) outright
 restructuring population by thanatological age \textit{as opposed to}
 chronological age\footnote{One could call the temporal ordering element to
 population structured by remaining years of life \textit{descending age} or
 \textit{reverse chronological age}.}, 2) exploring the implications for
 population growth under this new form of structure in general, and 3) exposing 
 a variety of two-sex extentions to the growth models, with special attention 
 to stable populations. We go well beyond a mortality-only or specific-use scope.

Demographers mark age as a linear function of newtonian time since
birth, an event which for all is necessarily in the past. For this reason we may
label this concept of age as backward-looking. The age structure of a population
is in this way a mirror to the past. Demographers also project forward, an activity 
informed by the experience of the past and present. To the
extent that projections of future population are also structured by age,
they are also projections of future mirrors to the past, i.e. still
backward-looking. Yet present populations may also be structured by an unknown
future, and this is what we propose to do. In this dissertation, we will derive
this forward-looking population structure based on data from the present, which of course are a reflection 
of the past. The activity is in this way necessary projective, but does not 
seek to be a projection in the proper demographic sense. It is rather an
application of synthetic, static, lifetable methods to a present population
under the assumption of constant mortality conditions. This time transformation
is revealing of a potential future-- a potential population structure, in the 
sense of \citet{hersch1944demographie}.

We will speculate that this notion of (potential) reverse chronological age may
also yield insights to all manner of demographic phenomenon. Specific
applications of the remaining-years perspective for the demographic study 
of the life course will not be discussed. Instead, we aim to make headway in 
the more fundamental terrain of population renewal and growth, and the practice of demography in general.
Insodoing, we hope to make available a set of tools to expand the present domain
of basic demographic analysis from an age-sex paradigm to include the
\textit{remaining years}-sex paradigm.

We begin by pointing out the key differences
between \textit{remaining years}-classified demographic data,
henceforth $e_y$-classified data\footnote{$e_y$ is distinct from $e_x$, in
this sense, since the later is defined as mean remaining lifetime by age,
whereas $e_y$, and the subscript $y$ in general will be used to refer to
remaining years as a classifying, or structuring, variable.}, and age-classified
data. We will first present a method to exactly redistribute population 
counts (events, exposures) according to remaining years of life, as determined
by the period lifetable. A reexamination of recent fertility patterns according
to remaining years of life will follow. The following chapters will build upon 
the idea of $e_y$-structured populations
to develop a new concept of population renewal and growth. First, the single-sex
model will be presented, followed by two-sex extensions. Both linear and
non-linear extensions will be considered. Results will be compared with those
from the age-classified system. Special attention will be given to the two-sex
problem throughout.


