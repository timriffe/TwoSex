 \FloatBarrier
Recall the observations of \citet{sanderson2005average}, more directly
relating to the popuation pyramid. These authors note that despite ageing in a
population, the mean \textit{remaining} years to be lived may increase. This is due 
to improvements in mortality offsetting (or more than offsetting) increases in
the mean age of a population. In general, a population viewed from the
perspective of life expectancy looks different from, behaves differently from,
and yields complimentary information to one looked at from the
perspective of chronological age. This observation is the point of departure for
the population models that will be presented in the remainder of this
dissertation.

This perspective, that of counting age as time until death rather than time
until birth, applied in the degree of formality and exactitude that is to
follow, has been lacking from the discipline of demography. It is known, albeit 
not fully established in other branches of social science, such as psychology.
\citet{carstensen2006influence,carstensen1999taking}, for instance, argue that
various aspects of emotion and cognition are best relativized to one's perception of time until
death than time since birth. Carstensen argues, inter alia, that the precision
of chronological age in measuring the life course loses precision in old age,
whereas one's intuition of remaining time gains in precision for marking various
kinds of cognitive transitions. We will speculate that this might also be
the case for the measurement of demographic phenomenon. Specific applications of
the remaining years perspective for the demographic study of the life course
will not be discussed. Instead, we aim to make headway in the more fundamental terrain of
population renewal and growth, and the practice of demography in general.
Insodoing, we hope to make available a set of tools to expand the present domain
of basic demographic analysis from an age-sex paradigm to include the
\textit{remaining years}-sex paradigm.

We begin by pointing out the key differences
between \textit{remaining years}-classified demographic data,
henceforth $e_y$-classified data\footnote{$e_y$ is distinct from $e_x$, in
this sense, since the later is defined as mean remaining lifetime by age,
whereas $e_y$, and the subscript $y$ in general will be used to refer to
remaining years as a classifying, or structuring, variable.}, and age-classified
data. We will first present a method to exactly redistribute population counts (events, exposures) according to period remaining years of life, rather than according to age per se. A reexamination of recent fertility patterns according to remaining years of life will follow. The following chapters will build upon the idea of $e_x$-structured populations
to develop a new concept of population renewal and growth. First, the single-sex
model will be presented, followed by two-sex extensions. Both linear and
non-linear extensions will be considered. Results will be compared with those
from the age-classified system. Special attention will be given to the two-sex
problem throughout.


