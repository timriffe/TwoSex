 \FloatBarrier
\label{sec:exdivergence}
It has been noted that the observed and expected distributions of births by
remaining years of life of mothers and fathers very closely resemble each other
(See Figure~\ref{fig:TotalVarobsexpex}), almost enough so that we could
approximate the observed distribution by a random distribution given only the
marginal distributions. In any case, the result would be a much closer fit to
observed data than would be the same excercise if undertaken with typical
age-classified data. 

Further, it has been noted that the population pyramid is
much more (6 times more) stable from year to year when classified by remaining
years of life than when classified by age. This is so because the
$e_x$-classified pyramid does not uniformly decrement in single-year steps, due
to well known and apparently stable trends of year-on-year mortality improvement
that have thus far shown no signs of abatement. Intuitively, the central bulge
in an $e_x$-classified population pyramid does not plummet to the base of the pyramid at a rate of 1
year per year, but rather much more slowly and smoothly, always leaving a
tapered base to the pyramid (the population expected to decrement soon). In this
way, the $e_x$-pyramid, at least in popualations that have radically reduced
infant and child mortality and have late-transition fertility
levels\footnote{These two characteristics typically co-occur
\citep{macinnes2009reproductive}, and both conditions hold for the US and Spain
in the years presented here}, tends to vary within a narrow range of what may,
with little room for discussion, be called a characteristic \textit{leaf} shape.

Since the $e_x$ population distribution can be predicted
with nearly equal facility and precision as the age-distributed population in
year $t+1$, one may ask whether, given the relative stability of underlying
exposures for both male and females, $e_x$-specific fertility rates are also more stable than
age-specific fertility rates from year to year. If this is so, then predicting 
birth counts $n$ years hence separately for the sexes based
on year $t$ $e_x$-SFR and year $t + n$ $e_x$-classified exposures has the
potential to entail lower disagreement in predicted birth counts derived
from male and female rates and exposures than does the age
anologue of this same exercise.

If results show that projected divergence in predicted birth counts, holding
single-sex fertility rates constant, is lesser for the $e_x$-classified data
than for age-classified data, then we can safely say that the two-sex problem
has been reduced in size, albeit not solved. In the case that the magnitude of
the problem has been reduced by this simple transformation, one further
concludes that whatever empirical or axiomatic drawbacks entailed by 
two-sex adjustment procedures presently in the literature will also be reduced,
thereby making the two-sex problem in practice less problematic. 

This exercise has been carried out for both the US and Spain with $n$ equal to
1, 5, 10 and 15. In summary, for the US (See Table~\ref{tab:USexDiscrepancy}),
the sex-discrepancy entailed by $e_x$-classified data is on the order of 5 times 
smaller than for age-classified
data, a welcome improvement . Further, the
$e_x$-method for the US entails sex discrepancies that vary roughly around zero, 
whereas age-classfied data were
always positively biased in the period studied. For Spain (See
Table~\ref{tab:ESexDiscrepancy}), we notice no difference in the magnitude of 
discrepancy, but indeed in the sign of
discrepancy.

% table of mean relative difference over whole span.
% US
\begin{table}
\caption{Relative discrepancy between single-sex projected births $n$ years
hence using $e_x$- vs age-classified data US, 1969-2009}
\label{tab:USexDiscrepancy}
\centering
\makebox[0pt][c]{\parbox{1.1\textwidth}{%
    \begin{minipage}[b]{0.45\hsize}
    \centering
        \caption*{Mean Relative Discrepancy}
        % latex table generated in R 2.15.2 by xtable 1.7-0 package
% Wed Feb 27 17:14:41 2013
\begin{tabular}{l|cc}
  \hline
 & $e_x$ & Age \\ 
  \hline
1-year & -0.0002 & 0.0038 \\ 
  5-year & -0.0024 & 0.0202 \\ 
  10-year & -0.0073 & 0.0444 \\ 
  15-year & -0.0131 & 0.0705 \\ 
   \hline
\end{tabular}

    \end{minipage}
    \hfill
    \begin{minipage}[b]{0.55\hsize}
    \centering
        \caption*{Mean Absolute Relative Discrepancy}
        % latex table generated in R 2.15.2 by xtable 1.7-0 package
% Wed Feb 27 17:14:42 2013
\begin{tabular}{rr}
  \hline
$e_x$ & Age \\ 
  \hline
0.0022 & 0.0049 \\ 
  0.0072 & 0.0248 \\ 
  0.0106 & 0.0505 \\ 
  0.0145 & 0.0743 \\ 
   \hline
\end{tabular}

    \end{minipage}
}}
\end{table}

% Spain
\begin{table}
\caption{Relative discrepancy between single-sex projected births $n$ years
hence using $e_x$- vs age-classified data, Spain 1975-2009}
\label{tab:ESexDiscrepancy}
\centering
\makebox[0pt][c]{\parbox{1.1\textwidth}{%
    \begin{minipage}[b]{0.45\hsize}
    \centering
        \caption*{Mean Relative Discrepancy}
        % latex table generated in R 2.15.2 by xtable 1.7-0 package
% Mon Mar  4 13:33:41 2013
\begin{tabular}{l|cc}
  \hline
 & $e_x$ & Age \\ 
  \hline
1-year & -0.0029 & 0.0036 \\ 
  5-year & -0.0168 & 0.0193 \\ 
  10-year & -0.0403 & 0.0401 \\ 
  15-year & -0.0641 & 0.0632 \\ 
   \hline
\end{tabular}

    \end{minipage}
    \hfill
    \begin{minipage}[b]{0.55\hsize}
    \centering
        \caption*{Mean Absolute Relative Discrepancy}
        % latex table generated in R 2.15.2 by xtable 1.7-0 package
% Mon Mar  4 13:33:41 2013
\begin{tabular}{rr}
  \hline
$e_x$ & Age \\ 
  \hline
0.0048 & 0.0047 \\ 
  0.0204 & 0.0238 \\ 
  0.0419 & 0.0437 \\ 
  0.0641 & 0.0633 \\ 
   \hline
\end{tabular}

    \end{minipage}
}}
\end{table}

%file.exists("home/triffe/git/DISS/latex/xtables/exDivergenceMeanAbsTableES.tex")

