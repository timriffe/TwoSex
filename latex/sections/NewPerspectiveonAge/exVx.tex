\FloatBarrier
\label{sec:fisher}
We wish to mention in passing that thanatologically structured fertility rates,
$F_y$, are just a short step away from a remaining-years version of Fisher's
\citep{fisher1999genetical} reproductive value\footnote{I thank Robert Chung
for suggesting that I think about this.}. Note that $y$ indicates the
temporal distance from death, and that therefore no decrement occurs unless $y=0$. 
Hence, $F_y$ is already in a sense discounted for mortality. The
reproductive value, $v_y$, in this setting becomes:

\begin{equation}
\label{eq:eyfisher}
v_y = \frac{\int _0 ^y F_y \dd y}{\int _0 ^\infty F_y \dd y} 
\end{equation}
In other words, $v_y$ is the proportion of reproduction that remains between
remaining years, $y$, and death, $y=0$. As with the age-specific
reproductive value, $v_x$, this value is the same in the present and stable
populations. The age-structured variant of this indicator has been of great 
value in the field of biology and in evolutionary theory, and one wonders
if the present definition might be of explanatory use. In the age-perspective,
we expect selective pressures on population to be greatest where the reproductive value,
$v_x$, is highest (equal to 1). Under the standard view, the ages
where $v_x = 1$ also are the ages where we observe mortality minima, as well as
negative senescence. Indeed, mortality is at its minimum right before the
onset of fertility (for females)-- a better bet for the species in terms of
reproductive output than for infants, though these have the same reproductive
value by the $v_x$. Under $v_y$, things are not so clear- namely in the highest
values of remaining years, the indicator will obtain the value of 1, and it will
fall off less precipitously with the approach of death, never fully dropping to
0, namely because remaining-years classes are heterogeous with respect to age
and therefore even very low values of $y$ will contain some reproduction.
Females, of course, do come much closer to $v_0 = 0$ than do males, due to
menopause. Figure~\ref{fig:vyUS1990} provides example male and female $v_x$ and
$v_y$ curves for the US in 1990.

 \begin{figure}
                \centering
                \caption{Age-specific and remaining-years specific reproductive
                values, US, 1990}
                \includegraphics{Figures/vyUS1990}
                \label{fig:vyUS1990}
\end{figure}

Indeed, the two-sex problem has been brought to bear on the concept of
reproductive values \citep{samuelson1977generalizing}. In brief, one may
reasonably ask what is the reproductive value of a 25-year-old male if there are
no females around?-- Hence the need for a two-sex solution. In-depth thinking 
on the consequences for reproductive values of the present age
transformation is beyond the scope of this dissertation, much less how two-sex
solutions may be of use to the definition of more comprehensive reproductive
values. We plant this seed and move on.

\FloatBarrier
