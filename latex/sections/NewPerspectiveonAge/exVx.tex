
We wish to mention in passing that thanatologically structured fertility rates,
$F_y$ are just a short step away from a remaining-years version of Fisher's
\citep{fisher1999genetical} reproductive value. Note that $y$ indicates the
temporal distance from death, and that therefore no decrement occurs 
unless $y=0$. Hence, $F_y$ is already in a sense discounted for mortality. The
reproductive value, $v_y$, in this setting becomes:

\begin{equation}
\label{eq:eyfisher}
v_y = \frac{\int _0 ^y F_y \dd y}{\int _0 ^\infty F_y \dd y} 
\end{equation}

In other words, the proportion of reproduction that remains between
remaining years, $y$, and death, $y=0$. This value is the same in the present
and stable populations. The age-structured variant of this indicator has been of
great value in the field of biology and in evolutionary theory, and one wonders
if the present definition might be of explanatory use. We plant this seed and
move on.
