\FloatBarrier
\label{sec:eximprov}
As we are dealing with a \textit{forward-looking}\citep{sanderson2007new}
transformation of structure, the reader will likely intuit that our redistribution 
method will err toward pessimism to the extent that
future improvements in mortality are ignored. To take account of future
mortality improvements is already a projections mindset, and we have made no
claim that the this structure is a projection- Rather it is a speedometer, to
use the analogy of \citet{coale1972growth}. In any case, one best accounts for
future mortality improvements by applying assumptions about the rate of
improvment in the age-specific motality hazard, $\mu_a$, rather than directly
manipulating $d_a$, which is our redistribution vector. To illustrate by
example, take the population of infants and the full present vector of $\mu_a$. It seems fair that
that $\mu_0$, or something close to it, will apply to these infants, yet by age
5, $\mu_5$ will likely be too high, and certainly by age 50
this same static $\mu_50$ will be too high. For the sake of simplicity, let us assume that the
rate of improvement, $\iota$, applies equally over all ages and future years,
and is equal to about 0.5\% per year (conservative for some ages, liberal
for others). We would prefer to have a multiplicative factor, so we define
$\iota = 1-.005 = 0.995$. Then we modify $\mu_a$ for these 0-year-olds, in
the following way

\begin{equation}
\mu_a^0 = \mu_a \prod _0 ^\infty \iota 
\end{equation}

where the superscript indicates that we have done this for persons of age 0. Now
one converts the $\mu _a^0$ to the new $d_a^0$ and redistributes the infant
population accordingly. For higher ages, one calculates $\mu_a^x$ as

\begin{equation}
\mu_a^x = \mu_a \prod _x ^\infty \iota 
\end{equation}
always with the first age $a$ equal to $x$.
Now one needs a direct relation of $\mu _a$ and $d_a$, and for this we use the
HMD methodology, namely assuming that the average age at death in each age
interval is 0.5 (except for age 0, which uses the Coale-Demeny rule of thumb),
deriving death probabilities, $q_a$, followed by the survival function, $l_a$,
and taking the product of these two to arrive at $d_a$\footnote{This is the
abbreviated version. See the HMD Methods Protocol for the full version, which
for this dissertation we have functionalized.}. This must be done for each age,
to create as many modified $\mu _a^x $ vectors as there are ages, and then one applies the same
Equation~\ref{eq:dxredist}, swapping the $\mu _a^x$ vector as one iterates over
ages. Alternatively, one could just take some future evolution of $\mu_a$ from a
standard projection method, such as the Lee-Carter method
\citep{lee1992modeling}, finding the mortality trajectory expected for each individual, converting
this to $d_a$, and then redistributing the population accordingly. 

Here we execute the simplest assumption, though there is ample room for
improvment in the method. The results of decreasing year 2009 mortality rates in each
successive year (iteration) by a multiplicative factor of 0.995 are displayed in
Figures~\ref{fig:exPyrUSimpr}~and~\ref{fig:exPyrESimpr}.

\begin{figure}
        \centering
        \begin{subfigure}
                \centering
                \caption{US population by remaining years under fixed
                mortality and constant improvement of 0.5\% per year.}
                \includegraphics[scale = .8]{Figures/exPyramidUSimpr}
                \label{fig:exPyrUSimpr}
        \end{subfigure}
        \begin{subfigure}
                \centering
                \caption{Spanish population by remaining years under fixed
                mortality and constant improvement of 0.5\% per year.}
                \includegraphics[scale = .8]{Figures/exPyramidESimpr}
                \label{fig:exPyrESimpr}
        \end{subfigure}
\end{figure}

Here, common practice would have been to start the rate of improvement stronger
and let is taper off with time, or to allow effects to work differently over
age, but the present exercise is meant to be illustrative. Here one appreciates
the slight maleability of population structure in light of potential future
mortality improvements. The impending mode, composed largely but not entirely of
baby-boomers, is absorbed to a certain extent, and the overall picture is even
more optimistic than the original (as one would expect) for both countries.

To incorporate uncertainty into this method, it is
recommended to allow variaton in $\mu_x$ per the strategy outlined in the previous
section~\ref{sec:structuncertainty}. At this time, we will depart from the
present line of development. For the rest of this dissertation we will deal only with static
period deaths distributions, and we leave the thoughtful incorporation of mortality 
improvements and addittional uncertainty into the method for future work.

\FloatBarrier