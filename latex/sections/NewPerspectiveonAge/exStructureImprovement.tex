\FloatBarrier
\label{sec:eximprov}
As we are dealing with a \textit{forward-looking} \citep{sanderson2007new}
transformation of structure, the reader will likely intuit that our redistribution 
method will err toward pessimism to the extent that
future improvements in mortality are ignored. The desire to take account of
future mortality improvements is already a projection mindset, and we have made
no claim that the this structure is a realistic projection -- Rather it is a
scenario -- a speedometer, to use the analogy of \citet{coale1972growth}. In any
case, one best accounts for future mortality improvements by applying assumptions about the rate of
improvement in the age-specific mortality \textit{hazard}, $\mu_a$, rather than
directly manipulating $d_a$, which is our redistribution vector. To illustrate by
example, take the population of infants and the full present vector of $\mu_a$. It seems fair that
that $\mu_0$, or something close to it, will apply to these infants, yet by age
5, $\mu_5$ will likely be too high, and certainly by age 50
this same static $\mu_{50}$ will be too high. For the sake of simplicity, let us
assume that the rate of improvement, $\iota$, applies equally over all ages and future years,
and is equal to about 0.5\% per year (conservative for some ages, liberal
for others). To use this as a multiplicative factor, we take $e^{\iota a}$,
where $\iota = -0.005$. Then we modify $\mu_a$ for these 0-year-olds, in the
following way
\begin{equation}
\mu_a^0 = \mu_a e^{\iota a} 
\end{equation}
where the superscript indicates that we have done this for persons of age 0, and
$\omega$ is the highest age attainable. Now one converts the $\mu _a^0$ to the
new $d_a^0$ and redistributes the infant population accordingly. The relation
between $\mu_a$ and $d_a$ in continuous terms is given by
\begin{equation}
d_a = \mu_a e^{-\int_0^\infty \mu_a \dd a}
\end{equation}
For the sake of completeness, we can respecify the entire redistribution
equation~\eqref{eq:dxredist} in terms of $\mu_a$ as
\begin{equation}
P_y = \int _{a=0}^\infty P_a \frac{\mu_{a+y} e^{-\int_{b=0}^{a+y} \mu_b \dd
b}}{e^{-\int_{b=0}^a \mu_b \dd b}} \dd a
\end{equation}
Then allowing for a constant rate of reduction in $\mu_a$ for ages
\textit{after} age $a$ in year $t$, $\iota$ (e.g., $-0.005$), we get
\begin{equation}
P_y = \int_{a=0}^\infty P_a \frac{\mu_{a+y}e^{\iota y} (e^{-\int_{b=0}^a\mu_b
\dd b}+e^{-\int_{b=a}^{a+y} \mu_b e^{\iota b} \dd b})}{e^{-\int_{b=0}^a \mu_b
\dd b}} \dd a
\end{equation}
which is likely in need of some explanation. The observed population of age $a$,
$P_a$, has already survived to its age $a$, so improvements for this set of
individuals must happen in later ages only (our assumption about the future).
Thus, we do not alter $l_a$ (the denominator) or the left side of the sum in
parentheses in the numerator (also equal to $l_a$). The part in the numerator in
parentheses is $l_{a+n}$ from Vaupel's Equation~\eqref{eq:vaupelredist}, but allowing
for improvements in $\mu$ starting with age $a$ only (the right side). In this
simple case, future $\mu_a$ are reduced by a factor of $e^{\iota n}$ where 
$n$ counts up from the age-group being redistributed. To allow for more flexible
improvements in $\mu_a$, we would need a separate vector of values for the
proportional reduction in $\mu_{a+n}$, for each change in $a$, or else a full
matrix of the future $\mu_a$ values that would apply to each age-group, taken
for example from a projection.

For our discretized example, the above formulas are not convenient. We use the
HMD methodology to derive to $d_a$ from $\mu_a$. This entails the following
steps. 1) Assume that the average proportion of the year completed at death in
each single-age interval is 0.5 (except for age 0, which uses the
sex-averaged Coale-Demeny rule of thumb). 2) Derive death probabilities, $q_a$,
using $\mu_a$ and the latter. 3) Derive the survival function, $l_a$, as the cumulative product of
the complement of $q_a$, with an initial value of 1 and a final value of 0. 4)
Finally, take the element-wise product of $q_a$ and $l_a$ to arrive at our
requisite $d_a$.\footnote{This is the abbreviated version. See the HMD Methods
Protocol \citep{wilmoth2007methods} for the full version, which for this
dissertation we have functionalized.} This must be done for each age, to create as many modified $\mu _a^x $ vectors as there 
are ages, and then one applies the same Equation~\eqref{eq:dxredist}, swapping
the $\mu _a^x$ vector as one iterates over ages. Alternatively, as mentioned above,
one could just take some future evolution of $\mu_a$ from a standard projection method, such as the Lee-Carter method
\citep{lee1992modeling}, finding the mortality trajectory expected for each individual, converting
this to $d_a$, and then redistributing the population accordingly. 

Here we execute the simplest assumption, though there is ample room for
improvement in the method. The results of decreasing year 2009 mortality rates
in each successive year (iteration) by a multiplicative factor of 0.995 (close to
$e^{-0.005}$) are displayed in
Figures~\ref{fig:exPyrUSimpr}~and~\ref{fig:exPyrESimpr}.

\begin{figure}
        \centering
        \begin{subfigure}
                \centering
                \caption{US population by remaining years under
                constant multipicative reduction in $\mu_a$ of 0.995 per year.}
                \includegraphics[scale = .8]{Figures/exPyramidUSimpr}
                \label{fig:exPyrUSimpr}
        \end{subfigure}
        \begin{subfigure}
                \centering
                \caption{Spanish population by remaining years under
                constant multipicative reduction in $\mu_a$ of 0.995 per year}
                \includegraphics[scale = .8]{Figures/exPyramidESimpr}
                \label{fig:exPyrESimpr}
        \end{subfigure}
\end{figure}

Common practice would have been to start the rate of improvement stronger
and let is taper off with time, or to allow effects to work differently over
age, but the present exercise is meant to be illustrative. Here one appreciates
the slight malleability of population structure in light of potential future
mortality improvements. The impending mode, composed largely but not entirely of
baby-boomers, is absorbed to a certain extent, and the overall picture is even
more optimistic than the original (as one would expect) for both countries.

To incorporate uncertainty into this method, it is
recommended to allow variation in $\mu_x$ per the strategy outlined in the
previous Section~\ref{sec:structuncertainty}. At this time, we depart from the
present line of development. For the rest of this dissertation we deal only with static
period deaths distributions, and we leave the thoughtful incorporation of mortality 
improvements and additional uncertainty into the method for future work.

\FloatBarrier