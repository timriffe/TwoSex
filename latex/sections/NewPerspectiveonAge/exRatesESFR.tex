 \FloatBarrier
The technique presented in Equation~\ref{eq:dxredist} and illustrated in
Figures~\ref{fig:exPyrUS}~and~\ref{fig:exPyrES} can indeed be used to reclassify
any age-distributed data, assuming that the appropriate lifetable is available.
We now apply this redistribution technique in order to calculate male and 
female $e_y$-specific fertility rates ($e$SFR). For any rate, the numerator 
and denominator require a common referent, thus both births and exposures are 
redistributed according to year $t$ mortality conditions. That
is to say, we take the extra step of moving the age-specific vector
of birth counts (by mothers' or fathers' age) into $e_y$-specific birth
vectors before dividing into $e_y$-specific exposures. Explicitly:

\begin{align}
\label{eq:eSFR}
F_{y} &= \frac{\int _{a = 0} ^{\infty} B_a \frac{d_{a + y}}{\int _{b = a)}
^{\infty} d_b\, \dd b} \;\dd a}{\int _{a = 0} ^{\infty} F_a \frac{d_{a + y}}{\int _{b = a} ^{\infty} d_b\, \dd b} \;\dd a} \\
 &= \frac{B_{y}}{E_{y}}
\end{align}
where $y$ indexes remaining years of life and $a$ indexes age, $B_a$ are
age-clsasified births, and $E_a$ are age-classified exposures. Remaining years
of life-specific rates cannot be directly compared with a typical age-specific
rate, since the time scales are different, but we 
can indeed apply some familiar tools in order to analyze this new curve.

The $e_y$-pattern of fertility is distinct from the age-pattern of fertility. 
In contemporary Western populations, female $e$SFR curves will be
further to the right than male curves for two reasons: 1) Female mortality is
almost universally lower than male mortality at (and beyond) any given age,
thus associating births at a given age with higher remaining life expectancies; 2)
female fertility is more tightly concentrated over young ages, partly due to the
upper bound defined by menopause, and partly due to prevailing hypergamy.
Figure~\ref{fig:eSFR2009} shows an example $e$SFR from 2009, for both the US and Spain.

\begin{figure}[ht!]
        \centering  
          \caption{Male and Female $e_y$-specific fertility rates, 2009, USA and
          Spain}
           % figure produced in /R/Parents_ex.R
           \includegraphics{Figures/eSFR2009}
          \label{fig:eSFR2009}
\end{figure}

One may question whether the curves shown in Figure~\ref{fig:eSFR2009} properly
represent rates. This author argues that the same definition of events in the
numerator and exposures in the denominator has been applied, only the
structuring variable has changed from \textit{time since birth} to \textit{time
until death} (of progenitor here). In this way, age-classified and
$e_y$-classified populations have structure in the same sense. As with any
demographic variable, we may wish to analyze the intensity of demographic
phenomena removed of the distorting effects of population structure.
Working with event-exposure rates are just one way of doing so, simple
decomposition is another, and indeed such rates and decompositions are possible
in the aggregate both with respect to age and with respect to
$e_y$.

This is, in the best case, a rough calculation, for several reasons. The
assumption of homogenous mortality is particularly consequential in the case of 
fertility, where health selection is self-evident, but not easily measurable.
It is for this reason to be expected that the left tails in
Figure~\ref{fig:eSFR2009} are too thick. 

Furthermore, exposure is taken from the \textit{entire} population, not merely
the populaton within reproductive ages. The rates could be thusly recalculated,
for instance using female ages $13-50$ and male age $15-65$, and would look like 
Figure~\ref{fig:eSFR2009limits}, in some instances a more reasonable if less
intelligible result\footnote{Rate surfaces based on $e_y$-specific fertility
data are calculated under a variety of reproductive spans in
Appendix~\ref{Appendix:reprospans}}.

\begin{figure}[ht!]
        \centering  
          \caption{Male and Female $e_y$-specific fertility rates, 2009, USA and
          Spain, with exposures redistributed using only female ages $13-50$ and
          male ages $15-65$}
           % figure produced in /R/Parents_ex.R
          \includegraphics{Figures/eSFR2009limits}
          \label{fig:eSFR2009limits}
\end{figure}

Comparing Figures~\ref{fig:eSFR2009} and~\ref{fig:eSFR2009limits} reminds of the
comments of \citet{gupta1978alternative} and \citet{mitra1976effect} on the difficulty of
defining an \textit{effective} population for use in exposures. Clearly, persons
outside the reproductive age range will conventionally be excluded from
exposures. Other kinds of risk heterogeneity are known to exist, such as age
patterns in fecundability, contraceptive use and sexual intercourse, that are
unaccounted for in standard fertility measures. 

With no claim of superiority over the more
restrictive exposures used for Figure~\ref{fig:eSFR2009limits}, we will proceed
in this section by using exposures derived from all ages. One could weakly
defend this choice by noting that we are attempting to measure the period
reproductivity of an \textit{entire} population, not just part of it. The
reproductive span was an outcome of evolution, varies greatly between
individuals and populations, and is mutable, both due to ongoing
population-level genetic, nutritional and hormonal changes and direct human
intervention. We will for the time being, be content to work with the cruder $e$SFR, and note
that this rate, as any other, is amenable to further disaggregation and
decomponsition.

\begin{figure}
        \centering
        \begin{subfigure}
                \centering
                \caption{Male and Female $e$SFR surfaces, 1969-2009, USA}
                \includegraphics[scale = .8]{Figures/eSFRsurfacesUS}
                \label{fig:exSFRsurfUS}
        \end{subfigure}
        \begin{subfigure}
                \centering
                \caption{Male and Female $e$SFR surfaces, 1975-2009, Spain}
                \includegraphics[scale = .8]{Figures/eSFRsurfacesES} 
                \label{fig:exSFRsurfES}
        \end{subfigure}
\end{figure}

As is visible in Figures~\ref{fig:exSFRsurfUS}~and~\ref{fig:exSFRsurfES}, 
$e$-SFR has changed its level and undergone a gradual displacement over 
time toward higher $e_y$, an altogether propitious development
as concerns human altriciality. The interpretation of this displacement is
entirely different from that of postponement in ASFR. Observed fertility 
postponement should shift $e$SFR unfavorably to higher mortality 
levels (lower $e_y$ levels), however mortality improvements have tended to 
offset this effect, acting to move the curve to higher
remaining life expectancies. 

 \FloatBarrier