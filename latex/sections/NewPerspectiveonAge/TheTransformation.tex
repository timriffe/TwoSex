 \FloatBarrier

The steps required to carry out the present data transformation are conceptually
simple, and easy to implement once understood. From a given
population and year extract the $d_x$ column from the corresponding lifetable of
radix ($l_0$) equal to 1\footnote{If the lifetable was calculcated with a
different radix, then simply divide the $d_x$ column by $l_0$}. Note that in
this case the $d_x$ column sums to 1, and is therefore a proper density function. 
$d_x$ can now be thought of as the probability of dying in any given age from the
 perspective of a 0-year-old, according to the given year's mortality
 experience. It follows that the observed population of age 0 can be
 redistributed according to $d_x$ and interpreted either as the expected death
 counts in each future year
$t+x$, or more intuitively as the distribution of persons currently-aged 0 according 
to remaining life expectancy. This can be done similarly for age 1, by ignoring the 
mortality experience of age 0, and rescaling $d_x$ to
sum to 1, or more generally redistributing each age and then summing to
$e_x$-specific totals:

\begin{align}
\label{eq:dxredist}
P_{y} &= \int _{a = 0} ^{\infty} P_a \frac{d_{a + y}}{\int _{b
= a} ^{\infty} d_b\, \dd b} \;\dd a
\\
&= \mathbf{E}(D_{t+y}) \notag
\end{align}
where $P_a$ is the population of age $a$, $d_a$ is the
lifetable density function and $\mathbf{E}(D_{t+y})$ is the expected number of
deaths $y$ years after the present year $t$, also understood as a vector of the
current population, redistributed into categories of remaining life expectancy,
$P_{y}$, our newly reclassified data.

The function of this formula is not original, as
\citet{miller2001increasing} and \citet{vaupel2009life} made use of a similar
identity:
\begin{equation}
\label{eq:vaupelredist}
f(n | a) = \mu (a+n) \frac{l(a+n)}{l(a)}
\end{equation}
where $f(n | a) $ is the probability of dying $n$ years in the future given
survival to age $a$, and $\mu$ is the force of mortality.
\citet{miller2001increasing} used the formula to look at death distributions of
particular ages in projecting health expenditures.
Formula~\eqref{eq:vaupelredist} can thus be used to weight age-classified data
as well. When then integrated over age for a given $n$ is equal to
Equation~\eqref{eq:dxredist}.

Formula~\eqref{eq:dxredist} is more convenient when 
discretized\footnote{Formula~\eqref{eq:dxredist} is more convenient due 1) to lifetable 
close-out issues and 2) because only one column from the lifetable is required instead 
of 3 columns ($\mu_x$, $l_x$, $L_x$) in
Equation~\eqref{eq:vaupelredist} }, although both are equally valid.
Equation~\ref{eq:dxredist} is equivalent to:

\begin{equation}
P_n = \int _{a=0} ^\infty P_a \mu_{a+n} \frac{l_{a+n}}{l_a} da
\end{equation}
where $n$ is treated as $y$ in \ref{eq:dxredist}.
The use of either formula in the way presented in this section is to this
author's knowledge novel. While Equation~\eqref{eq:vaupelredist} has
been used for particular purposes \citep{miller2001increasing}, population
structured by remaining years of time as given by Equation~\eqref{eq:dxredist}, structured
by Miller's \textit{thanatological age}, is the notion
to be developed further.

