 \FloatBarrier

The steps required to carry out the present data transformation are conceptually
simple, and easy to implement once understood. From a given
population and year extract the $d_x$ column from the corresponding lifetable of
radix ($l_0$) equal to 1\footnote{If the lifetable was calculcated with a
different radix, then simply divide the $d_x$ column by $l_0$}. Note that in
this case the $d_x$ column sums to 1, and is therefore a proper density function. 
$d_x$ can now be thought of as the probability of dying in any given age from the
 perspective of a 0-year-old, according to the given year's mortality
 experience. It follows that the observed population of age 0 can be
 redistributed according to $d_x$ and interpreted either as the expected death
 counts in each future year
$t+x$, or more intuitively as the distribution of persons currently-aged 0 according 
to remaining life expectancy. This can be done similarly for age 1, by ignoring the 
mortality experience of age 0, and rescaling $d_x$ to
sum to 1, or more generally redistributing each age and then summing to
$e_x$-specific totals:

\begin{align}
\label{eq:dxredist}
P_{y} &= \int _{a = 0} ^{\infty} P_a \frac{d_{a + y}}{\int _{b
= a} ^{\infty} d_b\, \dd b} \;\dd a
\\
&= \mathbf{E}(D_{t+y}) \notag
\end{align}
where $P_a$ is the population of age $a$, $d_a$ is the
lifetable density function and $\mathbf{E}(D_{t+y})$ is the expected number of
deaths $y$ years after the present year $t$, also understood as a vector of the
current population, redistributed into categories of remaining life expectancy,
$P_{y}$, our newly reclassified data.

The function of this formula is not original, as
\citet{miller2001increasing} and \citet{vaupel2009life} made use of a similar
identity:
\begin{equation}
\label{eq:vaupelredist}
f(n | a) = \mu (a+n) \frac{l(a+n}{l(a)}
\end{equation}
where $f(n | a) $ is the probability of dying $n$ years in the future given
survival to age $a$, and $\mu$ is the force of mortality.
\citet{miller2001increasing} used the formula to look at death distributions of
particular ages in projecting health expenditures.
Formula~\eqref{eq:vaupelredist} can thus be used to weight age-classified data
as well. When then integrated over age for a given $n$ is equal to
Equation~\eqref{eq:dxredist}.

Formula~\eqref{eq:dxredist} is more convenient when 
discretized\footnote{Formula~\eqref{eq:dxredist} is more convenient due 1) to lifetable 
close-out issues and 2) because only one column from the lifetable is required instead 
of 3 columns ($\mu_x$, $l_x$, $L_x$) in
Equation~\eqref{eq:vaupelredist} }, although both are equally valid.
Equation~\ref{eq:dxredist} is equivalent to:

\begin{equation}
P_n = \int _{a=0} ^\infty P_a \mu_{a+n} \frac{l_{a+n}}{l_a} da
\end{equation}
where $n$ is treated as $y$ in \ref{eq:dxredist}.
The use of either formula in the way presented in this section is to this
author's knowledge novel. While Equation~\eqref{eq:vaupelredist} has
been used for particular purposes \citep{miller2001increasing}, population
structured by remaining years of time as given by Equation~\eqref{eq:dxredist}, structured
by Miller's \textit{thanatological age}, is the notion
to be developed further.

\paragraph{Accounting for improvement in mortality: }
As we are dealing with a \textit{forward-looking}\citep{sanderson2007new}
transformation of structure, the reader will likely intuit that our redistribution 
method will err toward pessimism to the extent that
future improvements in mortality are ignored. To take account of future
mortality improvements is already a projections mindset, and we have made no
claim that the this structure is a projection- Rather it is a speedometer, to
use the analogy of \citet{coale1972growth}. In any case, one best accounts for
future mortality improvements by applying assumptions about the rate of
improvment in the age-specific motality hazard, $\mu_a$, rather than directly
manipulating $d_a$, which is our redistribution vector. To illustrate by
example, take the population of infants and the full present vector of $\mu_a$. It seems fair that
that $\mu_0$, or something close to it, will apply to these infants, yet by age
5, $\mu_5$ will likely be too high, and certainly by age 50
this same static $\mu_50$ will be too high. For the sake of simplicity, let us assume that the
rate of improvement, $\iota$, applies equally over all ages and future years,
and is equal to about 0.3\% per year (conservative for some ages, liberal
for others). We would prefer to have a multiplicative factor, so we define
$\iota = 1-.003 = 0.997$. Then we modify $\mu_a$ for these 0-year-olds, in
the following way

\begin{equation}
\mu_a^0 = \mu_a \prod _0 ^\infty \iota 
\end{equation}

where the superscript indicates that we have done this for persons of age 0. Now
one converts the $\mu _a^0$ to the new $d_a^0$ and redistributes the infant
population accordingly. For higher ages, one calculates $\mu_a^x$ as

\begin{equation}
\mu_a^x = \mu_a \prod _x ^\infty \iota 
\end{equation}
always with the first age $a$ equal to $x$
Now, one needs a direct relation of $\mu _a$ and $d_a$, and for this we use the
HMD methodology, namely assuming that the average age at death in each age
interval is 0.5 (except for age 0, which uses the Coale-Demeny rule of thumb),
deriving death probabilities, $q_a$, followed by the survival function, $l_a$,
which can be differenced to arrive at $d_a$\footnote{This is the abbreviated
version. See the HMD Methods Protocol for the full version, which we have
functionalized.}. This must be done for each age, to create as many modified
$\mu _a^x $ vectors as there are ages, and then one applies the same
Equation~\ref{eq:dxredist}, swapping the $\mu _a^x$ vector as one iterates over
ages. Alternatively, one could just take some future evolution of $\mu_a$ from a
standard projection method, like the Lee-Carter method \citep{lee1992modeling},
finding the mortality trajectory expected for each individual, and converting
this to $d_a$, and then redistributing. For the rest of this dissertation we
will deal only with static period deaths distributions, and we leave the
thoughtful incorporation of mortality improvements into the method for future
work.
