 \FloatBarrier
\label{sec:exobsexpected}
First, note that the observed bivariate $e_x$-distribution of birth counts is
very nearly identical to the expected distribution\footnote{The expected
distribution is defined as in Equation~\eqref{eq:expected}}.
Figure~\ref{fig:US1970obsexpex} compares these two distributions for birth 
counts in the USA in 1970 (compare with Figure~\ref{fig:US1970obsexp}). 

\begin{figure}[ht!]
        \centering  
          \caption{Observed versus expected bivariate distribution of birth
          counts by $e_x$-distribution of parents, 1970, USA}
           % figure produced in
           % /R/Parents_exCross.R
           \includegraphics{Figures/ObservedvsExpectedBexey}
          \label{fig:US1970obsexpex}
\end{figure}

It is
difficult to see any difference between the two surfaces in
Figure~\ref{fig:US1970obsexpex}, however we can measure the degree of
separation, $\theta$\footnote{See Equation~\eqref{eq:coefdiff}. Recall that 0 signifies
perfect overlap and 1 signifies perfect separation between the two
distributions}, just as for age-classified births (Compare with
Figure~\ref{fig:Theta}). One provisionally concludes that $e_x$-matching of
parents, at least with this level of approximation, appears to be very close to
random\footnote{Confidence bands used in Figure~\ref{fig:TotalVarobsexpex}, as
elsewhere in this dissertaiton for difference coefficients, represent the
central 95\% of randomly generated $\theta$ values using monte carlo
simulations. The present case differs from earlier simulated confidence bands in
that age-clssified death counts and age cross-classified birth counts 
are first drawn from poisson distributions, with
observed counts taken as $\lambda$. $\mu_x$ is then derived from the randomly
generated $D_x$ using exposures from the HMD, and $d_x$ is derived from $\mu_x$.
The simulated $d_x$ is then used to redistribute the randomly generated
$B_{xy}$ distribution, which is then compared with its own expected
distribution, producing the random $\theta$.}. When compared using the
Kolmogorov-Smirnov test, in fact, one cannot under even the most generous level
of significance, conclude that these two observed distributions come from
different theoretical distributions.

\begin{figure}[ht!]
        \centering  
          \caption{Departure from association-free bivariate distribution of
          birth counts cross-classified by $e_x$ of mother and father. USA,
          1969-2010 and Spain, 1975-2009}
           % figure produced in /R/Parents_exCross.R
           \includegraphics{Figures/TotalVariationObsvsExpectedexUSES}
          \label{fig:TotalVarobsexpex}
\end{figure}

Since the bivariate distribution by mothers' and fathers' $e_x$ is so close
to random, one could very closely replicate the full cross-classified matrix 
given only the two marginal
$e_x$ birth distributions by applying Equation~\eqref{eq:expected}. 

\FloatBarrier