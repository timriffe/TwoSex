\FloatBarrier

The reader will have been quick to notice that the present pyramid lends itself
to the calculation of some simple proportions akin to those offered by
\citet[e.g.][]{sanderson2005average}. For instance, one may calculate the
proportion of the pyramid with remaining life expectancy less than some
threshold, such as 15 (that commonly chosen by Sanderson and Scherbov). Where
$P^T$ is the total population and $P_y$ is the population with exact remaining
life expectancy $y$, we calculate the proportion of the population with 15 or
less remaining years of life expectancy, $P_{y \le 15}$, as
\begin{equation}
P_{y \le 15} = \frac{\int _{j=0}^{y} P_j \dd j}{P^T}
\end{equation}

The results from this calculation will differ from the method called the
``Prospective Old-Age Dependency Ratio''(POADR) for two reasons. 1) POADR is
calculated by indexing the lifetable $e_x$ to some reference
lifetable\footnote{This is explained nicely in \citet{sanderson2007new}}. 2)
we have lost the age information necessary to cut off the total population
below age 20, although this could be accounted for with little trouble. In
doing so, however, we would lose the deaths of all those below age 20 that
would happen within the next 15 years. 

To better compare with the notion of prospective age, look to the example given
in \citet{sanderson2007new}: The authors give the example of comparing
individuals from 1950 and 2000 who each have an average remaining life 
expectancy of 30. These two individuals will 
evidently have had different chronological ages. Imagine that in 1950 $e_x = 30$ 
occured at age 40, and in 2000 at age 50. Then using 1950 as our standard, we 
could say that the 50-year-old in 2000 has a prospective age of 40, in the sense
of ``50 is the new 40''. 

This basis for the method is quite different than that presented here because
indexing is done via the $e_x$ column of the lifetable, which is a weighted
average. Specifically, it is the weighted average of all possible remaining life
expectancies, where $d_x$ are the weights. Chronological age is in this way
never dispensed with, and it is hard to imagine how, for instance, this method
could do a good job of producing a fully structured population, such as that
seen in Figures~\ref{fig:exPyrUS}~and~\ref{fig:exPyrES}. Our method instead
combines ages via $d_x$ so as to arrive at clean breakpoints of $y$. We do not
offer a thorough comparison between such measures, although this is a ripe
avenue for further work.

Another measure of population ageing that falls out of our method is the
Proportion of Life Lived (PLL)\footnote{Again, this idea owes to a
particular conversation with John MacInnes, upon which I formalized the
concept.}. Take, for instance the same 50-year-old man in 2000 with a remaining life expectancy of
30. This person will in the end on average live 80 years, and so has completed
$\tfrac{50}{80} = 0.625$ or 62.5\% of his life. We can refine this a bit further
by using the deaths distribution at ages 50 and higher in 2000; some 50-year-old
males will die at age 50, others at age 51, and so on up until the highest age.
With respect to each potential age of death, we may then calculate a proportion
of life lived, and proceed to take a weighted average of these. Formally, we
calculate PLL for the whole population as

\begin{equation}
PLL = \frac{\int _{y = 0} ^\infty \int _{a = 0} ^\infty \frac{a}{a + y}
\left[ P_a
\frac{d_{a + y} }{ \int _{b = a} ^{\infty} d_b\, \dd b  }\right] \dd a \dd y }{
P^T }
\end{equation}
where $a$ indexes age and $y$ indexes remaining years of life. For the US and
Spain, the time trend of this indicator for our two populations and our
relatively narrow range of years is displayed in Figure~\ref{fig:PLL}
\begin{figure}
      \centering
      % Figure made in PyramidStability.R
      \caption{Population proportion of life lived, PLL, US, 1969-2009 and
      Spain, 1975-2009.}
         \includegraphics{Figures/PLL}
      \label{fig:PLL} 
\end{figure}

This trend will be seen to agree roughly with those presented by other
\textit{forward-looking} indicators of ageing. PLL, however, is particularly
intuitive, requiring no involved examples to explain to non-demographers, and
the trend seen here is particularly clear and consistent. One could
calculate PLL in like manner for particular ages, age-ranges, or remaining-years
classes. This indicator will surely yield much lower levels if improvements
are accounted for as discussed in Section~\ref{sec:eximprov}, and uncertainty
may be introduced using the monte-carlo strategy outlined in Section~\ref{sec:structuncertainty}

This dissertation offers no further discussion of the
potential ageing indicators implied by the present structuring of population.
This and the preceeding two sections on uncertainty and the incorporation of
mortality improvements into the present redistribution method have been intended
primarily to placate what the author considered to have been the most likely
initial doubts. As one sees, there is ample room for improvement in all methods
thus far presented. As has been stated, for the remainder of this dissertation,
we will work with the simplest deterministic assumption of fixed period rates.

\FloatBarrier