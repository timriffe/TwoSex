 \FloatBarrier
\label{sec:Decompr}
The main aspects of vital rates that contribute to the two-sex problem have by
now been illustrated, as has the maginitude of the problem, both in terms of
intrinsic growth rates and incongruous predictions of births. The primary
factors contributing to differences in $r$ have been indicated as mortality, fertility
and the sex ratio at birth. This section takes the extra step of segmenting and
quantifying differences between the intrinsic growth rates $r^m$ and $r^f$ into
consituent parts for fertility, mortality and the sex ratio at birth. Breaking the 
components to the gap, a pure data exercise, enables us to
visualize how the two-sex problem (in terms of $r$) has evolved over time, and
lends to a better understanding of why we observe the gap in the first place.

The exercise carried out is as follows: Equation~\eqref{eq:lotkaeqSRB} has been
functionalized and applied to the US and Spanish data for males and females,
with $r$ estimated using the method of \citet{coale1957new}. The inputs to the
function are the mortality hazard, $\mu_a$, from which the survival function,
$p_a$, is derived internally using the Human Mortality Database Methods
Protocol \citep{wilmoth2007methods}\footnote{Indeed it makes no difference how
mortality is specified, as the sum of the components that contribute to the
sex gap in $r$ will always be the same. The age distribution of the mortality
component of the decomposition will, however, depend on whether the
mortality input is specified as $\mu_x$, $d_x$, $l_x$ or directly as $L_x$ (the
discretized Lotka formula requires lifetable exposures, $L_x$, instead of the 
lifetable survival function $l_x$). While we do not display the age pattern of
any of the decomposition components, decomposing based on $mu_x$ would be the
most comparable in this instance, since the hazards in each age are independent
of other ages, which is not the case for $d_x$, $l_x$ or $L_x$.}, ASFR, $f_a$,
and $\varsigma _a$, the proportion of fertility by age that is girls for females, boys for males.
Each of these inputs is separate for males and females, and thus equation
Equation~\eqref{eq:lotkaeqSRB} is evaluated twice, once for males and again for
females. Each evaluation will therefore produce estimates of the year $t$
instrinsic growth rates $r^m$ and $r^f$, and it is the gap between
these ($r^m$ - $r^f$) that we wish to decompose.

\begin{figure}[h!]
        \centering
        \begin{subfigure}
                \centering
                \caption{Components to difference in single-sex intrinsic growth
                rates ($r^m - r^f$), US, 1969-2009}
                \includegraphics[scale = .8]{Figures/DecomprUS}
                \label{fig:DecomprUS}
        \end{subfigure}
        \begin{subfigure}
                \centering
                \caption{Components to difference in single-sex intrinsic growth
                rates ($r^m - r^f$), Spain, 1975-2009}
                \includegraphics[scale = .8]{Figures/DecomprES}  
                \label{fig:DecomprES}
        \end{subfigure}
\end{figure}

The decomposition itself is performed using the
pseudo-continuous approximation outlined in \citet{horiuchi2008decomposition}. 
This method allows for arbitrary reduction of error in the decomposition, and
virtually arbitarary specification of the function itself (here our Equation
~\eqref{eq:lotkaeqSRB} but with $p_a$ a function of $\mu_a$) as well as the
number and variety of parameters the function assumes (here $\mu _a$, $f_a$ and $\varsigma _a$). This is ideal for
the present case, since the functional form of the Lotka equation decomposed
here is \textit{somewhat} novel, and specification of a unique decomposition
formula would be potentially tedious. Output from the decomposition is given as
vectors of age-specific contributions from sex-differences in $\mu
_a$, $f_a$ and $\varsigma _a$ to the observed gap, $r^m$ - $r^f$. The values of these age-specific contributions to the observed gap may be either negative or
positive, but always sum to the observed gap, with a small arbitrary
error\footnote{In the present case, the author has ensured that the error of
decomposition is negligible and trivial. This is indeed computationally
intensive, but leaves no room for doubt in the interpretation of results.}. We
do not explore the age-patterns to the contributions in $r^m$ - $r^f$, but
rather sum the age-vectors for each of the three components, yielding a total of
three components to the sex gap in $r$, one for mortality, another for
fertility, and a third for the sex ratio at birth itself. The exercise is
repeated for each year of data and summarized in Figures~\ref{fig:DecomprUS}
and~\ref{fig:DecomprES}.

Positive values in Figures~\ref{fig:DecomprUS} and~\ref{fig:DecomprES} reflect
component-specific contributions act in the direction of $r^m > r^f$, while
negative values act in the direction of $r^m < r^f$. The sum of the three
components in each year is equal to the total observed gap. 

These results offer lessons. The sex ratio at birth, as
expected, consistently acts in favor of $r^m > r^f$. While this effect varies
subtly over time, decreasing on average in both countries, it is rather
consistent when compared to fertility and mortality. Just the reverse, and 
also as expected, mortality has consistently worked in
favor of $r^m < r^f$. This effect has tended to decline gradually over time in
both countries studied\footnote{The author offers no prediction about whether or
not we will one day observe a crossover in the mortality component to working
in favor of $r^m > r^f$, but such an observation would indeed be consistent
with the direction of change observed over the period studied in both the US
and Spain.}. 

The fertility component sheds more light on the observed gap than
the other two factors, as its direction of influence has been ambiguous, almost
sinusoidal in nature. One notes that in Spain, fertility contributed to $r^m >
r^f$ in the very years that the secular trend in fertility dropped to its lowest 
levels (as measured, say, by the trend
in TFR in Figure~\ref{fig:TFRseries}). In the US, fertility contributed to $r^m
> r^f$ until 1987, and has worked in favor of females since then. The current
trend would predict a neutral effect of fertility in the US by around 2020.
Indeed male and female fertility rates are calculated on the basis of the same
total number of births, and thus differences in rates are due primarily to the
interaction between the fertility distribution and differences in
exposure\footnote{i.e. if one measures the \textit{level} of fertility in
terms of total births, necessarily shared between males and females.}. One notes
that the decomposition could in this way continue ad infinitum, since 
observed exposures are the result of past fertility, mortality and sex
 ratios at birth. Indeed, an interactive two-sex model would also have fertility rates
themselves a function of exposures.

From these trends it should be clear that:
\begin{itemize}
  \item There are factors that work in favor of $r^m
> r^f$ and vice versa, and others that are ambiguous.
  \item The balance of these factors is dynamic.
  \item The sign of the sex gap in $r$ is ambiguous.
  \item The often-observed male advantage in $r$
is not necessary, though males have a strong positive bias in the form of the
sex ratio at birth
  \item Fertility is the main factor that changes the sign of the gap.
\end{itemize}






