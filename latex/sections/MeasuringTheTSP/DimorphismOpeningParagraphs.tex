Divergence between single-sex population models has been shown to be a problem
of both theoretical and practical significance for demographers, and it stems from
the fact that vital rates almost always differ between the sexes.
This characteristic of human populations, sexual dimorphism in vital rates,
is manifest in all subfields of demography. In observed populations,
fluctuations in vital rates are constantly underway, and can either magnify or
diminish differences between the sexes. In theoretical population models,
dimorphism is typically limited to fertility and mortality rates. Fertility 
is a two-sex interactive phenomenon, but in practice 
population models, such as projections, do not often treat it as
such, instead opting for female dominance in fertility rates. Mortality is
always modelled separately for both sexes, and can be reasonably thought of 
as independent for each sex. 

% not sure
Even if male and female
fertility schedules were in agreement, mortality differences would also
cause divergence between male and female single-sex models. 

% sex ratio assumptions mainly relevant for female dominant projections,
% not divergence
The sex ratio at
birth (SRB) varies from year to year, as well as over age, but it is usually
treated as a global variable, uniform over age and time. For the USA this is 
a harmless assumption, but for Spain, 
assumptions of a constant SRB would have been far off the mark. After
(hypothetically) forcing agreement between male and female fertility and
morality, fluctuations in the SRB would also lead to long run divergence.


\todo{rework the above paragraphs}