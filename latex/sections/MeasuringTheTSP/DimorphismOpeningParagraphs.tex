 \FloatBarrier
Divergence between single-sex population models has been shown to be a problem
of both theoretical and practical significance for demographers, and it stems from
the fact that vital rates almost always differ between the sexes.
This characteristic of human populations, sexual dimorphism in vital rates,
is manifest in all subfields of demography. In observed populations,
fluctuations in vital rates are constantly underway, and can either magnify or
diminish differences between single-sex intrinsic growth rates (or predicted
births). In population models, dimorphism is relevant as
it pertains to fertility and mortality rates, as well as the sex ratio at birth.

This section is exploratory and descriptive in nature. We seek here to
demonstrate 1) major differences between male and female rates and 2) the fact
that these gaps can and do change over time. We only touch upon rates that might
be relavant to the two-sex problem. The subsequent section~\ref{sec:dimorphASFR}
will quantify the contribution of the vital rates treated here to the size of the two-sex
problem.
