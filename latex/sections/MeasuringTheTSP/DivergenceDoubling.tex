
Differences in intrinsic growth rates are the essence of divergence in
stable populations, but these do not necessarily represent divergence in
projections, per se. Figure~\ref{fig:rSRdoubling} gives a more intuitive idea of
the magnitude of divergence implied by the vital rates in each studied year. The following
 exercise is carried out: Given each year's male and
female vital rates, how many years would it take for the total population of one
sex to be double the size of the other, always using the year $t$ population as
the initial conditions?\footnote{These figures were determined using projections
based on the two single-sex Leslie matrices that characterize male and female
vital rates each year.}

\begin{figure}[ht!]
        \centering  
          \caption{$log(\mathrm{Years})$ until one sex is twice the size as the
          other, given separate single-sex projections using annual vital rates and initial
          conditions, Spain and US, 1969-2009}
           \quad
           % /R/rm_rf_divergence
           \makebox[\textwidth]{\includegraphics{Figures/rSRdoubling}}
          \label{fig:rSRdoubling}
\end{figure}

Clearly the run of years in the United States where $r^f$ and $r^m$ were very
close (approx 1994-2001) imply such slow rates of divergence that we could, as a
matter of accident, safely ignore the two sex problem in those years. These
tended to be the same years where the greater growth rate oscillated between
male and females. However, any acceptability threshold is a matter of
convenience and taste: presumably the demographer would like age-specific 
population estimates to be much closer to truth than \textit{half} or \textit{twice} the ideal value.
Dropping the badness threshold would of course decrease the waiting time until
it is met in any given year. These are practical questions. More
stringent are the demands of theoretical stable populations, where
sex consistency is very desirable. Not a single year of data presented here
meets the requirements of a consistent stable population, and even if this were
to be observed, it would be coincidentally rather than essentially so. 
