 \FloatBarrier
\label{sec:divbirth}
Aside from divergence in the characteristic growth rates of the single-sex
stable models, single-sex separation is amenable to observation in the everyday
practice of demography. At the root of the two-sex problem is that the total
numbers of births predicted by male and female rates ought to, but never do,
agree, aside from in the jump-off year from which rates are initially derived, 
which is a tautology. Let us therefore design the following practical excercise:
Given the fertility rates of the present year $t$ and known exposures for
future years, both separate for males and females, how many total births do we
predict in $n$ years, where $n$ is equal to $[1, 5, 10, 15]$ based on male
versus female inputs? Figure~\ref{fig:BirthCountDivergenceAge} displays the
results of this exercise, where the value plotted in the relative difference
between total births predicted by male rates versus total births predicted by
female rates, divided by the average of the two predictions.

\begin{figure}[ht!]
        \centering  
          \caption{Relative difference (male - female) between predicted total
          birth counts in year $t+n$ based on year $t$ fertility rates and year $t+n$ exposures, US and Spain, 1969-2009.}
           % figure produced in
           % /R/BirthCountDivergenceAge.R
           \makebox[\textwidth]{\includegraphics{Figures/BirthCountDivergenceAge}}
          \label{fig:BirthCountDivergenceAge}
\end{figure}

Predicting births in year $t+1$ appears to entail a 1\% discrepancy in some
cases. In the first years for the US, the $t+15$ prediction (predicting
1984 births with 1969 rates) already entailed a 12\% relative difference
between the sexes ($B^M > B^F$), with separation between $t+15$ predictions
steadily falling over time. For Spain, $t+15$ predictions started (predicting 1990
births with 1975 rates) with little disagreement, but this has steadily grown to be as high
as 12\% in recent years.

Discrepancies illustrated here are net of observed secular changes in
fertility over time. That is to say, the relative differences in
Figure~\ref{fig:BirthCountDivergenceAge} are not prediction errors, but rather
the differences entailed betwen hypothetically choosing female or male
dominance. The short projection horizons tested here are well within the range
of horizons that demographers typically evaluate, and the magnitude of
discrepancy revealed here should give pause, even to the most ardent defender
of female dominance. The divergence of single sex
models has now been demonstrated for recent years in the US and Spain. Other
populations and years will show similar patterns, perhaps greater or lesser, as
the kind of divergence illustrated here is inherent in single-sex
population models.

