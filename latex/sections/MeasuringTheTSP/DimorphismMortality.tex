 \FloatBarrier
Sexual dimorphism in mortality is of primary significance to human reproduction.
Parents must survive in order to parent, and children must survive in order to
become parents. This later element, survival until reproductive ages, enters
directly into summary indicators such as the intrinsic growth rate, NRR, or
Fisher's reproductive value. Thus, differences in survival will also account for
part of the gap between male and female reproductivity, and are worth summarizing 
in light of sex-differences
that may contribute to sex divergence in population models.

Life expectancy at birth, $e_0$, is a synthetic indicator interpreted as the
average years that would be lived by the members of a cohort of individuals if
the mortality conditions of the present year were to be repeated each year until extinction of
the cohort. Sexual dimorphism in vital rates is commonly summarized using the
gap in $e_0$, in this case $e_0^F - e_0^M$, since female life expectancy tends
to be higher. For the data used in this dissertation, the gap is as displayed in
Figure~\ref{fig:e0gap}

\begin{figure}[!ht]
  \centering
    \caption{Gap between female and male life expectancy. USA,
    1969-2010 and Spain, 1975-2009.}
     % figure produced in /R/IllustrateDivergence.R
     \includegraphics{Figures/e0gap}
     \label{fig:e0gap}
\end{figure}

The gap in Figure~\ref{fig:e0gap} is amenable to various kinds of decomposition.
Such analyses have been done for Spain by age and cause \citep[pp 217-218 and
447]{amand2007thesis}. Blanes shows that over the period studied, the main cause-of-death components to the
gap in Spain were roughly the same over the period studied: cancers, heart
and circulatory disease, and lung disease. The hump in the middle of
Figure~\ref{fig:e0gap} for Spain is due to male excess mortality from external
causes, traffic accidents, ``other malignant tumors'', and AIDs, but it has
since declined. The gap component due to congenital conditions has also
decreased steadily over the entire period studied. Ages 40-80 account for the
majority of the gap over the whole period.

For the US, the components to the sex-gap in $e_0$ have also broken down rather
consistently over the period studied.\footnote{We have done our own age-cause
decomposition of the components to the difference in life expectancy at birth using the method of
\citet{andreev2002algorithm}, but we will not occupy too much space with this
in the dissertation. This method requires survival functions and age-specific
remaining life expectancies, as well as a matrix of the age-cause specific 
rates for males and females. All of these inputs were retrieved from
the Human Mortality Database (HMD) for the years 1970, 1980, 1990, 2000 and
2008. Cause of death data are at the time of this writing not yet publicly 
released by the HMD. Thanks 
to Magali Barbieri for providing me with these data in advance for
purposes of this decomposition. These data are not provided with the
dissertation, but requests may be directed to the author.} Most of the sex gap
over the period studied is due to mortality differences between the ages 50 
and 80. Until 1980 the age-pattern was more compact and centered on ages 60-70, but this hump has since
then spread out over a wider age-range. The male penalty in infant
mortality has decreased over this period, but has not changed since
2000. The specific causes that penalize males in comparison to females are 
heart disease, which explained a full three
years of the sex-gap in 1970, but which dropped steadily to 1.5 years in 2008.
External causes have fallen steadily from 1.8 years in 1970 to 1.4 years in
2008. Malignannt neoplasms climbed from 0.8 years of the gap in 1970 to 1.2 in
1990, but have since fallen back to around 0.9 years of penalty. 

This is all to say that the sex differences in life expectancy at birth are due
to mortality differences over particular causes of death and age-ranges. These
components break down differently over time according to population health,
technology, interventions and other factors. The contribution of these
particular causes and age-groups to sex differences in overall reproductivity 
are complex, dynamic, and sometimes considerable. In Section~\ref{sec:Decompr}
we decompose the sex-gap in intrinsic growth rates, $r$, including a mortality 
component, but in that instance we do not break down vital rate components further into
particular causes or age-groups. 

Life expectancy at birth does not provide all of the information that would help
us break down the contribution of mortality to sex-differences in
reproductivity. Also relevant to reproduction is the shape of mortality, since
reproduction happens in particular ages. $e_0$ is just the sum of the survival
function when $l_0 = 1$, where the negative unit change in the survival function gives the death distribution, $d_x$, and the ratio of these gives the mortality hazard, $\mu _x$. $\mu_x$ does not depend on what happens in other ages, whereas $l_x$ and $d_x$ do. Thus, the
most informative age-decomposition of $e_0$ should be based on differences in
$\mu _x$, whereas the most informative comparison of the overall shape of
survival as it might pertain to the differences in measures of reproduction
will be of overlap in the $l_x$ or $d_x$ distributions. We examine
sex-differences in the deaths distribution using $d_x$ simply
because it already sums to 1, which makes for simpler use of
Equation~\eqref{eq:coefdiff}. Thus the proportion of the two distributions that
is not in common, $\theta$, is displayed in Figure~\ref{fig:dxtheta}.

\begin{figure}[!ht]
  \centering
    \caption{Difference coefficient between male and female death distributions.
    USA, 1969-2010 and Spain, 1975-2009.}
     % figure produced in /R/IllustrateDivergence.R
     \includegraphics{Figures/dx_dissimilarity}
     \label{fig:dxtheta}
\end{figure}

From Figure~\ref{fig:dxtheta} we see that for the US, as the sex-specific $d_x$
distributions have approached each other the sex-gap in $e_0$ from the previous
Figure~\ref{fig:e0gap} has also narrowed. Likewise, for Spain, these two trends
have been roughly, but not entirely, synchronized. The remainder of the gap, the
part not explained by the trend in Figure~\ref{fig:dxtheta}, will be due to the
particular ages in which $d_x$-differences were observed, as $e_0$ may also be
conceived of as the $d_x$-weighted average of the ages in which persons died. We
will not investigate further into the age contributions that have led to this
gap, but will be content for now to note that, in general, the
contribution of mortality to the magnitude of the two-sex problem in the US has
declined over the period studied here, while for Spain it tended to increase 
into the 1990s and has since tapered off. In the following we conduct
a proper decomposition of the gap in intrinsic growth rates that places the
present and preceding sections on dimorphism in vital rates into context.

 \FloatBarrier
