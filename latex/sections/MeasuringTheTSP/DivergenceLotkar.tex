 \FloatBarrier
 \label{sec:divlotkar}
%* all figs here produced in IllustrateDivergence.R
As mentioned, divergence in this dissertation refers to the exponentially
increasing distance between single-sex male and female populations that unfolds
when they are simultaneously projected into the future -- or virtually projected
in the case of characteristic stable populations. The magnitude of separation increases
exponentially because males and females obtain different intrinsic 
growth rates, $r$, that are extracted from Lotka's fundamental equation
\citep{sharpe1911problem}:

\begin{equation}
\label{eq:lotkaeq}
1 = \int _0 ^\infty e^{-ra}p_a m_a \dd a 
\end{equation}
where $p_a$ are age-specific survival probabilities, $m_a$ are age-specific
single-sex fertility probabilities,\footnote{i.e., where $F_a^F$ is female
age-specific fertility,  $m_a = F_a^{F-F}$, which is female fertility calculated
using only daughters in the numerator, $F_a^{M-M}$ for males.} and $r$ is
the growth rate to be estimated.\footnote{In this dissertation, $r$ (and 
variations of $r$) are always estimated by using the (modified) strategy proposed 
by \citet{coale1957new}. Where modified, the new process is always described in full. In the present
case, we use Coale's version.} By ``single-sex'' it is meant that $m_a$ may
be specified either as the fertility of girls born to mothers or of boys born to
fathers. \citet{yellin1977comparison} prove that divergence is to be expected, as 
forced agreement between the male and female versions of Equation~\eqref{eq:lotkaeq}
 would imply an overdetermined system. In any instance
where single-sex $r$ estimates differ, projecting separately will result in sex
ratios that either grow toward infinity in the limit if $r^m
> r^f$ or decline to zero if $r^m < r^f$. If the gap between rates is large, this happens
quickly; if small, divergence is slower. This is in either case a modeling
problem of practical significance, and the crux of the two-sex problem. 

Single-sex intrinsic growth rates, $r^m$ and $r^f$, can be 
estimated from data. In looking at time series of 
 growth rates (see Figure~\ref{fig:rmf}), observe that the sex-gap has varied
 over time, that the male rate is typically higher than the female rate (aided greatly 
by the sex ratio at birth), and that there have been crossovers in the USA: 
$r^f > r^m$ in 1994-1996, and again briefly in 2001. 

\begin{figure}[ht!]
        \centering  
          \caption{Male and female intrinsic growth rates, Spain and US,
          1969-2009}
           % figure produced in
           % /R/rm_rf_divergence
           \includegraphics{Figures/rmf}
          \label{fig:rmf}
\end{figure}

Perhaps even more curious are occasions when $r^m$ and $r^f$ have been on
opposite sides of zero, i.e., exponential growth and exponential decay at the
same time. In the USA, this has happened many times in the period studied:
1972-1973, 1990, 2004-2005, and again recently in 2008. In Spain rates were 
briefly on opposites of zero in 1981-1982, in the middle of a period of sharp
decline in fertility. In all of these cases male growth rates were positive
while female growth rates were negative. Note that this does not mean that 
\textit{observed} year $t$ natural growth rates were of opposite signs, but
rather the intrinsic rate that characterizes the male and female stable
population models.
Figure~\ref{fig:rmfGap} again displays the information of interest, the size 
of the gap between $r^m$ and $r^f$ over time.

\begin{figure}[ht!]
        \centering  
          \caption{Gap between male and female intrinsic growth rates, Spain and
          US, 1969-2009}
           % figure produced in
           % /R/rm_rf_divergence
           \includegraphics{Figures/rmfGap}
          \label{fig:rmfGap}
\end{figure}
\label{par:coalermrf}
\citet[p. 57]{coale1972growth} points out that when $r^m > r^f$, as was
typically the case here, multiplying male exposures at each age by a factor equal to
$e^{(r^m - r^f)T^m}$, where $T^m$ is the male mean length of
generation,\footnote{where $T^m$ can be estimated as
$\frac{log(R_0^m)}{r^m}$} will bring $r^m$ in line with $r^f$. Alternatively,
$r^f$ can be aligned with $r^m$ by multiplying female exposures by a factor equal
 to $e^{-(r^m - r^f)T^f}$. This works in reverse when $r^f > r^m$.

 \FloatBarrier
