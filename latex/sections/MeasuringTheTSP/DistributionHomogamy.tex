 \FloatBarrier
The age combination of the male and female fertility schedules from
any given year varies greatly from the distribution that would be expected if
age and mother and age of father were selected randomly according to the two
single-sex distributions.

The expected cross-classified age distribution $\textbf{E}(B(x,y))$, that which
we would observe on average if age-mixing were random, is defined as:

\begin{equation}
\label{eq:expected}
\textbf{E}\left[B_{xy}\right] = \frac{B_x B_y}{\int _{x = \alpha} ^\beta \int
_{y = \alpha} ^\beta B_{xy} \; \dd x \;\dd y}
\end{equation}
where $x$ stands for age of father and $y$ stands for age of mother.

\begin{figure}[ht!]
        \centering  
          \caption{Observed versus expected bivariate age distribution of
          parents, 1970, USA}
           % figure produced in
           % /R/ObservedVsExpectedBivariateBirthDistribution.R
           \makebox[\textwidth]{\includegraphics{Figures/ObservedvsExpectedBxy}}
          %\includegraphics{Figures/ObservedvsExpectedBxy}
          \label{fig:US1970obsexp}
\end{figure}

Visual inspection of surfaces of the observed and expected birth counts in
Figure~\ref{fig:US1970obsexp} confirms they are indeed quite different: The
observed surface shows a stronger homogamy-hypergamy pattern than the expected surface. 
How similar are the
observed and association-free $B_{xy}$ distributions to each other? Again, we
can use a dissimilarity index, and re-apply Equation~\ref{eq:coefdiff} to the
present data, where $f_1$ is $B_{xy}$ and $f_2$ is $\textbf{E}(B_{xy})$, both scaled 
to sum to 1. $\theta$ is constrained to fall between 0 and 1, where 1 indicates that the
two distributions are separate and 0 indicates identical distributions. In 1970 USA,
$\theta$ was equal to $0.47$, a value which could be understood to stand for the
degree of residual preference. Precisely, it is the proportion of these two
distributions that is not shared. 47\% is rather high-- it means that the 1970
heterogamy pattern is far from random. If we further decide that marginal
age-distributions are not to be taken for granted, then 47\% is a lower limit to
the departure from randomness.

Note that that age-preference is an imprecise label for the variety
of preferences that may actually lead to observed age-combination biases. For
instance, preferences may reflect a third variable (e.g. socioeconomic
in nature) that covaries with age differently for the two sexes, so as to give
the appearance of age preferences. Furthermore, as \citet{bergstrom1994sweden}
demonstrate, pair matching may just as easily occur as a function of individual
preferences for event (mating, marriage) timing coupled with relative
availability, which follows partly from cohort size. This is consistent with
\citet{bhrolchain2001flexibility}, who argues that age preferences for
mates are highly adaptive.

Despite this ambiguity in mechanisms behind age combination patterns, one can
create a rough index of the strength of hypergamy or homogamy, based on the
matrices represented in Figure~\ref{fig:US1970obsexp}. Giving equal reproductive
bounds to the birth count matrix $B_{xy}$ makes a square matrix, from which we
can separate the upper and lower triangles. Here, the lower triangle, $L$,
of $B_{xy}$ contains births due to age-hypergamous (father's age > mother's
age) parents and the upper triangle $U$ contains births due to
age-hypogamous parents. Thus, a simple measure of total hypergamy, $\widehat{H}$, 
can be taken as a ratio of the total births in $L$ versus $U$, or in shorthand 
$\frac{B_{x>y}}{B_{x<y}}$, excluding single-age exact homogamy on the matrix diagonal.

\begin{equation}
\widehat{H} = \frac{\sum L}{\sum U} 
\end{equation}

In this case, the $\widehat{H}$ will be calculated for the observed and expected
birth matrices. US data from 1970 yields and observed $\widehat{H}$ of $7.37$
versus an expected $\textbf{E}(\widehat{H})$ of $1.75$. The later value is
possibly much higher than one would suspect, given that the $\textbf{E}(B_{xy})$
is purged of association. It is due, as mentioned above, to differences in the
shape and span of male and female single-sex fertility. For reference, I
will call this ``structural'' or ``latent'' hypergamy, as opposed to the
residual, or excess hypergamy, which is the ratio of observed (total) hypergamy to
structural hypergamy. For 1970 US data, excess hypergamy is $4.21$ times higher
than structural hypergamy. While these types of values do not enter, per se, 
into any of the thus-far mentioned two-sex models, they characterize the 
population in a basic way, and aid in understanding macro-level patterns. 

Let us then calculate two times series, one for total difference,
Figure~\ref{fig:Theta}, and another for our three measures of hypergamy,
Figure~\ref{fig:HypergamyStrength}:

\begin{figure}[!ht]
  \centering
    \caption{Departure from association-free bivariate distribution. USA,
    1969-2010 and Spain, 1975-2009}
     % figure produced in
     % /R/ObservedVsExpectedBivariateBirthDistribution.R
     \includegraphics{Figures/TotalVariationObsvsExpectedUSES}
     \label{fig:Theta}
\end{figure}

The bivariate age-distributions for both countries were far from being
association-free over the duration of the period studied. Since around
1979, Spain has undergone a roughly constant approach toward what would be the
expected distribution of births, random with respect to age of
partner. Since the decline in the departure from randomness in Spain 
may also be seen as closing a gap, one could just as
easily transform the data as such and view the secular change as one of an
\textit{accelerated} approach toward randomness\footnote{i.e. One could see the
acceleration by taking the lofit of the trend in $\theta$ shown.}. The US
underwent a similar approach toward randomness from 1969 until around 1985,
since which time the trend has gradually trended upward. In recent years, the
departure from randomness in the US has been considerably higher than in Spain.

Developments with respect to our rough indicators of hypergamy have been more
consistent between the two countries, both of which have undergone nearly
monotonic declines\footnote{Or perhaps more clearly \textit{monotonic
non-increases}} in all three hypergamy indicators, save for the US since the mid
1990s, which has held constant. The greatest drivers of the larger downward trend
have been declines in excess hypergamy: those more imaginably a result of
behavior. In both countries, excess hypergamy is greater than latent hypergamy,
though it would appear that this observation may not hold forever. The author
 speculates that we may one day see a crossover, with latent hypergamy-- that
which is more or less a product of sex-differences in fertility distributions,
and which owes in part to evolved differences in the reproductive span-- obtaining a
greater proportion of total hypergamy than excess hypergamy.

\begin{figure}[!ht]
  \centering
    % figure produced in
    % /R/ObservedVsExpectedBivariateBirthDistribution.R
    \caption{Strength of Hypergamy, $\frac{B_{x>y}}{B_{x<y}}$, total, structural
    and excess. USA, 1969-2010 and Spain, 1975-2009}
    \includegraphics{Figures/StrengthHypergamy}      
    \label{fig:HypergamyStrength}
\end{figure}

These trends, of substantive interest in their own right, will also be of
interest to the designer of two-sex reproductive models that incorporate
 assumptions about age-mixing. Such models have been known to make all
 manner of assumptions, from the simplicity of fixed age-matching, to
 sophisticated combinations of age-preferences interacting with availability
 conditions. Even the latent hypergamy indicator of
 Figure~\ref{fig:HypergamyStrength} does not contain information about how much
 of observed change is due to preference, say, in the age at childbearing, or
 relations between males and females with respect to the timing of childbirth.
 Nonetheless, it should be clear that the bivariate distribution with respect to
 ages of progenitors is far from random and often in a state of flux.
