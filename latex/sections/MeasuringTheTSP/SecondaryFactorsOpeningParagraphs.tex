Three factors that virtually always incede in two-sex models have
thus far been described and quantified for the two case-studies of Spain and
the United States: Fertility, mortality (survival) and the sex ratio at birth.
The degree to which these factors are pertinent also depends upon model 
specification. The previous decomposition exercise was based on a particular
model specification, the most simple design that is consistent with
established stable population theory and that incorporates our factors of
interest. 

Many proposed two-sex models make assumptions about age mixing between
mates as well as inter-age competition for mates. Let us loosely 
label such modelling considerations under the umbrella concept of
age-heterogamy. The lable is loose because the present
discussion does not deal with nuptiality, but rather directly with fertility.
The author will prefer to link the two concepts (fertility and nuptiality) via
the less binding concept of mating. Nuptiality, for this author, serves as a
statistical proxy for mating, and fertility is the result of presumed mating. No 
statistical analysis on the basis of
marriage data nor models that incorporate marriage as an intermediate state are
offered, per se, despite the fact that marriage and two-sex models have
been co-developed and for some are synonyms. To the extent that mating or
\textit{*gamy} enter into discussion in the paragraphs that follow, it is only
via inference from observed fertility patterns or as a rhetorical aid in interpreting observed fertility patterns.

Models may incorporate patterns of heterogamy along a
broad spectrum ranging from rigid, assuming a fixed age separation between mates--
 as in \citet{cabre1997tortulos}, \citet{karmel1947relations} or \citet{akers1967measuring}, 
 typically 2 or 3 years-- to flexible, which reaches its apogee in agent-based
modelling\footnote{The author claims this not because ABMs are more
sophisticated, but because aggregate-level patterns of mating in such models are
the result of potentially simple individual-level actions, which may not
necessarily follow an easily definable functional form or distribution.}.
Intermediate model varieties include those of \citet[e.g.]{gupta1972two} or
\citet{schoen1981harmonic}, which include either fixed matrices of age
combination distributions or a standard functional forms. Many model varieties
follow a similar strategy.  

The benefit to incorporating assumptions about age \textit{combinations} of
potential mates is that one need no longer assume that the marginal
distributions of male and female fertility are constant, but rather that
they adjust in some way to the relative abundance of mates in different
age-classes and/or to competition from other ages. Models can assure
that male and female marginal rates are in agreement to the extent that 
the same numbers of births are always predicted, but shift the
compromise (if any) between male and female rates to the less well-scrutinzed arena of
age-age-specific rates. Note that in this case, the model still holds something
constant: either a particular age-combination pattern, an exposure-dependant
mean function between constant sex-specific age-age-rates, or some other governing rule that finds
compromise. Marginal fertility distributions under such models-- models that
incorporate feedback into rates from changing population stocks--, as the
weighted average of age-age-specific rates, may change over time, but still 
be consistent with the condition of constancy of stable populations.

Two-sex models that contain such feedback are capable of either approaching
stability in the same sense as single-sex models- at which time marginal
distributions indeed become constant, or entering into a fixed
cycle or a cycle that gradually diminishes with time\citep{chung1994cycles}.
This author conceives of fixed cycles as another form of stability, dynamic 
stability. The present thesis does investiage 
this issue, that of feedback cycles, further, nor does it attempt to
quantify the potential affects of the exploratory analysis of age-matching that
follows. It is hoped that the present section will provide
occassion for empirically-based reflection on the appropriateness of
constant age-heterogamy assumptions in two-sex models. We will see that patterns
of age heterogamy have at time undergone sharp changes, and at other times have
held constant.

