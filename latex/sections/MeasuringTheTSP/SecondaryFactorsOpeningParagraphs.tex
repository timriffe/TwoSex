Three factors that virtually always incede in two-sex models have
thus far been described and quantified for the two cases treated in this
dissertation: Fertility, mortality (survival) and the sex ratio at birth. The
degree to which these factors are pertinent also depends upon model
specification. It is namely the case that many proposed two-sex models make
assumptions about age mixing between males and females as well as inter-age
competition for mates. Let us broadly label such modelling issues under the
umbrella concept of age-heterogamy. Models may incorporate patterns of
heterogamy along a broad spectrum ranging from rigid, assuming a fixed age
separation between mates-- as in \citet{cabre1997tortulos},
\citet{karmel1947relations} or \citet{akers1967measuring}, typically 2 or 3
years-- to flexible, which reaches its apogee in agent-based
modelling\footnote{The author claims this not because ABMs are more
sophisticated, but because aggregate-level patterns of mating in such models are
the result of potentially simple individual-level actions, which may not
necessarily follow an easily definable functional form or distribution.}.
Intermediate model varieties include those of \citet[e.g.]{gupta1972two} or
\citet{schoen1981harmonic}, which include either fixed matrices of age
combination distributions or a standard functional forms. Many model varieties
follow a similar strategy.  

The benefit to incorporating assumptions about age \textit{combinations} of
potential mates is that one need no longer assume that the marginal
distributions of male and female fertility are fixed, but may displace the
condition of \textit{stable} rates to this more flexible arena. \todo{revise,
expand on justification for section}






