 \FloatBarrier
 \label{sec:dimorphASFR}
 
It will later be seen that the effects of differential survival and the
sex ratio at birth on the maginitude of the two-sex problem are rather
consistent. This is not the case with fertility, which is, with respect
to the two-sex problem, volatile. To be explicit, fertility rates are in this
section (and previous sections) defined as births classified by age of
progenitor divided by person-years exposure classified by age of progenitor.
There are myriad ways to quantify fertility that demographers are well familiar
with. This section will only point out a few measures that are deemed by the
author to be relevant to the two-sex problem. Other factors that are known to
affect observed fertility, such as parity distributions, are not discussed. We
will biefly exposure differences between males and females as they pertain to
the maginitude and distribution of fertility rates. Magnitude is summarized in
terms of the total fertility rate (TFR), and much more attention is given to the
fertility distribution, which will be summarized by characterizing differences
in the age-pattern of male and female fertility, comparing the effective
age-bounds of male and female fertility, and creating a summary index of
distribution similarity between male and female fertility.

TFR is among the most well-known and understood demographic indicators, and all
demographers have intuition about how it has developed in recent decades. These
two statements are more true for female TFR than for male TFR, though the study
of male fertility is said to be on the increase in recent years.

 % TFR 1969- 2010
\begin{figure}[ht!]
        \centering  
          \caption{Male and Female Total Fertility Rates, 1969-2009, USA and
          Spain}
           % figure produced in /R/IllustrateDivergence.R
           \includegraphics{Figures/TFR}
          \label{fig:TFRseries}
\end{figure}

Figure~\ref{fig:TFRseries} shows in parallel the trends in male and female TFR
in the years studied for Spain and the US. Notes that
in the years of continuous decline, $TFR^M$ was higher than $TFR^F$, and in the years 
of gradual increase, $TFR^F$ crossed over to lead the way. In the United States,
this crossover was observed around 1988, and in Spain around 1998. 

% ASFR for 1975, both countries
\begin{figure}[ht!]
        \centering  
          \caption{Male and Female Age-Specific Fertility Rates, 1975, USA and
          Spain}
           % figure produced in /R/IllustrateDivergence.R
           \includegraphics{Figures/ASFR1975}  
          \label{fig:ASFR1975}
\end{figure}

\todo{add rate surfaces}

\todo{distribution overlap}

% ASFR bounds
\begin{figure}[ht!]
        \centering  
          \caption{Male and female fertility rate quantiles, 1969-2009, USA and
          Spain}
           % figure produced in
           % /R/IllstrateDivergence.R
           \includegraphics{Figures/ASFRbounds}
          \label{fig:TFRboundsseries}
\end{figure}

 \FloatBarrier