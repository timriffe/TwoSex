 \FloatBarrier
 \label{sec:dimorphASFR}
 
It will later be seen that the effects of differential survival and the
sex ratio at birth on the maginitude of the two-sex problem are rather
consistent. This is not the case with fertility, which is, with respect
to the two-sex problem, volatile. To be explicit, fertility rates are in this
section (and previous sections) defined as births classified by age of
progenitor divided by person-years exposure classified by age of progenitor.
There are myriad ways to quantify fertility that demographers are well familiar
with. This section will only point out a few measures that are deemed by the
author to be relevant to the two-sex problem. Other factors that are known to
affect observed fertility, such as parity distributions, are not discussed. We
will biefly exposure differences between males and females as they pertain to
the maginitude and distribution of fertility rates. Magnitude is summarized in
terms of the total fertility rate (TFR), and much more attention is given to the
fertility distribution, which will be summarized by characterizing differences
in the age-pattern of male and female fertility, comparing the effective
age-bounds of male and female fertility, and creating a summary index of
distribution similarity between male and female fertility.

TFR is among the most well-known and understood demographic indicators, and
demographers have intuition about how it has developed in recent decades. These
two statements are more true for female TFR than for male TFR, though the study
of male fertility is said to be on the increase in recent years.

 % TFR 1969- 2010
\begin{figure}[ht!]
        \centering  
          \caption{Male and Female Total Fertility Rates, 1969-2009, USA and
          Spain}
           % figure produced in /R/IllustrateDivergence.R
           \includegraphics{Figures/TFR}
          \label{fig:TFRseries}
\end{figure}

Figure~\ref{fig:TFRseries} shows in parallel the trends in male and female TFR
in the years studied for Spain and the US. Note that
in the years of continuous decline, $TFR^M$ tended to be higher than $TFR^F$,
and in the years of gradual increase, $TFR^F$ tended to be higher than $TFR^M$. In the United States,
this crossover was observed around 1988, and in Spain around 1998. 

% ASFR for 1975, both countries
\begin{figure}[ht!]
        \centering  
          \caption{Male and Female Age-Specific Fertility Rates, 1975, USA and
          Spain}
           % figure produced in /R/IllustrateDivergence.R
           \includegraphics{Figures/ASFR1975}  
          \label{fig:ASFR1975}
\end{figure}

The distribution of fertility rates over age also differs between males and
females. Figure~\ref{fig:ASFR1975} displays ASFR in 1975 for both Spain and the
US. The distributions have moved over time, but some stylized observations will
pertain in any year. Namely, the steep increase in fertility rates over young
ages follows a similar pattern for males and females, but begins some 4-6 years
later for males than for females in these two populations. Peak male fertility
will be around 7 years later than peak female fertility, and this spread widens
over the ages in which fertility declines, creating a longer and fatter
right-side tail for male ASFR than for female ASFR. 

% ASFR bounds
\begin{figure}[ht!]
        \centering  
          \caption{Male and female fertility rate quantiles, 1969-2009, USA and
          Spain}
           % figure produced in /R/IllstrateDivergence.R
           \includegraphics{Figures/ASFRbounds}
          \label{fig:TFRboundsseries}
\end{figure}

The physiological bounds to fertility, menarche and menopause for
females -- spermarche and andropause for males -- are well known. These may be
considered semi-rigid bounds. One might also derive bounds based on the ages
where fertility crosses some decided-upon threshold\footnote{i.e. take a
strategy similar to that proposed in \citet{coale1971age} for choosing the
starting age of marriage.}. Figure~\ref{fig:TFRboundsseries} displays the
results of choosing lower and upper bounds as those ages that contain 99\% of
all fertility, along with the median age\footnote{In other words, quantiles are
taken from the ASFR distribution, not observed birth counts. Non-integer results are derived from
discrete single-age ASFR by taking quantiles from ASFR after linear
interpolation between single-age midpoints, all assumed to be mid-interval.}. 
These statistical bounds fall within the physiological bounds, necessarily. 

In general, we note that the central ages of fertility have tended to shift more
over time than the upper and lower statistical bounds, particularly swiftly for
both males and females in Spain in the 1990s, though the upper bound for
Spanish males increased in parallel to the median over the same period. The
statistical upper bound applied here has been increasing in recent years for
both US and Spanish females, and by 2009 was about a half year higher than in
1969. The upper bound for Spanish females decreased about 2 years from 1975
to 1995, and has since increased to be just half a year lower than in 1975. Over
the period studied, median ages of ASFR have increased by around 5 years for
males and females in both countries. It is particularly noteworthy that Spanish
male and female mdeian ages and upper bounds diverged for much of the period
examined, much moreso than for the US.

One way to judge the overall dissimilarity of these two distributions is to
calculate a simple difference coefficient, $\theta$, namely:

\begin{equation}
\label{eq:coefdiff}
\theta = 1 - \int \;\int min(f_1, f_2)
\end{equation}
,where $f_1$ is male ASFR and $f_2$ is female ASFR, both scaled to sum
to 1. $\theta$ is constrained to fall between 0 and 1, where 1 indicates that the
two distributions are separate and 0 indicates identical distributions.
Figure~\ref{fig:ASFRdissimilarity} displays the results of applying this
indicator to each year of data for the US and Spain. 95\% simulated confidence
bands are presented, along with the direct estimate of
$\theta$. 

The indicator of overlap/divergence, $\theta$ is
used in several times in this dissertation. The author was unable to locate
an analytic solution for produce confidence estimates of this measure, but
some idea of variability presents complimentary information and may be of
interest. To approximate the level of uncertainty that might be present in the
data, the following procedure has been used: 1) Birth counts are drawn randomly
with replacement 1000 times from the poission distribution, with the parameter
$\lambda$ equal to the observed birth count. 2) Fertility rates are
re-calculated for males and females by dividing the simulated births by
exposures extracted from the HMD. 3) sex-specific ASFR is interpolated linearly
in age-steps of .01. 4) The 1000 interpolated ASFR series are each scaled to sum
to 1, and then compared (male vs female) using Equation~\ref{eq:coefdiff}, producing 1000 estimates of
$\theta$. 5) The represented condidence bands are the .025 and .975 quantiles
of the simualated $\theta$ distribution. This is the procedure used to represent
uncertainty in all later instances of this statistic as well, with
modifications noted accordingly. 

In the case of the US, confidence bands are in fact very narrow. $\theta$ has followed
a wave pattern in both the US and Spain in the years studied here, though 
quite differently between the two countries. US male and female fertiltiy rate
distributions are on the whole more similar than Spanish males and females. The
US underwent overall divergence until around 1980, then rates converged until
around 2003, since which time they have slowly begun to diverge again. Spanish
rates converged until 1980, then began to diverge until the early 1990s, since
which time they have begun again to converge. If simplistic visual biases are to
be given any weight, and without consulting other sources of information, one
might presume that male and female rates in both countries will begin to diverge
again over the next decade. However, it is unknown at this time whether the
longer pattern in this indicator would indeed be sinusoidal\footnote{Births by
age of mother and father are indeed available for a further 3 or so decades
before the start of this series, but these have not been converted to
useable data by this author.}.

\begin{figure}[ht!]
        \centering  
          \caption{Dissimilarity between male and female ASFR, 1969-2009, USA
          and Spain}
           % figure produced in /R/IllstrateDivergence.R
           \includegraphics{Figures/ASFRdissimilarity}
          \label{fig:ASFRdissimilarity}
\end{figure}

To reiterate, Figures~\ref{fig:ASFRdissimilarity}~and~\ref{fig:TFRboundsseries}
say nothing of relative levels of fertility between males and females, but
rather of distributions. These marginal distributions, will exert influence on
two-sex divergence even if all other factors, including TFR, are equal between
males and females. This is because fertility will be weighted differently along
the sex-specific survival curves. In the decomposition of the sex-gap in
intrinsic growth rates to be presented in a later section, we do not, however,
differentiate between fertility levels and fertility distributions, per se. 

 \FloatBarrier