 \FloatBarrier
Clearly another major factor contributing to divergence between the single sex
male and female stable population models will be non-unity sex ratios at
birth. Since sex ratios at birth are typically greater than one, ceteris paribus, males
are given a greater $l_0$. To a certain extent, this advantage in $l_0$ is
offset by greater attrition due to excess male mortality. In this way, effective
sex ratios in reproductive ages can be ambiguously greater than or less than 1,
depending both on the sex ratio at birth and on mortality conditions. The
single-sex Lotka Equation~\eqref{eq:lotkaeq} does not incorporate a third
variable for the sex ratio at birth, since we assume that rates can be
calculated separately by sex of birth. Equation~\eqref{eq:lotkaeq} could be
modified to incorporate such a variable, for instance, where $\varsigma$ is the
proportion male of births, $\tfrac{SRB}{1+SRB}$ -- $1 - \varsigma$ for
females -- and $m_a$ changes to either $f_a^F$ or $f_a^M$ to become either male
or female ASFR. For males, Equation~\ref{eq:lotkaeq} changes to:

\begin{equation}
\label{eq:lotkaeqSRB}
1 = \int _0 ^\infty e^{-ra}p_a^M \varsigma_a f_a^M \dd a 
\end{equation}
The female version is the same, with superscripts changed to $^M$. In
Equation~\eqref{eq:lotkaeqSRB}, the sex ratio at birth is not assumed to
be constant over age of mother or father, since SRB is known to decrease with
age, although in the stable population the total SRB does not change. Figure~\ref{fig:SRB1975}
demonstrates the age pattern (i.e., age of mother or father) for the US and
Spain in 1975.

\begin{figure}[ht!]
        \centering  
          \caption{Sex ratio at birth by age of progenitor, Spain
          and US, 1975}
           % figure produced in
           % /R/IllustrateDivergence.R
           \includegraphics{Figures/SRB1975}
          \label{fig:SRB1975}
\end{figure}

The age pattern to sex ratio at birth is susceptible to random
fluctuations. However, since the age-specific vector $\varsigma _a$ is summed
over age in~\eqref{eq:lotkaeqSRB}, these fluctuations are smoothed out, and in
fact results will be identical to those from \eqref{eq:lotkaeq}. That there is
an age pattern to the sex ratio at birth makes evident that the total sex ratio 
at birth is nothing more than the birth-weighted
average of the age-specific sex ratios at birth. Since in any projection, or virtual projection
(as in the case of the stable population model) the initial and final population 
structure will differ, one should not blindly assume or force a constant SRB valid 
for both the initial and stable states
if more information is available.\footnote{This latter condition was the basis
of the two-sex stable population model presented in \citet{mitra1982alternative,mitra1978derivation,mitra1976effect}, and is in the
opinion of this author an unreasonable condition.}

Aside from random fluctuations, especially evident in the oldest and youngest
ages, the age-pattern of SRB undergoes subtle changes over time. Further, there
are interactions in SRB by age of mother and age of father (the latter two
also being marginal distributions). These are aspects that may also be
considered if models rely upon fertility rates cross-classified by age of mother
and father. Therefore, to the extent that there is a trend over time in the SRB
(see Figure~\ref{fig:SRByears}), part of this will owe to changes in the
age-patterns of fertility.

\begin{figure}[ht!]
        \centering  
          \caption{Sex ratio at birth, US, 1969-2009 and Spain,
          1975-2009}
           % figure produced in
           % /R/IllustrateDivergence.R
           \includegraphics{Figures/SRByear}
          \label{fig:SRByears}
\end{figure}

Note that there has been a general downward trend in the SRB in both Spain and
the United States in the period studied. Spain has had a
higher\footnote{The difference between the US and Spain is also significant, not
shown.} SRB, peaking at over 1.09 in 1981,\footnote{These high figures for
Spain agree with tabulations from other sources, such as the INE itself, or the
Human Mortality Database. The spike around 1980 does not reflect the
preceding historical trend. There is ample evidence that such peaks in the SRB
are typical around wartime\citep{james2009variations}. The elevated levels of
domestic terrorism and counter-terrorism throughout Spain covary similarly with
this particular peak, and I speculate that the same mechanisms
that have been hypothesized for wartime SRB may have been behind this anomaly.} 
but falling ever since,
first precipitously then gradually. Since the population of Spain is smaller, the series 
is much more volatile, but the trend is nonetheless clear in both countries. It 
is particularly relevant to note that the assumption of a constant SRB of 1.05 in
 population projections in Spain would have been, and still would be, very far
 from observed values, and would affect the resulting population structure. This is 
 relevant not just for two-sex models,\footnote{Two-sex models are, however,
 especially advised to take special care with the SRB.} but also for standard female-dominant 
 projections, which treat males as a residual, splitting births based on some
assumption about the SRB.

This section is about dimorphism. The sex ratio at birth
falls in the domain of fertility, but is co-determined by unobserved mortality
(not treated here) because one of the determinants of the sex ratio at birth
must be sex-differentials in fetal mortality \citep{hassold1983sex}. This
variety of dimorphism is especially relevant for the ultimate sex structure of populations, since 
male and female survival curves are subject 
to differing radices (starting populations). For single-sex stable population
models, the male growth rate will necessarily be given an extra boost by SRB-inflated fertility
rates. This effect is separated in the decomposition presented in
Section~\ref{sec:Decompr}.
