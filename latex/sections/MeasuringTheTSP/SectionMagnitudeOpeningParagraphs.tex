
This section seeks to expose the maginitude of the two-sex problem. This will be
acheived in Section~\ref{sex:divlotkar} by measuring the gap between
male-specific and female-specific (canonical) intrinsic growth rates. Intrinsic growth rates are a
theoretical result-- an output of the application of stable population 
theory to data. If our treatment of the two-sex problem were limited to 
stable population theory, this would suffice. We will not, however, limit
ourselves to pointing out an inconsistency in an otherwise coherent and
self-contained set of mathematical abstractions. 

Applied demography is concerned with the more practical
busniess of population projections. Here too we will briefly
expose the magnitude of the problem by summarizing results in two more tangible
ways: 1) Section~\ref{sec:ageSRdoubling} presents the results of carrying out
simultaneous projections of male and female single-sex populations to an 
arbitrary point of absurdity; 2) Section~\ref{sec:divbirth} displays the
results of the even simpler task of projecting births at fixed time intervals
and measuring the size of the discrepancy between male and female predictions.

In this way, we summarize the major discrepancy in terms of an exponential
growth parameter, a waiting time, and a relativized count.

