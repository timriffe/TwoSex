\label{ch:Measuring}
The purpose of the present chapter is to describe and quantify the two sex
problem, both as a whole and in terms of its constituent parts. Purely
mathematical treatments of the two-sex problem have often been content to 
prove (or point out that it has been proven) that males and females,
if modeled separately, will obtain different growth rates, which leads to
absurd and inacceptable results. A notable exception is the early analysis in
\citet{karmel1948analysis}, which complements analytical work of the same author
\citep{karmel1947relations,karmel1948relations} by examining many of the vital
rate components to the two-sex problem in populations of that time. For a
complete historical and theoretical motivation for why demographers in general
ought be concerned with incorporating information from both sexes in the measurement of reproduction
and practice of demography in general, one need look no further than
\citet{francisco1996dissertation}. We here complement this brand of analysis
with a further decomposition of each source of the aggregate sex-gap.

Models that include both sexes must produce a single growth 
rate if they are meant to reflect observed human population renovation. 
This is true in the same way that mathematical identities are true, and to 
point this out, or to reproduce one of the proofs of the two-sex problem,
\textit{may} also have sufficed for the present dissertation. Here the aim is to produce intuition 
about the size and nature of the two-sex problem, and this will be
achieved by appealing to data. This intuition will tell us whether the 
problem is then trivial or worth accounting for in
population models. The conclusion will be that yes, it is usually worth our
while to account for the balance of sexes in projections and in models of human
population growth. In the scant instances where the two-sex problem would have
been trivial, the demographer incurs no penalty in accounting for it
nonetheless, and so it is advised to account for it.

The first task will be to measure the two-sex problem. This will be done in
three ways. 1) By calculating intrinsic growth rates separately for the sexes.
The gap between male and female growth rates determines the ultimate speed of
divergence between the males and females. 2) By projecting each sex separately
in order to estimate how many years would need to pass before one sex grows to
twice the size of the other sex. If the answer is a few decades, then this is
grave indeed, and if it is a few millenia, then we might not worry about the
two-sex problem in modeling. 3) By simply comparing predictions of births using
male versus female rates. The size of discordance between predictions of total
birth counts also serves as a measuring stick.

Having illustrated the magnitude of the problem, we will explore the primary
causes for the two-sex problem, namely sex-differences (dimorphism) in the vital rates
that determine population growth. Specifically, these include fertility, the sex
ratio at birth, and mortality. We present time series of these phenomena and
briefly describe the main respects in which males and females differ, to the
extent that is relevant in understanding the foundations of the two-sex problem.
We also illustrate how dimorphism has changed over time. The vital rates
used to estimate natural growth undergo changes, at times in different ways for
males and females. Outlining these changes makes clear that the nature and
composition of the two-sex problem also changes over time.

The presentation of dimorphism is followed by an explicit decomposition of the
gap between male and female growth rates into components due to fertility,
mortality, and the sex ratio at birth. This analytic exercise tells us the
weight that each relevant element of the sexual dimorphism in vital rates has
had in the two-sex problem. We see that the interplay between vital rates
in determining the size and direction of the sex-gap in intrinsic growth rates
is complex and inconsistent. Sex ratios consistently give males a head start in
growth rates in these two populations. This is offset slightly, but not
entirely, by female advantages in survival. The size and direction and of the effect of
fertility has changed dramatically over time.

Finally, further analysis and speculation is offered in how age-interactions may
also affect the size and nature of the two-sex problem. This section is more
suggestive than definitive in nature. However, such considerations are relevant
to two-sex models to the extent that age-interactions are allowed for or
controlled for. It will be shown that joint age distributions are very far
from random, that these distributions change over time, and that the degree of
age-hypergamy in fertility has changed over time. This paints a more complex
picture of fertility change than is visible by looking merely at marginal
distributions of age-specific rates.









