
The purpose of the present chapter is to describe and quantify the two sex
problem, both as a whole and in terms of its constituent parts. Purely
mathematical treatments of the two-sex problem have often been content to 
prove (or point out that it has been proven) that males and females,
if modelled separately, will obtain different growth rates, which leads to
absurd and inacceptable results. Models that include both sexes must produce a
single growth rate if they are meant to relect observed human population renovation. This is all
true in the same way that mathematical identities are true, and to point this
out, or reproduce one of the proofs of the two-sex problem, \textit{may} also have sufficed for the
present dissertation. However, via data we form intuition about
the size and nature of the two-sex problem. Here the aim is to produce such
intuition, and this will be acheived by appealing to data. This intuition will
tell us whether the problem is then trivial or worth accounting for in
population models. The conclusion will be that yes, it is usually worth our
while to account for the balance of sexes in projections and in models of human
population growth. In the scant instances where the two-sex problem would have
been trivial, the demographer incurs no penalty in accounting for it
nonetheless, and so it is advised to account for it-- somehow.








