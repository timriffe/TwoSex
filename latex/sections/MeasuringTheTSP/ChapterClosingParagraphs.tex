\FloatBarrier

The purpose of this chapter was to provide intuition into the nature of the
two-sex problem by means of data-driven illustration. Males and females differ
in the age patterns and levels of \textit{all} demographic phenomena that are
commonly used to gauge population reproductivity. These differences may partially reflect
differences in the evolutionary optimization of the life course, such as the sex 
ratio at birth and male excess mortality offsetting each
other. The magnitude of the effect of these two factors varies over time, but
the sign has proven consistent, at least for the two populations explored.
Fertility effects have been shown to be far less consistent. While differences
in the reproductive span, the age-bounds to reproduction, may also be attributed
to the relatively slow evolution of the life course, different locations on the
respective male and female marginal distributions are evidently malleable in
contemporary societies. Thus, we may observe inconsistent influence from the shape of
fertility on male versus female growth rates -- this, we note, was especially
the case for the Spanish population. Further, differences in the
overall level of fertility, which are separable from shape effects, will 
owe primarily to differences in male and female exposure levels, and hence will vary
from year to year depending on population structure, which is itself an outcome
of all manner of past demographic phenomena.

It is for this reason that fertility (marriage) balancing has been the primary
focus of methods intended to account for the two-sex problem in demography.
Fertility is the source of new generations in iterative population models, i.e.,
the starting point in a population model. One may conceivably, and will virtually
always in practice, conceive of male and female mortality as mutually
exclusive forces. Therefore, once a new cohort is produced in a population
model, the rest is taken care of by the respective sex-specific mortality
schedules. That is to say, no balancing is necessary for mortality schedules
because we have no obligation to maintain any particular population
proportion via mortality. This leaves the sex ratio at birth and fertility to
be thoughtfully dealt with in models, and this is the topic of the following
chapter.
