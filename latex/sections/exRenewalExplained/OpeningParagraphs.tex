 \FloatBarrier
\label{sec:exstructuredrenewal}
By now it has been demonstrated that the vital rates corresponding to an
$e_y$-structured population have a different overall shape and behavior from
those that belong to age-structured populations. This is because 1) $e_y$-classified rates
are calculated over the entire population, 2) $e_y$ fertility rates respond to
both fertility and mortality changes, 3) the underlying $e_y$-structured
population ranges close to its ultimate stable form, which means that
the effects of population structure are typically minor and almost never
abrupt\footnote{Wars, famines and other potential large-scale shocks do cause
abrupt changes to the $e_y$-structured pyramid, but such deformations usually
disappear within a year, as the  $e?y$-structured pyramid has very little
\textit{memory}.}. This later point will be demonstrated in greater depth later
in this dissertation.

Before proposing two-sex models that take advantage of the properties of
$e_y$-structured rates and populations, we must first define how to conceive of
reproduction under this new regimen of structure. Indeed, the basics are the
same as those for the age-structure Lotka system, as the only entrance into the
population is via birth, and the only exit is via death. What differs are the
distributions of the pertinent vital rates and population stocks, which indeed
will lead to a new formula for population growth. This new system will first be
presented so that it may then be expanded upon in the typical two-sex fashion
that lays at the heart of this dissertation. The present chapter provides an
overview, hopefully intuitive, for how population cycles through the
$e_y$-structured system. The following chapter on single-sex renewal formalizes
the ideas explained here.

First, note that much of what we know about age-structured populations has been
conditioned by our instruments of observation. Age is, nowadays in any case,
known by individuals, and is recorded by statistical aparatuses. Remaining life
expectancy is not recorded as such for individuals, but is rather calculated
based on age-classified data. So it is that data classified by remaining
life expectancy rely on age-classified data and not vice versa. The description of 
reproduction for population classified by remaining years will therefore
ocassionally borrow concepts from age-clsasified data. In particular, the deaths
distribution, $d_a$, is never fully precinded of, as it is essentially a direct
mirror of deaths classified by remaining years $d_y$, which is
iteratively derived from the former as in Equation~\eqref{eq:dxredist}.

Aside from $d_x$, one may conceive of reproduction in an $e_y$-structured
population without periodic reversion to the familar ground of age-structured populations.
Intuitively, imagine the two varieties of pyramid that correspond to the
(closed) population in question. 