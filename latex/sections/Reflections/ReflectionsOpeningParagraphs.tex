\FloatBarrier
\label{sec:reflections}

\begin{singlespace}
\begin{quote}
Wenn einer kratzt, wo es ihn juckt, mu{\ss} ein Fortschritt zu sehen sein? Ist
es sonst ekin echtes Kratzen oder kein echtes Jucken? Und kann diese Reaktion
auf die Reizung lange Zeit nicht so weitergehen, ehe ein Mittel gegen das Jucken
gefunden wird?
\citep{wittgenstein1984culture}
\end{quote}

English translation:
\begin{quote}
 If somebody scratches the spot where he has an itch, do we have to see 
 some progress? Isn't genuine scratching otherwise, or genuine itching 
itching? And can't this reaction to an irritation continue in the 
same way for a long time before a cure for the itching is discovered?
\citep{wittgenstein1984culture}
\end{quote}
\end{singlespace}

Many reflections and discussions about the findings of this dissertation have
been planted in the text where deemed appropriate. However, we attempt a
synthesis of the knowledge produced from this dissertation, and so expand on
where it might take demography as a whole. 

In this dissertation we aimed to investigate and compare models of
population growth. A guiding principle has been that population renewal models
should thoughtfully incorporate both sexes. We started by doing this in the
traditional way in Part 1, modeling populations structured by both age and sex. In
Chapter~\ref{ch:Measuring} we showed that the practical consequences of
neglecting the two-sex problem in fertility measurement and projections are
often non-negligible. The discrepancies between predictions/models based on male
and female rates result from a complex mix of the shapes (over age) and
magnitudes of all vital rates. In Chapter~\ref{sec:modelingapproaches} we showed
that the common two-sex adjustment methods designed to deal with these
discrepancies tend to produce very similar results in the aggregate despite
differences in properties. Fertility is best modeled as a function of male and
female fertility rates for both sexes of birth, and additional flexibility can
be gained when fertility information comes from the joint distribution of births by ages of males and females. The stable sex-ratio at birth need not be equal to the initial sex ratio at birth.

in Parts 2 and 3 we replaced age-structure with remaining-years structure
for population renewal modes and laid the groundwork for a demography
based on remaining-years structure. Part 2  In Chapter~\ref{ch:newpersp} we saw that any
age-structured demographic phenomenon can instead be structured in terms of
remaining-years of life by means of a simple transformation. The
remaining-years perspective is not new, but remaining-years structure is indeed
a new idea with implications for and beyond the practice of demography are many.
In Chapter~\ref{sec:exstructuredrenewal} we showed that reproduction in
populations structured by remaining-years of life can
summarized in a parsimonious single-sex renewal equation akin to the Lotka
equation for age-structured single-sex populations. Of course, the two-sex
problem nonetheless persists in populations structured by remaining years of
life, but the pace of divergence is usually slower than in the case of
age-structured populations, and the components to sex differences break down
differently.

In the Chapters 6 through 9 we showed that common two-sex
methods for age-structured populations are amenable to translation
to the remaining-years perspective, and that these maintain the same properties.
Further, remaining-years structured populations are usually more stable (in
different senses of the concept) than age-structured populations. Population renewal models structured
by remaining-years are incongruous with models structured by age, just as male and 
female single-sex models produce incongruous results. As a result, it is
possible for one and the same population to be both growing and shrinking
according to the renewal models of each perspective. We call this the two-age problem.


