\FloatBarrier
\label{sec:reflections}

\todo{Chapter under development!}

\begin{singlespace}
\begin{quote}
 If somebody scratches the spot where he has an itch, do we have to see 
 some progress? Isn't genuine scratching otherwise, or genuine itching 
itching? And can't this reaction to an irritation continue in the 
same way for a long time before a cure for the itching is discovered?
\citep{wittgenstein1984culture}
\end{quote}
\end{singlespace}

Many reflections on the findings of this dissertation have been planted in the
text where this seemed approriate. Here we attempt a synthesis of the
knowledge produced from the process of writing this dissertation and from the
dissertation itself.

What do we know that we did not know before?

The common two-sex methods for age-structured populations tend to produce very
similar results. Of the small set of methods presented in this dissertation, for
insertion into a population projection engine, we recommend either:

1) the weighted-dominance method of \citet{goodman1967age}, probably with a
dominance parameter of $0.5$, such that equal information is taken from male and female
weights. This method wins on ease of implementation, and it produces reasonable
reasults for population structures typically observed (i.e. without 0s in
reproductive ages). It is also containable in a static Leslie matrix, although
we did not explain this construction for age-structured two-sex populations. The
method has a drawback in that the model itself is less appealing, as it does 
not allow for proper interaction between the sexes, or between ages.

2) the mean method, with the mean either set to harmonic, logrithmic,
geometric or an unnamed mean in that approximate range and that has the property
of falling to zero if one sex is absent. This method is appealing because the
male and female marginal fertility rates for a given year are determined
dynamically by what is happening in each age-combination of males and females.


