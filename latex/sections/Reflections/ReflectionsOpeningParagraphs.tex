\FloatBarrier
\label{sec:reflections}

\begin{singlespace}
\begin{quote}
Wenn einer kratzt, wo es ihn juckt, mu{\ss} ein Fortschritt zu sehen sein? Ist
es sonst ekin echtes Kratzen oder kein echtes Jucken? Und kann diese Reaktion
auf die Reizung lange Zeit nicht so weitergehen, ehe ein Mittel gegen das Jucken
gefunden wird?
\citep{wittgenstein1984culture}
\end{quote}

English translation:
\begin{quote}
 If somebody scratches the spot where he has an itch, do we have to see 
 some progress? Isn't genuine scratching otherwise, or genuine itching 
itching? And can't this reaction to an irritation continue in the 
same way for a long time before a cure for the itching is discovered?
\citep{wittgenstein1984culture}
\end{quote}
\end{singlespace}

Many reflections on the findings of this dissertation have been planted in the
text where deemed appropriate. Here we attempt a synthesis of the
knowledge produced from this dissertation. We condense our primary findings in
the following.

In Chapter~\ref{ch:Measuring} we showed that the practical consequences of
neglecting the two-sex problem in fertility measurement and projections are often non-negligible. The discrepancy between 
predictions/models based on male and female rates are
due to a complex mix of the shapes (over age) and magnitudes of all vital rates. 
In Chapter~\ref{sec:modelingapproaches} we showed that the common two-sex
adjustment methods designed to deal with this disrepancy tend to produce very
similar results in the aggregate despite differences in properties. Fertility is
best modeled as a function of male and female fertility rates for both sexes of
birth, and further flexibility may be gained when fertility information comes
from the joint distribution of births by ages of males and females. The stable
sex-ratio at birth need not be equal to the initial sex ratio at birth.

In Chapter~\ref{ch:newpersp} we showed that any age-structured demographic
phenomenon can also be viewed in terms of remaining-years structure by means of a simple 
transformation. Remaining-years structure implies a perspective
change that directly yields new insights in and beyond the field of demography.
Reproduction in remaining-years structured populations may be summarized in a
parsimonious single-sex renewal equation akin to the Lotka equation
for age-structured single-sex populations. The two-sex problem
persists in populations structured by remaining years of life, but the pace of
divergence is usually slower and the components to sex differences break down
differently than in the case of age-structured populations. 

In the chapters of Part 3 we showed that common two-sex
methods for the case of age-structured populations are amenable to translation
in the remaining-years perspective, and that these maintain the same properties.
Remaining-years structured populations are usually more stable (in different
senses of the concept) than age-structured populations. Population renewal models structured
by remaining-years are incongruous with models structured by age, just as male and 
female single-sex models produce incongruous results. This means that it is
possible for one and the same population to be growing and shrinking according
to the renewal models of each perspective. We call this the two-age problem.
