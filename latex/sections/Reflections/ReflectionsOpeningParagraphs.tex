\FloatBarrier
\label{sec:reflections}

\todo{Chapter under development!}

\begin{singlespace}
\begin{quote}
 If somebody scratches the spot where he has an itch, do we have to see 
 some progress? Isn't genuine scratching otherwise, or genuine itching 
itching? And can't this reaction to an irritation continue in the 
same way for a long time before a cure for the itching is discovered?
\citep{wittgenstein1984culture}
\end{quote}
\end{singlespace}

Many reflections on the findings of this dissertation have been planted in the
text where deemed appropriate. Here we attempt a synthesis of the
knowledge produced from the process of writing this dissertation and from the
dissertation itself. The common two-sex methods for age- or remaining-years
structured models tend to produce very similar results. However, remaining-years
and age-structured models are also incongruous, just as male and female
single-sex models produce incongruous results. We discuss the complimentary definitions 
of age later. 

First we consider some practical aspects of two-sex models. As mentioned
previously, the two-sex problem comes to the fore when projecting population into the
future. The most common practice to avoid disagreement between the sexes is to
use only the age-specific fertility rates of females, which is just the same as
using the \citet{goodman1967age} model with the dominance parameter set to
accept 100\% of model information from female fertility rates. There are
very few projections that do not make this assumption. Assuming female dominance
in one way or another is so widespread that newcomers to demography often perceive 
it as a given, or in the worst case adopt the
practice dogmatically. For others, the implementation is too complex or
else the requisite transition rates are not available. Both of these later
two obstacles may be overcome by removing nuptial states from the model
altogether, as we have done in this dissertation. 

Formal demographers, however, have long taken the discrepancy seriously, at
least for purposes of consistent model design. These mathematical models
have been the driving force behind this dissertation effort. We have at
times simplified two-sex models from their original form, and one offshoot of
this choice is that these models may be of potential use in applied demography.
The \texttt{R} code used to produce our results should be recyclable or else may
easily be quarried to such ends. It therefore behooves us to recommend from
amongst the methods explored in this dissertation. In order to account for sex
divergence in projections or self-contained models, of those methods treated in this dissertation, 
we recommend from the following three adjustment strategies:

\begin{enumerate}
\item The weighted-dominance method of \citet{goodman1967age}, probably with
a dominance parameter of $0.5$, such that equal information is taken from male and female
weights is a reasonable choice. This method wins on ease of implementation, and
it produces reasonable results for population structures typically observed (i.e. without 0s in
reproductive ages). It is also containable in a static Leslie matrix, although
we only explained this construction for $e_y$-structured populations. The
method has a drawback in that the model itself is less appealing, as it does 
not allow for proper interaction between the sexes, or between ages, but for
purposes of age-sex population projections on a 10-40 year horizon, the simple
model is well worth further consideration in our opinion.

\item The mean method from Sections~\ref{sec:ageharmonic} and
\ref{sec:ex2sexschoen} for age and remaining-years structured models
respectively, is also a reasonable choice. In this case, we recommend
implementation with the mean either set to harmonic, logarithmic, geometric or
an unnamed mean in that approximate range and that has the property of 
falling to zero if one sex is absent. We opine that there is no point in
quibbling over which of these means is best, as observed human populations do
not exhibit the extreme sex ratios required to produce results that are
meaningfully different. This method is appealing because the male and female
marginal fertility rates for a given year are determined dynamically by 
what is happening in each age-combination of males and females, and the range
of means listed here allows for some degree of \textit{bottlenecking} due to
the minority sex. The model is also of parsimonious design, easy to understand,
and straightforward to implement.

\item The IPF method is the most flexible of these three, because it
incorporates inter-age competition and substitution. That the method is
iterative presents no real drawback, as any of the above methods is best 
implemented in a scripted language, and the method is
not perceptibly slower than the alternatives. The properties of IPF are easy to
demonstrate but difficult to prove, and so it has received less attention from
mathematicians and ODE modellers. For the demographer designing a projection
apparatus, IPF is nonetheless a convenient choice. This method is in need of
comparison with the recent contribution from \citet{choo2006estimating}, which
has similar properties.
\end{enumerate}

Each of these three methods has advantages and disadvantages, and it is up to
the demographer to evaluate the optimal choice for the particular projective or
modelling scenario. All three of these methods work equally well in
age-structured and remaining-years structured projections. Further, these three
adjustment techniques work just the same for projections that incorporate
assumptions about future developments in fertility. In this case, the
demographer makes assumptions about the male and female paths of fertility rate development and adjusts in each
iteration to force agreement in results. For the method (2) based on a mean of
age- (or remaining-years) specific combinations of male and female exposures,
the incorporation of sophisticated assumptions entails more care, as these must be
distributed over a matrix. If this proposition is inconvenient, one may prefer
either IPF or dominance weighting, which rely only on marginal fertility
distributions for adjustments beyond the initial year.

Further, we have demonstrated that mean-based (2) and IPF (3) methods each
entail changes to the marginal male and female rates after adjustment, and one
may wonder whether these adjusted rates have any predictive power per se-- for
instance adequately adjusting for foreseen changes in population structure--, or
whether they are a modelling artifact to be disregarded in favor of the total 
(unstructured) birth count. We have demonstrated that this feature exists,
and we have shown instances where the two methods make predictions that are at
odds. We do not follow up this observation with an empirical comparison
to check whether (2) or (3) hit the mark closer in terms of distribution
prediction. This therefore remains an intriguing question that could tip the
balance in favor of preferring one of these two methods over the other. A
priori, we expect IPF (3) to display more appropriate sensitivity amidst abrupt
changes in cohort size, but we do not know whether the magnitude of adjustment is
appropriate. 

Of course, the demographer may also compare with two-sex models not treated
in this dissertation, incorporate nuptial states into the model, and so
forth. In this case, the two-sex method is transferred to nuptiality
(match-making, pairing, marriage) as the event being predicted, but the
adjustment procedures are one and the same. Such a projection would
entail more sophisticated construction, more data inputs, and the incorporation of more hypotheses,
namely hypotheses about changes in marriage rates as well as changes in marital
and extramarital fertility. For populations with high proportions of
extramarital fertility, extra data are required to approximate the formation of
non-marital mated pairs, e.g. transitions into and out of cohabitation, as well
as fertility fertility rates that apply to this subpopulation (and mortality
rates if supposed different). This is to say that the addition of further state
considerations to fertility assumptions greatly increases the model complexity
and data requirements.

To the extent that fertility rates and the sex ratio at
birth vary along the path to stability, one may wonder whether any of the
\textit{interactive} two sex models are at odds with the notion of
rate invariance in stable populations. Once in the state of stability, of
course, both population structure and marginal male and female fertility rates are invariant, which
implies that the two-sex problem itself vanishes. In this case, for both the
mean and the IPF methods, the stable adjusted marginal fertility rates become
invariant, and the male and female rates yield the same results, making the
population tautologically \textit{dominance-indifferent}. In any of the
interactive models the element held fixed prior to stability are not rates, but
some standard. For IPF, the element held constant in our 
description is the original cross-classified birth matrix and the corresponding 
male and female marginal rates. For the mean-based method, one holds constant 
the standard rate matrix, as well as the mean function itself, but the marginal 
rates produced by these standards have been shown to indeed change over time
under these modelling assumptions.

This author has provided no proof that the stable rate of growth
attained in the various two-sex age-structured and remaining-years structured
models are unique, necessary or ergodic. There is a possibility that under some
real conditions the stabilizing trajectory arrives in a limit cycle, which this
author considers a particular variety of stability. \citet{wijewickrema1980weak}
and \citet{chung1990phd, chung1994cycles} explore the possibility of such cycles
and bifurcations in two-sex models, but this and many other dynamic
properties remain to be explored for remaining-years structured two-sex
populations. We have omitted any sensitivity analysis, although this would
enhance our ability to compare age-classified and remaining-years
classified models. We have in some cases measured the total amount of
oscillation in population structure between the initial and stable states, but
we have not examined its path. We have also not checked whether the end state
is truly ergodic or somehow dependent on initial conditions. These and most
other transient properties of the models presented in this dissertation have
been ignored, and are particularly ripe for exploration for the new family of
remaining-years structured models that we propose.

We have shown that populations may be projected while structured by
remaining years of life. We have also revealed that populations under these two
structures, calculated using data from one and the same population, even one and the same
sex, imply different results. Specifically, for the data used in this
dissertation, remaining-years structured data tended to imply higher
reproductive values than age-structured data($r$, $R_0$, TFR, and so
on). At first glance, one would expect the direction of difference to be
negative rather than positive because we took the simplest possible course of
redistributing population counts (or exposures) from all age classes rather than
only from age classes in ages typically identified as reproductive ages. The
outcome is positive because the age-ditributions of marginal birth counts and
population counts are different, but redistributed by the same lifetable death
distribution. We ought not expect differing intitial distributions to wind up
proportionally in the same remaining years classes. Hence, the sum of these
ratios (events to exposures) over remaining years is also different. TFR, for
instance, is higher for remaining-years structured data than the age-equivalent
measure because on average there is less exposure per birth in the newly
redistributed data.

One may question whether there is any sense in allowing non-reproductive ages
into remaining-years exposures, and to this we have three responses. The first
comes by manner of analogy to age-specific fertility distributions, which
have tails that are longer than most people feel comfortable imagining, 
both for males and females. For males and
females there are so-called central ages of reproduction and there are
less-common ages of reproduction. Even for ages typically measured, say
45-49 for females, not all persons captured in these exposures are truly
exposed to the risk of fertility, and everyone knows this. Indeed a minority
of females in this age group are truly at risk of fertility. This does not imply
that the rate calculated is invalid, but rather that the rates is\ldots low.
This is one reason why a large portion of fertility studies deal with the
measurement of the proximate determinants of fertility-- these
factors differentiate exposure for purposes of calculating more specific rates.
Age-specific rates are valid without such differentiation, as they help purge
our measurement of distortions from a certain degree of population structure. 

As \citet{stolnitz1949recent} so eloquently describe, demographic rates are
never fully purged of population structure. We may nonetheless \textit{de}structure in
any of myriad ways, and remaining-years structure is the way that we have shed
light upon in this dissertation. No claim is made about whether chronological or
thanatological age are the more efficient classifying variable for population
data, nor that one is more pure than the other. In sum, we have in this
dissertation provided yet another alternative structuring variable, and if one
questions the validity of a rate calculated on the basis of blended age groups,
one may ask: 1) how many persons are expected to die in $y$ years and 2) how
many births this year were to persons expected to die in $y$ years, and there we
have everything needed to calculate a rate. This is what we have done. Further
refinements are possible and are worthy of exploration.

The second response to the potentialempirical regularity

Third - awareness (find citations from Ryand Edwards)


