\FloatBarrier
\label{sec:reflections}

\begin{singlespace}
\begin{quote}
Wenn einer kratzt, wo es ihn juckt, mu{\ss} ein Fortschritt zu sehen sein? Ist
es sonst ekin echtes Kratzen oder kein echtes Jucken? Und kann diese Reaktion
auf die Reizung lange Zeit nicht so weitergehen, ehe ein Mittel gegen das Jucken
gefunden wird?
\citep{wittgenstein1984culture}
\end{quote}

English translation:
\begin{quote}
 If somebody scratches the spot where he has an itch, do we have to see 
 some progress? Isn't genuine scratching otherwise, or genuine itching 
itching? And can't this reaction to an irritation continue in the 
same way for a long time before a cure for the itching is discovered?
\citep{wittgenstein1984culture}
\end{quote}
\end{singlespace}

Many reflections on the findings of this dissertation have been planted in the
text where deemed appropriate. Here we attempt a synthesis of the
knowledge produced from the process of writing this dissertation and from the
dissertation itself. The common two-sex methods for age- or remaining-years
structured models tend to produce very similar results. However, remaining-years
and age-structured models are also incongruous, just as male and female
single-sex models produce incongruous results. We discuss the complimentary definitions 
of age later. 



