\FloatBarrier
\label{sec:reflections}

\todo{Chapter under development!}

\begin{singlespace}
\begin{quote}
 If somebody scratches the spot where he has an itch, do we have to see 
 some progress? Isn't genuine scratching otherwise, or genuine itching 
itching? And can't this reaction to an irritation continue in the 
same way for a long time before a cure for the itching is discovered?
\citep{wittgenstein1984culture}
\end{quote}
\end{singlespace}

Many reflections on the findings of this dissertation have been planted in the
text where deemed approriate. Here we attempt a synthesis of the
knowledge produced from the process of writing this dissertation and from the
dissertation itself.

What do we know that we did not know before?

The common two-sex methods for age-structured or remaining-years tend to produce
very similar results, and the common two-sex methods for
remaining-years-structured or remaining-years tend to produce very similar
results. However, remaining years models and age-structured models will
produce different results, just as male and female single-sex models produce
different and conflicting results. We will reflect on the conflicting
definitions of age later. In order to account for sex divergence in
projections or self-contained models, of those methods treated in
this dissertation, we recommend from the following three adjustment strategies:

\begin{enumerate}
\item The weighted-dominance method of \citet{goodman1967age}, probably with
a dominance parameter of $0.5$, such that equal information is taken from male and female
weights. This method wins on ease of implementation, and it produces reasonable
reasults for population structures typically observed (i.e. without 0s in
reproductive ages). It is also containable in a static Leslie matrix, although
we did not explain this construction for age-structured two-sex populations. The
method has a drawback in that the model itself is less appealing, as it does 
not allow for proper interaction between the sexes, or between ages.

\item The mean method, with the mean either set to harmonic, logrithmic,
geometric or an unnamed mean in that approximate range and that has the property
of falling to zero if one sex is absent. This method is appealing because the
male and female marginal fertility rates for a given year are determined
dynamically by what is happening in each age-combination of males and females.

\item The IPF method is the most flexible, as it incorporates inter-age
competition. That the method is iterative presents no real drawback, as any of
the above methods is best implemented in a scripted language, and the method is
not perceptibly slower than the alternatives. The properties of IPF are easy to
demonstrate but difficult to prove, and so it has received less attention from
mathematicians and ODE modellers. For the demographer designing a projection
aparatus, IPF is nonetheless a convenient choice. This method is in need of
comparison with the recent contribution from \citet{choo2006estimating}, which
has been show to satisfy the same axioms.
\end{enumerate}

Each of these three methods has advantages and disadvantages, and it is up to
the demographer to evaluate the optimal choice for the particular projective or
modelling scenario. All three of these methods work equally well in
age-structured and remaining-years structured projections. These three
adjustment techniques work just the same for projections that incorporate
assumptions about future developments in fertility. In this case, the
demographer makes assumptions about the male and female paths of fertility rate development and adjusts in each
iteration to force agreement in results. For the method based on a mean of age
(or remaining years) - specific combinations of male and female exposures, the
incorporation of sophisticated assumptions entails more care, as these must be
distributed over a matrix. If this proposition is inconvenient, one may prefer
either IPF or dominance weighting, which rely on on marginal fertility
distributions for adjustments beyond the initial year.




We have shown that populations may be projected while structured by
remaining years of life. We have also shown that if oneThese considerations
transfer to the case of remaining-years structured populations as well. 

To the extent that fertility rates and the sex ratio at
birth vary along the path to stability, one may wonder whether any of the
\textit{interactive} two sex models are at odds with the notion of invariance 
in stable populations. Once in the state of stability, of course, both population 
structure and marginal male and female fertility rates are invariant, which
means that the two-sex problem itself vanishes. In any of the interactive models
