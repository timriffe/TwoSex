
We have shown that populations may be projected while structured by
remaining years of life. We have also demonstrated that populations under each
of our two age-structures, calculated using data from one and the same
population, even one and the same sex, imply different results. Specifically, for the data used in this
dissertation, remaining-years structured data tend to imply higher
reproductive values ($r$, $R_0$, TFR, and so on) than age-structured data. At
first glance, one would expect the direction of difference to be negative 
rather than positive because we took the simplest possible course of
redistributing population counts (or exposures) from all age classes rather than
only from age classes in ages typically identified as reproductive ages. The
outcome is positive because the age-distributions of marginal birth counts and
population counts are different, but redistributed by the same lifetable death
distribution. We ought not expect differing initial distributions to wind up
proportionally in the same remaining years classes. Hence, the sum of these
ratios (events to exposures) over remaining years is also different. TFR, for
instance, is higher for remaining-years structured data than the age-equivalent
measure because on average there is less exposure per birth in the newly
redistributed data.

One may question whether there is any sense in allowing non-reproductive ages
into remaining-years exposures, and to this we have two responses. The first
comes by manner of analogy to age-specific fertility distributions, which
have tails that are longer than most people feel comfortable imagining, 
both for males and females. For males and
females there are so-called central ages of reproduction and there are
less-common ages of reproduction. Even for ages typically measured, say
45-49 for females, not all persons captured in these exposures are truly
exposed to the risk of fertility, and everyone knows this. Indeed a minority
of females in this age group are truly at risk of fertility. This does not imply
that the rate calculated is invalid, but rather that the rates are both low and
potentially subject to further decomposition. This is one reason why a large
portion of fertility studies deal with the measurement of the proximate determinants of fertility-- these
factors differentiate exposure for purposes of calculating more specific rates.
Age-specific rates are valid without such differentiation, as they help purge
our measurement of distortions from a certain degree of population structure. 

As \citet{stolnitz1949recent} so eloquently describe, demographic rates are
never fully purged of population structure. We may nonetheless \textit{de}structure in
any of myriad ways, and remaining-years structure is the way that we have shed
light upon in this dissertation. No claim is made about whether chronological or
thanatological age are the more efficient classifying variable for population
data, nor that one is more pure than the other. In sum, we have in this
dissertation provided yet another alternative structuring variable, and if one
questions the validity of a rate calculated on the basis of blended age groups,
one may ask: 1) how many persons are expected to die in $y$ years and 2) how
many births this year were to persons expected to die in $y$ years, and there we
have everything needed to calculate a rate. This is what we have done. Further
refinements are possible and are worthy of exploration.

The second response to the potential criticism about overly permissive
heterogeneity within remaining years classes for purposes of producing
fertility rates is that these rates nonetheless appear to obey a
certain degree of empirical regularity. The range of shapes possible for
fertility rates by remaining life expectancy, at least absent a mortality
crisis, is narrow. This we confidently claim, but the claim is in need of
further empirical support beyond this dissertation. The same observation may be made of
the underlying population structure by remaining years of life-- the range of 
shapes that we have observed this structure to obtain in Western countries, 
excluding mortality crises, is
narrow. These observations compel one to conclude that
remaining-years-specific fertility rates ($e$SFR) describe a substantive
underlying phenomena. If age-heterogeneity in remaining years classes 
were problematic in this sense, we would expect more erratic patterns in
$e$SFR over time and between populations, but this is not so. Instead, $e$SFR
invites comparison between populations and over time, and such comparisons are
decomposible into mortality and fertility components. We have not
taken this latter exploratory step, but this remains an exciting question, since
a stable $e$SFR pattern may hold over time even as fertility age patterns and
mortality levels change.

Most empirical findings arising from this thesis are based only on data
from the years 1969-2009 for the United States and 1975-2009 for
Spain\footnote{The primary exception is the relative \textit{stability} of
observed remaining-years-structured populations versus age-structured
populations, which we verified using the entire HMD.}. Evidently, any
novel findings produced therefrom, especially those dealing with fertility, 
are in need of verification from a wider array of
populations, so as to be placed into context. This is to say that there is ample opportunity to test
and refine the hypotheses produced here. Certainly $e$SFR will move over time,
but we suppose that this movement is meaningful and will aid our understanding
of population processes.

Net of the potential criticism of age-heterogeneity in remaining-years classes,
there are reasons to suspect that fertility timing and levels might also be
a function of remaining years of life. Although one's own age at death is
unknown, individuals do have a \textit{sense} of their own remaining life
expectancies, and this may condition fertility. This is not to say that people 
predict their own lifespans without bias. \citet{payne2013life}, for example,
concludes that the framing of the very questions used to asses subjective
remaining life expectancy in survey respondents can bias results greatly. This
source of bias has more to do with eliciting respondents' beliefs and
perceptions, although there are also biases in subjective predictions that vary
over age. The important thing is that people's subjective expectations of
remaining life \textit{correlate} positively with real remaining life
expectancies, and much evidence has been produced that would support
this \citep{hurd1995evaluation, mirowsky1999subjective, hurd2002predictive,
perozek2008using, delavande2011differential, post2012longevity}. These authors
tend to explain individuals' predictive power in terms of awareness of
their own particular genetic and environmental situation. Most people
understand the general mortality increase over age, and are able to
weigh these factors out to produce a rough idea of future death 
probabilities in certain age intervals. These findings have largely come from
health and retirement surveys of older persons beyond reproductive ages. Of
course, it is difficult to measure how well persons in \textit{reproductive}
ages are at predicting their own ages at death because panels would need to
run over a very long range of years in order to asses the accuracy of
predictions of the relatively young. 

Another particular variety of finding that lends support to our case are studies
of risky behavior in youth. \citet{borowsky2009health} show that risky behavior
in teenagers increases with high perceived risks of early death, and these
behaviors include risky sexual behavior. \citet{wilson1997life} conclude that
high real mortality risks for youth due to \textit{perceived} factors such as 
accidents, suicide and homicide, predict reductions in the age of fertility, which at
least on the surface, is consistent with the idea of $e$SFR actually describing 
some sort of underlying pattern. 

None of the aforementioned studies provide
direct support of our hypothesis that we ought expect that fertility behavior is
somehow a function of remaining life expectancy, but they do help our case.
The kind of data that would help shed direct light on this question would be a
panel study that includes questions about expected probabilities of survival
until (or death by) certain future ages, as well as the usual battery of 
fertility questions. This could be incorporated, for example, in future waves of
longitudinal youth surveys. To suit our ends directly, this survey would follow
up with actual ages of death. Evidently, much patience will be required to reap
any results in this direction, and there is much uncertainty at this preliminary
stage (we are only imaging such a survey at the moment) about whether results
would provide an estimate of true remaining-years fertility curves, and whether
foresight of one's own mortality moves this curve. Removing subjective
remaining life expectancy from our demands, one may reconstruct cohort $e$SFR on 
the basis of long-running population registers from countries such as Denmark or
Sweden.

Absent a well-rounded base of psychological, sociological and demographic
evidence in support of the belief that fertility patterns ought be a function of
remaining years of life (as well as a function of age), we rely on our own
set of evidence that simply show these patterns to be stable. Stable demographic
patterns are desirable, as they are useful for predicting. We encourage
the exploration of the predictive power of remaining-years-specific
demographic rates, just as economists have been keen to look for patterns in
retirement saving and investment as a function of perceived remaining years of
life.
