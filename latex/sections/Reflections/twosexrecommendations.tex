
Population projections rarely incorporate males into fertility assumptions. This
assumption, that fertility depends only on females, is called female-dominance,
and in Chapter~\ref{ch:Measuring} we have shown that female dominance is not an
innocuous assumption for the case of age-structured models. In
Chapter~\ref{sec:exstructuredrenewal} we showed that female dominance is also
not an onnocuous assumption for the case of remaining-years structured
populations. In Chapter~\ref{sec:modelingapproaches} for chronological age and
the chapters of Part 3 for thanatological age, we set out to provide a suite of
standard working examples of solutions of two-sex adjustment methods so that 
methods may be compared and implemented by the demographer if desired. This 
is a practical concern that has implications on the way
contemporary demography is practiced. In the this section we provide some
guidance by recommending from among the methods presented.

As mentioned previously, the two-sex problem comes to the fore when projecting
population into the future. The most common practice to avoid disagreement between the sexes is to
assume female dominance, which characterizes the female single-sex Lotka
renewal model, and is a degenerate case of the
\citet{goodman1967age} two-sex model from Section~\ref{sec:googmanage} with the
dominance parameter set to accept 100\% of model information from female fertility rates. 
There are very few population projections produced for consumption beyond
two-sex studies themselves that do not make this assumption. Assuming female dominance
in one way or another is so widespread that newcomers to demography often 
perceive it as a given, or in the worst case adopt the
practice dogmatically. For others, the implementation is too complex or
else the requisite transition rates are not available. Both of these later
two obstacles may be overcome by removing nuptial states from the model
altogether, as we have done in this dissertation. 

Formal demographers have long taken the sex-discrepancy seriously, at
least for purposes of consistent model design. These mathematical models
have been the driving force behind this dissertation effort. We have at
times simplified two-sex models from their original form. One offshoot of
this choice is that our palette of models may be of potential use in applied
demography. The \texttt{R} code used to produce our results should be recyclable 
or else may easily be quarried to such ends. It therefore behooves us to recommend from
amongst the methods explored in this dissertation. In order to account for sex
divergence in projections or self-contained models, of those methods treated in this dissertation, 
we recommend from the following three adjustment strategies:

\begin{enumerate}
\item The weighted-dominance method of \citet{goodman1967age} from
Section~\ref{sec:googmanage} and Chapter~\ref{sec:ex2sexdomweights} is a
reasonable choice, probably with a dominance parameter of $0.5$, 
such that equal information is taken from male and female
weights. This method wins on parsimony, ease of implementation and it has the
simplest data inputs. It produces reasonable results for population
structures typically observed (i.e. without 0s in reproductive ages). It is also 
containable in a static Leslie matrix, although
we only explained this construction for populations structured by
thanatological age. The method has a drawback in that the model itself is less
appealing, as it does not allow for proper interaction between the sexes, or
between ages. However, for purposes of production age-sex population projections
within a 40-year horizon, the simplest model is well worth consideration in our opinion.

\item The mean method from Sections~\ref{sec:ageharmonic} and
\ref{sec:ex2sexschoen} for age and remaining-years structured models
respectively, is also a reasonable choice. In this case, we recommend
implementation with the mean either set to harmonic, logarithmic, geometric or
an unnamed general mean in that approximate range (with the property of 
falling to zero if one sex is absent). We opine that there is no point in
quibbling over which of these means is best, as observed human populations do
not exhibit the extreme sex ratios required to produce results that are
meaningfully different. If one were to segment the population further into
groups on which partner preference occurs, then more extreme sex ratios would
be possible, but then the whole model design also becomes more complex. This
method is appealing because the male and female marginal fertility rates 
for a given year are determined dynamically by changes in each 
age-combination (remaining-years combination) of males and females, and the
range of means listed here allows for some degree of \textit{bottlenecking} due
to the minority sex in a particular combination. The model is also of
parsimonious design, easy to understand, and straightforward to implement. This
model does not allow for competition or substitution between ages (remaining
years).

\item The iterative proportional fitting method (IPF) is the most flexible of
these three, because it incorporates competition and substitution between ages (remaining
years). That the
method is iterative presents no real drawback, as any of the above methods is best 
implemented in a scripted language, and the method is
not perceptibly computationally slower than the alternatives. The properties of
IPF are easy to demonstrate but difficult to prove, and so it has received less attention from
mathematicians and continuous equation modelers. For the demographer designing
a projection apparatus, IPF is nonetheless a convenient choice. This method is in need of
comparison with the recent contribution from \citet{choo2006estimating}, which
has similar properties.
\end{enumerate}

Each of these three methods has advantages and disadvantages, and it is up to
the demographer to evaluate the optimal choice for the particular projective or
modeling scenario. All three of these methods work equally well in
age-structured and remaining-years structured projections. Further, these three
adjustment techniques work just the same for projections that incorporate
assumptions about future developments in fertility. In this case, the
demographer makes assumptions about the male and female paths of fertility 
rate development and adjusts in each
iteration to force agreement in results. For the method (2) based on a mean of
age- (or remaining-years) specific combinations of male and female exposures,
the incorporation of sophisticated assumptions entails more care, as these must be
distributed over a matrix. If this proposition is inconvenient, one may prefer
either IPF or dominance weighting, which rely only on marginal fertility
distributions for adjustments beyond the initial year.

Further, we have demonstrated that mean-based (2) and IPF (3) methods each
entail changes to the marginal male and female rates after adjustment, and one
may wonder whether these adjusted rates have any predictive power per se-- for
instance adequately adjusting for foreseen changes in population structure--, or
whether they are a modeling artifact to be disregarded in favor of the total 
(unstructured) birth count. We have demonstrated that this feature exists,
and we have shown instances where the two methods make predictions that are at
odds. We do not follow up this observation with an empirical comparison
to check whether (2) or (3) hit the mark closer in terms of
fertility distribution prediction. This therefore remains an intriguing question
that has been ignored thus far and that could tip the balance in favor of
preferring one of these two methods over the other. A priori, we expect IPF (3) 
to display more appropriate sensitivity amidst abrupt
changes in cohort size, but we do not know whether the magnitude and
distribution of adjustment is appropriate. 

Of course, the demographer may also compare with two-sex models not treated
in this dissertation, incorporate nuptial states into the model, and so
forth. In this case, the two-sex method is transferred to nuptiality
(match-making, pairing, marriage) as the event being predicted, but the
adjustment procedures are one and the same. Such a projection would
entail more sophisticated construction, more data inputs, and the incorporation of more hypotheses,
namely hypotheses about changes in marriage rates as well as changes in marital
and extramarital fertility. For populations with high proportions of
extramarital fertility, extra data are required to approximate the formation of
non-marital mated pairs, e.g., transitions into and out of cohabitation, as well
as fertility fertility rates that apply to this subpopulation (and mortality
rates if supposed different). This is to say that the addition of further state
considerations to fertility assumptions greatly increases the model complexity
and data requirements and so is not likely to be appealing to projection
designers unless the states themselves are necessary and interesting.

To the extent that fertility rates and the sex ratio at
birth vary along the path to stability, one may wonder whether any of the
\textit{interactive} two sex models are at odds with the notion of
rate invariance in stable populations. Once in the state of stability, of
course, both population structure and marginal male and female fertility rates are invariant, which
implies that the two-sex problem itself vanishes. In this case, for both the
mean and the IPF methods, the stable adjusted marginal fertility rates become
invariant, and the male and female rates yield the same results, making the
population tautologically \textit{dominance-indifferent}. In any of the
interactive models the element held fixed prior to stability are not rates, but
some standard. For IPF, the element held constant in our 
description is the original cross-classified birth matrix and the corresponding 
male and female marginal rates. For the mean-based method, one holds constant 
the standard rate matrix, as well as the mean function itself, but the marginal 
rates produced by these standards have been shown to indeed change over time
under these modeling assumptions.




