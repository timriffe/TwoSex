\FloatBarrier

\begin{singlespace}
\begin{quote}
 If somebody scratches the spot where he has an itch, do we have to see 
 some progress? Isn't genuine scratching otherwise, or genuine itching 
itching? And can't this reaction to an irritation continue in the 
same way for a long time before a cure for the itching is discovered?
\citep{wittgenstein1984culture}
\end{quote}
\end{singlespace}

It should be clear enough that we have not \textit{solved} the two-sex problem
in the sense of laying the issue to rest once and for once all. With some
luck we will have succeeded in making the problem more tangible for some,
accessible to the extent that results have been made reproducible,Such a claim
would be misplaced and unjustified. Rather, in exploring a subset of



In the name of headway on the two-sex problem, the
present work took a deviation into much hitherto unexplored territory. Indeed,
demographers have known that this territory existed for some time, but have
only scratched its surface by way of approximate indicators. Certainly none had
ever explored the consequences for fertility of thanatological age, and none had
ever reframed demography's most enduring population models in terms of
thanatological age. As a result of this latter excercise, the two-sex problem
has not been solved, but rather been shown in a new light. Further, a new discrepancy 
has been uncovered, namely divergence between the age-structured and $e_y$-structured 
renewal processes, which is oftentimes much greater than divergence between 
the sexes per se. This author makes no claims about the legitimacy of 
the $e_y$-structured renewal model(s) presented here- only of their 
consistency and potential utility. 

As with the two-sex case, there are no
discrepancies per se in the year from which data were used to derive vital
rates, only in future projected years. As such, neither \textit{problem} deals with lived populations, but rather of
modelled populations. Why complicate the practice of population modelling with yet
another perspective? Might it not be the case that transforming a given
population to the $e_y$-perspective is fruitful, but not projecting one? At this
time, results fresh, and we may best wait and contemplate before coming to rash
conclusions. Upon the first illustrations of the two-sex problem (e.g.
\cite{kuczynski1932fertility}), the practice of demography centered itself on
the assumption of female dominance, whereas the science of demography took up
the hobby, still very much alive\footnote{At least one publication on the
topic have been released so far in 2013 at the time of this writing (\citet{Matthews2013})}



