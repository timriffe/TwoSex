Suggestions for future research have been offered throughout this dissertation
when it seemed appropriate. Here we summarize such recommendations into a
well-rounded research agenda. Most of these lines of research stem
from our formalization of the remaining-years perspective, the novel aspect of
this dissertation. Due to the definition of remaining-years population
structure, research areas already interested in the remaining-years perspective,
but without formally recognizing this tool, will yield more well-founded
results. These include studies thatdeal directly with
population structure, such as studies of dependency and population aging. 

In Section~\ref{sec:exageing} we defined two exceedingly simple indices
that derive from the remaining-years population structure. A worthy task will
be to more fully relate these two indices to already-existing indices that
attempt to measure the same underlying quantity. A second aspect of this
research direction is to derive descriptive results from these indicators,
and examine whether any conclusions change from our present state of knowledge.
These issues are of inherent public, economic, and policy-making interest,
and thus it is worth grooming the graphical tool used to communicate this new
kind of population structure: the population pyramids as depicted in
Section~\ref{sec:expopstruct1} and elsewhere in this dissertation, which we
think will make this material palatable to a wider audience. Specifically, the
visualization of a remaining-years pyramid communicates more about the future
than an age-classified pyramid, and so circulation of such images (and ensuring
that they are interpreted correctly) would likely be more useful to
non-demographer policy-makers. For instance, the baby-boomer aging bubble is rather smoothed and
absorbed in the remaining-years pyramid. 

Other aspects of this redistribution method in need of further research are
conceptual design decisions regarding whether the method is best carried out
using the static period deaths distribution (i.e., finding greater utility as a
period indicator itself), or whether the redistribution should be treated as a
 projection and should therefore take mortality improvements into account.
In the latter case, it will be necessary to simultaneously depict uncertainty in
the resulting population structure. In
Sections~\ref{sec:structuncertainty} and~\ref{sec:eximprov} we offered
preliminary work in these two directions, but this preliminary work is ripe for
greater formalization and application.

Also a bridge beyond applied demography, in Section~\ref{sec:fisher} we
defined a remaining-years version of Fisher's reproductive value. We
have given the indicator, but have not related it to the existing foundation of
biological and life course theory that has been based on Fisher's reproductive
value. The question is left begging whether the timing of certain
life course transitions such as menopause, or the existence of curiosities such 
as the human post-reproductive lifespan -- (which have been determined by
evolution) -- are best related to the remaining-years transformed reproductive
value.

Of interest to demography in general, but perhaps especially to evolutionary
demographers, is our definition of remaining-years fertility rates, $e$SFR. We
have claimed many times that these fertility rates exhibit
regularity and are likely meaningful. However, whether such rates have a good
substantive interpretation is a matter for discussion beyond the confines of
this dissertation. In order to stimulate such discussion we hypothesize that fertility rates are a
function of time until death in two ways. 

First, to the extent that fertility is
a volitional demographic phenomenon, and to the extent that individuals have a
sense of their remaining years of life\footnote{In
Section~\ref{sec:esfrreflections} we list several studies suggesting that people
do have a rough sense of their remaining years of life.}, there may be a calculus of fertility that adjusts to
this subjective estimate. The direction of this calculus need not be simple. One
could advance reproduction in the face of a likely early death, so as to ensure
procreation. One could advance fertility despite foreseeing death far in the
future so as to ensure maximal years of overlapping life with offspring (to
ensure that offspring themselves survive to reproductive ages). One could
postpone fertility in foreseeing death far in the future so as to accumulate resources prior to
reproduction. Clearly saving is a function of one's foreseeable years
remaining until death, but also of projected childbearing, and so forth. There
is ample room for exploration of the volitional aspects of fertility and
subjective remaining life expectancy. In any case, this hypothetical
relationship will be reflected in remaining-years structured fertility only to
the extent that one's personal projection of time until death is roughly
accurate. 

Second, there may be subconscious population-endogenous mechanisms as work.
Such a mechanism is difficult to define, and is motivated by the
observation (not presented in this dissertation) of remarkable stability in the shape of remaining-years population pyramids over long time series in certain
populations. The hypothesis is that the population \textit{attempts} to 
maintain a particular shape with respect to remaining years of life, and one of
the levers in this process is the fertility level rather than shape. This could 
just as likely be an analogy, however; fertility rates have
long been hypothesized to adjust after mortality changes, and the uncanny
stability of remaining-years population structure could be an artifact of these
other mechanisms. 

The majority of this dissertation has yielded formal results from
two-sex population renewal models. The age-structured models (or some variant
of them) already existed in the literature, but the remaining-years structured
models did not. In the first place, the single-sex model from
Section~\ref{sec:ex2sexequation} must be complemented with a mathematical proof 
of a unique real solution, but otherwise these formal results are ready to ship
in a self-contained formal article. The corresponding projection matrix is also
well-defined, and completes the product, although it has as-yet 
unexplored properties that would best be described apart. While the two-sex
extensions are of interest, of more immediate concern is an explanation for
the discrepancy between results from chronological and thantalogical age. A good
place to start is the admittedly superficial solution of simply summing the
chronological-age and thanatological-age renewal models and optimizing for the
value of $r$ that makes the two models sum to two. This value of $r$ will be
intermediate, of course, but it also sheds no light on the problem. The same
solution would work for any of the analogous two-sex models that we
present in this dissertation. Of course, a real solution could be derived from
this species of musing.

Also unexplored in this dissertation are the kinds of aspects of the
remaining-years model that \citet{caswell2001matrix} describes for age and/or
stage-structured matrix population models. Specifically, in
Section~\ref{sec:ex2sexdomweightsstabstruct} we explored the transient 
dynamics of the remaining-years model in only the most summary ways. A more
detailed examination of the path to stability as compared with that of the analogous
age-structured model will be informative. Further, we have undertaken no
sensitivity tests, nor have we examined the elasticity of these models.
Our models have been deterministic, and there are most certainly
stochastic extensions of these models to be designed and explored. In exploring
these aspects of the thanatological reproduction model we propose,
comparisons alongside like analyses of the age-structured model will be
informative and add to our understanding of population models and of population
dynamics in general. As is typical of theoretical work, we
have succeeded in producing more questions than we have answered. We believe
that the new questions are good ones and invite demographers to consider the
material we propose.





