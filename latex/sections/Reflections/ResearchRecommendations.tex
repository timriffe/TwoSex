Suggestions for future research have been made throughout this dissertation when
this seemed appropriate. Here we summarize these recommendations into a
well-rounded research agenda. Most of these lines of research stem
from our formalization of the remaining-years perspective, the novel aspect of
this dissertation. Due to the definition of remaining years population
structure, research areas already interested in the remaining-years perspective,
but that did not formaly recognize this tool, will yield more well-founded
results. These include those studies that most often deal directly with
population structure, such as studies of dependancy and population aging. 

In Section~\ref{sec:exageing} we defined two exceedingly simple indices
that derive from the remaining-years populaiton structure. An urgent task will
be to more fully relate these two indices to already-existing indices that
attempt to measure the same underlying quantity. A second aspect of this
research direction will be to derive descriptive results from these indicators
and examine whether any conclusions change from our present state of knowledge.
These issues are of inherent public, economic, and policy-making interest, thus
it will be worth grooming the graphical tool used to communicate this new kind
of population structure -- the population pyramids as depicted in
Section~\ref{sec:expopstruct1} and elsewhere in this dissertation, which we
think will make this material palatable to a wider audience. Specifically, the
visualization of a remaining-years pyramid communicates more about the future
than an age-classified pyramid, and so circulation of such images (an ensuring
that they are interpreted correctly) would appear more useful to non-demographer
policy-makers. For instance, the baby-boomer aging bubble is rather smoothed and
absorbed in the remaining-years pyramid. 

Other aspects of this redistribution method in need of further research are
conceptual design decisions regarding whether the method is best carried out
using the static period deaths distribution, i.e., finding greater utility as a
period indicator itself, or whether the redistribution should be treated as a
 projection and should therefore take mortality improvements into account.
In the latter case, it will be necessary to simultaneously depict uncertainty in
the resulting population structure. In
Sections~\ref{sec:structuncertainty} and~\ref{sec:eximprov} we offered
preliminary work in these two directions, but this preliminary work is ripe for
greater formalization and application.

Also a bridge beyond applied demography, in Section~\ref{sec:fisher} we
defined a remaining-years oriented version of Fisher's reproductive value. We
have given the idicator, but we have not related it to the existing base of
biological and lifecourse theory that has been based on Fisher's reproductive
value. The question is left begging as to whether the timing of certain
lifecourse transitions such as menopause or the existence of certain
lifecourse curiosities such as the human post-reproductive lifespan, that have
been determined by evolution is best related to the remaining-years transformed 
reproductive value. 

Of interest to demography in general, but perhaps especially to evolutionary
demographers, is our definition of remaining-years fertility rates, $e$SFR. We
have claimed many times in this dissertation that these fertility rates exhibit
regularity and are likely meaningful. Whether such rates have a good substantive
interpretation is a matter in need of discussion beyond this dissertation.
In order to stimulate such discussion we hypothesize that fertility rates are a
function of time until death in two ways. 

First, To the extent that fertility is
a volitional demographic phenomena, and to the extent that individuals have a
sense of their remaining years of life\footnote{In
Section~\ref{sec:esfrreflections} we list several studies that suggest that people do have a rough sense of their
remaining years of life.}, there may be a calculus of fertility that adjusts to
this subjective estimate. The direction of this calculus need no be simple. One
could advance reproduction in the face of a likely early death, so as to ensure
procreation. One could advance fertility despite forseeing death far in the
future so as to ensure maximal years of overlapping life with offspring to ensure 
offspring's survival to reproductive ages. One could postpone fertility in
forseeing death far in the future so as to accumulate resources prior to
reproduction. Clearly saving is a function of one's foreseeable years
remaining until death, but also of projected childbearing, and so forth. There
is ample room for exploration of the volitional aspects of fertility and
subjective remaining life expectancy. In any case, this hypothetical
relationship will only be reflected in remaining-years structured fertility to
the extent that one's personal projection of time until death is roughly
accurate. 

Second, there may be subconscious population-endogenous mechanisms as work. This
latter mechanism is difficult to define, and it is motivated by the observation
(not presented in this dissertation) of remarkable stability in the shape of
remaining-years population pyramids over long time series in certain
populations. The hypothesis here is that the population \textit{attempts} to 
maintain a particular shape with respect to remaining years of life, and one of
the levers in this process are fertility levels rather than shape. This
hypothesis could just as likely be an analogy, however, as fertility rates have
long been hypothesized to adjust after mortality changes, and the uncanny
stability of remaining-years population structure could be an artefact of these
other mechanisms. 

The majority of this dissertation has yielded formal results from
two-sex population renewal models. The age-structured models (or some variant
of them) already existed in the literature, but the remaining-years structured
models did not. In the first place, the single-sex model from
Section~\ref{sec:ex2sexequation} must be complemented with a mathematical proof 
of a unique real solution, but otherwise these formal results are ready to ship
in a self-contained formal article. The corresponding projection matrix is also
well-defined, and will complete the product, although it has as
yet unexplored properties that would best be described apart. While the two-sex
extensions are of interest, of more immediate concern is an explanation of
the discrepancy between results from chronological and thantological age. A good
place to start will be the admittedly superficial solution of simply summing the
chronological-age and thanatological-age renewal models and optimizing for the
value of $r$ that makes the two models sum to two. This value of $r$ will be
intermediate, of course, but it also sheds no light on the problem. The same
solution would work for any of the analagous two-sex models that we have
presented in this dissertation. Of course, a real solution could be derived from
this species of musing.

Also unexplored in this dissertation are the kinds of aspects of the
remaining-years model that \citet{caswell2001matrix} describes for age and/or
stage-structured matrix population models. Specifically, in
Section~\ref{sec:ex2sexdomweightsstabstruct} we have only explored the transient 
dynamics of the remaining-years model in the most summary ways. A more detailed
examination of the path to stability as compared with that of the analagous
age-structured model will be informative. Further, we have undertaken no
sensitivity tests. Nor have we examined the elasticity of these models.
Furthermore, our models have been deterministic, and there are most certainly
stochastic extentions of these models to be designed and explored. In exploring
each of these aspects of the thanatological reproduction model we propose,
comparisons alongside like analyses of the age-structured model will be
informative and add to our understanding of population models and population
dynamics in general.





