
Only a single step is required to incorporate thanatological age
instead of chronological age into a population renewal
model, but there are many ways to incorporate both sexes into the fertility
component of a population renewal model, and so the two-sex component 
requires extra attention under either kind of age-structure. Consideration
of two-sex solutions has occupied a majority of this dissertation, and explains the first part of 
the title: ``the two-sex problem.'' The primary
novelty of this dissertation, however, has been to do demography in general --
and model population renewal in particular -- using thanatologically structured
demographic data. The latter has been our objective and the former
an intermediary, but we are not sure which will prove of to be greater utility
beyond this dissertation. In producing
two-sex models under this new structure, we have concluded that the same two-sex 
strategies that work for age-structured populations also work for
remaining-years structured populations with much the same properties.

We have also concluded that the year-to-year stability of observed
thanatological age-structure, using our redistribution method, is
greater than that of the same population structured by
chronological age. This finding is valid for the four decades of US and
Spanish population data used throughout this dissertation, and has been further
confirmed on the basis of all 46 populations in the Human Mortality Database.
The same kind of stability might hold for fertility rates as well, but this
remains to be confirmed for other periods and populations. One of the
implications of this finding, however, is that the pace of divergence between
male and female predictions of future births is slower -- we observe less such
divergence because of the relative stability in population structure by
remaining years. This means that the gap in total births predicted on the basis of male and female 
fertility rates and future exposures is smaller when these rates and exposures 
are structured by thanatological age than when structured by chronological age. However,
sex-divergence in predictions is not eliminated, and both sexes should still
be considered together when modeling or projecting populations structured
by remaining years of life.

Of equal or greater importance in this dissertation is the finding that models
of population renewal yield incongruous results when structured by
chronological versus thanatological age. This is a new problem to 
which we have offered no solution, and from which we hope to inspire future
formal demographic work. The only exceptions to this incongruity are rare
coincidences and the tautological cases of the initial population state and 
the theoretical stable state. We can therefore make a manner of recommendation
for future research toward 1) refining the remaining-years-structured population
model where necessary, 2) reconciling the conflicting results obtained from
these two definitions of age and 3) reconciling both sexes and both
definitions of age in a single model of population renewal. We do not claim that
it will be possible to model the two definitions of age together in a true and
necessary way, but we expect that the attempt to do so will surrender insights
into population dynamics, as has been the case with the long history of two-sex
modeling.

In the following section we make several recommendations regarding two-sex
methods. In the final section we outline a set of specific future research
directions based on the results of this dissertation.

