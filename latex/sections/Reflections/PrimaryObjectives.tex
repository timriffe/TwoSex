
In this dissertation we have aimed to investigate and compare models of
population growth. This we have done in the traditional way, modeling populations 
structured by both age and sex in Part 1. We then replaced age-structure with remaining-years
structure for population renewal modes in Parts 2 and 3. Remaining-years
structure is a new idea, and so the implications for the practice of demography are many. Part
2 laid the groundwork for a demography based on remaining-years
structure, and then reformulated the canonical single-sex model of population
growth under this new structure. A guiding principle has been that
population renewal models thoughtfully incorporate both sexes. 

Only a single step is required to incorporate chronological age versus
thanatological age in a population renewal model. There are many ways to incorporate both sexes
into the fertility component of a population renewal model, and so the 
two-sex component requires great consideration under either
variety of age-structure. The consideration of two-sex solutions has occupied
the majority of this dissertation, and explains the first part of 
the title: ``the two-sex problem.'' The primary
novelty of this dissertation, however, as been to do demography in general, and
model population renewal in particular, using thanatologically structured
demographic data. The latter achievement has been our objective and the former
an intermediary, although we are not sure which will prove of greater utility
beyond this dissertation. Part 3 rounded out our investigation by producing
two-sex models under this new structure. We have concluded that the same two-sex 
strategies that work for age-structured populations also work for
remaining-years-structured populations with much the same properties.

We have also concluded that the year-to-year stability of observed
thanatological age-structure, using our redistribution method, is
greater than that of the same population when structured by
chronological age. This finding is valid for the four decades of US and
Spanish population data used throughout this dissertation, and it has been
confirmed on the basis of all 46 populations in the Human Mortality Database.
One of the implications of this finding is that the pace of
divergence between male and female predictions of future births is slower. This
means that the gap in total births predicted on the basis of male and female fertiltiy rates
and future exposures is smaller when these rates and exposures are structured
by thanatological age than when structured by chronological age. However,
sex-divergence in predictions is not eliminated, and both sexes still ought to
be considered together when modeling or projecting populations structured
by remaining years of life.

Of equal or greater importance in this dissertation is the finding that
population renewal models yield incongruous results when structured by
chronological versus thanatological age. This is a new problem to 
which we have offered no solution and from which we hope to inspire future
formal demographic work. The only exceptions to this incongruity are rare
coincidences and the tautological cases of the initial population state and 
the theoretical stable state. We may therefore make a manner of recommendation for
future research toward 1) refining the remaining-years-structured population
model where necessary, 2) reconciling the conflicting results obtained from
these two definitions of age and 3) reconciling both sexes and both
definitions of age in a single model of population renewal. We do not claim that
it will be possible to model the two definitions of age together in a true and
necessary way, but we expect that the attempt to do so will surrender insights
on population dynamics, as has been the case with the long history of two-sex
modeling.

In the following section we make some recommendations regarding two-sex methods.
We then address some doubts on the validity of fertility rates structured by
remaining years of life that may have arisen in the course of reading Part 2 of
this dissertation, where remaining-years structure and the
single-sex model were introduced. Finally, we discuss some technical doubts that
may have arisen in the reading of Part 3 of this dissertation, as well as
several areas in need of future research. As is typical of theoretical work, we
have suceeded in producing more questions than we have answered. We believe that
the new questions are good questions and invite demographers to consider the
material we propose.

