
\subsection{Formal results}

\subsubsection{The basic transformation}
The basic transformation underlying all \textit{original} findings in this
dissertation is that of redistributing data classified by chronological age into
remaining years classes, or thanatological age, by means of the period
lifetable (see Equation~\eqref{eq:dxredist} on page~\pageref{eq:dxredist}). This
feat is note genuinely new, as all the parts to the requisite formula were 
already present in the literature, and indeed \citep{miller2001increasing} had
already used the basic component to this transformation in forecasting health
expenditures. The tiny step of integrating over age, so as to arrive at data
\textit{structured} by remaining years of life appears to be new. Further, the
extention of this principle to various kinds of age-structured data, especially
fertility data, but also population stocks in general, is also new.

The graphical device that falls out of this transformation,
akin to the age-structured population pyramid, the remaining years population
\textit{leaf}, is scattered throughout this dissertation. It is essentially the
same as a population pyramid and exhibits no particular innovation as far as
visual communication is concerned. The only difference for demographers will be
to imagine generations shifting downward with the passage of time instead of
upward. While the graphical device is not new, the content-- a picture of a
population reshuffled to transmit information about the death cohorts that would
unfold according to a given lifetable-- certainly is. The population leaf
transmits a much higher density of information than a line graph of a given time
trend in an ageing indicator. It is complementary to the population pyramid in
terms of understanding population ageing and has the potential the make


\subsubsection{Transformation of vital rates}
\subsubsection{The notion of thanatological renovation}
\subsubsection{Development of the single-sex model}
\subsubsection{Translation of two-sex perspectives}
\subsubsection{A new two-sex method for the thanatological perspective}

\subsection{Empirical findings}
All empirical findings arising from this thesis have been based only on data
from the years 1969-2009 for the United States and 1975-2009 for Spain.
Evidently, any novel findings produced therefrom are in need of verification
from a wider array of populations, so as to be placed into context. For the most
basic age-transformation, data are readily available from such
large-scale datasources as the Human Mortality
Database\footnote{\url{www.mortality.org/}}, the Human Lifetable
Database\footnote{\url{www.lifetable.de/}}, the Human Fertility
Database\footnote{\url{www.humanfertlity.org/}}, the World Health
Organization\footnote{\url{http://www.who.int/healthinfo/statistics/mortality/en/index.html}},
and the United Nations Population
Division\footnote{\url{http://esa.un.org/unpd/wpp/unpp/panel_population.htm}},
and other data sources. This is to say that there is ample opportunity to test
and refine many of the hypotheses or empirical findings presented here.
\subsubsection{Stability of thanatological population structure}
Populations structured by remaining years of life have been found 
\subsubsection{Population rejuvenation}
\subsubsection{Difference in estimated growth rates under two age definitions}
